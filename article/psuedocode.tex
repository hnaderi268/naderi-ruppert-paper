\documentclass[10pt]{article}
\renewcommand{\baselinestretch}{1.8}

\usepackage[b4paper,left=0.8in,right=0.8in,top=0.8in,bottom=0.8in]{geometry}
\usepackage{tikz-qtree}
\usepackage{algorithm}
\usepackage{algpseudocode}
\makeatletter
\renewcommand{\ALG@beginalgorithmic}{\footnotesize}
\makeatother
\usepackage{graphicx}
\usepackage{subcaption}
\usepackage{showkeys}
\usepackage{amsmath}


\usepackage{multicol}
\setlength\columnsep{24pt}

\algnewcommand\algorithmicforeach{\bf{for each}}
\algdef{S}[FOR]{ForEach}[1]{\algorithmicforeach\ #1\ \algorithmicdo}

\algdef{S}[FUNCTION]{Function}
   [3]{{\tt{\sl{#1}}} \textproc{\tt{#2}}\ifthenelse{\equal{#3}{}}{}{\tt{(#3)}}}
  
\algdef{E}[FUNCTION]{EndFunction}
   [1]{\algorithmicend\ \tt{\textproc{#1}}}

\algrenewcommand\Call[2]{\tt{\textproc{#1}\ifthenelse{\equal{#2}{}}{}{(#2)}}}
   
\newcommand\keywordfont{\sffamily\bfseries}
\algrenewcommand\algorithmicend{{\keywordfont end}}
\algrenewcommand\algorithmicfor{{\keywordfont for}}
\algrenewcommand\algorithmicforeach{{\keywordfont for each}}
\algrenewcommand\algorithmicdo{{\keywordfont do}}
\algrenewcommand\algorithmicuntil{{\keywordfont until}}
\algrenewcommand\algorithmicfunction{{\keywordfont function}}
\algrenewcommand\algorithmicif{{\keywordfont if}}
\algrenewcommand\algorithmicthen{{\keywordfont then}}
\algrenewcommand\algorithmicelse{{\keywordfont else}}
\algrenewcommand\algorithmicreturn{{\keywordfont return}}

\renewcommand\thealgorithm{}
\newcommand{\setalglineno}[1]{%
  \setcounter{ALC@line}{\numexpr#1-1}}

\newcommand{\sub}[1]{\textsubscript{#1}}
\renewcommand{\tt}[1]{\texttt{#1}}
\renewcommand{\sl}[1]{\textsl{#1}}
\renewcommand{\it}[1]{\textit{#1}}
\renewcommand{\bf}[1]{\textbf{#1}}
\newcommand{\cmt}[1]{\Comment{#1}}

\usepackage{amsmath,amssymb,amsthm}
\newtheorem{theorem}{Theorem}
\newtheorem{lemma}[theorem]{Lemma}
\theoremstyle{definition}
\newtheorem{definition}[theorem]{Definition}

\begin{document}

\begin{algorithm}
\caption{Fields description}
\begin{algorithmic}[1]
\setcounter{ALG@line}{100}
\begin{multicols}{2}

%\Statex \bf{Structure}

\Statex $\diamondsuit$ \tt{\sl{Shared}}
\begin{itemize}
\item \tt{\sl{Tree} tree} \textsf{: A binary tree of \tt{Node}s. \tt{root} is a pointer to the root node.}
%such that \tt{root} is the root and the left child and the right child and the parent of \tt{Tree[i]} are \tt{Tree[2i+1]}, \tt{Tree[2i+2]} and \tt{Tree[i/2]}.
\end{itemize}

\Statex

\Statex $\diamondsuit$ \tt{\sl{Local}}
\begin{itemize}
\item \tt{\sl{*Node} leaf:} \sf{ a pointer to the process's leaf in the tree.}
\end{itemize}

\Statex
\Statex $\diamondsuit$ \tt{\sl{Structures}}

\Statex $\blacktriangleright$ \tt{\sl{Node}}
\begin{itemize}
\item \tt{\sl{*Node} left, right, parent} \textsf{: initialized  when creating the tree.}
\item \tt{\sl{BlockList} blocks}
\item \tt{\sl{int} size= 1}\textsf{: \#\tt{block}s in \tt{blocks}.}
\item \tt{\sl{int} num\sub{propagated}= 0}\textsf{} \textsf{: \# groups of blocks that have been propagated from the node to its parent. Since it is incremented after propagating, it may be behind by 1.}
\item \tt{\sl{int[]} super}\textsf{: \tt{super[i]} stores an approximate index of the superblock of the blocks in \tt{blocks} whose \tt{group} field have value \tt{i}.}
\end{itemize}

\Statex $\blacktriangleright$ \tt{\sl{Root} extends \sl{Node}}
\begin{itemize}
  
\item \tt{\sl{PBRT} blocks}

  \textsf{\tt{BlockList} is implemented with a persistent red-black tree.}
  
\end{itemize}

\Statex $\blacktriangleright$ \tt{\sl{NonRootNode} extends \sl{Node}}
\begin{itemize}
  
\item \tt{\sl{Block[]} blocks}

  \textsf{\tt{BlockList} is implemented with an array.}
  
\end{itemize}

\Statex $\blacktriangleright$ \tt{\sl{Leaf} extends \sl{NonRootNode}}
\begin{itemize}
  
  \item \tt{\sl{int} \tt{last\sub{done}}}

  \textsf{Stores the index of the block in the root such that the process that owns this leaf has most recently finished the. A block is finished if all of its operations are finished. \tt{enqueue(e)} is finished if \tt{e} is returned by some \tt{dequeue()} and \tt{dequeue()} is finished when it computes its response. \it{put the definitions before the pseudocode}}
  
\end{itemize}
\pagebreak

\Statex $\blacktriangleright$ \tt{\sl{Block}} 
%\cmt{If \tt{b} is \tt{blocks[i](i!=0)} then \tt{b[-1] is blocks[i-1].}}
\cmt{For a \tt{block} in a \tt{blocklist} we define \it{the prefix for the block} to be the blocks in the \tt{BlockList} up to and including the \tt{block}. \it{put the definitions before the pseudocode}}

\begin{itemize}
%  \item \tt{\sl{int} num\sub{enq}, num\sub{deq}}
%  \textsf{: \# enqueue, dequeue operations in the block}
%  \item \tt{\sl{int} num, sum}
%  \textsf{: total \# operations in block, prefix sum of \tt{num}}

  \item \tt{\sl{int} group}
  \textsf{: the value read from \tt{num\sub{propagated}} when appending this block to the node.}
\end{itemize}


\Statex $\blacktriangleright$ \tt{\sl{LeafBlock} extends \sl{Block}}
\begin{itemize}
  \item \tt{\sl{Object} element}
  \textsf{: Each block in a leaf represents a single operation. For \tt{enqueue} operations element is the input of the \tt{enqueue} and for \tt{dequeue} operations it is \tt{null}.}
  
    \item \tt{\sl{Object} response}
  \textsf{: stores the response of the operation in the \tt{LeafBlock}.}
  
    \item \tt{\sl{int} sum\sub{enq}, sum\sub{deq}}
  \textsf{: \# enqueue, dequeue operations in the prefix for the block}
  
\end{itemize}

\Statex $\blacktriangleright$ \tt{\sl{InternalBlock} extends \sl{Block}}
\begin{itemize}
    \item \tt{\sl{int} end\sub{left}, end\sub{right}}
  \textsf{:~~index of the last subblock of the block in the left and right child}
  \item \tt{\sl{int} sum\sub{enq-left}}
  \textsf{: \# enqueue operations in the prefix for \tt{left.blocks[end\sub{left}]}}
  \item \tt{\sl{int} sum\sub{deq-left}}
  \textsf{: \# dequeue operations in the prefix for \tt{left.blocks[end\sub{left}]}}
  \item \tt{\sl{int} sum\sub{enq-right}}
  \textsf{: \# enqueue operations in the prefix for \tt{right.blocks[end\sub{right}]}}
  \item \tt{\sl{int} sum\sub{deq-right}}
  \textsf{: \# dequeue operations in the prefix for \tt{right.blocks[end\sub{right}]}}
\end{itemize}


\Statex $\blacktriangleright$ \tt{\sl{RootBlock} extends \sl{InternalBlock}}
\begin{itemize}
  \item \tt{\sl{int} length}
  \textsf{: length of the queue after performing all operations in the prefix for this block}
%  \item \tt{\sl{int} sum\sub{non-null deq}}
%  \textsf{: count of non-null dequeus up to this block}
  \item \tt{\sl{counter} num\sub{finished}}
  \textsf{: number of finished operations in the block}
    \item \tt{\sl{int} order}
  \textsf{: the index of the block in the \tt{BlockList} containing the block.}
\end{itemize}

%\Statex $\blacktriangleright$ \tt{Conventions}
%\begin{itemize}
%  \item \tt{i} : index of ith operation in the tree
%  \item \tt{j} : index of jth operation in a node
%  \item \tt{b\sub{n}} : index of the block containing the operation n based on the scope
%  \item Also we are not going to refer to blocks directly, only with their indices. Except while constructing a new block.
%\end{itemize}


\end{multicols}
\end{algorithmic}
\end{algorithm}


\begin{footnotesize}
  
\it{Variable naming:}
\begin{itemize}
  \item \tt{b\sub{op}}: index of the block containing operation\tt{op}
  \item \tt{r\sub{op}}: rank of operation \tt{op} i.e. the ordering among the operations of its type according to linearization ordering
\end{itemize}

\it{Abbreviations:}
\begin{itemize}
 \item \tt{blocks[b].sum\sub{x}=blocks[b].sum\sub{x-left}+blocks[b].sum\sub{x-right}}  \tt{ (for b$\geq$0 and x $\in$ \{enq, deq\}})
 \item \tt{blocks[b].sum=blocks[b].sum\sub{enq}+blocks[b].sum\sub{deq}}  \tt{ (for b$\geq$0})
  \item \tt{blocks[b].num\sub{x}=blocks[b].sum\sub{x}-blocks[b-1].sum\sub{x}} \\ \tt{(for b>0 and x $\in$ \{$\emptyset$, enq, deq, enq-left, enq-right, deq-left, deq-right\}, blocks[0].num\sub{x}=0})
\end{itemize}
\end{footnotesize}



%##########################################


\begin{algorithm}
\caption{\tt{\sl{Queue}}}
\begin{algorithmic}[1]
\setcounter{ALG@line}{200}
\begin{multicols}{2}

\Function{void}{Enqueue}{\sl{Object} e} \cmt{Creates a block with element \tt{e} and appends it to the tree.}
\State \tt{block newBlock= \Call{new}{\sl{LeafBlock}}}
\State \tt{newBlock.element= e}
%\State \tt{b.num\sub{enq}=1}
\State \tt{newBlock.sum\sub{enq}= leaf.blocks[leaf.size].sum\sub{enq}+1}
\State \tt{newBlock.sum\sub{deq}= leaf.blocks[leaf.size].sum\sub{deq}}
\State \tt{leaf.}\Call{Append}{newBlock}
\EndFunction{Enqueue}

\Statex
\Function{Object}{Dequeue()}{}
\State \tt{block newBlock= \Call{new}{\sl{LeafBlock}}} \cmt{Creates a block with null value element, appends it to the tree, computes its order among operations, then computes and returns its response.}
\State \tt{newBlock.element= null}
\State \tt{newBlock.sum\sub{enq}= leaf.blocks[leaf.size].sum\sub{enq}}
\State \tt{newBlock.sum\sub{deq}= leaf.blocks[leaf.size].sum\sub{deq}+1}
\State \tt{leaf.}\Call{Append}{newBlock}
\State \Return\tt{leaf.}\Call{HelpDequeue()}{}
\EndFunction{Dequeue}

\Statex

\Function{int, int}{FindResponse}{int i, int b}\cmt{Computes the rank  and index of the block in the root of the enqueue that is the response of the \tt{i}th dequeue in the root's \tt{b}th block. Returns \tt{<-1,-->} if the queue is empty.}
\If{\tt{root.blocks[b-1].length + root.blocks[b].num\sub{enq} - i $<$ 0}}
\State \Return \tt{<-1,-->}
\Else
\Statex \cmt{We call the dequeues that return a value \it{non-null dequeues}. $r$th non-null dequeue returns the element of th $r$th enqueue. We can compute \# non-null dequeues in the prefix for a block this way: \#non-null dequeues= length - \#enqueues. Note that the $i$th dequeue in the given block is not a non-null dequeue.}
\State \tt{r\sub{enq}= root.blocks[b-1].sum\sub{enq}- root.blocks[b-1].length + i}
\State \Return \tt{<root.blocks.get(enq, r\sub{enq}).order, r\sub{enq}>}
\EndIf
\EndFunction{FindResponse}

\end{multicols}
\end{algorithmic}
\end{algorithm}


%##########################################


\begin{algorithm}
\caption{\tt{\sl{Node}}}
\begin{algorithmic}[1]
\setcounter{ALG@line}{300}
\begin{multicols}{2}

\Function{void}{Propagate()}{}
\If{\bf{not} \Call{Refresh()}{}}
\State \Call{Refresh()}{} \cmt{Lemma Double Refresh}
\EndIf
\If{\tt{this} \bf{is not} \tt{root}}
\State \tt{parent.}\Call{Propagate()}{}
\EndIf
\EndFunction{Propagate}

\Statex

\Function{boolean}{Refresh()}{}
\State \tt{s= size}
%\If{\tt{n.blocks[h]!=null?}} \tt{h+=1}\EndIf
\State \tt{<new, np\sub{left}, np\sub{right}>= \Call{CreateBlock}{h}} \cmt{\tt{np\sub{left}, np\sub{right}} are the values read from the children's \tt{num\sub{propagated}} feild.}
\If{\tt{new.num==0}} \Return{\tt{true}} \cmt{The block contains nothing.}
\ElsIf{\tt{blocks.tryAppend(new, s)}}
\ForEach{\tt{dir} {\keywordfont{in}} \tt{\{left, right\}}} \label{okcas}
\State \tt{\Call{CAS}{dir.super[np\sub{dir}], null, h+1}} \cmt{Write would work too.}
\State \tt{\Call{CAS}{dir.num\sub{propagated}, np\sub{dir}, np\sub{dir}+1}}
\EndFor
\State \tt{\Call{CAS}{size, h, h+1}}
\State \Return{\tt{true}}
\Else
\State \tt{\Call{CAS}{size, h, h+1}} \cmt{Even if another process wins, help to increase the size. The winner might have fallen sleep before increasing \tt{size}.}
\State \Return{ \tt{false}}
\EndIf
\EndFunction{Refresh}

\Statex

\Statex $\leadsto$ \textsf{Precondition: \tt{blocks[start..end]} contains a block with field \tt{f} $\geq$ \tt{i}}
\Function{int}{BSearch}{\sl{field} f, \sl{int} i, \sl{int} start, \sl{int} end}

\Statex \cmt{\textmd{Does binary search for~the value \tt{i} of the given prefix sum \tt{field}. Returns the index of the leftmost block in \tt{blocks[start..end]} whose \sl{field} \tt{f} is $\geq$ \tt{i}}.}
%\State \Return \tt{result block's index}
\EndFunction{BSearch}

\pagebreak

\Function{<Block, int, int>}{CreateBlock}{\sl{int} i} 
\Statex\cmt{Creates a block to be inserted into as $i$th \tt{block} in \tt{blocks}. Returns the created block as well as values read from each child's num\sub{propagated} field. These values are used for incrementing the children's num\sub{propagated} field if the block was appended to \tt{blocks} successfully.}
\State \tt{block newBlock= \Call{new}{\sl{block}}}
\State \tt{newBlock.group= num\sub{propagated}}
\State \tt{newBlock.order= i}
\ForEach{\tt{dir} {\keywordfont{in}} \tt{\{left, right\}}}
\State \tt{index\sub{last}= dir.size} \label{lastLine}
\State \tt{index\sub{prev}= blocks[i-1].end\sub{dir}} \label{prevLine}
\State \tt{newBlock.end\sub{dir}= index\sub{last}}
\State \tt{block\sub{last}= dir.blocks[index\sub{last}]}
\State \tt{block\sub{prev}= dir.blocks[index\sub{prev}]}
\State \cmt{\tt{newBlock} includes \tt{dir.blocks[index\sub{prev}+1..index\sub{last}]}.}
\State \tt{this\sub{dir}= dir.num\sub{propagated}}
\State \tt{newBlock.sum\sub{enq-dir}= blocks[i-1].sum\sub{enq-dir} + block\sub{last}.sum\sub{enq} - block\sub{prev}.sum\sub{enq}}
\State \tt{newBlock.sum\sub{deq-dir}= blocks[i-1].sum\sub{deq-dir} + block\sub{last}.sum\sub{deq} - block\sub{prev}.sum\sub{deq}}
\EndFor
%\State \tt{b.num\sub{enq}= b.num\sub{enq-left} + b.num\sub{enq-right}}
%\State \tt{b.num\sub{deq}= b.num\sub{deq-left} + b.num\sub{deq-right}}
%\State \tt{b.num= b.num\sub{enq} + b.num\sub{deq}}
%\State \tt{b.sum= n.blocks[i-1].sum + b.num}

\If{\tt{this} \bf{is} \tt{root}}
\State \tt{newBlock.length= max(root.blocks[i-1].length + b.num\sub{enq} - b.num\sub{deq}, 0)}
%\State \tt{b.sum\sub{non-null deq}= root.blocks[i-1].sum\sub{non-null deq} + max( b.num\sub{deq} - root.blocks[i-1].length - b.num\sub{enq}, 0)}
\EndIf

\State \Return \tt{<b, np\sub{left}, np\sub{right}>}
\EndFunction{CreateBlock}

\end{multicols}
\end{algorithmic}
\end{algorithm}

%##########################################

\begin{algorithm}
\caption{Node}
\begin{algorithmic}[1]
\setcounter{ALG@line}{400}

\Statex $\leadsto$ \textsf{Precondition:~\tt{blocks[b].num\sub{enq}$\geq$i}}
\Function{element}{GetEnq}{\sl{int} b, \sl{int} i} 
\If{\tt{this} \bf{is} \tt{leaf}}
\State\Return \tt{blocks[b].element}
\ElsIf{\tt{i $\leq$ blocks[b].num\sub{enq-left}}} \cmt{\tt{i} exists in the left child of this node}
\State \tt{subBlock= left.\Call{BSearch}{sum\sub{enq}, i, blocks[b-1].end\sub{left}+1, blocks[b].end\sub{left}}} \cmt{Search range of left child's subblocks of \tt{blocks[b]}.}
\State \Return\tt{left.}\Call{Get}{i-left.blocks[subBlock-1].sum\sub{enq}, subBlock} 
\Else
\State \tt{i= i-blocks[b].num\sub{enq-left}}
\State\tt{subBlock= right.\Call{BSearch}{sum\sub{enq}, i, blocks[b-1].end\sub{right}+1, blocks[b].end\sub{right}}} \cmt{Search range of right child's subblocks of \tt{blocks[b]}.}
\State \Return\tt{right.}\Call{Get}{i-right.blocks[subBlock-1].sum\sub{enq}, subBlock} 
\EndIf
\EndFunction{GetEnq}

\Statex
\Statex $\leadsto$ \textsf{Precondition: \tt{b}th block of the node has propagated up to the root and \tt{blocks[b].num\sub{enq}$\geq$i}.}
\Function{<int, int>}{IndexDeq}{\sl{int} b, \sl{int} i} \cmt{Returns the rank of $i$th dequeue in the $b$th block of the node, among the dequeues in the root.}
\If{\tt{this} \bf{is} \tt{root}}
\State\Return \tt{<b, i>}
\Else
\State \tt{dir= (parent.left==n)? left: right} \cmt{check if a left or a right child}
\State \tt{superBlock= parent.\Call{BSearch}{sum\sub{deq-dir}, i, super[blocks[b].group]-p, super[blocks[b].group]+p}} 
\Statex\cmt{superblock's group has at most $p$ difference with the value stored in \tt{super[]}.}
\If{\tt{dir {\keywordfont is} right}}
%\State \tt{i+= parent.blocks[superBlock-1].sum\sub{deq-right}}
\State \tt{i+= blocks[superBlock].sum\sub{deq-left}} \cmt{consider the dequeues from the right child}
\EndIf
\State \Return\tt{this.parent.}\Call{IndexDeq}{superBlock, i}
\EndIf
\EndFunction{Index}

\end{algorithmic}
\end{algorithm}


%##########################################

\begin{algorithm}
\caption{Leaf}
\begin{algorithmic}[1]
\setcounter{ALG@line}{500}

\Function{void}{Append}{\sl{block} blk} \cmt{Append is only called by the owner of the leaf.}
\State \tt{size+=1} \label{appendEnd} 
\State \tt{blk.group= size} \label{appendStart} 
\State \tt{blocks[size]= blk} 
\State \tt{parent.}\Call{Propagate()}{} 
\EndFunction{Append}

\Statex

\Function{Object}{HelpDequeue()}{}
\State \tt{<b\sub{deq}, r\sub{deq}>=} \Call{IndexDeq}{leaf.size, 1}\cmt{\tt{r} is the rank among the dequeues of the dequeue of  the \tt{b\sub{deq}}th block in the root containing.}
\State \tt{<r\sub{enq}, r\sub{enq}>=} \Call{FindResponse}{r\sub{deq}, b\sub{deq}} 
 \cmt{\tt{r\sub{enq}} is the rank of the enqueue whose element is the response to the dequeue in the block containing it and \tt{b\sub{deq}} is the index of that block of it in the blocklist. If the response is \tt{null} then r\sub{deq} is -1.} \label{deqRest}
\If{\tt{r\sub{enq}==-1}}
\State \tt{output= null}
\State \tt{root.blocks[b\sub{deq}].num\sub{finished}.inc()} \cmt{shared counter}
\If{\tt{root.blocks[b\sub{deq}].num\sub{finished}==root.blocks[b\sub{deq}].num}}
\State \tt{last\sub{done}= b\sub{deq}}
\EndIf
\pagebreak
\Else
\State \tt{output= \Call{GetEnq}{b\sub{enq}, r\sub{enq}}} \cmt{getting the reponse's \tt{element}.}
\State \tt{root.blocks[b\sub{enq}].num\sub{finished}.inc()}
\State \tt{root.blocks[b\sub{enq}].num\sub{finished}.inc()}
\If{\tt{root.blocks[b\sub{deq}].num\sub{finished}==root.blocks[b\sub{deq}].num}} 
\State \tt{last\sub{done}= b\sub{deq}}
\ElsIf{\tt{root.blocks[b\sub{enq}].num\sub{finished}==root.blocks[b\sub{enq}].num}}
\State \tt{last\sub{done}= b\sub{enq}}
\EndIf
\EndIf
\State \Return{\tt{output}}
\EndFunction{Dequeue}


\end{algorithmic}
\end{algorithm}


%##########################################

\begin{algorithm}
\caption{BlockList}
\begin{algorithmic}[1]
\setcounter{ALG@line}{600}

\Statex \cmt{\textsf{: Supports two operations \tt{blocks.tryAppend(Block b), blocks[i]}. Initially  empty, when \tt{blocks.tryAppend(b, n)} returns true \tt{b} is appended to \tt{blocks[n]} and \tt{blocks[i]} returns $i$th block in the \tt{blocks}. If some instance of \tt{blocks.tryAppend(b, n)} returns \tt{false} there is a concurrent instance of \tt{blocks.tryAppend(b$^\prime$, n)} which has returned \tt{true}.\tt{blocks[0]} contains an empty block with all fields equal to 0 and \tt{end\sub{left}, end\sub{right}} pointers to the first block of the corresponding children.}}

\Statex
\Statex $\diamondsuit$ \tt{\sl{PBRT implementation}}
  \Statex \sf{A persistant red-black tree supporting \tt{append(b, key),get(key=i),split(j)}}.
  \tt{append(b, key) returns \tt{true} in case successful. Since \tt{order, sum\sub{enq}}are both strictly increasing we can use one of them for another.}
\Statex \tt{root: }\sf{pointer to the root of the PBRT}
\Function{boolean}{TryAppend}{\sl{block} blk, \sl{int} n} \cmt{\textsf{adds block b to the \tt{root.blocks[n]}}}
\If{\tt{root.size\%$p^2$==0}} \cmt{Help every often $p^2$ operations appended to the root.}
\State \tt{Help()}
\State \tt{CollectGarbage()}
\EndIf
\State \tt{blk.num\sub{finished}= 0}
\State \tt{*oldRoot= \&root.blocks.root}
\State \tt{*newRoot= root.blocks.Append(blk).root}
\State \Return \tt{CAS(root, oldRoot, newRoot)}
\EndFunction{TryAppend}

\Statex
\Statex $\diamondsuit$ \tt{\sl{Array implementation}}
\Statex \tt{blocks[]: }\sf{array of blocks}
\Function{boolean}{TryAppend}{\sl{block} blk, \sl{int} n} 
\State \Return{\tt{CAS(blocks[n], null, blk)}}
\EndFunction{TryAppend}

\end{algorithmic}
\end{algorithm}

%##########################################


\begin{algorithm}
\begin{algorithmic}[1]
\setcounter{ALG@line}{800}
\begin{multicols}{2}

\Function{void}{Help}{}\cmt{Helps pending operations}
\For{\tt{leaf l} \bf{in leaves}}
\State{\tt{last= l.size-1}}
\cmt{\tt{l.blocks[last]} can not be \tt{null} because size increases after appending, see lines \ref{appendStart}-\ref{appendEnd}.}
\If{\tt{l.blocks[last].element==null}} \cmt{operation is dequeue}
\State \tt{l.blocks[last].response= l.}\Call{HelpDequeue()}{}
\EndIf
\EndFor
\EndFunction{Help}
\Statex

\Function{void}{CollectGarbage}{}\cmt{Collects the root blocks that are done.}
\State \tt{s=FindMostRecentDone(Root.Blocks.root)}  \cmt{Lemma: If block b is done after helping then all blocks before b are done as well.}
\State \tt{t1,t2= RBT.split(order, s)}
\State \tt{RBTRoot.CAS(t2.root)}
\EndFunction{CollectGarbage}

\Statex

\Function{Block}{FindMostRecentDone}{b}
\For{\tt{leaf l} \bf{in leaves}}
\State\tt{max= Max(l.maxOld, max)}
\EndFor
\State\Return\tt{max} \cmt{This snapshot suffies.}
\EndFunction{FindMostRecentDone}

\Statex

\Function{response}{FallBack}{op i} \cmt{\it{how to use as exception handling? by adding try catch in all the methods reading the root?}}

\If{root.blocks.get(num\sub{enq}), i is null} \cmt{this enqueue was already finished}

\State \Return \tt{this.leaf.response(block.order)}
\EndIf

\EndFunction{FallBack}

\end{multicols}
\end{algorithmic}
\end{algorithm}




\end{document}






