% !TEX root =  podc-submission.tex

\section{Detailed Description of the Bounded-Space Implementation}
\label{reducing-details}


\renewcommand{\algorithmiccomment}[1]{\hfill\eqparbox{COMMENTSINGLEAPP}{\com\ #1}}

Here, we give more details of the bounded-space construction sketched in Section \ref{reducing}.
We first describe the modifications to the implementation.
To avoid confusion, we use nodes to refer to the nodes of the ordering tree, and blocks to refer
to the nodes of a RBT (since the RBT stores blocks).

The space-bounded implementation uses two additional shared arrays:
the \var{leaf} array allows processes to access one another's leaves to perform helping, and
the \var{last} array is used to determine which blocks are safe to discard.
\begin{itemize}
\item \typ{Node[]} \var{leaf}$[1..p]$ \Comment{\var{leaf}$[k]$ is the leaf assigned to process $k$}
\item \typ{int[]} \var{last}$[1..p]$ \Comment{\var{last}$[k]$ is the largest \fld{index} of a block in the root  that process $k$}\\
\mbox{ }\Comment{has observed to contain either an enqueue whose value has}\\
\mbox{ }\Comment{been dequeued or a null dequeue }
\end{itemize}

The \fld{blocks} field of each node in the \ordering\ tree is implemented as a pointer to the root of a RBT of \typ{Blocks} rather than an infinite array.  
Each RBT is initialized with an empty block with index~0.
Any access to an entry of the \fld{blocks} array is replaced by a search in the RBT.
The node's \head\ field, which previously gave the next position to insert into the \fld{blocks} array is no
longer needed; we can instead simply find the maximum \fld{index} of any block in the RBT.
To facilitate this, \op{MaxBlock} is a query operation on the RBT that 
returns the block with the maximum \fld{index}.
We can store, in the root of the RBT, a pointer to the maximum block so that \op{MaxBlock}
can be done in constant time, without affecting the time of other RBT operations.
Similarly, a \op{MinBlock} query finds the block with the minimum \fld{index} in a RBT.

Blocks no longer require the \fld{super} field.  This was used to quickly find the blocks 
superblock in the parent node's \fld{blocks} array, but this can now be done efficiently 
by searching the parent's \fld{blocks} RBT instead.
Each \typ{Block} has an additional field.
\begin{itemize}
\item \typ{int} \var{index} \Comment{position this block would have in the \fld{blocks} array}
\end{itemize}
To facilitate helping, each \typ{Block} in a leaf has one more additional field.
\begin{itemize}
\item \typ{Object} \var{response} \Comment{response of the operation in the block, if it is a dequeue}
\end{itemize}

\here{Talk about what happens if a block is not found in an RBT}

Pseudocode for the space-bounded implementation appears in Figure \ref{pseudocode1GC}--\ref{pseudocode2GC}.
New or modified code appears in blue.
The \op{Propagate} and \op{GetEnqueue} routines are unchanged.
A few lines have been added to \op{FindResponse} to update the \var{last} array
to ensure that it stores the value described above.
Minor modifications have also
been made to \op{Enqueue}, \op{Dequeue}, \op{CreateBlock}, \op{Refresh} and \op{Append}
to accommodate  the switch from an array of blocks to a RBT of blocks (and the corresponding disappearance
of the \head\ field).
In addition, the second half of the \op{Dequeue} routine is now in a
separate routine called \op{CompleteDeq} so that it can also be used by other processes
helping to complete the operation.
The \op{Refresh} routine no longer needs to set the \fld{super} field of blocks since
that field has been removed.
The \op{IndexDequeue} routine, which must trace the location of a dequeue along
a path from its leaf to the root has a minor modification to search the \fld{blocks} RBT at 
each level instead of using the \fld{super} field.

The new routines, \op{AddBlock}, \op{SplitBlock}, \op{Help} and \op{Propagated} are
used to implement GC.
When a \op{Refresh} or \op{Append} wants to add a new block to a node's \fld{blocks} RBT,
it calls the new \opemph{AddBlock} routine.
Before attempting to add the block to a node's RBT, \op{AddBlock} triggers a GC phase on the RBT if 
the new block's \fld{index} is a multiple of the constant $G$, which we choose to be $p\ceil{\log p}$.
This ensures that obsolete blocks are removed from the RBT once every $G$ times a new block is added to it.
The GC phase uses \op{SplitBlock} to determine the index $s$ of the oldest block to keep,
calls \op{Help} to help all pending dequeues that have been propagated to the root
(to ensure that all blocks before $s$ can safely be discarded),
uses the standard RBT \op{Split} routine \cite{Tar83}
to remove all blocks with \fld{index} less than $s$,
inserts the new block
and finally performs a \op{CAS} to swing the node's \fld{blocks} pointer to the new RBT.

To determine the oldest block in a node \var{v} to keep, the \opemph{SplitBlock} routine  first
finds the most recent block $B_{root}$ in the root that contains either an enqueue that has been dequeued
or a null dequeue using the \var{last} array.  By the FIFO property of queues, all enqueues in blocks before $B_{root}$ are either
dequeued or will be dequeued by a dequeue that is currently in progress.  Once those pending dequeues have been
helped to complete by line \ref{invokeHelp},
it is safe to discard any blocks in the root older than $B_{root}$, as well
as their subblocks.\footnote{If we used the more conservative approach of discarding blocks whose indices 
are smaller than the \emph{minimum} entry of \var{last} instead of the maximum, helping would be unnecessary, but then one slow process could prevent GC from discarding any blocks, so the space would not be bounded.}
The \op{SplitBlock} uses the \eleft\ and \eright\ fields to find the last subblock in \var{v}
of $B_{root}$ (or any older block in the root).
As \op{SplitBlock} is in progress, it is possible that some block that it needs in a node $\var{v}'$
along the path from \var{v} to the root is discarded by another GC phase.
In this case, \op{SplitBlock} uses the last subblock in \var{v} of the oldest block in $\var{v}'$ instead
(since $\var{v}'$ determined that all blocks older than that are safe to discard).

The \opemph{Help} routine is fairly straightforward:  it loops through all leaves and
helps the dequeue that is in progress there if it has already been propagated to the root.
The \opemph{Propagated} function is used to determine whether the dequeue has propagated to the root.

\renewcommand{\algorithmiccomment}[1]{\hfill\eqparbox{COMMENTDOUBLE}{\com\ #1}}

\begin{figure}
\begin{minipage}[t]{0.465\textwidth}
\begin{algorithmic}[1]
\setcounter{ALG@line}{200}

\Function{void}{Enqueue}{\typ{Object} \var{e}} 
    \gc{
    \State \var{h} \assign \Call{MaxBlock}{\var{leaf}.\fld{blocks}}.\fld{index}+1
	\State \hangbox{let \var{B} be a new \typ{Block} with \fld{element} \assign\ \var{e},\\
		$\fld{sum\sub{enq}} \assign\ \var{leaf}.\fld{blocks}[h-1].\fld{sum\sub{enq}}+1,$\\
		$\fld{sum\sub{deq}} \assign\ \var{leaf}.\fld{blocks}[h-1].\fld{sum\sub{deq}}$,\\
		$\fld{index}\assign h$}\label{enqNewGC}
	}
    \State \Call{Append}{\var{B}}
\EndFunction{Enqueue}

\spac

\Function{Object}{Dequeue()}{} 
    \gc{
    \State \var{h} \assign \Call{MaxBlock}{\var{leaf}.\fld{blocks}}.\fld{index}+1
    \State \hangbox{let \var{B} be a new \typ{Block} with \fld{element} \assign\ \nl,\\
	    $\fld{sum\sub{enq}} \assign\ \var{leaf}.\fld{blocks}[h-1].\fld{sum\sub{enq}},$\\
	    $\fld{sum\sub{deq}} \assign\ \var{leaf}.\fld{blocks}[h-1].\fld{sum\sub{deq}}+1$,\\
	    $\fld{index}\assign h$}\label{deqNewGC}
	}
    \State \Call{Append}{\var{B}}
    \gc{\State \Return \Call{CompleteDeq}{\var{leaf}, $h$}}
\EndFunction{Dequeue}

\spac

\gc{
\Function{Object}{CompleteDeq}{\var{leaf}, $h$}
    \State \linecomment Finish propagated dequeue in $\var{leaf}.\fld{blocks}[h]$
    \State $\langle \var{b}, \var{i}\rangle \assign\ \Call{IndexDequeue}{\var{leaf}, h, 1}$
    \State $\var{res} \assign \Call{FindResponse}{b, i}$
	\State \Return{\var{res}}
\EndFunction{CompleteDeq}
}

\spac

\Function{void}{Append}{\typ{Block} \var{B}} 
    \State \linecomment{append block to leaf and propagate to root}
    \gc{\State \var{leaf}.\fld{blocks} \assign \Call{AddBlock}{\var{leaf}, \var{leaf}.\fld{blocks}, \var{B}}\label{appendLeafGC}}
    \State \Call{Propagate}{\var{leaf}.\fld{parent}} 
\EndFunction{Append}

\spac

\gc{
\Function{RBT}{AddBlock}{\typ{Node} \var{v}, \typ{RBT} \var{T}, \typ{Block} \var{B}}
    \State \linecomment{add block $\var{B}\neq\nl$ to \var{T}, doing GC if necessary}
    \If{\var{B}.\fld{index} is a multiple of $G$}
    	\State \linecomment{Do garbage collection}\label{GCstart}
        \State $s \assign \Call{SplitBlock}{\var{v}}.\var{index}$\label{callSplitBlock}
        \State \Call{Help}{}\label{invokeHelp}
        \State $T' \assign\ \Call{Split}{T,s}$\label{splitGC}
        \State \linecomment{\op{Split} removes blocks with $\fld{index} < s$}
        \State \Return \Call{Insert}{$T',B$}\label{GCend}
	\Else\ \Return \Call{Insert}{$T, B$}
    \EndIf
\EndFunction{AddBlock}
}

\spac

\gc{
\Function{Block}{SplitBlock}{\typ{Node} v}
	\State \linecomment{figure out where to split \var{v}'s RBT}
	\If{$\var{v}=\var{root}$}
		\State $m \assign 0$
		\For{$k\assign 1..p$} $m\assign \max(m, \var{v}.\fld{last}[k])$
		\EndFor
		\State $B\assign \var{root}.\fld{blocks}[m-1]$
	\Else
		\State $B_p \assign \Call{SplitBlock}{\var{v}.\fld{parent}}$\label{SBrecurse}
		\State \var{dir} \assign ($\var{v}=\var{v}.\fld{parent.left}$ ? \fld{left} : \fld{right})
		\State $B \assign \var{v}.\fld{blocks}[B_p.\edir]$
	\EndIf
	\State \linecomment If $B$ was discarded, use leftmost block instead
	\State \Return ($B = \nl$ ? $\Call{MinBlock}{\var{v}.\var{blocks}}$ : $B$) \label{substitute}
\EndFunction{SplitBlock}
}


\end{algorithmic}
\end{minipage}
\begin{minipage}[t]{0.529\textwidth}

\begin{algorithmic}[1]
\setcounter{ALG@line}{249}

\Function{void}{Propagate}{\typ{Node} \var{v}}
    \State \linecomment{propagate blocks from \var{v}'s children to root}
    \If{\bf{not} \Call{Refresh}{\var{v}}} \label{firstRefreshGC}  \hfill \com\ double refresh
        \State \Call{Refresh}{\var{v}} \label{secondRefreshGC}
    \EndIf
    \If{$\var{v} \neq \var{root}$} \hfill \com\ recurse up tree
        \State \Call{Propagate}{\var{v}.\fld{parent}}
    \EndIf
\EndFunction{Propagate}


\spac

\Function{boolean}{Refresh}{\typ{Node} \var{v}}
	\State \linecomment{try to append a new block to $\var{v}.\fld{blocks}$}
	\gc{
	\State $T \assign \var{v}.\fld{blocks}$\label{refreshReadTGC}
    \State $\var{h} \assign \Call{MaxBlock}{T}.\fld{index}+1$ \label{refreshReadMax}
    }
    \State \var{new} \assign\ \Call{CreateBlock}{\var{v, h}} \label{invokeCreateBlockGC}
    \If{\var{new = \nl}} \Return{\tr} \label{addOPGC}
    \Else  
    	\gc{
	    \State $T' \assign \Call{AddBlock}{v, T, \var{new}}$\label{refreshABGC}
		\State \Return \Call{CAS}{\var{v}.\fld{blocks}, $T$, $T'$}\label{casGC}
		}
    \EndIf

\EndFunction{Refresh}

\spac


\gc{
\Function{boolean}{Propagated}{\typ{Node} \var{v}, \typ{int} \var{b}}
    \State \linecomment{check if $\var{v}.\fld{blocks}[b]$ has  propagated to \var{root}}
    \State \linecomment{Precondition:  $\var{v}.\fld{blocks}[b]$ exists}
    \If {$\var{v} = \var{root}$} \Return{true}
    \Else
    	\State $T\assign \var{v}.\fld{parent.blocks}$
		\State \fld{dir} \assign\ (\var{v}.\fld{parent.left} = \var{v} ? \fld{left} : \fld{right}) 
		\If{\Call{MaxBlock}{$T$}.\edir < b} \Return \fa
		\Else
			\State $B_p$ \assign min block in $T$ with $\edir \geq b$\label{searchsuper3}
			\State \Return \Call{Propagated}{\var{v}.\fld{parent}, $B_p$.\fld{index}}
		\EndIf
	\EndIf
\EndFunction{Propagated}
}

\spac

\Function{$\langle\typ{int}, \typ{int}\rangle$}{IndexDequeue}{\typ{Node} \var{v}, \typ{int} \var{b}, \typ{int} \var{i}}
    \State \linecomment{return $\langle\var{x, y}\rangle$ such that \var{i}th dequeue in $D(\var{v}.\fld{blocks}[\var{b}])$}
    \State \linecomment{is \var{y}th dequeue of $D(\var{root}.\fld{blocks}[\var{x}])$}
    \State \linecomment{Precondition: \var{v}.\fld{blocks}[\var{b}] exists and has propagated}
    \State \linecomment{to root and $D(\var{v}.\fld{blocks}[\var{b}])$ has at least \var{i} dequeues}
    \If{$\var{v} = \var{root}$} \Return $\langle\var{b, i}\rangle$ \label{indexBaseCaseGC}
    \Else
	    \State \fld{dir} \assign\ (\var{v}.\fld{parent.left} = \var{v} ? \fld{left} : \fld{right}) 
    	\gc{
		\State $T\assign \var{v}.\fld{parent.blocks}$
		\State $B_p$ \assign min block in $T$ with $\edir \geq b$\label{searchsuper1}
		\State $B_p'$ \assign max block in $T$ with $\edir < b$\label{searchsuper2}
	    }
	    \State \linecomment{compute index \var{i} of dequeue in superblock $B_p$}
	    \State \hangbox{\var{i} += $\var{v}.\fld{blocks}[\var{b}-1].\fld{sum\sub{deq}} -$ \\
	    		$\var{v}.\fld{blocks}[\gc{B_p'}.\edir].\fld{sum\sub{deq}}$}
        \If{$\fld{dir} = \fld{right}$} 
        	\State \hangbox{\var{i} += $\var{v}.\fld{blocks}[\gc{B_p}.\eleft].\fld{sum\sub{deq}} - \mbox{ }$\\
					$\var{v}.\fld{blocks}[\gc{B_p'}.\eleft].\fld{sum\sub{deq}}$}\label{considerLeftBeforeRightGC}
        \EndIf \label{computeISuperEndGC}
        \State \Return\Call{IndexDequeue}{\var{v}.\fld{parent}, $\gc{B_p.\fld{index}}$, \var{i}}
    \EndIf
\EndFunction{IndexDequeue}


\end{algorithmic}
\end{minipage}
\vspace*{-4mm}
\caption{Bounded-space queue implementation.\label{pseudocode1GC}  $G$ is a constant, which we choose to be $p^2 \ceil{\log p}$.}
\end{figure}

\begin{figure}
\begin{minipage}[t]{0.465\textwidth}
\begin{algorithmic}[1]
\setcounter{ALG@line}{302}

\gc{
\Function{void}{Help}{}
    \State \linecomment{help pending operations}
    
    \For{$\ell$ in $\var{leaf}[1..k]$}
        \State $\var{B} \assign \Call{LastBlock}{\ell.\fld{blocks}}$
    	%\State \Call{Propagate}{$\ell.\fld{parent}$}   <--- don't do this because recursive helping could be infinite loop
        \If{$\var{B}.\fld{element} = \nl$ and $\var{B}.\fld{index}>0$ 
        					\State \hspace*{4mm} and \Call{Propagated}{$\ell, \var{B}.\fld{index}$}} 
            \State \linecomment{operation is a propagated dequeue}
            \State $\var{B}.\fld{response} \assign\ \Call{CompleteDeq}{\ell, \var{B}.\fld{index}}$ 
        \EndIf
    \EndFor
\EndFunction{Help}
}
\spac

\Function{Block}{CreateBlock}{\typ{Node} \var{v}, \typ{int} \var{i}} 
    \State\linecomment{create new block  to install in \var{v}.\fld{blocks}[\var{i}]}
    \State let \var{new} be a new \typ{Block} \label{initNewBlockGC}
    \gc{
    \State \var{new}.\eleft \assign\ $\Call{MaxBlock}{\var{v}.\fld{left}.\fld{blocks}}.\fld{index}$\label{createEndLeftGC}
    \State \var{new}.\eright \assign\ $\Call{MaxBlock}{\var{v}.\fld{right}.\fld{blocks}}.\fld{index}$\label{createEndRightGC}
    \State $\var{new}.\fld{index} \assign i$\label{createIndexGC}
    }
	\State \hangbox{\var{new}.\fld{sum\sub{enq}} \assign\ \var{v}.\fld{left.blocks}[\var{new}.\eleft].\fld{sum\sub{enq}} + \\
			\var{v}.\fld{right.blocks}[\var{new}.\eright].\fld{sum\sub{enq}}}
	\State \hangbox{\var{new}.\fld{sum\sub{deq}} \assign\ \var{v}.\fld{left.blocks}[\var{new}.\eleft].\fld{sum\sub{deq}} + \\
			\var{v}.\fld{right.blocks}[\var{new}.\eright].\fld{sum\sub{deq}}}
    \State \var{num\sub{enq}} \assign\ $\var{new}.\fld{sum\sub{enq}} - \var{v}.\fld{blocks}[\var{i}-1].\fld{sum\sub{enq}}$\label{computeNumEnqGC}
    \State \var{num\sub{deq}} \assign\ $\var{new}.\fld{sum\sub{deq}} - \var{v}.\fld{blocks}[\var{i}-1].\fld{sum\sub{deq}}$
    \If{$\var{v} = \var{root}$}
        \State \hangbox{\var{new}.\fld{size} \assign\ max(0, $\var{v}.\fld{blocks}[\var{i}-1].\size\ + \\
        	\var{num\sub{enq}} - \var{num\sub{deq}}$)}\label{computeLengthGC}
    \EndIf
    \If{$\var{num\sub{enq}} + \var{num\sub{deq}} = 0$}\label{testEmptyGC}
        \State \Return \nl \hfill \com\ no blocks to be propagate to \var{v}\label{CBnullGC}
    \Else
        \State \Return \var{new}
    \EndIf
\EndFunction{CreateBlock}




\end{algorithmic}
\end{minipage}
\begin{minipage}[t]{0.529\textwidth}
\begin{algorithmic}[1]
\setcounter{ALG@line}{352}
\Function{element}{FindResponse}{\typ{int} \var{b}, \typ{int} \var{i}}
    \State \linecomment{find response to \var{i}th dequeue in $D(\var{root}.\fld{blocks}[\var{b}])$}
    \State \linecomment{Precondition:  $1\leq i\leq |D(\var{root}.\fld{blocks}[\var{b}])|$}
    \State \hangbox{\var{num\sub{enq}} \assign\ $\var{root}.\fld{blocks}[\var{b}].\fld{sum\sub{enq}} - \mbox{ }$\\
    		$\var{root}.\fld{blocks}[\var{b}-1].\fld{sum\sub{enq}}$}
    \If{$\var{root}.\fld{blocks}[\var{b}-1].\size + \var{num\sub{enq}} < \var{i}$}\label{checkEmptyGC}
    	\State \linecomment{ queue is empty when dequeue occurs}
        \gc{
        \If{$b > \var{last}[\var{id}]$} $\var{last}[\var{id}]\assign b$
        \EndIf
        }
        \State \Return \nl \hfill \label{returnNullGC}
    \Else \ \linecomment{response is the \var{e}th enqueue in the root}
        \State \var{e} \assign\ \var{i} + \var{root}.\fld{blocks}[\var{b}-1].\fld{sum\sub{enq}} - 
			\var{root}.\fld{blocks}[\var{b}-1].\size\label{computeEGC}
		\State \linecomment{compute enqueue's block using binary search}
		\State find min $b_e \leq \var{b}$ with $\var{root}.\fld{blocks}[b_e].\fld{sum\sub{enq}} \geq \var{e}$\label{FRsearchGC}
		\State \linecomment{find rank of enqueue within its block}
		\State $i_e \assign\ \var{e} - \var{root}.\fld{blocks}[b_e-1].\fld{sum\sub{enq}}$
		\gc{
        \State \var{res} \assign \Call{GetEnqueue}{\var{root}, $b_e$, $i_e$}\label{findAnswerGC}
        \If {$b_e > \var{last}[\var{id}]$} $\var{last}[\var{id}]\assign b_e$
        \EndIf
        \State \Return \var{res}
		}
    \EndIf
\EndFunction{FindResponse}


\end{algorithmic}
\end{minipage}

\spac

\begin{algorithmic}[1]
\setcounter{ALG@line}{332}

\Function{element}{GetEnqueue}{\typ{Node} \var{v}, \typ{int} \var{b}, \typ{int} \var{i}} \Comment{returns argument of \var{i}th enqueue in $E(\var{v}.\fld{blocks}[\var{b}])$}
    \State \linecomment{Preconditions: $\var{i}\geq 1$ and \var{v}.\fld{blocks}[\var{b}] exists and contains at least \var{i} enqueues}
    \If{\var{v} is a leaf node}
        \State\Return \var{v}.\fld{blocks}[\var{b}].\fld{element} \label{getBaseCaseGC}
    \Else 
        \State \var{sum\sub{left}} \assign\ \var{v}.\fld{left.blocks}[\var{v}.\fld{blocks}[\var{b}].\eleft].\fld{sum\sub{enq}} \Comment{\#\ of enqueues in \var{v}.\fld{blocks}[1..$\var{b}$] from left child}
        \State \var{prev\sub{left}} \assign\ \var{v}.\fld{left.blocks}[\var{v}.\fld{blocks}[$\var{b}-1$].\eleft].\fld{sum\sub{enq}} \Comment{\#\ of enqueues in \var{v}.\fld{blocks}[1..$\var{b}-1$] from left child}
        \State \var{prev\sub{right}} \assign\ \var{v}.\fld{right.blocks}[\var{v}.\fld{blocks}[$\var{b}-1$].\eright].\fld{sum\sub{enq}} \Comment{\#\ of enqueues in \var{v}.\fld{blocks}[1..$\var{b}-1$] from right child}
        \If{$\var{i} \leq \var{sum\sub{left}} - \var{prev\sub{left}}$} \label{leftOrRightGC} \cmt{required enqueue is in \var{v}.\fld{left}}
            \State \fld{dir} \assign\ \fld{left}
        \Else
            \State \fld{dir} \assign\ \fld{right}
            \State $\var{i}\ \assign\ \var{i} - (\var{sum\sub{left}} - \var{prev\sub{left}})$
        \EndIf
        \State \linecomment{Use binary search to find enqueue's block in \var{v}.\fld{dir.blocks} and its rank within block}
        \State find minimum $\var{b}'$ in range [\var{v}.\fld{blocks}[\var{b}-1].\edir+1..\var{v}.\fld{blocks}[\var{b}].\edir] s.t. $\var{v}.\fld{dir.blocks}[\var{b}'].\fld{sum\sub{enq}} \geq \var{i} + \var{prev\sub{dir}}$\label{getChildGC}
        \State $\var{i}'$ \assign\ $\var{i} - (\var{v}.\fld{dir.blocks}[\var{b}'-1].\fld{sum\sub{enq}} - \var{prev\sub{dir}})$
        \State \Return\Call{GetEnqueue}{\var{v}.\fld{dir}, $\var{b}'$, $\var{i}'$}
    \EndIf
\EndFunction{GetEnqueue}

\end{algorithmic}

\caption{Bounded-space queue implementation, continued.\label{pseudocode2GC}}
\end{figure}



%\begin{algorithm}
%\caption{\tt{\sl{Node}}}
%\begin{algorithmic}[1]
%\setcounter{ALG@line}{25}
%
%\Statex $\leadsto$ \textsf{Precondition: \tt{blocks[start..end]} contains a block with \tt{sum\sub{enq}} greater than or equal to \tt{x}}
%\Statex \com\ \textmd{Does a binary search for~the value \tt{x} of \tt{sum\sub{enq}} field and returns the index of the leftmost block in\\
%\com\ \tt{blocks[start..end]} whose \tt{sum\sub{enq}} is $\geq$ \tt{x}}.
%\Function{int}{BinarySearch}{\sl{int} x, \sl{int} start, \sl{int} end}
% \State \Return \tt{min\{j: blocks[j].sum\sub{enq}$\geq$x\}}
%\While{\nf{start<end}}
%\State \tt{\sl{int}} \tt{mid \assign\ floor((start+end)/2)}
%\If{\nf{blocks[mid].sum\sub{enq}<x}}
%\State \nf{start \assign\ mid+1}
%\Else
%\State \nf{end \assign\ mid}
%\EndIf
%\EndWhile
%\State\Return \nf{start}
%\EndFunction{BinarySearch}
%
%\end{algorithmic}
%\end{algorithm}

%\begin{algorithm}
%\caption{\tt{\sl{Root}}}
%\begin{algorithmic}[1]
%\setcounter{ALG@line}{36}
%\Statex
%\Statex $\leadsto$ \textsf{Precondition: \tt{root.blocks[end].sum\sub{enq} $\geq$ \tt{e}}}
%\Statex \com\ \textmd{Returns \tt{<b,i>} such that $E_\nf{e}(\nf{root})$ is $E_\nf{i}(\nf{root},\nf{b})$, i.e., the \nf{e}th \nf{Enqueue} in the \nf{root} is the \nf{i}th \nf{Enqueue} within \\
%\com\ the \nf{b}th block in the \nf{root}.}
%
%\Function{<int, int>}{DoublingSearch}{\sl{int} e, \sl{int} end}
%\State \tt{start \assign\ end-1} \label{dsearchStart}
%\While{\tt{root.blocks[start].sum\sub{enq}}$\geq$\tt{e}}
%\State \tt{start \assign\ max(start-(end-start), 0)} \label{doubling}
%\EndWhile \label{dsearchEnd}
%\State \tt{b \assign\ root.BinarySearch(e, start, end)} \label{dsearchBinarySearch}
%\State \tt{i \assign\ e- root.blocks[b-1].sum\sub{enq}} \label{DSearchComputei}
%\State\Return \tt{<b,i>}
%\EndFunction{DoublingSearch}
%\end{algorithmic}
%\end{algorithm}

