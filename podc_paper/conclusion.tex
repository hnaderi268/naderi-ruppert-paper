% !TEX root =  podc-submission.tex

\section{Future Directions}

Our focus in constructing our queue was on optimizing step complexity for worst-case executions.
However, our approach does have a higher cost in the best case (for example, when an operation
runs in isolation).
It may be possible to make our queue adaptive by having an operation capture a starting node
in the tournament tree (as in \cite{??}) rather than having a statically assigned leaf as its starting point.
Another way to make our queue more practical might be to use it as the slow path in the
fast-path slow-path methodology of Kogan and Petrank \cite{10.1145/2370036.2145835} to
get a queue that has good performance in practice while also having good worst-case step complexity.

There is a gap between our implementation, which takes $O(\log^2 p + \log q)$ steps per operation,
and Attiya and Fouren's $\Omega(\min(c,\log\log p))$ lower bound \cite{DBLP:conf/opodis/AttiyaF17}.
It would be interesting to determine how the true step complexity of lock-free queues (or, more generally, bags)
depends on $p$.

We believe the approach used here to implement a lock-free queue 
could also be applied to obtain other lock-free
data structures with a polylogarithmic number of steps per operation.
For example, we could implement a kind of vector data structure that stores a sequence and
provides three operations: \opa{append}{e} to add an element \var{e} to the end of the sequence,
\opa{get}{i} to read the \var{i}th element in the sequence, and
\opa{index}{e} to compute the position of element \var{e} in the sequence.
\here{Can omit rest of this paragraph to save space}
An \opa{append}{e} is implemented like \opa{Enqueue}{e} in $O(\log p)$ steps.  
A \opa{get}{i} calls \nf{DoublingSearch} but with a \nf{BinarySearch} on the entire \nf{root.blocks} array, taking $O(\log n + \log^2p)$ when the vector has $n$ elements.  
An \opa{index}{e} is similar to \op{IndexDequeue} (except operating on enqueues instead of dequeues) and takes $O(\log p)$ steps.
\here{doublecheck all preceding claims for accuracy.}
As future work we would like to investigate whether a similar approach could be used for stacks or deques.

