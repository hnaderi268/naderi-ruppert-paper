\documentclass{article}
\usepackage[utf8]{inputenc}
\renewcommand{\baselinestretch}{1.8}
\usepackage[dvipsnames]{xcolor}
\usepackage[letterpaper, margin=1in]{geometry}
\usepackage{tikz-qtree}
\usepackage{algorithm}
\usepackage{algpseudocode}
\makeatletter
\renewcommand{\ALG@beginalgorithmic}{\footnotesize}
\makeatother
\usepackage{graphicx}
\usepackage{subcaption}
\usepackage{hyperref}
\usepackage{amsmath}
\usepackage{relsize}
\usepackage{enumitem}
\usepackage{bold-extra}
\renewcommand*\contentsname{Table of Contents}

\usepackage{multicol}
\setlength\columnsep{24pt}

\algnewcommand\algorithmicforeach{\bf{for each}}
\algdef{S}[FOR]{ForEach}[1]{\algorithmicforeach\ #1\ \algorithmicdo}

\algdef{S}[FUNCTION]{Function}
   [3]{{\tt{\sl{#1}}} {\tt{#2}}\ifthenelse{\equal{#3}{}}{}{\tt{(#3)}}}
  
\algdef{E}[FUNCTION]{EndFunction}
   [1]{\algorithmicend\ \tt{{#1}}}

\algrenewcommand\Call[2]{\tt{{#1}\ifthenelse{\equal{#2}{}}{}{(#2)}}}
   
\newcommand\keywordfont{\sffamily\bfseries}
\algrenewcommand\algorithmicend{{\keywordfont end}}
\algrenewcommand\algorithmicfor{{\keywordfont for}}
\algrenewcommand\algorithmicforeach{{\keywordfont for each}}
\algrenewcommand\algorithmicdo{{\keywordfont do}}
\algrenewcommand\algorithmicuntil{{\keywordfont until}}
\algrenewcommand\algorithmicfunction{{\keywordfont function}}
\algrenewcommand\algorithmicif{{\keywordfont if}}
\algrenewcommand\algorithmicthen{{\keywordfont then}}
\algrenewcommand\algorithmicelse{{\keywordfont else}}
\algrenewcommand\algorithmicreturn{{\keywordfont return}}

\renewcommand\thealgorithm{}
\newcommand{\setalglineno}[1]{
  \setcounter{ALC@line}{\numexpr#1-1}}

\newcommand{\sub}[1]{\textsubscript{#1}}
\renewcommand{\tt}[1]{\texttt{#1}}
\renewcommand{\sl}[1]{\textsl{#1}}
\renewcommand{\it}[1]{\textit{#1}}
\renewcommand{\sc}[1]{\textsc{#1}}
\renewcommand{\bf}[1]{\textbf{#1}}
\newcommand{\nf}[1]{{\normalfont{\texttt{#1}}}}
\newcommand{\cmt}[1]{\Comment{#1}}
\newcommand{\head}{head}
\newcommand{\size}{size }

\usepackage{amsmath,amssymb,amsthm}
\newtheorem{theorem}{Theorem}
\newtheorem{lemma}[theorem]{Lemma}
\newtheorem{corollary}[theorem]{Corollary}
\newtheorem{observation}[theorem]{Observation}
\theoremstyle{definition}
\newtheorem{definition}[theorem]{Definition}
\newtheorem{invariant}[theorem]{Invariant}
\newtheorem{proposition}[theorem]{Proposition}


\begin{document}

\section{Feilds}
\begin{algorithm}
\caption{Tree Fields Description}
\begin{algorithmic}[1]
\setcounter{ALG@line}{1}


\Statex $\diamondsuit$ \tt{\sl{Shared}}
\begin{itemize}
\item \textsf{A binary tree of \tt{Node}s with one \tt{leaf} for each process. \tt{root} is the root \nf{node}.}
\item {\color{red}\tt{\sl{MaxbyProcess} lastDequeuedFrom} \textsf{Index of the most recent block in the root that has been deqeueued from.}}
\end{itemize}

\Statex $\diamondsuit$ \tt{\sl{Local}}
\begin{itemize}
\item \tt{\sl{Node} leaf:} \sf{ process's leaf in the tree.}
% \item {\color{red}\tt{\sl{int} garbageCollectRound}}
\end{itemize}

\Statex $\blacktriangleright$ \tt{\sl{Node}}
\begin{itemize}
\item \tt{\sl{*Node} left, right, parent} \textsf{: Initialized  when creating the tree.}
\item {\color{red}\tt{\sl{PBRT} blocks} \textsf{: Initially \tt{blocks[0]} contains an empty block with all fields equal to 0.}}
\item \tt{\sl{int} \head= 1}\textsf{: \#\tt{block}s in \tt{blocks}. \tt{blocks[0]} is a block with all integer fields equal to zero.}
\end{itemize}

\Statex $\blacktriangleright$ \tt{\sl{Block}} 

\begin{itemize}
  \item \tt{\sl{int} super}
  \textsf{: approximate index of the superblock, read from \tt{parent.head} when appending the block to the node}
\end{itemize}



\Statex $\blacktriangleright$ \tt{\sl{InternalBlock} extends \sl{Block}}
\begin{itemize}
    \item \tt{\sl{int} end\sub{left}, end\sub{right}}
  \textsf{:~~indices of the last subblock of the block in the left and right child}
  \item \tt{\sl{int} sum\sub{enq-left}}\textsf{: \#enqueues in \tt{left.blocks[1..end\sub{left}]}}
  \item \tt{\sl{int} sum\sub{deq-left}}\textsf{: \#dequeues in \tt{left.blocks[1..end\sub{left}]}}
  \item \tt{\sl{int} sum\sub{enq-right}}\textsf{: \#enqueues in \tt{right.blocks[1..end\sub{right}]}}
  \item \tt{\sl{int} sum\sub{deq-right}}\textsf{: \#dequeues in \tt{right.blocks[1..end\sub{right}]}}
\end{itemize}

\Statex $\blacktriangleright$ \tt{\sl{LeafBlock} extends \sl{Block}}
\begin{itemize}
  \item \tt{\sl{Object} element}
  \textsf{: Each block in a leaf represents a single operation. If the operation is \tt{enqueue(x)} then \tt{element=x}, otherwise \tt{element=null}.}
  
    \item \tt{\sl{int} sum\sub{enq}, sum\sub{deq}}
  \textsf{: \# enqueue, dequeue operations in this block and its previous blocks in the leaf}

  \item {\color{red}\tt{\sl{object} response}}
\end{itemize}

\Statex $\blacktriangleright$ \tt{\sl{RootBlock} extends \sl{InternalBlock}}
\begin{itemize}
  \item \tt{\sl{int} \size}
  \textsf{: size of the queue after performing all operations in this block and its previous blocks in the root}
\end{itemize}

\end{algorithmic}
\end{algorithm}

\section{Queue}
\begin{algorithm}
\caption{\tt{\sl{Queue}}}
\begin{algorithmic}[1]
\setcounter{ALG@line}{0}


\Function{void}{Enqueue}{\sl{Object} e} \cmt{Creates a \tt{block} with element \tt{e} and adds it to the tree.}
\State \tt{block newBlock= \Call{new}{\sl{LeafBlock}}}
\State \tt{newBlock.element= e}
\State \tt{newBlock.sum\sub{enq}= leaf.blocks[leaf.\head].sum\sub{enq}+1}
\State \tt{newBlock.sum\sub{deq}= leaf.blocks[leaf.\head].sum\sub{deq}}
\State \tt{leaf.}\Call{Append}{newBlock}
\EndFunction{Enqueue}

\Statex

\Statex $\triangleright$ Creates a block with \nf{null} value element, appends it to the tree and returns its response.
\Function{Object}{Dequeue()}{} 
\State \tt{block newBlock= \Call{new}{\sl{LeafBlock}}} 
\State \tt{newBlock.element= null}
\State \tt{newBlock.sum\sub{enq}= leaf.blocks[leaf.\head].sum\sub{enq}}
\State \tt{newBlock.sum\sub{deq}= leaf.blocks[leaf.\head].sum\sub{deq}+1}
\State \tt{leaf.}\Call{Append}{newBlock}
\State \tt{<b, i>=} \Call{IndexDequeue}{leaf.\head, 1}
\State \tt{output=} \Call{FindResponse}{b, i} 
\label{deqRest}
\State \Return{\tt{output}}
\EndFunction{Dequeue}

\Statex

\Statex $\triangleright$ Returns the response to $D_i(root,b)$, the \nf{i}th \nf{Dequeue} in \nf{root.blocks[b]}.
\Function{element}{FindResponse}{\sl{int} b, \sl{int} i}
\If{\tt{ root.blocks[b-1].\size}\tt{ + root.blocks[b].num\sub{enq} - i $<$ 0}} \label{checkEmpty}\cmt{Check if the queue is empty.}
\State \tt{ \color{red} lastDequeudFrom.update(b)}
\State \Return \tt{null} \label{returnNull}
\Else \cmt{The response is $E_e(root)$, the \nf{e}th \nf{Enqueue} in the root.}
\State \tt{e= i + (root.blocks[b-1].sum\sub{enq}-root.blocks[b-1].size)} \label{computeE}
\State {\color{red}\tt{<x, y>= root.\Call{DoublingSearch}{e, b}}}
\State {\color{red}\tt{lastDequeudFrom.update(x)}}
\State \Return \tt{root.GetEnqueue(x,y)}\label{findAnswer}
\EndIf
\EndFunction{FindResponse}

\end{algorithmic}
\end{algorithm}


\section{Search+Append}
\begin{algorithm}
\caption{\tt{\sl{Node}}}
\begin{algorithmic}[1]
\setcounter{ALG@line}{25}

\Statex $\leadsto$ \textsf{Precondition: \tt{blocks[start..end]} contains a block with \tt{sum\sub{enq}} greater than or equal to \tt{x}}
\Statex $\triangleright$ \textmd{\color{red} Update needed: search on RBT does not need start and end, we can search over whole the red-black tree.}.
\Function{int}{BinarySearch}{\sl{int} x}
\State \Return \tt{\color{red} min\{j: blocks[j].sum\sub{enq}$\geq$x\}}
% \While{\nf{start<end}}
% \State \tt{\sl{int}} \tt{mid= floor((start+end)/2)}
% \If{\nf{blocks[mid].sum\sub{enq}<x}}
% \State \nf{start= mid+1}
% \Else
% \State \nf{end= mid}
% \EndIf
% \EndWhile
% \State\Return \nf{start}
\EndFunction{BinarySearch}

\end{algorithmic}
\end{algorithm}

\begin{algorithm}
\caption{\tt{\sl{Root}}}
\begin{algorithmic}[1]
\setcounter{ALG@line}{36}
\Statex
\Statex $\leadsto$ \textsf{Precondition: \tt{root.blocks[end].sum\sub{enq} $\geq$ \tt{e}}}
\Statex $\triangleright$ \textmd{Returns \tt{<b,i>} such that $E_\nf{e}(\nf{root})=E_\nf{i}(\nf{root},\nf{b})$, i.e., the \nf{e}th \nf{Enqueue} in the \nf{root} is the \nf{i}th \nf{Enqueue} within $\triangleright$~the \nf{b}th block in the \nf{root}.}

\Function{<int, int>}{DoublingSearch}{\sl{int} e, \sl{int} end} \cmt{{\color{red} I think with garbage collection, doubling search is not needed any more and a binary search on \nf{root.blocks} would be good enough.}}
% \State \tt{start= end-1} \label{dsearchStart} 
% \While{\tt{root.blocks[start].sum\sub{enq}}$>=$\tt{e}}
% \State \tt{start= max(start-(end-start), 0)} \label{doubling}
% \EndWhile \label{dsearchEnd}
% \State \tt{b= root.BinarySearch(e, start, end)} \label{dsearchBinarySearch}
% \State \tt{i= e- root.blocks[b-1].sum\sub{enq}} \label{DSearchComputei}
% \State\Return \tt{<b,i>}
\EndFunction{DoublingSearch}
\end{algorithmic}
\end{algorithm}


\begin{algorithm}
\caption{\tt{\sl{Leaf}}}
\begin{algorithmic}[1]
\setcounter{ALG@line}{45}

\Function{void}{Append}{\sl{block} B} \cmt{Only called by the owner of the leaf.}
\State \tt{\color{red} blocks.TryAppend(B, head)} \label{appendLeaf}
\State \tt{\head= \head+1} \label{appendEnd} 
\State \tt{parent.}\Call{Propagate()}{} 
\EndFunction{Append}

\end{algorithmic}
\end{algorithm}






\section{Propagate}
\begin{algorithm}
\caption{\tt{\sl{Node}}}
\begin{algorithmic}[1]
\setcounter{ALG@line}{50}

\Statex $\triangleright$ \textmd{\nf{$n$.Propagate} propagates operations  in \nf{this}.children up to \nf{this} when it terminates.}
\Function{void}{Propagate()}{}
\If{\bf{not} \Call{Refresh()}{}} \label{firstRefresh}
\State \Call{Refresh()}{} \label{secondRefresh}
\EndIf
\If{\tt{this} \bf{is not} \tt{root}}
\State \tt{parent.}\Call{Propagate()}{}
\EndIf
\EndFunction{Propagate}

\Statex

\Statex $\triangleright$ \textmd{Creates a block containing new operations of \nf{this.}children, and then tries to append it to \nf{this}.}
\Function{boolean}{Refresh()}{}
\State \tt{h= \head} \label{readHead}
\ForEach{\tt{dir} {\keywordfont{in}} \tt{\{left, right\}}} \label{startHelpChild1}
\State \tt{h\sub{dir}= dir.\head} \label{readChildHead}
\If{\nf{dir.blocks[h\sub{dir}]!=null}} \label{ifHeadnotNull}
\State{\tt{dir.\Call{Advance}{h\sub{dir}}}} \label{helpAdvance}
\EndIf
\EndFor \label{endHelpChild1}
\State \tt{new= \Call{CreateBlock}{h}} \label{invokeCreateBlock}
\If{\tt{new.num==0}} \Return{\tt{true}} \label{addOP} 
\EndIf
\State{\tt{\color{red} result= blocks.TryAppend(new, h)}} \label{cas}
\State{\tt{this.\Call{Advance}{h}}} \label{advance}
\State \Return{ \tt{result}}

\EndFunction{Refresh}


\end{algorithmic}
\end{algorithm}

\begin{algorithm}
\caption{\tt{\sl{Node}}}
\begin{algorithmic}[1]
\setcounter{ALG@line}{73}

\Function{void}{Advance}{\sl{int} h} \cmt{Sets \nf{blocks[h].super} and increments \nf{head} from \nf{h} to \nf{h+1}.}
\State \tt{h\sub{p}= parent.\head} \label{readParentHead}
\State \tt{blocks[h].super.CAS(null, h\sub{p})} \label{setSuper1}
\State \tt{head.CAS(h, h+1)} \label{incrementHead}
\EndFunction{Advance}

\Statex

\Function{Block}{CreateBlock}{\sl{int} i} \cmt{Creates and returns the block to be installed in \tt{blocks[i]}.}
\State \tt{block new= \Call{new}{\sl{InternalBlock}}} \label{initNewBlock}
\ForEach{\tt{dir} {\keywordfont{in}} \tt{\{left, right\}}}
\State \tt{index\sub{prev}= blocks[i-1].end\sub{dir}} \label{prevLine}
\State \tt{new.end\sub{dir}= dir.\head-1} \label{lastLine} \cmt{\nf{new} contains \tt{dir.blocks[blocks[i-1].end\sub{dir}..dir.\head-1]}.}
\State \tt{block\sub{prev}= dir.blocks[index\sub{prev}]}
\State \tt{block\sub{last}= dir.blocks[new.end\sub{dir}]}
\State \tt{new.sum\sub{enq-dir}= blocks[i-1].sum\sub{enq-dir} + block\sub{last}.sum\sub{enq} - block\sub{prev}.sum\sub{enq}} \label{setSumEnqLeft}
\State \tt{new.sum\sub{deq-dir}= blocks[i-1].sum\sub{deq-dir} + block\sub{last}.sum\sub{deq} - block\sub{prev}.sum\sub{deq}} \label{setSumEnqRight}
\EndFor
\If{\tt{this} \bf{is} \tt{root}}
\State \tt{new.type= \sl{InternalBlock}-->\sl{RootBlock}}
\State \tt{new.size= max(root.blocks[i-1].\size { }+ new.num\sub{enq}- new.num\sub{deq}, 0)}\label{computeLength}
\EndIf

\State \Return \tt{new}
\EndFunction{CreateBlock}

\Statex
\color{red}
\Function{int}{GetLastDequeuedFrom}{}{\cmt{Returns the index that is safe to remove the blocks before that in the node.}}
\State \tt{x= lastDequeuedFrom.Get()-1}
\State \tt{n= root}
\State \bf{while} \tt{n!=this} \bf{do}
\State \tt{dir= left (if this is in left subtree of n): otherwise dir=right}
\State \tt{    x=n.blocks[x].end\sub{dir}}
\State\bf{end while}
\EndFunction{GetLastDequeuedFrom}

\end{algorithmic}
\end{algorithm}

\section{Index+Get}

\begin{algorithm}
\caption{\tt{\sl{Node}}}
\begin{algorithmic}[1]
\setcounter{ALG@line}{94}

\Statex $\leadsto$ \textsf{Precondition:~\tt{blocks[b].num\sub{enq}$\geq$i$\geq 1$}}
\Function{element}{GetEnqueue}{\sl{int} b, \sl{int} i} \cmt{Returns the \tt{element} of $E_\tt{i}(\tt{this},\tt{b})$.}
\If{\tt{this} \bf{is} \tt{leaf}}
\State\Return \tt{blocks[b].element} \label{getBaseCase}
\ElsIf{\tt{i <= blocks[b].num\sub{enq-left}}} \label{leftOrRight} \cmt{$E_\tt{i}(\tt{this},\tt{b})$ is in the left child of this node.}
\State \tt{subblockIndex= left.BinarySearch(i+blocks[b-1].sum\sub{enq-left}, blocks[b-1].end\sub{left}+1,}  \label{leftChildGet}
\Statex \hspace{10.7em}\tt{blocks[b].end\sub{left})} \cmt{{\color{red} start and end values are not needed anymore?}}
\State \Return\tt{left.}\Call{GetEnqueue}{subblockIndex, i} 
\Else
\State \tt{i= i-blocks[b].num\sub{enq-left}}
\State\tt{subblockIndex= right.BinarySearch(i+blocks[b-1].sum\sub{enq-right}, blocks[b-1].end\sub{right}+1,} \label{rightChildGet}
\Statex \hspace{11.1em}\tt{blocks[b].end\sub{right})} \cmt{{\color{red} start and end values are not needed anymore?}}
\State \Return\tt{right.}\Call{GetEnqueue}{subblockIndex, i} 
\EndIf
\EndFunction{GetEnqueue}

\Statex
\Statex $\leadsto$ \textsf{Precondition: \tt{b}th block of the node has propagated up to the root and \tt{blocks[b].num\sub{deq}$\geq$i}.}
\Function{<int, int>}{IndexDequeue}{\sl{int} b, \sl{int} i} \cmt{{\color{red} Update needed: return null when superblock in the root was not found.}}
\If{\tt{this} \bf{is} \tt{root}}
\State\Return \tt{<b, i>} \label{indexBaseCase}
\Else
\State \tt{dir= (parent.left==n ? left: right)} 
\State \tt{superblockIndex= parent.blocks[blocks[b].super].sum\sub{deq-dir} > blocks[b].sum\sub{deq} ? \label{computeSuper} 
\Statex \hspace{11.3em} blocks[b].super: blocks[b].super+1} \cmt{{\color{red} Preconditions might be not met.}}

\If{\tt{dir {\keywordfont is} left}} \label{computeISuperStart}
\State \tt{i+= blocks[b-1].sum\sub{deq}-parent.blocks[superblockIndex-1].sum\sub{deq-left}} \label{considerPreviousLeft}
\Else \label{considerRight}
\State \tt{i+= blocks[b-1].sum\sub{deq}-parent.blocks[superblockIndex-1].sum\sub{deq-right}}  \label{considerPreviousRight}
\State \tt{i+= parent.blocks[superblockIndex].num\sub{deq-left}}  \label{considerLeftBeforeRight}
\EndIf \label{computeISuperEnd}
\State \Return\tt{this.parent.}\Call{IndexDequeue}{superblockIndex, i}
\EndIf
\EndFunction{IndexDequeue}

\end{algorithmic}
\end{algorithm}

\color{red}
\section{MaxByProcess}

\begin{algorithm}
\color{red}
\caption{\tt{\sl{\color{red}MaxByProcess}}}
\begin{algorithmic}[1]
\setcounter{ALG@line}{121}

\State \tt{\sl{int[p]} lastDequeuedbyProcess}
\Statex

\Function{int}{Get}{}
\State \Return \tt{max(lastDequeuedbyProcess)}
\EndFunction{Get}

\Statex
\Function{}{Update}{\sl{int} \tt{b}}
\If{\tt{lastDequeuedbyProcess[pid]<b}}
\State \tt{lastDequeuedbyProcess[pid]=b}
\EndIf
\EndFunction{Update}

\end{algorithmic}
\end{algorithm}


\section{Help}

\begin{algorithm}
\color{red}
\caption{\tt{\sl{Tree}}}
\begin{algorithmic}[1]
\setcounter{ALG@line}{130}

\Function{int}{Help}{}
\State \bf{for each} \sl{process} \tt{P}
\State \tt{h=P.leaf.head}
\If{\tt{P.leaf.blocks[h].num\sub{deq}==1 and P.leaf.IndexDequeue(h,1)!=null}}
\State \tt{<b, i>=} \Call{IndexDequeue}{h, 1}
\State \tt{output=} \Call{FindResponse}{b, i} 
\State \tt{P.leaf.blocks[h].response= output}
\EndIf
\State \bf{end for}
\EndFunction{Help}

\end{algorithmic}
\end{algorithm}


\section{FreeMemory}

\begin{algorithm}
\color{red}
\caption{\tt{\sl{Node}}}
\begin{algorithmic}[1]
\setcounter{ALG@line}{140}

\Function{}{FreeMemory}{\sl{int} \tt{b}}
\If{not leaf}
\State \tt{left.FreeMemory(blocks[i].end\sub{left}-1)}
\State \tt{right.FreeMemory(blocks[i].end\sub{right}-1)}
\EndIf
\State \tt{blocks= blocks.splitGreater(i)} \cmt{I think CAS is not needed.}
\EndFunction{FreeMemory}

\end{algorithmic}
\end{algorithm}





\section{PBRT}

\begin{algorithm}
\color{red}
\caption{\tt{\sl{PBRT}}}
\begin{algorithmic}[1]
\setcounter{ALG@line}{140}

\Statex \tt{\sl{PRBT} prbt}
\Statex \sf{nodes store \tt{<key, sum\sub{enq}-> block}}
\Statex \tt{[i] -> GetByBlock(i)}

\Statex
\Function{}{GetByBlock}{\sl{int} \tt{i}}
\State \Return \tt{rbt.get(i)}
\If{not found}
\State \tt{return written response}
\EndIf
\EndFunction{GetByBlock}

\Statex
\Function{}{TryAppend}{\sl{block} \tt{B}, \sl{int} \tt{i}}
\Statex {\color{red} Tries to append \nf{B} with key \nf{i} to the red-black tree.}
{\color{red}\If{\tt{i\%p\textsuperscript{2}=0}}
\State \Call{Help()}{}
\State \tt{root.}\Call{FreeMemory()}{lastDequeuedFrom.Get()-1}
\State \tt{garbageCollectRound=floor(root.head/p\textsuperscript{2})}
\EndIf}
\EndFunction{FreeMemory}

\end{algorithmic}
\end{algorithm}

\pagebreak
\section{Description}

\color{black}
In our original algorithm an \nf{Enqueue} or a \nf{Dequeue} remains in the \nf{blocks} array in the tree nodes even after they terminate. This makes the space used by the algorithm factor of the number of operations of invoked on the queue. In this section here we want to free the memory allocated by the operations that are no longer needed and make the space used polynomial of $p+q$.

One way of handling garbage collection is to reallocate the space taken by each operation right after it is not needed anymore, but it might cost too much to do that. So we garbage collect by batching. If we garbage collect the unnecessary blocks in a node every $p^2$ block appended to the node, the garbage collection cost is amortized over $p^2$ blocks which is $O(p^3)$ and $\Omega(p^2)$.

In our design a process garbage collects a node when it attempts to append a block in the $kp^2$ position in the. \nf{Garbage Collect} corresponded to the $kp^2$th block in node $n$ is called the $k$th round of garbage collect on $n$. Every $p^2$  block appended to a node at least one garbage collect happens because the node's \nf{head} cannot advance until the garbage collect is done (see Lines ).

\begin{lemma}
The number of the blocks in the \nf{blocks} is $O(p^2+q)$.
\end{lemma}


An \nf{Enqueue} operation can be removed from the tree after its \nf{element} has been computed to be the response of some \nf{Dequeue}. It is safe to remove a \nf{Dequeue} operation from the tree after the \nf{Dequeue} is terminated, but a \nf{Dequeue} might sleep for a long time and prevents to be garbage collected. In that situation other processes can help the \nf{Dequeue} to compute its response and wtite down its response somewhere. If we remove a \nf{Dequeue} from the tree after it is helped the \nf{Dequeue} can read the helped response writtern when it encountered a problem (computing the \nf{Dequeue} index in the \nf{root} or getting the response \nf{Enqueue}). A \nf{Dequeue} operation can be removed from the tree after its response has been computed. We say a block  is \it{finished} if all of its operations can be removed.

\begin{lemma}
There are at most $O(p^2+q)$ unfinished blocks in a node's \nf{blocks} at a time.
\end{lemma}

If the $i$th \nf{Enqueue} gets dequeued in a FIFO queue, it means $1$ to $i-1$th \nf{Enqueue} operations are also dequeued. This gives us the idea that if a block is finished then all the blocks before it are also finished. If an operation in a block goes to sleep for a long time then other processes help the operation so the block get finished. Note that when an \nf{Enqueue} operation in a block is not finished the block cannot be finished until some \nf{Dequeue} dequeues that \nf{Enqueue}. Since there are at most $p$ idle operations, we can help them before garbage collection and then remove all the finished blocks safely.

\begin{lemma}
    If all current operations are helped, then there is a block in the root that all of its previous blocks are finished. That block is the most recent block that has been dequeued from.
\end{lemma}

The idea above leads us to a poly-log data structure that supports throwing away all the blocks with keys smaller than an index. Red-black trees do this for us. \nf{Get(i)}, \nf{Append()} and \nf{Split(i)} are logarithmic in block trees.
We can create a shared red-black tree just creating a new path for the operation and then using \nf{CAS} to change the root of the tree. See [this] for more.

\begin{observation}
PBRT supports poly-log operations ....
\end{observation}

\begin{lemma}
    If we replace the arrays we used to implement \nf{blocks} with red-black trees the amortized complexity of the algorithm would be $PolyLog(p,q)$. And also the algorithm is correct.
\end{lemma}

We can help a \nf{Dequeue} by computing its response and writing it down. If the process in future failed to execute, it can read the helped value written down.

\begin{lemma}
The \nf{response} written is correct.
\end{lemma}

But how can we know which blocks in each node are finished or not? 

\begin{lemma}
    If all current operations are helped, then the blocks before the newest block that some \nf{Enqueue} has been dequeued from is safe to remove. If the most current \nf{Dequeue} returned \nf{null} then all the blocks before the block containing the \nf{Dequeue} can be removed.
\end{lemma}

There is a shared array among processes which they write the last block dequeued from in it. 


\begin{lemma}
    $\nf{GetLastDequeuedFrom($n$} - \text{index of the last finished block}$ in the node $n$ is $O(p)$.
\end{lemma}

\begin{lemma}
    After \nf{FreeMemory}, the space taken by each node of the tree is $O(p^2+q)$. The total space in the tree is $PolyLog(p+q)$
\end{lemma}


\end{document}

