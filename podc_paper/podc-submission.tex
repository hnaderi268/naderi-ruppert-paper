%%
%% This is file generated with the docstrip utility from the source file: 
%%
%% samples.dtx  (with options: `acmsmall-conf')
%% With some modifications for PODC 2023.
%% For the copyright see the source file.
%%
%%
%% Commands for TeXCount
%TC:macro \cite [option:text,text]
%TC:macro \citep [option:text,text]
%TC:macro \citet [option:text,text]
%TC:envir table 0 1
%TC:envir table* 0 1
%TC:envir tabular [ignore] word
%TC:envir displaymath 0 word
%TC:envir math 0 word
%TC:envir comment 0 0
%%
%%
%% The first command in your LaTeX source must be the \documentclass
%% command.
%%
%% For submission and review of your manuscript please change the
%% command to \documentclass[manuscript, screen, review]{acmart}.
%%
%% When submitting camera ready or to TAPS, please change the command
%% to \documentclass[sigconf]{acmart} or whichever template is required
%% for your publication.
%%
%%
\documentclass[acmsmall,nonacm,anonymous]{acmart}

\usepackage{tikz-qtree}
\usepackage{algorithm}
\usepackage{algpseudocode}
\makeatletter
\renewcommand{\ALG@beginalgorithmic}{\footnotesize}
\makeatother
\usepackage{graphicx}
\usepackage{subcaption}
\usepackage{hyperref}
\usepackage{amsmath}
\usepackage{relsize}
\usepackage{enumitem}
\usepackage{bold-extra}
\usepackage{showkeys}
\usepackage{cleveref}
\renewcommand*\contentsname{Table of Contents}

\usepackage{multicol}
\setlength\columnsep{24pt}

\numberwithin{equation}{section}

\def\draft{1}  % set to 1 for draft (include comments, color edits), 0 for clean final copy (omit comments, edits in black)
\if\draft 1
\newcommand{\mycomment}[3]{{\color{#2}{[\bf{#1: #3}]}}}
\newcommand{\myedit}[2]{{\color{#1}{#2}}\normalcolor}
\newcommand{\rebuttaledit}[2]{{\color{#1}{#2}}\normalcolor}
\newcommand{\here}[1]{\bf{[[[#1]]]}}
%\usepackage{showkeys}
\else
\newcommand{\mycomment}[3]{}
\newcommand{\rebuttaledit}[2]{#2}
\newcommand{\myedit}[2]{#2}
\newcommand{\here}[1]{}  % \here's include comments to ourselves that readers should not see.
\fi

\newcommand{\Eric}[1]{\mycomment{Eric}{magenta}{#1}}
\newcommand{\Hossein}[1]{\mycomment{Hossein}{blue}{#1}}
\newcommand{\er}[1]{\myedit{magenta}{#1}}

\newcommand{\node}{node}
\newcommand{\nodes}{nodes}
\newcommand{\block}{block}
\newcommand{\blocks}{blocks}

\algnewcommand\algorithmicforeach{\bf{for each}}
\algdef{S}[FOR]{ForEach}[1]{\algorithmicforeach\ #1\ \algorithmicdo}

\algdef{S}[FUNCTION]{Function}
   [3]{{\typ{{#1}}} {\op{#2}}\ifthenelse{\equal{#3}{}}{}{(\var{#3})}}
  
\algdef{E}[FUNCTION]{EndFunction}
   [1]{\algorithmicend\ \op{{#1}}}

\algrenewcommand\Call[2]{\op{#1}\ifthenelse{\equal{#2}{}}{}{(#2)}}
\usepackage{eqparbox}

\newcommand\keywordfont{\sffamily\bfseries}
\algrenewcommand\algorithmicend{{\keywordfont end}}
\algrenewcommand\algorithmicfor{{\keywordfont for}}
\algrenewcommand\algorithmicforeach{{\keywordfont for each}}
\algrenewcommand\algorithmicdo{{\keywordfont do}}
\algrenewcommand\algorithmicuntil{{\keywordfont until}}
\algrenewcommand\algorithmicfunction{{\keywordfont function}}
\algrenewcommand\algorithmicif{{\keywordfont if}}
\algrenewcommand\algorithmicthen{{\keywordfont then}}
\algrenewcommand\algorithmicelse{{\keywordfont else}}
\algrenewcommand\algorithmicreturn{{\keywordfont return}}



\renewcommand\thealgorithm{}
\newcommand{\setalglineno}[1]{%
  \setcounter{ALC@line}{\numexpr#1-1}}

\newcommand{\sub}[1]{\textsubscript{#1}}
\renewcommand{\tt}[1]{\texttt{#1}}
\renewcommand{\sl}[1]{\textsl{#1}}
\renewcommand{\it}[1]{\textit{#1}}
\renewcommand{\sc}[1]{\textsc{#1}}
\renewcommand{\bf}[1]{\textbf{#1}}
\newcommand{\nf}[1]{{\normalfont{\texttt{#1}}}}
\newcommand{\cmt}[1]{\Comment{#1}}
\newcommand{\head}{\fld{head}}
\newcommand{\size}{\fld{size}}
\newcommand{\eleft}{\fld{end\sub{left}}}
\newcommand{\eright}{\fld{end\sub{right}}}
\newcommand{\edir}{\fld{end\sub{dir}}}

\newcommand{\var}[1]{\mbox{\textnormal{\it{#1}}}}
\newcommand{\fld}[1]{\mbox{\it{#1}}}
\newcommand{\typ}[1]{\mbox{\sf #1}}
\newcommand{\ceil}[1]{\lceil #1 \rceil}
\newcommand{\nl}{\mbox{\sf null}}
\newcommand{\tr}{\mbox{\sf true}}
\newcommand{\opa}[2]{\op{#1(\var{#2})}}
\newcommand{\op}[1]{{\sf #1}}
\newcommand{\opemph}[1]{{\sf \bf{#1}}}
\newcommand{\com}{$\triangleright$}
\newcommand{\linecomment}[1]{\com\ #1}

\usepackage{amsmath,amsthm}
\newtheorem{theorem}{Theorem}
\newtheorem{lemma}[theorem]{Lemma}
\newtheorem{corollary}[theorem]{Corollary}
\newtheorem{observation}[theorem]{Observation}
%\theoremstyle{definition}
\newtheorem{definition}[theorem]{Definition}
\newtheorem{invariant}[theorem]{Invariant}
\newtheorem{proposition}[theorem]{Proposition}

%%
%% \BibTeX command to typeset BibTeX logo in the docs
\AtBeginDocument{%
  \providecommand\BibTeX{{%
    Bib\TeX}}}

%% Rights management information.  This information is sent to you
%% when you complete the rights form.  These commands have SAMPLE
%% values in them; it is your responsibility as an author to replace
%% the commands and values with those provided to you when you
%% complete the rights form.
\setcopyright{none}  %FIX FOR CAMERA READY
%\copyrightyear{2023}
%\acmYear{2023}
\acmDOI{XXXXXXX.XXXXXXX}
\fancyfoot{} % REMOVE THIS LINE FOR CAMERA READY

%% These commands are for a PROCEEDINGS abstract or paper.
\acmConference[PODC 2023]{ACM Symposium on Principles of Distributed Computing}{June 19--23,
  2023}{Orlando, FL}
%%
%%  Uncomment \acmBooktitle if the title of the proceedings is different
%%  from ``Proceedings of ...''!
%%
%%\acmBooktitle{Woodstock '18: ACM Symposium on Neural Gaze Detection,
%%  June 03--05, 2018, Woodstock, NY}
%\acmPrice{15.00}
%\acmISBN{978-1-4503-XXXX-X/18/06}


%%
%% Submission ID.
%% Use this when submitting an article to a sponsored event. You'll
%% receive a unique submission ID from the organizers
%% of the event, and this ID should be used as the parameter to this command.
%%\acmSubmissionID{123-A56-BU3}

%%
%% For managing citations, it is recommended to use bibliography
%% files in BibTeX format.
%%
%% You can then either use BibTeX with the ACM-Reference-Format style,
%% or BibLaTeX with the acmnumeric or acmauthoryear sytles, that include
%% support for advanced citation of software artefact from the
%% biblatex-software package, also separately available on CTAN.
%%
%% Look at the sample-*-biblatex.tex files for templates showcasing
%% the biblatex styles.
%%

%%
%% The majority of ACM publications use numbered citations and
%% references.  The command \citestyle{authoryear} switches to the
%% "author year" style.
%%

%%
%% end of the preamble, start of the body of the document source.
\begin{document}

\title{A Wait-free Queue with Polylogarithmic Step Complexity}


%%
%% The "author" command and its associated commands are used to define
%% the authors and their affiliations.
%% Of note is the shared affiliation of the first two authors, and the
%% "authornote" and "authornotemark" commands
%% used to denote shared contribution to the research.
\author{Eric Ruppert}
\affiliation{%
  \institution{York University}
  \streetaddress{P.O. Box 1212}
  \city{Toronto}
  \state{Ontario}
  \country{Canada}
  \postcode{43017-6221}
}
\email{ruppert@cse.yorku.ca}
\orcid{1234-5678-9012}

\author{Hossein Naderibeni}
\email{hnaderi268@gmail.com}
\affiliation{%
  \institution{York University}
  \streetaddress{P.O. Box 1212}
  \city{Toronto}
  \state{Ontario}
  \country{Canada}
  \postcode{43017-6221}
}

\settopmatter{printacmref=false}  % REMOVE THIS LINE FOR CAMERA READY

\begin{abstract}
We introduce a novel linearizable wait-free queue implementation using single-word 
CAS instructions.
Previous lock-free queues all take $\Omega(p)$ steps per operation in the worst case, 
where $p$ is the number of processes that can access the queue.
We achieve $O(\log^2 p +\log q)$ steps per operation, where $q$ is the size of the queue.
\here{mention space usage; update time bound if space-efficient version takes more time per op}
\end{abstract}

%%
%% The code below is generated by the tool at http://dl.acm.org/ccs.cfm.
%% Please copy and paste the code instead of the example below.
%%

%%
%% Keywords. The author(s) should pick words that accurately describe
%% the work being presented. Separate the keywords with commas.
%\keywords{concurrent data structures, wait-free queues}

%%
%% This command processes the author and affiliation and title
%% information and builds the first part of the formatted document.
\maketitle

% !TEX root =  podc-submission.tex

\section{Introduction}

There has been a great deal of research in the past several decades on the design of linearizable, lock-free queues.
Besides being a fundamental data structure, queues are used in
significant concurrent applications, including OS kernels \cite{MP91}, memory management (e.g., \cite{BBFRSW21}\here{try to find an older, more canonical reference}),
packet processing \cite{DPDK}, synchronization \cite{KAE23},\here{is this a good citation for this?}
and sharing resources or tasks.
The lock-free MS-queue of Michael and Scott \cite{MS98} is a classic shared queue implementation.
It uses a singly-linked list with pointers to the front and back nodes.
To dequeue or enqueue an element, the front or back pointer is updated by a 
compare-and-swap (CAS) instruction.
If this CAS fails, the operation must retry.
In the worst case, this means that each successful CAS may cause all other processes to
fail and retry, leading to an amortized step complexity of $\Omega(p)$ per operation in a system of $p$ processes.
(To measure amortized step complexity of a lock-free implementation, we consider all possible finite executions
and divide the number of steps in the execution by the number of operations  in the execution.)
Numerous papers have suggested modifications to the MS-queue \cite{DBLP:conf/opodis/HoffmanSS07,DBLP:conf/podc/KoganH14,DBLP:conf/ppopp/KoganP11,DBLP:journals/dc/Ladan-MozesS08,MKLLP22,DBLP:conf/spaa/MoirNSS05,RC17}, but 
all still have $\Omega(p)$ amortized step complexity as a result of
contention on the front and back of the queue.
Morrison and Afek \cite{DBLP:conf/ppopp/MorrisonA13} called this the \emph{CAS retry problem}.
The same problem occurs in array-based implementations of queues \cite{DBLP:conf/iceccs/ColvinG05,DBLP:conf/icdcn/Shafiei09,DBLP:conf/spaa/TsigasZ01,DBLP:conf/opodis/GidenstamST10}.
Solutions that tried to sidestep this problem using fetch\&increment \cite{DBLP:conf/ppopp/MorrisonA13,DBLP:conf/ppopp/YangM16,Nik19,10.1145/3490148.3538572}
rely on slower mechanisms to handle worst-case executions and still have $\Omega(p)$ step complexity.

Many concurrent data structures that keep track of a set of elements also have an $\Omega(p)$ term in their step complexity, as observed by Ruppert \cite{Rup16}.
For example, lock-free lists \cite{FR04,Sha15}, stacks \cite{Tre86} and search trees \cite{EFHR14} 
have an $\Omega(c)$ term in their step complexity, where $c$ represents contention,
the number of processes that access the data structure concurrently, which can be $p$ in the worst case.
Attiya and Fouren \cite{DBLP:conf/opodis/AttiyaF17} proved 
that amortized $\Omega(c)$ steps per operation are indeed necessary
for any CAS-based implementation of a lock-free bag data structure, which provides operations
to insert an element or remove an arbitrary element (chosen non-deterministically).
Since a queue trivially implements a bag, this lower bound also holds for queues.
At first glance, this would seem to settle the question of the step complexity of lock-free queues.
However, the lower bound leaves a loophole:  it holds only if $c$ is $O(\log\log p)$.
Thus, the lower bound could be stated more precisely as an amortized bound of $\Omega(\min(c,\log\log p))$ steps per operation.

We exploit this loophole.  We show  it is, in fact, possible to design a linearizable, wait-free queue
with step complexity sublinear in $p$.
Our queue is the first whose step complexity  is polylogarithmic in $p$ and in $q$, the number of elements in the queue.
For ease of presentation, we first give an unbounded-space construction where enqueues take $O(\log p)$ steps and
dequeues take $O(\log^2 p + \log q)$ steps,
and then modify it to bound the space
while  having $O(\log p\log( p+ q))$ amortized step complexity  per operation.
Both versions use single-word CAS on %reasonably-sized 
small words.
Moreover, each operation does $O(\log p)$ CAS instructions in the worst case, whereas previous
lock-free queues use 
%an unbounded number of CAS instructions in the worst case and 
$\Omega(p)$ CAS instructions, even in an amortized sense.
For the space-bounded version, we unlink unneeded objects from our data structure.
We do not address the orthogonal problem of reclaiming memory; we assume a safe
garbage collector, such as the highly optimized one that Java provides.

Our queue uses a binary tree, called the \emph{\ordering\ tree}, where each process has its own leaf.
Each process adds its operations to its leaf.
As in previous work (e.g., \cite{DBLP:conf/stoc/AfekDT95,DBLP:conf/fsttcs/JayantiP05}), operations are propagated from the leaves up to the root in a cooperative way that ensures wait-freedom
and avoids the CAS retry problem.
Operations in the root are ordered, 
and this order is used to linearize the operations and compute their responses.
\here{Either here or in related work section, talk about previous usage of ordering tree and how ours differs from it}
Explicitly storing all operations in the tree nodes would be too costly.
Instead, we use a novel implicit representation of sets
of operations that allows us to quickly merge two sets from the children of a node,
and quickly access any  operation in a~set.
A preliminary version of this work appeared in~\cite{Nad22}.
\here{Maybe say a little more about techniques used in the implementation, if space permits}


% !TEX root =  podc-submission.tex

\section{Related Work}

\paragraph{List-based Queues}
The MS-queue \cite{MS98} is a lock-free queue that has stood the test of time.
A version of it is
included in the standard Java Concurrency Package.  %java.util.concurrent.ConcurrentLinkedQueue
See the paper that introduced it for a survey of the early history of concurrent queues.
As mentioned above, the MS-queue suffers from the CAS retry problem because of contention at the front and back of the queue.
Thus, it is lock-free but not wait-free and has an amortized step complexity
of $\Theta(p)$ per operation.

A number of papers have described ways to reduce contention in the MS-queue.
Moir et al.~\cite{DBLP:conf/spaa/MoirNSS05} 
added an elimination array that allows an enqueue to pass its enqueued value directly
to a concurrent dequeue when the queue is empty or when concurrent operations 
can be linearized to empty the queue.
However, when there are $p$ concurrent enqueues (and no dequeues), the CAS retry problem
is still present.
The baskets queue of
Hoffman, Shalev, and Shavit~\cite{DBLP:conf/opodis/HoffmanSS07} 
attempts to reduce contention by grouping concurrent enqueues into baskets.
An enqueue that fails its CAS is put in the basket with the eqnueue that succeeded.
Enqueues within a basket determine their order among themselves without having to access the back of the queue.
However, if $p$ concurrent enqueues are in the same basket
the CAS retry problem occurs when they order themselves using CAS instructions.
Both modifications still have $\Omega(p)$ amortized step complexity.

Kogan and Herlihy \cite{DBLP:conf/podc/KoganH14} described how to use futures to improve
the performance of the MS-queue.
Operations return future
objects instead of responses. Later, when an operation's response is needed, it
can be evaluated using the corresponding future object.
This allows batches of enqueues or dequeues to be done at once on an MS-queue.
However, the implementation satisfies a weaker correctness condition than linearizability.
Milman-Sela et al.~\cite{MKLLP22} extended this approach to allow batches
to mix enqueues and dequeues.
\here{Is this worth saying:
They use some properties of the queue size before and after a batch, similar to a part of our work.}
In the worst case, where operations require their response right away,
batches have size 1, and both of these implementations behave like a standard MS-queue.

Ladan-Mozes and Shavit~\cite{DBLP:journals/dc/Ladan-MozesS08}
presented an optimistic  queue implementation. 
%MS-queue uses
%two \nf{CAS}es to do an enqueue: one to change the tail to the new
%node and another one to change the next pointer of the previous node
%to the new node. 
In the MS-queue, an enqueue requires two CAS steps.
The optimistic queue uses a doubly-linked list to reduce the number of
\texttt{CAS} instructions to one in the best case. 
Pointers in the doubly-linked list can be inconsistent, but are fixed when needed by traversing the list.
Although this fixing is rare in practice, it yields an amortized complexity of $\Omega(qp)$ 
steps per operation for worst-case executions.

Kogan and Petrank~\cite{DBLP:conf/ppopp/KoganP11} 
added Herlihy's helping
technique~\cite{10.1145/114005.102808} to the MS-queue to obtain
a wait-free queue.
Ramalhete and Correia \cite{RC17} added a different helping mechanism.
In both cases, the helping mechanisms only add to the step complexity of the MS-queue.


-----

% OMIT, since it is not lock-free:
%Hendler et al.~\cite{DBLP:conf/spaa/HendlerIST10} proposed a new
%paradigm called flat combining. The key idea behind flat combining is
%to allow a combiner who has acquired the global lock on the data
%structure to learn about the requests of threads on the queue, combine
%them and apply the combined results on the data structure. Their queue
%is linearizable but not lock-free and they present experiments that
%show their algorithm performs well in some situations. 

Gidenstam, Sundell, and Tsigas~\cite{DBLP:conf/opodis/GidenstamST10}
introduced a new algorithm using a linked list of arrays. The queue is
stored in a shared array where head and tail pointers point to the
current elements in the queue. When the array is full, an empty array
is linked to the array and tail pointers are updated. A global head
points to the array containing the first element in the queue, and
each process has a local head index that points to the first element
in that array. Global tail and local tail pointers are similar. A
process updates the position of the pointers after it does an
operation. One process might go to sleep before setting the pointers,
so the pointers might be behind their real places. They mention how to
scan the arrays to update pointers while doing an operation. A process
writes an element in the location head by a \nf{CAS} instruction, so
if $p$ processes try to enqueue simultaneously, the amortized step
(and \nf{CAS}) complexity remains $\Omega(p)$. Their design is
lock-free but not wait-free. 



Nikolaev and Ravindran~\cite{10.1145/3490148.3538572} present a
wait-free queue that uses the fast-path slow-path methodology
introduced by Kogan and Petrank~\cite{10.1145/2370036.2145835}. Their
work is based on a circular queue using bounded memory. When a process
wishes to do an enqueue or a dequeue, it starts two paths. The fast
path  ensures good performance while the slow path ensures
termination. They show that these two paths do not affect each other
and the queue remains consistent. If a process makes no progress,
other processes help its slow path to finish. The helping phase
suffers from the CAS Retry Problem because processes compete in a
\nf{CAS} loop to decide which succeeds to help. Because of this, the
amortized complexity cannot be better than $\Omega(p)$. 

The CAS Retry Problem is not limited to list-based queues; array-based
queues also share
it~\cite{DBLP:conf/iceccs/ColvinG05,DBLP:conf/icdcn/Shafiei09,DBLP:conf/spaa/TsigasZ01}.
Our motivation is to overcome this problem and present a wait-free
sublinear queue. 

\here{Check:
John Giacomoni et al. FastForward
https://dl-acm-org.ezproxy.library.yorku.ca/doi/10.1145/1345206.1345215
}

\paragraph{Other Primitives and Restricted Queues}

David introduced the first sublinear-time queue
\cite{DBLP:conf/wdag/David04}, but it works only for a single enqueuer.
It uses fetch\&increment and swap primitive instructions and takes constant time per operation, but
uses unbounded memory.  A modification to use bounded space
is described, but it increases the time per operation to $\Omega(p)$.

Jayanti and Petrovic introduced a wait-free poly-logarithmic
queue~\cite{DBLP:conf/fsttcs/JayantiP05}, but it only works for a single dequeuer. 
Our implementation uses their idea of having
a tournament tree among processes to agree on the linearization of
operations.

\here{new result I found:}
Khanchandani and Wattenhofer \cite{KW18} gave a wait-free queue implementation
with $O(\sqrt{p})$ step complexity using non-standard synchronization primitives
called half-increment and half-max, which can be viewed as particular kinds of
double-word read-modify-write operations.
They use this as evidence that their primitives can be more efficient than CAS
since previous CAS-based queues all required $\Omega(p)$ step complexity.
Our new implementation counters this argument.


\subsection{Universal Constructions and Other Poly-log Time Data Structures}
A \textit{universal construction} is an algorithm that can implement a
shared version of any given sequential object. The first universal
construction was introduced by
Herlihy~\cite{10.1145/114005.102808}. We can implement a concurrent
queue using a universal construction. Jayanti proved an $\Omega(\log
p)$ lower bound on the worst-case shared-access time complexity of
$p$-process universal
constructions~\cite{DBLP:conf/podc/Jayanti98a}. He also mentions that
the universal construction by Afek, Dauber, and
Touitou~\cite{DBLP:conf/stoc/AfekDT95} can be modified to $O(\log p)$
worst-case step complexity, using atomic access to $\Omega(p \log
p)$-bit words. Chandra, Jayanti and Tan introduced a semi-universal
construction that achieves $\textsc{O}(\log^2 p)$ shared
accesses~\cite{DBLP:conf/podc/ChandraJT98}. However, their algorithm
cannot be used to create a queue. We mention a non-practical universal
construction with a poly-log number of \nf{CAS} instructions in the
last paragraph of page 13. 

Ellen and Woelfel introduced an implementation of a Fetch\&Inc object
with step complexity of $O(\log p)$ using $O(\log n$)-bit
\texttt{LL/SC} objects, where $n$ is the number of
operations~\cite{10.1007/978-3-642-41527-2_20}. Their idea to achieve
logarithmic complexity is to use a tree storing the Fetch\&Inc
operations invoked by processes. When a process wants to do a
Fetch\&Inc it adds its Fetch\&Inc to the tree and returns the number
of elements in the tree. There are some similarities between designing
a queue and a Fetch\&Inc object. A Fetch\&Inc object can be
constructed from a queue. The algorithm by Ellen and Woelfel is
interesting because of the similarities between Fetch\&Inc objects and
queues. Also, it is one of the few wait-free data structures achieving
poly-logarithmic complexity. 



% !TEX root =  podc-submission.tex

\section{Queue Implementation} \label{DescriptQ}

\subsection{Overview of Implementation}
We use a \emph{tournament tree} to agree on a total ordering of the operations performed on the queue.
The tree is a static binary tree of height $\ceil{\log_2 p}$ with one leaf 
assigned to each process. 
Each tree node  stores an array of \emph{blocks}, where each block represents a 
sequence of enqueues and a sequence of dequeues.
See Figure \ref{tournament} for an example.
In this section, we assume for simplicity that each node has an infinite blocks array.
Section \ref{reducing} describes how to replace the infinite array by a representation that uses bounded space.


To perform an operation on the queue, a process $P$ creates a block containing that single 
operation and appends it to the blocks array in $P$'s leaf.
Then, $P$ attempts to propagate the operation to each node along the path from that leaf to the root of the tree.
We shall define a total order on all operations that have been propagated to the root, which 
will serve as the linearization ordering of the operations.

To propagate an operation to a node $v$ in the tree, $P$ first observes
the blocks in both of $v$'s children that are not already in $v$,
creates a new block by combining information from those blocks, and attempts to append this 
new block to $v$'s blocks array using a CAS instruction.
Following \cite{}, we call this a 3-step sequence a
\op{refresh} on $v$. %(see Figure \ref{fig::propagstep}).
A successful \op{refresh} by $P$ may propagate pending operations by several other processes to $v$.
This cooperative approach is sufficient to ensure that if $P$ fails to CAS a 
new block into $v$'s array twice,
then its operation has been propagated to $v$ by some other process, so $P$ can continue 
onwards towards the root.

Now suppose $P$'s operation has been propagated all the way to the root.
If $P$'s operation is an enqueue, it has obtained a place in the linearization ordering and can terminate.
If $P$'s operation is a dequeue, $P$ must use information in the tree to compute the value that the
dequeue must return.  To do this, $P$ first determines which block in the root contains its operation
(since the operation may have been propagated to the root by some other process).
Then, $P$ determines whether the queue is empty when its dequeue is linearized. 
If so, it returns \var{null} and we call it a \emph{null dequeue}.
If not, $P$ computes the rank\footnote{We say that the $r$th element in a sequence has rank $r$ within that sequence.} $r$ of its dequeue in the linearization ordering
among all non-null dequeues,
finds the $r$th enqueue in the linearization, and returns that enqueue's value.

The primary challenge is thus figuring out what information to store in each block so that 
the following tasks can be done efficiently (in a polylogarithmic number of steps).
\begin{enumerate}[label={(T\arabic*)}]
\item
\label{construct}
Construct a block for node $v$ that represents the operations contained in $O(p)$ consecutive blocks in $v$'s children, as required for a \op{refresh}.
\item
\label{findinroot}
Given a dequeue in a leaf that has been propagated to the root, find that operation's position in the root's blocks array.
\item
\label{findrank}
Given a dequeue's position in the root, determine whether it is a null dequeue (i.e., whether the queue is empty when it is linearized)
or determine the rank $r$ of the enqueue whose value it should return.
\item
\label{findenqueue}
Find the $r$th enqueue in the linearization ordering.
\end{enumerate}
Since these tasks depends on the linearization ordering, we describe that ordering next.

\begin{figure}[tb]
\input{tournamentTree.pdf_t}
\caption{An example tournament tree with four processes. 
We show explicitly the enqueue sequence and dequeue sequence represented by each block in the \var{blocks} arrays of the seven nodes.  The leftmost element of each \var{blocks} array is a dummy block.
Arrows represent the indices stored in \eleft\ and \eright\ fields of blocks (as described in Section \ref{sec:fields}).
The fourth process's Deq\sub{6} is still in progress.
The linearization order for this tree is
Enq(a) Enq(e) Deq\sub{2} $\mid$ Enq(b) Deq\sub{4} Deq\sub{5} $\mid$ Enq(d) Enq(f) Enq(h) Deq\sub{1} $\mid$ Enq(c) Deq\sub{3} $\mid$ Enq(g), where vertical bars indicate boundaries of blocks in the root.\label{tournament}}
\end{figure}

\begin{figure}
\input{implicit.pdf_t}
\caption{\label{implicit}The actual, implicit representation of the tree shown in Figure \ref{tournament}.
The leaf blocks simply show the \var{element} field.
Internal blocks show the \var{sum\sub{enq}} and \var{sum\sub{deq}} fields,
and \eleft\ and \eright\ fields are shown using arrows as before.
Root blocks also contain the additional \var{size} field.
The \var{super} field is not shown.}
\end{figure}

\subsection{Linearization Ordering}

An operation can terminate only after a block containing it has been appended to the root's blocks array.
So, if an operation $op_1$ terminates before another operation $op_2$ begins, 
$op_1$ will be in an earlier block than $op_2$ in root's blocks array.
Thus, our linearization orders operations according to the block they belong to in the root's blocks array.
Operations that appear in the same block are necessarily concurrent, so we can choose how to order them.

Each block in a leaf represents a single operation.
Each block $B$ in an internal node $v$ results from merging
several consecutive blocks from each of $v$'s children.
The blocks in $v$'s children are called the \emph{direct subblocks} of $B$.
A block $B'$ in a descendant of $v$ is a \emph{subblock} of $B$ if it is a direct subblock of $B$
or a subblock of a direct subblock of $B$.
A block $B$ represents the set of operations in all of $B$'s subblocks in leaves of the tree.

The operations propagated by a \op{refresh} are all pending when the \op{refresh} occurs,
so there is at most one operation per process.
Hence, a block represents at most $p$ operations in total.  
Moreover, we never append empty blocks, so 
each block represents at least one operation and it follows that a block can have at most $p$ direct subblocks.

As mentioned above, we are free to order operations within a block however we like.
For convenience, we order the enqueues and dequeues separately, and put the 
operations that were propagated from the left child before the operations from the right child.
More formally, we inductively define an sequence $E(B)$ of the enqueues represented by a block $B$.
If $B$ is a block in a leaf representing an enqueue operation, its enqueue sequence $E(B)$ is either the single
enqueue represented by the block (if $B$ stores an enqueue), or the empty sequence (if $B$ stores a dequeue).
If $B$ is a block in an internal node $v$ with direct subblocks $B^L_1, \ldots, B^L_\ell$ from the left child of $v$
and $B^R_1,\ldots,B^R_r$ from the right child of $v$, then $B$'s enqueue sequence is the concatenation $E(B^L_1)\cdots E(B^L_\ell)\cdot E(B^R_1) \cdots E(B^R_r)$.
The dequeue sequence $D(B)$ of a block $B$ is defined symmetrically.

Finally, when designing a total order for the operations propagated to the root, we choose
to put each block's enqueues before its dequeues.
Thus, if the root's blocks array contains blocks $B_1, \ldots, B_k$, the 
linearization ordering is 
$E(B_1)\cdot D(B_1) \cdot E(B_2) \cdot D(B_2) \cdots E(B_k) \cdot D(B_k)$.



\subsection{Designing a Block Representation to Solve Tasks \ref{construct} to \ref{findenqueue}}
\label{sec:fields}

Each node has an infinite array called \var{blocks}.
To simplify the code, each node has an empty block in \var{blocks[0]}.
Each node also has a \var{head} index that stores the position in the \var{blocks} array to be used
for the next attempt to append a block to this node.

If a block contained an explicit representation of its sequences of enqueues and dequeues,
it would take $\Omega(p)$ time to construct a block, which would be too slow for task \ref{construct}.
Instead, the block stores an implicit representation of the sequences.
We now explain how we designed the fields for this implicit representation. 
Refer to Figure \ref{implicit} for an example showing how the tree in Figure \ref{tournament} is actually represented, and Figure \ref{object-fields} for the definitions of the fields of blocks and tree nodes.

A block in a leaf represents a single enqueue or dequeue.  The \var{element} field stores the value
enqueued for an enqueue operation, or \var{null} if the operation is a dequeue.

\here{If space permits, we might want to add some examples in the following paragraphs that refer back to Figure \ref{implicit}.}

Each block in an internal node $v$ stores the indices of the last direct subblock in $v$'s left and right child in the fields \eleft\ and \eright.  This allows us to navigate to the direct subblocks of any block easily.
Blocks also store prefix sums of the numbers of enqueues and dequeues:
the block in $v.\var{blocks}[i]$ has two fields \var{sum\sub{enq}} and \var{sum\sub{deq}}
that store the total number of enqueues and dequeues in $v.\var{blocks[1..i]}$.
These fields allow us to pinpoint the exact location of an operation among the subblocks of a given block.
For example, consider finding the $r$th enqueue in the linearization ordering.
We use the \var{sum\sub{enq}} fields of the root's blocks to do a binary search
to locate which block in the root contains the operation.
If we know which block $B$ in some node $v$ contains the enqueue,
we can use the \var{sum\sub{enq}} field again to determine which child of $v$ contains the enqueue
and then to do a binary search
among the direct subblocks of $B$ in that child.
Thus, we work our way down the tree until we find a leaf block, which explicitly stores 
the enqueue.
%Now, suppose we want to find the $r$th enqueue in the linearization ordering for task \ref{findenqueue}.Let $B_1, B_2, \ldots, B_k$ be the blocks in the root.
%First, we need prefix sums of the number of enqueues in $E(B_1)\cdot E(B_2)\cdots E(B_i)$
%so that we can do a binary search for the block $B_e$ that contains the $r$th enqueue.
%This prefix sum also allows us to know the rank $r'$ within $E(B_e)$ of the $r$th enqueue.
%Once we have $r'$, we need the number of enqueues that $B_e$ received from its left child
%to determine whether the enqueue came from the left or right child of the root.
%Suppose the enqueue came from the right child $v_r$.
%Then, we know that the index of the block in $v_r$ that contains the enqueue
%is between $B_{e-1}.\eright + 1$ and $B_e$.\eright.
%We can again do a binary search within this range.
%For this, we can again use the prefix sums of the number of enqueues in any prefix of the array $v_r.blocks$.
%We can then continue in this way down the tree until reaching a leaf where the enqueue is stored explicitly.
We shall show that the binary search in the root can be done in $O(\log p + \log q)$ steps,
and the binary search within each other node along the path to a leaf can be done in $O(\log p)$ steps,
for a total of $O(\log^2 p + \log q)$ steps to complete task \ref{findenqueue}.
All of the information needed for this search process can be derived from the 
\eleft, \eright\ and \var{sum\sub{enq}} fields.

To facilitate task \ref{findinroot}, each block has a field \var{super} that contains
the (approximate) index of its superblock in the parent node's blocks array.
We will ensure that this field's value differs from the true index of the superblock by at most 1.
This allows a process to determine the true location of the superblock by checking the \eleft\ or \eright\ values of just two blocks in the parent node.
Thus, starting from an operation in a leaf's block, one can use these indices to track the 
operation all the way up the path to the root, and determine the operation's location in a root block
in $O(\log p)$ time.

We now consider task \ref{findrank}.
First, we must determine whether the queue is empty when a dequeue occurs.
To facilitate this, each block in the root stores a \var{size} field that stores the number of elements
in the queue after all operations in the linearization ordering up to that block (inclusive) 
have been performed.
We can easily determine which dequeues in a block $B_d$ in the root are null dequeues using
$B_{d-1}.\var{size}$, which stores the size of the queue just before $B_d$'s operations are performed, and the number of enqueues and dequeues in $B_d$.
We can also compute the number of non-null dequeues in blocks $B_1, \ldots, B_{d-1}$ 
as $B_{d-1}.\var{sum\sub{enq}}-B_{d-1}.\var{size}$.
Then, for any non-null dequeue in $B_d$, we can use this information to determine its
rank among all non-null dequeues in the linearization ordering, and hence the rank of the enqueue
whose value it should return (among all enqueues).

Now that we have defined the fields required for tasks \ref{findinroot}, \ref{findrank} and \ref{findenqueue},
we can easily see how to construct a new block $B$ during a \op{refresh} in $O(1)$ time.
A \op{refresh} on node $v$ reads the value $h$ of the \var{head} field of each of $v$'s children and stores 
$h-1$ in the $B.\eleft$ and $B.\eright$ fields of the new block.
Then, we can compute $B.\var{sum\sub{enq}}$ as $v.\var{left}.\var{blocks}[B.\eleft].\var{sum\sub{enq}} + v.\var{right}.\var{blocks}[B.\eright].\var{sum\sub{enq}}$.
For a block $B$ in the root, $B.\var{size}$ can be computed using the \var{size} field of the previous block $B'$ and
the number of enqueues and dequeues in $B$:
$B.\var{size} = \max(0, B'.\var{size} + (B.\var{sum\sub{enq}}-B'.\var{sum\sub{enq}}) - (B.\var{sum\sub{deq}} - B'.\var{sum\sub{deq}}))$.

The only remaining field is $B.\var{super}$.  When the block 
$B$ is created for a node $v$, we do not yet know where its
superblock will eventually be installed in $B$'s parent.
So, we leave this field blank.  After $B$ is installed 
in $v.\var{blocks[i]}$, processes cooperate to fill it in 
when they attempt to advance $v.\var{head}$ from $i$ to $i+1$.
They use the value read from the \var{head} field of $v$'s parent at that time.
As mentioned above, this might not be exactly the right index for $B$'s superblock, but we
shall prove that it is close.


\begin{figure}
\here{Does \var{leaf} have to be shared for helping when doing GC?}
\begin{algorithmic}[1]
\setcounter{ALG@line}{1}
\Statex $\diamondsuit$ \tt{\sl{Shared variable}}
\begin{itemize}
\item \tt{\sl{Node} root} \Comment{root of a binary tree of \tt{Node}s with one leaf for each process}
\end{itemize}

\Statex $\diamondsuit$ \tt{\sl{Thread-local variable}} 
\begin{itemize}
\item \tt{\sl{Node} leaf} \Comment{process's leaf in the tree}
\end{itemize}

\Statex $\blacktriangleright$ \tt{\sl{Node}}
\begin{itemize}
\item \tt{\sl{Node} left, right, parent} \Comment{children and parent pointers initialized  when creating the tree}
\item \tt{\sl{Block[0..$\infty$]} blocks} \Comment{initially \tt{blocks[0]} contains an empty block with all integer fields equal to 0}
\item \tt{\sl{int} \head} \Comment{position to attempt appending next \tt{block} to \tt{blocks}, initially 1}
\end{itemize}

\Statex $\blacktriangleright$ \tt{\sl{Block}} 

\begin{itemize}
  	\item \tt{\sl{int} sum\sub{enq}, sum\sub{deq}}
  		\Comment{number of enqueue, dequeue operations in \var{blocks} array up to this block (inclusive)}
  	\item \tt{\sl{int} super}
  		\Comment{approximate index of the block's superblock in \var{parent.blocks}}
	\item[\com] Blocks in internal nodes have the following additional fields
	\item \tt{int} \eleft, \eright
  		\Comment{index of last direct subblock in the left and right child}
  	\item[\com] Blocks in leaf nodes have the following additional field
  	\item \tt{\sl{Object} element}
  		\Comment{if the block's operation is \tt{enqueue(x)} then \tt{element=x}, otherwise \tt{element=null}.}
	\item[\com] Blocks in the root node have the following additional field
	\item \tt{\sl{int} \size}%
  		\Comment{size of the queue after performing all operations up to the end of this block}
\end{itemize}

%\Statex {\com\  Blocks in internal nodes have the following additional fields}
%\Statex $\blacktriangleright$ \tt{\sl{InternalBlock} extends \sl{Block}} \sf{\com\ the following additional fields are used only for blocks in internal nodes}
%\begin{itemize}
%	\item \tt{int} \eleft, \eright
%  		\Comment{index of last direct subblock in the left and right child}
%\end{itemize}

%\Statex {\com\ Blocks in leaf nodes have the following additional field}
%\Statex $\blacktriangleright$ \tt{\sl{LeafBlock} extends \sl{Block}} \sf{\com\ the following additional field is used only for blocks in leaves}
%\begin{itemize}
%  \item \tt{\sl{Object} element}
%  \Comment{if the block's operation is \tt{enqueue(x)} then \tt{element=x}, otherwise \tt{element=null}.}
%\end{itemize}

%\Statex {\com\ Blocks in the root node have the following additional field}
%\Statex $\blacktriangleright$ \tt{\sl{RootBlock} extends \sl{InternalBlock}} \sf{\com\ the following additional field is used only for blocks in the root}
%\begin{itemize}
%  \item \tt{\sl{int} \size}%
%  \Comment{size of the queue after performing all operations up to the end of this block}
%\end{itemize}

\end{algorithmic}
\caption{Objects used in tournament tree data structure \label{object-fields}}
\end{figure}

\subsection{Details of the Implementation}
Section \ref{algQ} gives the pseudocode for the queue
implementation. It uses the following two types of objects. 
\paragraph{\tt{Node}} 
 In each \texttt{Node} we store pointers to its parent and its left
 and right child, an array of \nf{Block}s called \texttt{block}s and
 the index \nf{head} of the first empty entry in \texttt{blocks}. 

\paragraph{\tt{Block}}
 The information stored in a \nf{Block} depends on whether  the
 \nf{Block} is in an internal node or a leaf. If it is in a leaf, we
 use a \nf{LeafBlock} which stores one operation. If a block $B$ is in
 an internal node $n$, then it contains subblocks in the left and
 right children of $n$. The left subblocks of $B$ are some consecutive
 blocks in the left child of $n$ starting from where the left
 subblocks of the block prior to $B$ ended. The right subblocks of $B$
 are defined similarly. In each \texttt{block} we store four essential
 fields that implicitly summarize which operations are in the block
 \texttt{sum\textsubscript{enq-left}},
 \texttt{sum\textsubscript{deq-left}},
 \texttt{sum\textsubscript{enq-right}},
 \texttt{sum\textsubscript{deq-right}}. The
 \texttt{sum\textsubscript{enq-left}} field is the total number of
 \nf{Enqueue} operations in the blocks before the last subblock of $B$
 in the left child. The other fields' semantics are similar. The
 \nf{end\sub{left}} and \nf{end\sub{right}} field store the index of
 the last subblock of a block in the left and the right child,
 respectively. The approximate index of the superblock of non-root
 blocks is stored in their \nf{super} field. The \texttt{size} field
 in a block in the root node stores the size of the queue after the
 operations in the block have been performed.  

We now describe the routines used in the implementation.

\paragraph{\tt{Enqueue($e$)}}
An \nf{Enqueue} operation does not return a response, so it is
sufficient to propagate the \nf{Enqueue} operation to the root and
then use its position in the linearization for future \nf{Dequeue}
operations. \nf{Enqueue($e$)} creates a \nf{LeafBlock} with
$\nf{element}=e$, sets its \nf{sum\sub{enq}} and \nf{sum\sub{deq}}
fields and then appends it to the tree. 

\paragraph{\tt{Dequeue()}}
\nf{Dequeue} creates a \nf{LeafBlock}, sets its \nf{sum\sub{enq}} and
\nf{sum\sub{deq}} fields and appends it to the tree. Then, it computes
the position of the appended \nf{Dequeue} operation in the root using
\nf{IndexDequeue} and after that finds the response of the
\nf{Dequeue} by calling \nf{FindResponse}. 

\paragraph{\tt{Append($B$)}}
The \nf{head} field is the index of the first empty slot in
\nf{blocks} in a \nf{LeafBlock}. There are no multiple write accesses
on \nf{head} and \nf{blocks} in a leaf because only the process that
the leaf belongs to appends to it. \nf{Append($B$)} adds $B$ to the
end of the \nf{blocks} field in the leaf, increments \nf{head} and
then calls \nf{Propagate} on the leaf's \nf{parent}. When
\nf{Propagate} terminates, it is guaranteed that the appended block is
a subblock of a block in the \nf{root}.  

\paragraph{\tt{Propagate()}}
\nf{Propagate} on node $n$ uses the double refresh idea described
earlier and invokes two \nf{Refresh}es on $n$ in Lines
\ref{firstRefresh} and \ref{secondRefresh}. Then, it invokes
\nf{Propagate} on \nf{$n$.parent} recursively until it reaches the
root.  

\paragraph{\tt{Refresh()} and \tt{Advance()}}
The goal of a \nf{Refresh} on node $n$ is to create a block of $n$'s
children's new blocks and append it to $n$\nf{.blocks}. The variable
\nf{h} is read from $n$\nf{.head} at Line \ref{readHead}. The new
block created by \nf{Refresh} will be inserted into
$n\nf{.blocks[h]}$. Lines \ref{startHelpChild1}--\ref{endHelpChild1}
of \nf{$n$.Refresh} help to \nf{Advance} $n$'s children. \nf{Advance}
increments the children's \nf{head} if necessary and sets the
\nf{super} field of their most recently appended blocks. The reason
behind this helping is explained later when we discuss
\nf{IndexDequeue}. After helping to \nf{Advance} the children, a new
block called \nf{new} is created in Line
\ref{invokeCreateBlock}. Then, if \nf{new} is empty, \nf{Refresh}
returns \nf{true} because there are no new operations to propagate,
and it is unnecessary to add an empty block to the tree. Later we will
use the fact that all blocks contain at least one operation. Line
\ref{cas} tries to install \nf{new}. If it was successful, all is
good. If not, it means someone else has already put a block in
$n$\nf{.blocks[h]}. In this case, \nf{Refresh} helps advance
$n$\nf{.head} to \nf{h+1} and update the \nf{super} field of
$n$\nf{.blocks[h]} at Line \ref{advance}. 


\paragraph{\tt{CreateBlock()}} \texttt{$n$.CreateBlock($h$)} is used
by \nf{Refresh} to construct a block containing new operations of
$n$'s children. 
The block \nf{new} is created in Line \ref{initNewBlock} and its
fields are filled similarly for both left and right directions. The
variable  \nf{index\sub{prev}} is the index of the block preceding the
first subblock in the child in direction \nf{dir} that is aggregated
into \nf{new}. Field \texttt{new.end\textsubscript{dir}} stores the
index of the rightmost subblock of \nf{new} in the child. Then
\nf{sum\sub{enq-dir}} is computed from the sum of the number of
\nf{Enqueue} operations in the \nf{new} block from direction \nf{dir}
and the value stored in \texttt{$n$.blocks[h-1].sum\sub{enq-dir}}. The
field \nf{sum\sub{deq-dir}} is computed similarly. Then, if \nf{new}
is going to be installed in the \nf{root}, the \nf{size} field is also
computed. 

\paragraph{\tt{IndexDequeue($b,i$)}}

A call to $n$\nf{.IndexDequeue($b,i$)} computes the block and the rank
within the block in the root of the $i$th \nf{Dequeue} of the $b$th
block of $n$. Let $R_n$ be the successful  \nf{Refresh} on node $n$
that did a successful \nf{CAS(null, $B$)} into \nf{n.blocks[$b$]}. Let
$par$ be $n$\nf{.parent}. Without loss of generality, assume for the
rest of this section that ${n}$ is the left child of $par$. Let
$R_{par}$ be the first successful \nf{$par$.Refresh} that reads some
value greater than ${b}$ for \nf{left.head} and therefore contains
${B}$ as a subblock of its created block in Line
\ref{invokeCreateBlock}. Let $j$ be the index of the block that
$R_{par}$ puts in $par$\nf{.blocks}. 
 
Since the index of the superblock of ${B}$ is not known until $B$ is
propagated, $R_n$ cannot set the \nf{super} field of ${B}$ while
creating $B$. One approach for $R_{par}$ is to set the \nf{super}
field of ${B}$ after propagating $B$ to $par$. This solution would not
be efficient because there might be up to $p$ subblocks that $R_{par}$
propagated, which need to update their \nf{super} field. However,
intuitively, once $B$ is installed, its superblock is going to be
close to \nf{$n$.parent.head} at the time of installation. If we know
the approximate position of the superblock of $B$ then we can search
for the real superblock when it is needed. Thus, \nf{$B$.super} does
not have to be the exact location of  the superblock of $B$, but we
want it to be close to $j$. We can set \nf{$B$.super} to
\nf{$par$.head} while creating $B$, but the problem is that there
might be many \nf{Refresh}es on $par$ that could happen after $R_n$
reads $par$\nf{.head} and before propagating $B$ to $par$. If $R_n$
sets \nf{$B$.super} to \nf{$par$.head} after appending $B$ to
\nf{$n$.blocks} (Line \ref{setSuper1}), $R_n$ might go to sleep at
some time after installing $B$ and before setting \nf{$B$.super}. In
this case, the next \nf{Refresh}es on $n$ and $par$ help fill in the
value of \nf{$B$.super}. 

Block $B$ is appended to \nf{$n$.blocks[$b$]} on Line \ref{cas}. After
appending $B$, \nf{$B$.super} is set on Line \ref{setSuper1} of a call
to \nf{Advance} from \nf{$n$.Refresh} by the same or another process
or by Line \ref{helpAdvance} of a \nf{$n$.parent.Refresh}. We shall
show that this is sufficient to ensure that \nf{$B$.super} differs
from the index of $B$'s superblock by at most~1. 

\paragraph{\tt{FindResponse($b, i$)}}
To compute the response of the $i$th \nf{Dequeue} in the $b$th block
of the root Line \ref{checkEmpty} computes whether the queue is empty
or not. If there are more \nf{Dequeue}s than \nf{Enqueue}s the queue
would become empty before the requested \nf{Dequeue}. If the queue is
not empty, Line \ref{computeE} computes the rank $e$ of the
\nf{Enqueue} whose argument is the response to the
\nf{Dequeue}. Knowing the response is the $e$th \nf{Enqueue} in the
root (which is before the $b$th block), we find the block and position
containing the \nf{Enqueue} operation using \nf{DoublingSearch} and
after that \nf{GetEnqueue} finds its \nf{element}. 

\paragraph{\tt{GetEnqueue($b,i$)} and \tt{DoublingSearch($e, end$)}}
We can describe an operation in a node in two ways: the rank of the
operation among all the operations in the node or the index of the
block containing the operation in the node and the rank of the
operation within that block. If we know the block and rank within the
block of an operation, we can find the subblock containing the
operation and the operation's rank within that subblock in poly-log
time. To find the response of a \nf{Dequeue}, we know about the  rank
of the response \nf{Enqueue} in the root (\nf{e} in Line
\ref{computeE}). 
We also know the \nf{e}th \nf{Enqueue} is in
\nf{root.blocks[1..end]}. \nf{DoublingSearch} uses doubling to find
the range that contains the answer block (Lines
\ref{dsearchStart}--\ref{dsearchEnd}) and then tries to find the
required indices with a binary search (Line
\ref{dsearchBinarySearch}). 
A call to \nf{$n$.GetEnqueue($b,i$)} returns the \nf{element} of the
$i$th enqueue in the $b$th block of $n$. The range of subblocks of a
block is determined using the \nf{end\sub{left}} and
\nf{end\sub{right}} fields of the block and its previous block. Then,
the subblock is found using binary search on the \nf{sum\sub{enq}}
field (Lines \ref{leftChildGet} and \ref{rightChildGet}). 



% !TEX root =  podc-submission.tex

\section{Proof of Correctness}


\here{need to update this outline later}
First, we define and prove some facts about blocks and the node's \nf{head} field. Then, we introduce  the linearization ordering formally. Next, we prove double \nf{Refresh} on a node is enough to propagate its children's new operations up to the node, which is used to prove (1). After this, we prove some claims about the size and operations of each block, which we use to prove the correctness of \nf{DoublingSearch()}, \nf{GetEnqueue()} and \nf{IndexDequeue()}. Finally, we prove the correctness of the way we compute the response of a dequeue, which establishes (2).

\subsection{Basic Properties}

A \typ{Block} object's fields, except for \fld{super}, are immutable:  they are written only 
when the block is created at line \ref{enqNew} or \ref{deqNew} (for a leaf's block) or lines \ref{initNewBlock}--\ref{computeLength} (for an internal node's block).  
Moreover, only a \op{CAS} at line \ref{setSuper1} can modify the \fld{super} field 
(from \nl\ to a non-\nl\ value), so it remains fixed once a value is stored in it.
Similarly, only a \op{CAS} at line \ref{cas} can modify an element of a node's \fld{blocks} array 
(from \nl\ to a non-\nl\ value), so once a block is stored in a node, it remains there forever.
Only a \op{CAS} at line \ref{incrementHead} can update a node's \head\ field by incrementing it,
which implies the following.

\begin{observation} \label{nonDecreasingHead}
For each node \var{v},  \var{v}.\fld{head} is non-decreasing over time.
\end{observation}

\begin{lemma} \label{lem::headInc}
Let $R$ be an instance of \opa{Refresh}{v} whose call to \op{CreateBlock} returns a non-\nl\ block.  When $R$ terminates, \var{v}.\head\ is greater than the value $R$ reads from it at line \ref{readHead}.
\end{lemma}
\begin{proof}
After $R$'s \op{CAS} at line \ref{incrementHead}, \var{v}.\head\ is no longer equal to the value \var{h}
read at line \ref{readHead}.  The claim follows from \Cref{nonDecreasingHead}.
\end{proof}

Now we show $v.\fld{blocks}[v.\head]$ is either the last non-\nl\ block or the first \nl\ block in node $v$.

\begin{invariant}\label{lem::headPosition} 
For $0 \leq i < v.\head$, $v.\fld{blocks}[i]\neq\nl$.  For $i>v.\head$, $v.\fld{blocks}[i]=\nl$.
If $v\neq \var{root}$,  $v.\fld{blocks}[i].super \neq \nl$ for $0<i<v.head$.
\end{invariant}

\begin{proof}
Initially, $v.\head=1$, $v.\fld{blocks}[0]\neq\nl$  and $v.\fld{blocks}[i]=\nl$ for  $i>0$, so the claims~hold.

Assume the claim holds before a change to $v.\fld{blocks}$, which can be made only
by a successful \op{CAS} at line \ref{cas}.
The \op{CAS} changes $v.\fld{blocks}[h]$ from \nl\ to a non-\nl\ value.
Since $v.\fld{blocks}[h]$ is \nl\ before the CAS, $v.\head \leq h$ by the hypothesis.
Since $h$ was read from $v.\fld{blocks}[h]$ earlier at line \ref{readHead}, 
$v.\head \geq h$ by \Cref{lem::headPosition}.
So, $h=v.\head$ and a change to $v.\fld{blocks}[v.\head]$ preserves the invariant.

Now, assume the claim holds before a change to $v.\head$, which can only be an increment from $h$ to $h+1$
by a successful \op{CAS} at line \ref{incrementHead} of \op{Advance}.
For the first two claims, it suffices to show that $v.\fld{blocks}[head] \neq \nl$.
\nf{Advance} is called either at line \ref{helpAdvance} 
after testing that $v.\fld{blocks}[h]\neq\nl$ at line \ref{ifHeadnotNull},
or at line \ref{advance} after the \op{CAS} at line \ref{cas} ensures $v.\fld{blocks}[h]\neq\nl$.
For the third claim, observe that prior to incrementing $v.\head$ at line \ref{incrementHead},
the \op{CAS} at line \ref{setSuper1} ensures that $v.\fld{blocks}[i].super\neq \nl$.
\end{proof}

It follows that blocks accessed by the \op{Enqueue}, \op{Dequeue} and \op{CreateBlock} routines are non-\nl.

The following two lemmas show that no operation appears in more than one block of the root.
\begin{lemma} \label{lem::headProgress}
 If $b>0$ and $v.\fld{blocks}[b] \neq \nl$ then 
 $v.\fld{blocks}[b-1].\fld{end\sub{left}} \leq v.\fld{blocks}[b].\fld{end\sub{left}}$ and 
 $v.\fld{blocks}[b-1].\fld{end\sub{right}} \leq v.\fld{blocks}[b].\fld{end\sub{right}}$.
\end{lemma}
\begin{proof}
Let $B$ be the block in $v.\fld{blocks}[b]$.
Before creating $B$ at line \ref{invokeCreateBlock}, the \op{Refresh} that installed $B$
read $b$ from $v.\head$ at line \ref{readHead}.
At that time, $v.\fld{blocks}[b-1]$ contained a block $B'$, by \Cref{lem::headPosition}.
Thus, the \op{CreateBlock}($v,b-1$) that created $B'$ terminated before the \op{CreateBlock}($v,b$) that
created $B$ started.
It follows from \Cref{nonDecreasingHead} that the value that 
line \ref{createEndLeft} of \op{CreateBlock}($v,b-1$) stores in $B'.\fld{end\sub{left}}$   
is less than or equal to the value that line \ref{createEndLeft} of \op{CreateBlock}($v,b$) 
stores in $B.\fld{end\sub{left}}$.
Similarly, the values stored in $B'.\eright$ and $B.\eright$ at line \ref{createEndRight} 
%of these calls to \op{CreateBlock} 
satisfy the claim.
\end{proof}

\begin{lemma} \label{lem::subblocksDistinct}
If $B$ and $B'$ are two blocks in nodes at the same depth, their sets of subblocks are disjoint.
\end{lemma}
\begin{proof}
We prove the lemma by reverse induction.
If $B$ and $B'$ are in leaves, they have no subblocks, so the claim is true.
Assume the claim is true for nodes at depth $d+1$ and let $B$ and $B'$ be two blocks in nodes at depth $d$.
Consider the direct subblocks of $B$ and $B'$ defined by (\ref{defsubblock}).
If $B$ and $B'$ are in different nodes at depth $d$, then their direct subblocks are disjoint.
If $B$ and $B'$ are in the same node, it follows from \Cref{lem::headProgress} that their direct subblocks are disjoint.
Either way, their direct subblocks (at depth $d+1$) are disjoint, so the claim follows from the induction hypothesis.
\end{proof}

It follows that each block has at most one superblock.
Moreover, we can now prove each operation is contained in at most one block of each node,
and hence appears at most once in the linearization~$L$.

\begin{corollary}\label{lem::noDuplicates}
For  $i\neq j$, $v.\fld{blocks}[i]$ and $v.\fld{blocks}[j]$ cannot both contain the same operation.
\end{corollary}
\begin{proof}
The operations contained in a block $B$ are those that appear in subblocks of $B$ in the leaves of the tree.
Since each process puts each of its operations in only one block of its own leaf, an operation 
cannot be in two different leaf blocks. 
By \Cref{lem::subblocksDistinct}, $v.\fld{blocks}[i]$ and $v.\fld{blocks}[j]$ have no subblocks in common, so the claim follows.
\end{proof}



%\begin{definition}
%$n\nf{.blocks[}i\nf{]}$ is \emph{established} if $n\nf{.head}>i$. An operation is \it{established} in node $n$ 
%if it is in an established block of $n$. $EST^t_n$ is the set of established operations in node $n$ at time $t$.
%\end{definition}
%
%Now we want to say that blocks of a node grow over time.
%\begin{observation}\label{lem::blocksOrder}
%  If  time $t<$ time $t^\prime$ ($t$ is before $t^\prime$), then $ops(n.blocks)$ at time $t$ is a subset of 
%$ops(n.blocks)$ at time $t^\prime$.
%\end{observation}
%\begin{proof}
%Blocks are only appended (not modified) with \nf{CAS} to $n\nf{.blocks[}n\nf{.head]}$, so the set of the blocks of a node after the \nf{CAS} contains the set of the blocks before the \nf{CAS}.
%\end{proof}

% \begin{corollary}\label{lem::establishedOrder}
%   If  time $t<$ time $t^\prime$, then $EST_n^t\subseteq EST_n^{t^\prime}$.
% \end{corollary}
% \begin{proof}
% From Observations \ref{nonDecreasingHead}, \ref{lem::blocksOrder}.  
% \end{proof}

The following shows that the values stored in \fld{sum\sub{enq}} and \fld{sum\sub{deq}} fields are accurate.
\here{If space is tight, could easily move next proof to appendix and just say it follows easily from the definition of subblocks and the values stored in the fields on lines \ref{enqNew}, \ref{deqNew}, \ref{createSumEnq}, \ref{createSumDeq}.}

\begin{invariant}\label{lem::sum}
If $B$ is a block stored in $v.\fld{blocks}[i]$,
$B.\fld{sum\sub{enq}} = | E(v.\fld{blocks}[0])\cdots E(v.\fld{blocks}[i]) |$ and
$B.\fld{sum\sub{deq}} = | D(v.\fld{blocks}[0])\cdots D(v.\fld{blocks}[i]) |$.
\end{invariant}
\begin{proof}
Initially, each \fld{blocks} array only contains an empty block $B_0$ in location 0.
By definition, $E(B_0)$ and $D(B_0)$ are empty sequences.
Moreover, $B_0.\fld{sum\sub{enq}} = B_0.\fld{sum\sub{deq}} = 0$, so the claim is true.

We show that each installation of a block $B$ into some location $v.\fld{blocks}[i]$ preserves the claim,
assuming the claim holds before this installation.  We consider two cases.

If $v$ is a leaf, $B$ was created at line \ref{enqNew} or \ref{deqNew}.
For line \ref{enqNew}, $B$ represents a single enqueue operation, so $|E(B)|=1$ and $|D(B)|=0$.
Moreover, $B.\fld{sum\sub{enq}}$ is set to $v.\fld{blocks}[i-1]+1$ and
$B.\fld{sum\sub{deq}}$ is set to $v.\fld{blocks}[i-1]$, so the claim follows from the hypothesis.
The proof for line \ref{deqNew}, where $B$ represents a single dequeue operation is similar.

Now suppose $v$ is an internal node. By the definition of subblocks in (\ref{defsubblock}), the
subblocks of $v.\fld{blocks}[1..i]$ are $v.\fld{left.blocks}[1..B.\eleft]$ 
and $v.\fld{right.blocks}[1..B.\eright]$.
Thus, the enqueues in $E(v.\fld{blocks}[0])\cdots E(v.\fld{blocks}[i])$ are those in
$E(v.\fld{left.blocks}[0]) \cdots E(v.\fld{left.blocks}[B.\eleft])$ and
$E(v.\fld{left.blocks}[0]) \cdots E(v.\fld{left.blocks}[B.\eright])$.
By the hypothesis, the total number of these enqueues is $v.\fld{left.blocks}[B.\eleft].\fld{sum\sub{enq}} + v.\fld{right.blocks}[B.\eright].\fld{sum\sub{enq}}$, which is the value that line \ref{createSumEnq} stored in $B.\fld{sum\sub{enq}}$ when $B$ was created.
The proof for \fld{sum\sub{deq}} (stored on line~\ref{createSumDeq}) is similar.
\end{proof}

This allows us to prove that every block a Refresh installs contains at least one operation.

\begin{corollary}\label{blockNotEmpty}
If a block $B$ is in $v.\fld{blocks}[i]$ where $i>0$, then $E(B)$ and $D(B)$ are not both empty.
\end{corollary}
\begin{proof}
The \op{Refresh} that installed $B$ got $B$ as the response to its call to \op{CreateBlock} on line \ref{invokeCreateBlock}.
Thus, at line \ref{testEmpty} $\var{num\sub{enq}}+\var{num\sub{deq}}\neq 0$.
By \Cref{lem::sum}, $\var{num\sub{enq}} = |E(B)|$ and $\var{num\sub{deq}} = |D(B)|$,
so these sequences cannot both be empty.
\end{proof}



\subsection{Propagating Operations to the Root}
Next, we show two \op{Refresh}es suffice to propagate operations from a child to its parent.
We say that node $v$ \emph{contains} an operation if some block in $v.\fld{blocks}$ contains the operation.
\here{move this defn earlier and just recall it here?}
Once a block is added to a node, it remains there forever.  Thus, if $v$ contains an operation at some time, it contains the operation at all later times too.

\begin{lemma}\label{successfulRefresh}
Let $R$ be a call to \op{Refresh}($v$) that performs a successful \op{CAS} on line \ref{cas} (or terminates at line \ref{addOP}).
In the configuration after that CAS (or termination, respectively), $v$ contains all operations that $v$'s children contained 
when $R$ executed line~\ref{readHead}.
\end{lemma}
\begin{proof}
Suppose $v$'s child (without loss of generality, $v.\fld{left}$) contained an operation $op$ 
in the configuration $C$ immediately before $R$ executed line \ref{readHead}.
Then some block $B$ in $v.\fld{left.blocks}[i]$ contains $op$.
By \Cref{nonDecreasingHead} and \Cref{lem::headProgress}, the value of $childHead$ that $R$ reads from
$v.\fld{left.head}$ in line \ref{readChildHead} is at least $i$.
If it is equal to $i$, $R$ calls \op{Advance} at line \ref{helpAdvance}, which ensures that 
$v.\fld{left.head} > i$.
Then, $R$ calls \op{CreateBlock}($v,h$) in line \ref{invokeCreateBlock}, where $h$ is the value $R$ reads at line \ref{readHead}.
\op{CreateBlock} reads a value greater than $i$ from $v.\fld{left.head}$ at line \ref{createEndLeft}.
Thus, $new.\eleft \geq i$.  We consider two cases.

Suppose $R$'s call to \op{CreateBlock} returns the new block $B'$ and $R$'s \op{CAS} at line \ref{cas} 
installs $B'$ in $v.\fld{blocks}$.
Then, $B$ is a subblock of some block in $v$, since  $B'.\eleft$ is greater than or equal to $B$'s index
in $v.\fld{left.blocks}$.
Hence $B'$ contains $op$, as required.

\Eric{I think this case was missing from the proof in the thesis}
Now suppose $R$'s call to \op{CreateBlock} returns \nl, causing $R$ to terminate at line \ref{addOP}.
Intuitively, since there are no operations in $v$'s children to promote, $op$ is already in $v$.
We formalize this intuition.
The value computed at line \ref{createSumEnq} is
\begin{eqnarray*}
\var{num\sub{enq}} 
&=& v.\fld{left.blocks}[new.\eleft].\fld{sum\sub{enq}} + v.\fld{right.blocks}[new.\eright].\fld{sum\sub{enq}} - v.\fld{blocks}[h-1].\fld{sum\sub{enq}} \\
&=& v.\fld{left.blocks}[new.\eleft].\fld{sum\sub{enq}} + v.\fld{right.blocks}[new.\eright].\fld{sum\sub{enq}} \\
&&\mbox{ }- v.\fld{left.blocks}[v.\fld{blocks}[h-1].\eleft].\fld{sum\sub{enq}} - v.\fld{right.blocks}[v.\fld{blocks}[h-1].\eright].\fld{sum\sub{enq}}
\end{eqnarray*}
It follows from \Cref{lem::sum} that $num\sub{enq}$ is the total number of enqueues contained in the blocks
$v.\fld{left.blocks}[v.\fld{blocks}[h-1].\eleft+1..new.\eleft]$ and
$v.\fld{right.blocks}[v.\fld{blocks}[h-1].\eright+1..new.\eright]$.
Similarly, $num\sub{deq}$ is the total number of dequeues contained in these blocks.
Since $num\sub{enq}+num\sub{deq}=0$ at line \ref{testEmpty},
these blocks contain no operations.
By \Cref{blockNotEmpty}, this means the ranges of blocks are empty, so that $v.\fld{blocks}[h-1].\eleft \geq new.\eleft \geq i$.
Hence, $B$ is already a subblock of some block in $v$, so $v$ contains $op$.
\end{proof}

We now show that if a process fails its \op{Refresh}($v$) twice, some other process must have succeeded.

\begin{lemma}\label{lem::doubleRefresh}
Consider two consecutive terminating calls $R_1$, $R_2$ to \nf{Refresh}($v$) by the same process.
When $R_2$ terminates, $v$ contains all operations that $v$'s children contained when $R_1$ was invoked.
\end{lemma}
\begin{proof}
If either $R_1$ or $R_2$ performs a successful \op{CAS} at line \ref{cas} or terminates at line \ref{addOP}, the claim follows
from \Cref{successfulRefresh}.
So suppose both $R_1$ and $R_2$ perform a failed \op{CAS} at line \ref{cas}.
Let $h_1$ and $h_2$ be the values $R_1$ and $R_2$ read from $v.\head$ at line \ref{readHead}.
By \Cref{lem::headInc}, $h_2>h_1$.
By \Cref{lem::headProgress}, $v.blocks[h_2]=\nl$ when $R_1$ executes line \ref{readHead}.
Since $R_2$ fails its \op{CAS} on $v.blocks[h_2]$, some other \op{Refresh} $R_3$ must have done
a successful \op{CAS} on $v.blocks[h_2]$ before $R_2$'s \op{CAS}.
$R_3$ must have executed line \ref{readHead} after $R_1$, since $R_3$ read the value $h_2$ from $v.\head$ and the value of $v.\head$ is non-decreasing, by \Cref{nonDecreasingHead}.
Thus, all operations contained in $v$'s children when $R_1$ begins
are also contained in $v$'s children when $R_3$ later executes line \ref{readHead}.
By \Cref{successfulRefresh}, these operations are contained in $v$ when $R_3$ performs its successful \op{CAS},
which is before $R_2$ terminates.
\end{proof}

\begin{lemma} \label{lem::appendExactlyOnce}
When an \op{Append}($B$) terminates, $B$'s operation is contained in exactly one block in each node along the path from the process's leaf to the root.
\end{lemma}
\begin{proof}
The \op{Append} adds $B$ to the process's leaf and calls \op{Propagate}, which
performs a double \op{Refresh} on each node along the path from the leaf's parent to the root.
By \Cref{lem::doubleRefresh}, this ensures $B$'s operation is contained in each node along this path.
By \Cref{lem::noDuplicates}, it is contained in exactly one block in each node.
\end{proof}

\subsection{Correctness of Helping Routines}

We first show that \op{getEnqueue} works correctly.

\begin{lemma}\label{lem::get}
If $1\leq i\leq |E(v.\fld{blocks}[b])|$ then \op{getEnqueue}($v,b,i$) returns the argument of the $i$th enqueue in $E(v.\fld{blocks}[b])$.
\end{lemma}
\begin{proof}
We prove the claim by induction on the height of node $v$.
If $v$ is a leaf, the hypothesis implies that $i=1$ and the block $v.\fld{blocks}[b]$ represents 
an enqueue whose argument is stored in $v.\fld{blocks}[b].\fld{element}$.
\op{GetEnqueue} returns the argument of this enqueue at line \ref{getBaseCase}.

Assuming the claim holds for $v$'s children, we prove it for $v$.
Let $B$ be $v.\fld{blocks}[b]$.
By (\ref{defSeqs}),
$E(B)$ is obtained by concatenating the enqueue sequences of the direct subblocks
of $B$, which are listed in (\ref{defsubblock}).
By \Cref{lem::sum}, $\var{sum\sub{left}}-\var{prev\sub{left}}$ is the number
of enqueues in $E(B)$ that come from $B$'s subblocks in $v$'s left child.
Thus, $dir$ is set to the direction for the child of $v$ that contains the required enqueue.
Moreover, when line \ref{endChooseDir} is reached, $i$ is the position of the required enqueue within the portion $E'$ of $E(B)$ that comes from that child.
Thus,  line \ref{getChild} finds the index $b'$ of the subblock $B'$ containing the required enqueue.
By \Cref{lem::sum}, $v.\var{dir.blocks}[b'-1].\var{sum\sub{enq}} - \var{prev\sub{dir}}$ is the number of 
enqueues in $E'$ before the enqueues of block $B'$, so
the value $i'$ computed on line \ref{getChildIndex} is the position of the required enqueue within $E(B')$.
Thus, the recursive call on line \ref{getRecurse} satisfies its precondition, and 
returns the required result, by the induction hypothesis.
\end{proof}

Next, we prove the \fld{super} field of a block is within one of the true index of the block's superblock.
\here{An alternative would be modify the code
so that the correct index of the superblock is computed before storing it in $B.super$.
In Advance, we could check if $h>v.parent.blocks[h_p].end_{left/right}$ and if so write $h_p+1$ instead of $h_p$ in $v.blocks[h].super$.  Then, in IndexDequeue there would be no need to correct the value read from the super field.  If we do this, the following lemma would be modified to prove that the value is perfectly accurate.}

\begin{lemma}\label{superRelation}
Let $B=v.\var{blocks}[b]$.
  If $v.\fld{parent.blocks}[s]$ is the superblock of $B$ then $s-1\leq B.\fld{super}\leq s$.
\end{lemma}
\begin{proof}
We first show $B.\fld{super}\leq s$.
Let $R_s$ be the instance of \op{Refresh}($v.\fld{parent}$) that installs the superblock of $B$ 
in $v.\fld{parent.blocks}[s]$.
By the definition of subblocks (\ref{defsubblock}), $R_s$'s read $r$ of $v.head$ at line \ref{createEndLeft} or \ref{createEndRight} obtains a value greater than $b$.
By \Cref{lem::headPosition}, $B.\fld{super} \neq \nl$ when $r$ occurs, which means
that $B.\fld{super}$ was set (by line~\ref{setSuper1}) to a value read from $v.\fld{parent}.\head$ before $r$.
When $r$ occurs, $v.\fld{parent.blocks}[s] = \nl$, since the later \op{CAS} by $R_s$ at line
\ref{cas} succeeds.
So, by \Cref{lem::headPosition}, $v.\fld{parent}.\head \leq s$ when $r$ occurs.
Since the value stored $B.\fld{super}$ was read from $v.\fld{parent.head}$ before $r$ and the \head\ field is non-decreasing by \Cref{nonDecreasingHead}, it follows that $B.super\leq s$.

Next, we show that $B.\fld{super}\geq s-1$.
The value stored in $B.\fld{super}$ at line \ref{setSuper1} is read from $v.\fld{parent}.\head$ at line \ref{readParentHead} and \head\ values are always at least 1, so $B.\fld{super} \geq 1$.
So, if $s\leq 2$, the claim is trivial.  Assume $s>2$ for the remainder of the proof.
By \Cref{lem::headProgress}, $v.\fld{parent.blocks}[s-1]\neq \nl$.  Let $R_{s-1}$ be the call to
$\op{Refresh}(v.\fld{parent})$ that installed the block in $v.\fld{parent.blocks}[s-1]$.
Let $r'$ be the step when $R_{s-1}$ reads $s-1$ in $v.\fld{parent}.\head$ at line \ref{readHead}.
This read $r'$ must be before $B$ is installed in $v$;
otherwise, \Cref{successfulRefresh} would imply that $B$ is a subblock of one of 
$v.\fld{parent.blocks}[1..s-1]$, contrary to the hypothesis.
Now, consider the call to \op{Advance}($v, b$) that writes $B.\fld{super}$.
It is invoked either 
at line \ref{helpAdvance} after seeing $v.\fld{blocks}[h]\neq \nl$ at line \ref{ifHeadnotNull}
or at line \ref{advance} after ensuring $v.\fld{blocks}[h]\neq \nl$ at line~\ref{cas}.
Either way, the \op{Advance} is invoked after $B$ is installed, and therefore after $r'$.
By \Cref{nonDecreasingHead}, $v.\fld{parent}.\head$ is non-decreasing, so 
the value this \op{Advance} reads in $v.\fld{parent}.\head$ and
writes in $B.\fld{super}$ is greater than or equal to the value $s-1$ that $r'$ reads in $v.\fld{parent}.\head$.
\end{proof}

%The reader may wonder when the case $b\nf{.super}=s$ happens. This can happen when $
%\nf{$n$.parent.blocks[$B$.super]}=\nf{null}$ when $B$\nf{.super} is written and $R_p$ puts its created block 
%into \nf{$n$.parent.blocks[$B$\nf{.super}]} afterwards.

We prove \op{IndexDequeue}'s correctness using \Cref{superRelation} on each step of the \op{IndexDequeue}.
\here{mention somewhere why precondition of IndexDequeue is true; also check that block of parent indexed by sup in that routine is non-null}

\begin{lemma}\label{lem::indexDequeue}
If $v.\fld{blocks}[b]$ has been propagated to the root and $1\leq i\leq |D(v.\fld{blocks}[b])|$, 
 then \op{IndexDequeue}($v, b, i$) returns $\langle b',i' \rangle$ such that the \var{i}th dequeue in $D(\var{v}.\fld{blocks}[\var{b}])$ is the $(i')$th dequeue of $D(\var{root}.\fld{blocks}[b'])$.
\end{lemma}
\begin{proof}
We prove the claim by induction on the depth of node $v$. The base case where $v$ is the root is trivial (see Line \ref{indexBaseCase}).
Assuming the claim holds for $v$'s parent, we prove it for $v$.
Let $B=v.\fld{blocks}[b]$ and $B'$ be the superblock of $B$.
\op{IndexDequeue}($v, b, i$) first computes the index $sup$ of $B'$ in $v.\fld{parent}$.
By \Cref{superRelation}, this index is either $B.super$ or $B.super+1$.
The correct index is determined by testing on line \ref{supertest} whether $B$ is not a subblock of $v.\fld{parent.blocks}[B.\fld{super}]$.

Next, the position of the required dequeue in $D(B')$ is computed in 
lines \ref{computeISuperStart}--\ref{computeISuperEnd}. 
We first add the number of dequeues in the subblocks of $B'$ in $v$ that preced $B$ on line \ref{computeISuperStart}.
If $v$ is the right child of its parent, then all of the subblocks of $B'$ from $v$'s left sibling
also precede the required dequeue, so we add the number of dequeues in those subblocks in line \ref{considerLeftBeforeRight}.

Finally, \op{IndexDequeue} is called recursively on $v$'s parent.
Since $B$ has been propagated to the root, so has its superblock $B'$.
Thus, all preconditions of the recursive call are met.
By the induction hypothesis, the recursive call returns the location of the required dequeue in the root.\end{proof}

\subsection{Linearizability}

Consider any finite execution.  We must show that the linearization ordering $L$ defined in 
(\ref{linearization}) is a legal permutation of a subset of the operations in 
the execution, i.e., that it includes all operations that terminate and 
if one operation $op_1$ terminates before another operation $op_2$ begins, then $op_2$ does appear
before $op_1$ in $L$.  Then we must show that each dequeue that terminates returns the 
same response as it would in the sequential execution described by $L$.

\begin{lemma} \label{linearSat}
$L$ is a legal linearization ordering.
\end{lemma}
\begin{proof}
By \Cref{lem::noDuplicates}, $L$ is a permutation of a subset of the operations in the execution.
Since each operation creates a block $B$ containing the operation and then calls \op{Append}($B$), 
the operation is propagated to the root before the operation terminates, by \Cref{lem::appendExactlyOnce},
so it appears in $L$.
Also, if $op_{1}$ terminates before $op_{2}$ begins, then $op_{1}$ 
is propagated to the root before $op_2$ begins, so $op_2$ appears before $op_2$ in $L$.
\end{proof}

In the next lemma, we show that the \fld{size} field in each block in the root is computed correctly.
\begin{lemma}\label{sizeCorrectness}
If the operations of $\var{root}.\fld{blocks}[0..b]$ are applied sequentially in the order of~$L$ on an initially empty queue, the resulting queue has $\var{root}.\fld{blocks}[b].\size$ elements.  
\end{lemma}
\begin{proof}
We prove the claim by induction on $b$. 
The base case when ${b=0}$ is trivial, since the queue is initially empty and 
$\var{root}.\fld{blocks}[0]$ contains an empty block whose \size\ field is $0$. 
Assuming the claim holds for $b-1$, we prove it for $b$.
The \fld{size} field of the block $B$ installed in $\var{root}.\fld{blocks}[b]$ is computed
at line \ref{computeLength} of a call to \op{CreateBlock}(\var{root, b}).
By the induction hypothesis, $root.\fld{blocks}[b-1].\size$ gives the size of the queue before the operations
of block $B$ are performed.
By \Cref{lem::sum}, the values of \var{num\sub{enq}} and \var{num\sub{deq}}
are the number of enqueues and dequeues contained in $B$.
Hence, the size of the queue after the operations of $B$ are performed (with enqueues before dequeues as specified by $L$)
is $\max(0, \var{root}.\fld{blocks}[b-1].\size + \var{num\sub{enq}} - \var{num\sub{deq}})$.
\end{proof}

Next, we show operations return the same response as they would in the sequential execution $L$.

\begin{lemma}\label{linearCorrect}
Each terminating dequeue returns the response it would in the sequential execution $L$.
\end{lemma}
\begin{proof}
If a dequeue $D$ terminates, it is contained in some block in the root, by \Cref{lem::appendExactlyOnce}.
By \Cref{lem::indexDequeue}, $D$'s call to \op{IndexDequeue} on line \ref{invokeIndexDequeue}
returns a pair $\langle b,i\rangle$ such that $D$ is the $i$th dequeue in the block 
$B=\var{root}.\fld{blocks}[b]$.
$D$ then calls \op{FindResponse}($b,i$) on line \ref{deqRest}.
By \Cref{sizeCorrectness}, the queue contains $\var{root}.\fld{blocks}[b-1].\fld{size}$ elements
after the operations in $\var{root}.\fld{blocks}[1..b-1]$ are performed sequentially 
in the order given by $L$.
By \Cref{lem::sum}, the value of \var{num\sub{enq}} computed on line \ref{FRNum}
is the number of enqueues in $B$.
Since the enqueues in block $B$ precede the dequeues,
the queue is empty when the $i$th dequeue of $B$ occurs if 
$\var{root}.\fld{blocks}[b-1].\fld{size} + \var{num\sub{enq}} < i$.
So $D$ returns \nl\ on line \ref{returnNull} if and only if it would do so in the sequential
execution $L$.
Otherwise, the size of the queue after doing the operations in $\var{root}.\fld{blocks}[0..b-1]$
in the sequential execution $L$ is $\var{root}.\fld{blocks}[b-1].\fld{sum\sub{enq}}$ minus
the number of non-\nl\ dequeues in that prefix of $L$.
Hence, line \ref{computeE} sets $e$ to the rank of $D$ among all the non-\nl\ dequeues in $L$.
Thus, in the sequential execution~$L$, $D$ returns the value enqueued by the $e$th enqueue in $L$.
By \Cref{lem::sum}, this enqueue is the $i_e$th enqueue 
in $E(\var{root}.\fld{blocks}[b_e])$, where
$b_e$ and $i_e$ are the values $D$ computes on line \ref{FRb} and \ref{FRi}.
By \Cref{lem::get}, the call to \op{GetEnqueue} returns the argument of the required enqueue.
\end{proof}

Combining \Cref{linearSat} and \Cref{linearCorrect} provides our main result.

\begin{theorem}
The queue implementation is linearizable.
\end{theorem}


% !TEX root =  podc-submission.tex

\section{Analysis}
In this section, we analyze the number of steps and the number of \op{CAS} instructions performed by operations.
First, we prove some claims about the size and operations of a block. 

\begin{proposition}\label{casbound}
Each \nf{Enqueue} or \nf{Dequeue} operation performs $O(\log p)$ \nf{CAS} instructions.
\end{proposition}
\begin{proof}
An operation invokes \nf{Refresh} at most twice at each of the $\ceil{\log_2 p}$ levels of the tree.
A \nf{Refresh} does at most 7 \nf{CAS} steps: one in line \ref{cas} and two from each \nf{Advance} in line \ref{helpAdvance} or~\ref{advance}.
\end{proof}



\begin{lemma}\label{dSearchTime}
The search that \op{FindResponse}$(b,i)$ does at line \ref{FRb} to find the index $b_e$ of the block in the root containing the $e$th enqueue takes $O(\log (\var{root}.\fld{blocks}[b_e].\size + \var{root}.\fld{blocks}[b-1].\size))$ steps.
\end{lemma}
\begin{proof}
Let the blocks in the root be $B_1, \ldots, B_\ell$.
The doubling search used to compute $b_e$ takes $O(\log (b-b_e))$ steps,
so we prove $b - b_e \leq 2 \cdot B_{b_e}.\size + B_{b-1}.\size + 1$.
If $b \leq b_e+1$, then this is trivial, so assume for the rest of the proof that $b>b_e+1$.

As shown in \Cref{linearCorrect}, the dequeue that calls \op{FindResponse} is in $B_b$ and is supposed to return an enqueue in $B_{b_e}$.
Thus, there can be at most $B_{b_e}.\size$ dequeues in 
$D(B_{b_e+1}) \cdots D(B_{b-1})$; otherwise in the sequential execution $L$,
all elements enqueued before the end of
$E(B_{b_e})$ would be dequeued before $D(B_b)$. 
Furthermore, by \Cref{sizeCorrectness}, the size of the queue  after the prefix of $L$ corresponding to 
$B_1,\ldots,B_{b-1}$  is 
$B_{b-1}.\size \geq B_{b_e}.size + |E(B_{b_e+1})\cdots E(B_{b-1})| - |D(B_{b_e+1})\cdots D(B_{b-1})|$.
Thus, $|E(B_{b_e+1})\cdots E(B_{b-1})| \leq B_{b-1}.\size + |D(B_{b_e+1})\cdots D(B_{b-1})| \leq B_{b-1}.\size + B_{b_e}.\size$.
So, the total number of operations in $B_{b_e+1}, \ldots, B_{b-1}$ is at most
$B_{b-1}.\size + 2\cdot B_{b_e}.\size$.
Each of these $b-1-b_e$ blocks contains at least one operation, by \Cref{blockNotEmpty}.
So, $b-1-b_e \leq B_{b-1}.\size + 2\cdot B_{b_e}.\size$.
\end{proof}

The following lemma is useful in bounding the time to \op{GetEnqueue}
\begin{lemma}\label{blockSize}
Each block $B$ in each node contains at most one operation of each process.
If $c$ is the execution's maximum point contention, $B$ has at most $c$ direct~subblocks.
\end{lemma}
\begin{proof}
Suppose $B$ contains an operation of process $p$.
Let $op$ be the earliest operation by $p$ contained in $B$.
When $op$ terminates, $op$ is contained in $B$ by \Cref{lem::appendExactlyOnce}.
Thus, $B$ cannot contain any later operations by $p$, since $B$ is created before
those operations are invoked.

Let $t$ be the earliest termination of any operation contained in $B$.
By \Cref{lem::appendExactlyOnce}, $B$ is created before $t$, so all operations contained in $B$
are invoked before $t$.  Thus, all are  running concurrently at $t$, so $B$ contains at most $c$ operations.
By definition, the direct subblocks of $B$ contain these $c$ operations, and each operation is contained
in exactly one of these subblocks, by Lemma \ref{lem::subblocksDistinct}.
By \Cref{blockNotEmpty}, each direct subblock of $B$ contains at least one operation,
$B$ has at most $c$ direct subblocks.
\end{proof}

We now bound step complexity in terms of the number of processes $p$, the maximum contention $c\leq p$ and the size of the queue. 

\begin{mytheorem}\label{enqDeqTime}
Each \op{Enqueue} and null Dequeue takes $O(\log p)$ steps 
and each non-null \op{Dequeue} takes
$O(\log p\log c + \log q_e+ \log q_d)$ steps,
where $q_d$ is the size of the queue when the \nf{Dequeue} is linearized and 
$q_e$ is the size of the queue when the \op{Enqueue} of the value returned is linearized.
\end{mytheorem}
\begin{proof}
An \op{Enqueue} or null \op{Dequeue} creates a block, appends it to the process's 
leaf and propagates it to the root.  The \op{Propagate}  does $O(1)$ steps 
at each node on the path from the process's leaf to the root.
A null \op{Dequeue} additionally calls \op{IndexDequeue}, which also does $O(1)$ steps
at each node on this path. 
So, the total number of steps for either type of operation is $O(\log p)$.

A non-null \op{Dequeue} must also search at line \ref{FRb} and call \op{GetEnqueue}
at line \ref{findAnswer}.
By \Cref{dSearchTime}, the doubling search takes $O(\log(q_e+q_d+p))$ steps,
since the size of the queue can change by at most $p$ within one block (by \Cref{blockSize}).
\op{GetEnqueue} does a binary search within each node on a path from the root to a leaf.
Each node $v$'s search is within the subblocks of one block in $v$'s parent.
By \Cref{blockSize}, each search takes $O(\log c)$ steps, for a total of $O(\log p\log c)$ steps.
\end{proof}

\begin{corollary}
The queue implementation is wait-free.
\end{corollary}


% !TEX root =  queue.tex

\section{Bounding Space Usage}
\label{reducing}

Operations remain in the \fld{blocks} arrays forever. 
This uses space proportional to the number of enqueues that have been invoked.
Now, we modify the implementation to remove blocks that are no longer needed, so that space usage is
polynomial in $p$ and $q$, while (amortized) step complexity is still polylogarithmic. We replace the \fld{blocks} array in each node by a red-black tree (RBT)
that stores the blocks.
Each block has an additional \fld{index} field
that represents its position within the original \fld{blocks} array, and
blocks in a RBT are sorted by \fld{index}.
The attempt to install a new block in $\fld{blocks}[i]$  on line \ref{cas}
is replaced by an attempt to insert a new block with index $i$ into the RBT.
Accessing the block in $\fld{blocks}[i]$ is 
replaced by searching the RBT for the  index $i$.
The binary searches for a block in line \ref{FRb} and \ref{getChild} can simply search the RBT
using the \fld{sum\sub{enq}} field, since the RBT is also sorted with respect to this field, by Invariant \ref{inv::sum}.
 
Known lock-free search trees have step complexity that includes a term linear in $p$ \cite{EFHR14,Ko20}.  
However, we do not require all the standard search tree operations.
Instead of a standard insertion, we allow a \op{Refresh}'s insertion to fail if another
concurrent \op{Refresh} succeeds in inserting a block, just as the \op{CAS} on line \ref{cas}
can fail if a concurrent \op{Refresh} does a successful \op{CAS}.
Moreover, the insertion should succeed only if the newly inserted block has a larger index than any other block in the RBT.
Thus, we can use a particularly simple concurrent RBT implementation.
A sequential RBT can be made persistent using the classic node-copying technique of 
Driscoll et al.~\cite{DSST89}:  all RBT nodes are immutable, and operations on the 
RBT make a new copy of each RBT node $x$ that must be modified, as well
as each RBT node along the path from the RBT's root to~$x$.
The RBT reachable from the new copy of the root is the result of applying the RBT operation.
This only adds a constant factor to the running time of any routine designed for a (sequential) RBT.
Once a process has performed an update to the RBT representing the blocks of a node 
$\var{v}$ in the \ordering\ tree, 
it uses a CAS to swing $\var{v}$'s pointer from the previous RBT root to the new RBT root.
A search in the RBT can simply read the pointer to the RBT root and perform a standard
sequential search on it.
Bashari and Woelfel ~\cite{DBLP:conf/podc/BashariW21} used persistent RBTs in a similar way for a snapshot data structure.

To prevent RBTs from growing without bound, we would like to discard
blocks that are no longer needed.
Ensuring the size of the RBT is polynomial in $p$ and $q$ will 
also keep the running time of our operations polylogarithmic.
Blocks should be kept if they contain operations still in progress.
Moreover, a block containing an \op{Enqueue}($x$) must be kept until $x$ is dequeued.

To maintain good amortized time, we periodically do a garbage collection (GC) phase.
If a \op{Refresh} on a node adds a block whose \var{index} is a multiple of $G=p^2\ceil{\log p}$, it does GC to remove obsolete blocks from the node's RBT.
To determine which blocks can be thrown away, we use a global array $\var{last}[1..p]$ where 
each process writes the index of the last block in the root
containing a null dequeue or an enqueue whose element it dequeued.
To perform GC, a process reads $\var{last}[1..p]$ and finds the maximum entry $m$.
Then, it helps complete every other process's pending dequeue 
by computing the dequeue's response and writing it in the block in the leaf that represents the dequeue.
Once this helping is complete, it follows from the FIFO property of the queue that elements enqueued 
in $\fld{root.blocks}[1..m-1]$ have all been dequeued, so GC can discard all subblocks of those.
Fortunately, there is an  RBT \op{Split} operation that can remove
these obsolete blocks from an RBT in logarithmic time \cite[Sec.~4.2]{Tar83}.

An operation $op$'s search of a RBT may fail to find the required block $B$ that has been removed 
by another process's GC phase.  If $op$
is a dequeue, $op$ must have been helped before $B$ was discarded, so $op$ can simply read its response from
its own leaf.  If $op$ is an enqueue, it can simply terminate.

Our GC scheme ensures each RBT has  $O(q+p^2\log p)$ blocks, so RBT operations take $O(\log(p+q))$ time.
Excluding GC, an operation does $O(1)$ operations on RBTs at each level of the tree for a total of
$O(\log p\log(p + q))$ steps.
A GC phase takes $O(p\log p \log(p+q))$ steps to help complete all pending operations.
If all processes carry out this GC phase, it takes a total of $O(p^2\log p\log(p+q))=O(G\log(p+q))$ steps.
Since there are at least $G$ operations between two GC phases, each node contributes $O(\log(p+q))$ steps to each operation in an amortized sense.
Adding up over all nodes an operation may have to do GC on, 
an operation spends $O(\log p\log(p+q))$ steps doing GC in an amortized sense.
So the total amortized step complexity is  $O(\log p \log(p+q))$ per operation.
%Doing GC less frequently would yield better constant factors in the amortized number of steps per operation but use more space.



% !TEX root =  queue.tex

\section{Future Directions}

Our focus was on optimizing step complexity for worst-case executions.
However, our queue has a higher cost than the MS-queue in the best case (when an operation
runs by itself).
Perhaps our queue could be made adaptive by having an operation capture a starting node
in the \ordering\ tree (as in \cite{DBLP:conf/stoc/AfekDT95}) rather than starting at a statically assigned leaf.
A possible application of our queue  might be to use it as the slow path in the
fast-path slow-path methodology  \cite{10.1145/2370036.2145835} to
get a queue that has good performance in practice while also having good worst-case step complexity.

It would be interesting to close the gap that remains between our queue, which takes $O(\log^2 p + \log q)$ steps per operation,
and Jayanti, Tarjan and Boix-Adser\`{a}'s $\Omega(\log p)$ lower bound \cite{JTB19}.
For the more relaxed bag data structure the gap is larger between the  $\Omega(\min(c,\log\log p))$ lower bound \cite{DBLP:conf/opodis/AttiyaF17} and our upper bound of $O(\log^2 p + \log q)$.
Could the complexity for either queues or bags be made polylogarithmic in $p$ while being independent of the size $q$ of the data structure?

The approach used here to implement a lock-free queue 
could be applied to obtain other lock-free
data structures with a polylogarithmic step complexity.
For example, we can easily adapt the routines of the implementation in Section \ref{DescriptQ} 
to implement a restricted kind of vector data structure that stores a sequence and
provides three operations: \opa{Append}{e} to add an element \var{e} to the end of the sequence,
\opa{Get}{i} to read the \var{i}th element in the sequence, and
\opa{Index}{e} to compute the position of element \var{e} in the sequence.
Only the \op{Append} operations need to be propagated to the root of the ordering tree
since the other two operations do not affect the state of the object.
An \opa{Append}{e} is implemented like \opa{Enqueue}{e} in $O(\log p)$ steps.  
A \opa{Get}{i} is similar to \op{GetEnqueue}, taking $O(\log n + \log^2p)$ steps when the vector has $n$ elements.  
An \op{Index} is similar to \op{IndexDequeue} (except operating on \enqueues\ instead of \dequeues) and would take $O(\log p)$ steps if the argument is a pointer to the leaf block that contains the element~$e$.

Building on the work described in this paper, Asbell and Ruppert \cite{AR23} have
designed a doubly-ended queue with polylogarithmic amortized step
complexity, which also yields a stack as a special case.  
This required a substantially different representation of the data stored in the ordering tree.
Whether the ordering tree could also be used to obtain a priority queue with polylogarithmic step complexity
remains an open question.



\newpage

\bibliographystyle{plain}
\bibliography{queues.bib}

\appendix
\end{document}
