% !TEX root =  queue.tex

\section{Future Directions}

Our focus was on optimizing step complexity for worst-case executions.
However, our queue has a higher cost than the MS-queue in the best case (when an operation
runs by itself).
Perhaps our queue could be made adaptive by having an operation capture a starting node
in the \ordering\ tree (as in \cite{DBLP:conf/stoc/AfekDT95}) rather than starting at a statically assigned leaf.
A possible application of our queue  might be to use it as the slow path in the
fast-path slow-path methodology  \cite{10.1145/2370036.2145835} to
get a queue that has good performance in practice while also having good worst-case step complexity.

It would be interesting to close the gap that remains between our queue, which takes $O(\log^2 p + \log q)$ steps per operation,
and Jayanti, Tarjan and Boix-Adser\`{a}'s $\Omega(\log p)$ lower bound \cite{JTB19}.
For the more relaxed bag data structure the gap is larger between the  $\Omega(\min(c,\log\log p))$ lower bound \cite{DBLP:conf/opodis/AttiyaF17} and our upper bound of $O(\log^2 p + \log q)$.
Could the complexity for either queues or bags be made polylogarithmic in $p$ while being independent of the size $q$ of the data structure?

We believe the approach used here to implement a lock-free queue 
could be applied to obtain other lock-free
data structures with a polylogarithmic step complexity.
For example, we can easily adapt our routines to implement a  vector data structure that stores a sequence and
provides three operations: \opa{Append}{e} to add an element \var{e} to the end of the sequence,
\opa{Get}{i} to read the \var{i}th element in the sequence, and
\opa{Index}{e} to compute the position of element \var{e} in the sequence.
%\here{Can omit rest of this paragraph to save space}
%An \opa{append}{e} is implemented like \opa{Enqueue}{e} in $O(\log p)$ steps.  
%A \opa{get}{i} is similar to \op{GetEnqueue}, taking $O(\log n + \log^2p)$ steps when the vector has $n$ elements.  
%An \opa{index}{e} is similar to \op{IndexDequeue} (except operating on enqueues instead of dequeues) and would take $O(\log p)$ steps.
\here{doublecheck all preceding claims for accuracy.}
Building on the work described in this paper, Asbell and Ruppert \cite{AR23} have
designed a doubly-ended queue with polylogarithmic amortized step
complexity, which also yields a stack as a special case.  
This required a substantially different representation of the data stored in the ordering tree.
Whether the ordering tree could also be used to obtain a priority queue with polylogarithmic step complexity
remains an open question.

