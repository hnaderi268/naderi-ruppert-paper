% !TEX root =  podc-submission.tex

\section{Introduction}

There has been a great deal of research in the past several decades on the design of linearizable, lock-free queues.
Besides being a fundamental data structure, queues are used in
significant concurrent applications, including OS kernels \cite{MP91}, memory reclamation \cite{?},
packet processing \cite{DPDK},
and sharing resources or tasks.
The lock-free MS-queue of Michael and Scott \cite{MS98} is a classic shared queue implementation.
It uses a singly-linked list with pointers to the front and back nodes.
To dequeue or enqueue an element, the front or back pointer is updated by a 
compare-and-swap (CAS) instruction.
If this CAS fails, the operation must retry.
In the worst case, this means that each successful CAS may cause all other processes to
fail and retry, leading to an amortized step complexity of $\Omega(p)$ per operation.
(To measure amortized step complexity of a lock-free implementation, we consider all possible finite executions
and divide the number of steps in the execution by the number of operations initated in the execution.)
Numerous papers have suggested modifications to the MS-queue \cite{DBLP:conf/opodis/HoffmanSS07,DBLP:conf/podc/KoganH14,DBLP:conf/ppopp/KoganP11,DBLP:journals/dc/Ladan-MozesS08,MKLLP22,DBLP:conf/spaa/MoirNSS05,RC17}, but 
all still have $\Omega(p)$ amortized step complexity as a result of
contention on the front and back of the queue.
Morrison and Afek \cite{DBLP:conf/ppopp/MorrisonA13} called this the CAS retry problem.
The same problem occurs in array-based implementations of queues \cite{DBLP:conf/iceccs/ColvinG05,DBLP:conf/icdcn/Shafiei09,DBLP:conf/spaa/TsigasZ01,DBLP:conf/opodis/GidenstamST10}.
Solutions that tried to sidestep this problem using fetch\&increment \cite{DBLP:conf/ppopp/MorrisonA13,DBLP:conf/ppopp/YangM16,Nik19,10.1145/3490148.3538572}
rely on slower mechanisms to handle worst-case executions and still have $\Omega(p)$ step complexity.

Many concurrent data structures that keep track of a set of elements also have an $\Omega(p)$ term in their step complexity, as observed by Ruppert \cite{Rup16}.
For example, lock-free lists \cite{FR04,Sha15}, stacks \cite{Tre86} and search trees \cite{EFHR14} 
have an $\Omega(c)$ term in their step complexity, where $c$ represents contention,
the number of processes that access the data structure concurrently (which can be $p$ in the worst case).
Attiya and Fouren \cite{DBLP:conf/opodis/AttiyaF17} proved 
that amortized $\Omega(c)$ steps per operation are indeed necessary
for CAS-based implementations of lock-free bag data structures, which provide operations
to insert an element or remove an arbitrary element (chosen non-deterministically).
Since a queue trivially implements a bag, this lower bound also holds for queues.
At first glance, this would seem to settle the question of the step complexity of lock-free queues.
However, the lower bound leaves a loophole:  it holds only if $c$ is $O(\log\log p)$.
Thus, the lower bound could be stated more precisely as an amortized bound of $\Omega(\min(c,\log\log p))$ steps per operation.

In this paper, we exploit this loophole.  We show that it is in fact possible to implement a linearizable, lock-free queue
whose step complexity does not depend linearly on $p$.
Our implementation is the first to have step complexity that is polylogarithmic in $p$ and in $q$, the number of elements
stored in the queue.
For simplicity of presentation, we first give an unbounded-space construction that uses
$O(\log^2 p + \log q)$ steps per operation.
Then, we show how to modify the implementation to use 
\here{$O(?)$ space} and \here{$O(?)$ steps per operation}.
Both implementations use single-word CAS instructions and reasonably-sized words.
Moreover, they use $O(\log p)$ CAS instructions per operation in the worst case, whereas previous
lock-free queues use 
%an unbounded number of CAS instructions in the worst case and 
$\Omega(p)$ CAS instructions in an amortized sense.
For the space-bounded implementation, we unlink unneeded memory from our data structure.
We do not address the orthogonal problem of reclaiming unlinked memory; we assume a safe
garbage collector, such as the highly optimized one that Java provides.

Our queue uses a tournament tree where each process has its own leaf.
Each process propagates its operations along the path from its leaf up to the root.
The root of the tree provides an ordering for all the operations,
and this ordering is used to linearize the operations and compute their responses.
\here{Either here or in related work section, talk about previous usage of tournament tree and how ours differs from it}
To ensure wait-freedom, we use a helping mechanism:  a process propagating its operation 
to a node $n$ helps propagate all operations from both of $n$'s children.
If we explicitly stored all operations in the nodes, this helping would become very costly.
Instead, we use an implicit representation of collections
of operations that allows for quickly merging two collections from the two children of a node,
and for quickly accessing any particular operation in the collection.
\here{Maybe say a little more about techniques used in the implementation, if space permits}
