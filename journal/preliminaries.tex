% !TEX root =  queue.tex

\section{Model}

\rev{
We use a  standard model of an asynchronous, shared-memory system.
A collection of $p$ processes can communicate by accessing
shared memory locations using read, write and compare-and-swap (CAS) instructions.
A \op{CAS}(\var{X}, \var{old}, \var{new}) instruction updates a variable \var{X} to the value \var{new} and returns \tr\ if \var{X}'s value is \var{old}; otherwise the \op{CAS} leaves \var{X} unchanged and returns \fa.
We also assume processes can allocate a block of shared memory to represent an object.

We are concerned with implementing a first-in, first-out queue in such a system,
which requires an algorithm that a process can execute to perform
an enqueue or dequeue operation.
We treat individual memory instructions as atomic, so an execution
of such an implementation can be thought of as a sequence of steps by processes
where in each step a process performs at most one shared-memory instruction 
(and receives the response of that instruction).
The subsequence of steps performed by a process should follow that process's
algorithm for performing a sequence of enqueue and dequeue operations.
The order in which the steps of processes are interleaved is arbitrary, and the implementation must be correct for all
possible interleavings.

We use the standard correctness requirement of linearizability \cite{HW90} for 
our queue.  This means that for every execution of our queue implementation,
there should be a sequential ordering (called the linearization) of 
a subset of the operations, including all completed operations,
satisfying two properties.  First, if one operation terminates before another
begins, they must appear in the same order in the linearization.
Second, each completed operation must return the same response that it would
if the operations were done sequentially in their linearization order.
(In fact, our implementation also satisfies the stronger requirement of being
strongly linearizable \cite{GHW11}.)
The termination condition that our queue satisfies is wait-freedom:
no operation by a process can take infinitely many steps of the process to complete.

The step complexity of an operation is the number of steps it performs.
The amortized step complexity is the total number of steps by all processes
divided by the number of operations they perform.
More formally, to show that the amortized step complexity is $O(T_e)$ per enqueue
and $O(T_d)$ per dequeue, we must show that in every execution, 
if $n_e$ enqueues and $n_d$ dequeues are invoked, the total number
of steps taken by all processes is $O(n_e\cdot T_e + n_d\cdot T_d)$.
}

