% !TEX root =  podc-submission.tex

\section{Queue Implementation} \label{DescriptQ}

\subsection{Overview}
Our \emph{\ordering\ tree} data structure is used to agree on a total ordering of the operations performed on the queue.
It is a static binary tree of height $\ceil{\log_2 p}$ with one leaf 
for each process. 
Each tree node  stores an array of \emph{blocks}, where each block represents a 
sequence of enqueues and a sequence of dequeues.
See Figure \ref{orderingtree} for an example.
In this section, we use an infinite array of blocks in each node.
Section \ref{reducing} describes how to replace the infinite array by a representation with bounded space.

To perform an operation on the queue, a process $P$ appends a new block containing that  
operation to the \fld{blocks} array in $P$'s leaf.
Then, $P$ attempts to propagate the operation to each node along the path from that leaf to the root of the tree.
We shall define a total order on all operations that have been propagated to the root, which 
will serve as the linearization ordering of the operations.

\Eric{Reworded following parag after Hossein's comments of Jan 4}
To propagate operations from a node $\var{v}$'s children to $\var{v}$, $P$ first observes
the blocks in both of $\var{v}$'s children that are not already in $\var{v}$,
creates a new block by combining information from those blocks, and attempts to append this 
new block to $\var{v}$'s \fld{blocks} array using a \op{CAS} instruction.
Following \cite{DBLP:conf/fsttcs/JayantiP05}, we call this a 3-step sequence a
\op{Refresh} on $\var{v}$. %(see Figure \ref{fig::propagstep}).
A \op{Refresh}'s \op{CAS} may fail if there is another concurrent \op{Refresh} on $\var{v}$.
However, since a successful \op{Refresh} propagates multiple pending operations 
from $\var{v}$'s children to $\var{v}$,
we can prove that if two \op{Refresh}es by $P$ on $\var{v}$ fail,
then $P$'s operation has been propagated to $\var{v}$ by some other process, so $P$ can continue 
onwards towards the~root.

Now suppose $P$'s operation has been propagated all the way to the root.
If $P$'s operation is an enqueue, it has obtained a place in the linearization ordering and can terminate.
If $P$'s operation is a dequeue, $P$ must use information in the tree to compute the value that the
dequeue must return.  To do this, $P$ first determines which block in the root contains its operation
(since the operation may have been propagated to the root by some other process).
Then, $P$ determines whether the queue is empty when its dequeue is linearized. 
If so, it returns \nl\ and we call it a \emph{null dequeue}.
If not, $P$ computes the rank\footnote{We say that the $r$th element in a sequence has rank $r$ within that sequence.} $r$ of its dequeue among all non-null dequeues in the linearization ordering,
finds the $r$th enqueue in the linearization, and returns that enqueue's value.

We must choose what to store in each block so that the following tasks can be done efficiently.
\begin{enumerate}[label={(T\arabic*)}]
\item
\label{construct}
Construct a block for node $\var{v}$ that represents the operations contained in consecutive blocks in $\var{v}$'s children, as required for a \op{Refresh}.
\item
\label{findinroot}
Given a dequeue in a leaf that has been propagated to the root, find that operation's position in the root's \fld{blocks} array.
\item
\label{findrank}
Given a dequeue's position in the root, determine if it is a null dequeue (i.e., if the queue is empty when it is linearized)
or determine the rank $r$ of the enqueue whose value it should return.
\item
\label{findenqueue}
Find the $r$th enqueue in the linearization ordering.
\end{enumerate}
Since these tasks depends on the linearization ordering, we describe that ordering next.

\begin{figure*}[t]
\input{tournamentTree.pdf_t}
\caption{An example \ordering\ tree with four processes. 
We show explicitly the enqueue sequence and dequeue sequence represented by each block in the \fld{blocks} arrays of the seven nodes.  The leftmost element of each \fld{blocks} array is a dummy block.
Arrows represent the indices stored in \eleft\ and \eright\ fields of blocks (as described in Section \ref{sec:fields}).
The fourth process's Deq\sub{6} is still propagating towards the root.
The linearization order for this tree is
Enq(a) Enq(e) Deq\sub{2} $\mid$ Enq(b) Deq\sub{4} Deq\sub{5} $\mid$ Enq(d) Enq(f) Enq(h) Deq\sub{1} $\mid$ Enq(c) Deq\sub{3} $\mid$ Enq(g), where vertical bars indicate boundaries of blocks in the root.\label{orderingtree}}
\medskip
\input{implicit.pdf_t}
\caption{\label{implicit}The actual, implicit representation of the tree shown in Figure \ref{orderingtree}.
The leaf blocks simply show the \fld{element} field.
Internal blocks show the \fld{sum\sub{enq}} and \fld{sum\sub{deq}} fields,
and \eleft\ and \eright\ fields are shown using arrows as before.
Root blocks also have the additional \fld{size} field.
The \fld{super} field is not shown.}
\end{figure*}

\subsection{Linearization Ordering}

Performing a double refresh at each node along the path from the leaf to the root ensures 
a block containing the operation is appended to the root before the operation completes.
So, if an operation $op_1$ terminates before another operation $op_2$ begins, 
$op_1$ will be in an earlier block than $op_2$ in the root's blocks array.
Thus, we linearize operations according to the block they belong to in the root's array.
We can choose how to order operations in the same block, since they must be concurrent.

Each block in a leaf represents one operation.
Each block $B$ in an internal node $\var{v}$ results from merging
several consecutive blocks from each of $\var{v}$'s children.
The merged blocks in $\var{v}$'s children are called the \emph{direct subblocks} of $B$.
A block $B'$ is a \emph{subblock} of $B$ if it is a direct subblock of $B$
or a subblock of a direct subblock of $B$.
A block $B$ represents the set of operations in all of $B$'s subblocks in leaves of the tree.
The operations propagated by a \op{Refresh} are all pending when the \op{Refresh} occurs,
so there is at most one operation per process.
Hence, a block represents at most $p$ operations in total.  
Moreover, we never append empty blocks, so 
each block represents at least one operation and it follows that a block can have at most $p$ direct subblocks.

As mentioned above, we are free to order operations within a block however we like.
We order the enqueues and dequeues separately, and put the 
operations propagated from the left child before the operations from the right child.
More formally, we inductively define sequences $E(B)$ and $D(B)$ of the enqueues and dequeues
represented by a block $B$.
If $B$ is a block in a leaf representing an enqueue operation, its enqueue sequence $E(B)$ is that operation
and its dequeue sequence $D(B)$ is empty.  If $B$ is a block in a leaf representing a dequeue, $D(B)$ is that single operation and $E(B)$ is empty.
If $B$ is a block in an internal node $\var{v}$ with direct subblocks $B^L_1, \ldots, B^L_\ell$ from 
$\var{v}$'s left child
and $B^R_1,\ldots,B^R_r$ from $\var{v}$'s right child, then $B$'s operation sequences are defined by the concatenations 
\begin{eqnarray}
E(B) &=& E(B^L_1)\cdots E(B^L_\ell)\cdot E(B^R_1) \cdots E(B^R_r) \mbox{ and }\nonumber\\
D(B) &=& D(B^L_1)\cdots D(B^L_\ell)\cdot D(B^R_1) \cdots D(B^R_r)\label{defSeqs}
\end{eqnarray}
We say the block $B$ \emph{contains} the operations in $E(B)$ and $D(B)$.

Finally, when designing a total order for the operations propagated to the root, we choose
to put each block's enqueues before its dequeues.
Thus, if the root's blocks array contains blocks $B_1, \ldots, B_k$, the 
linearization ordering is 
\begin{equation}
L=E(B_1)\cdot D(B_1) \cdot E(B_2) \cdot D(B_2) \cdots E(B_k) \cdot D(B_k).
\label{linearization}
\end{equation}

\subsection{Designing a Block Representation to Solve Tasks \ref{construct} to \ref{findenqueue}}
\label{sec:fields}

\renewcommand{\algorithmiccomment}[1]{\hfill\eqparbox{COMMENTSINGLE}{\com\ #1}}
\begin{figure}
\begin{algorithmic}
\setcounter{ALG@line}{1}
\Statex \linecomment{Shared variable}
\begin{itemize}
\item \typ{Node} \var{root} \Comment{root of binary tree of \tt{Node}s with one leaf per process}
\end{itemize}

\Statex \linecomment{Thread-local variable}
\begin{itemize}
\item \typ{Node} \var{leaf} \Comment{process's leaf in the tree}
\end{itemize}

\Statex $\blacktriangleright$ \typ{Node}
\begin{itemize}
\item \typ{Node} \fld{left}, \fld{right}, \fld{parent} \Comment{tree pointers initialized  when creating the tree}
\item \typ{Block}[0..$\infty$] \fld{blocks} \Comment{blocks that have been propagated to this node;}\linebreak
	\mbox{ }\Comment{\var{blocks}[0] is empty block whose integer fields are 0}
\item \typ{int} \head \Comment{position to append next \block\ to \fld{blocks}, initially 1}
\end{itemize}

\Statex $\blacktriangleright$ \typ{Block} 

\begin{itemize}
  	\item \typ{int} \fld{sum\sub{enq}, sum\sub{deq}}
  		\Comment{number of enqueues, dequeues in \fld{blocks} array}\linebreak
		\mbox{ }\Comment{up to this block (inclusive)}
  	\item \typ{int} \fld{super}
  		\Comment{approximate index of superblock in \fld{parent.blocks}}
	\item[\com] Blocks in internal \nodes\ have the following additional fields
	\begin{itemize}[leftmargin=3mm]
		\item[$\bullet$] \typ{int} \eleft, \eright
  		\Comment{index of last direct subblock in the left and right child}
	\end{itemize}
  	\item[\com] Blocks in leaf \nodes\ have the following additional field
  	\begin{itemize}[leftmargin=3mm]
		\item[$\bullet$] \typ{Object} \fld{element}
  		\Comment{\var{x} for \opa{Enqueue}{x} operation; otherwise \nl}
	\end{itemize}
	\item[\com] Blocks in the root \node\ have the following additional field
	\begin{itemize}[leftmargin=3mm]
		\item[$\bullet$] \typ{int} \size%
  		\Comment{size of queue after performing all operations up }\linebreak
		\mbox{ }\Comment{to the end of this block}
	\end{itemize}
\end{itemize}

%\Statex {\com\  Blocks in internal nodes have the following additional fields}
%\Statex $\blacktriangleright$ \tt{\sl{InternalBlock} extends \sl{Block}} \sf{\com\ the following additional fields are used only for blocks in internal nodes}
%\begin{itemize}
%	\item \tt{int} \eleft, \eright
%  		\Comment{index of last direct subblock in the left and right child}
%\end{itemize}

%\Statex {\com\ Blocks in leaf nodes have the following additional field}
%\Statex $\blacktriangleright$ \tt{\sl{LeafBlock} extends \sl{Block}} \sf{\com\ the following additional field is used only for blocks in leaves}
%\begin{itemize}
%  \item \tt{\sl{Object} element}
%  \Comment{if the block's operation is \tt{enqueue(x)} then \tt{element=x}, otherwise \tt{element=null}.}
%\end{itemize}

%\Statex {\com\ Blocks in the root node have the following additional field}
%\Statex $\blacktriangleright$ \tt{\sl{RootBlock} extends \sl{InternalBlock}} \sf{\com\ the following additional field is used only for blocks in the root}
%\begin{itemize}
%  \item \tt{\sl{int} \size}%
%  \Comment{size of the queue after performing all operations up to the end of this block}
%\end{itemize}

\end{algorithmic}
\caption{Objects used in the \ordering\ tree data structure.\label{object-fields}\Eric{Fix indentation}}
\end{figure}

Each \node\ of the \ordering\ tree has an infinite array called \fld{blocks}.
To simplify the code, \fld{blocks}[0] is initialized with an empty \block\ $B_0$, 
where $E(B_0)$ and $D(B_0)$ are empty sequences.
Each \node's \fld{head} index  stores the position in the \fld{blocks} array to be used
for the next attempt to append a \block.

If a block contained an explicit representation of its sequences of enqueues and dequeues,
it would take $\Omega(p)$ time to construct a block, which would be too slow for task \ref{construct}.
Instead, the block stores an implicit representation of the sequences.
We now explain how we designed the fields for this implicit representation. 
Refer to Figure \ref{implicit} for an example showing how the tree in Figure \ref{orderingtree} is actually represented, and Figure \ref{object-fields} for the definitions of the fields of \blocks\ and \nodes.

%The information stored in a \block\ depends on whether it is in an internal node or a leaf.
A block in a leaf represents a single enqueue or dequeue.  The block's \fld{element} field stores the value
enqueued if the operation is an enqueue, or \nl\ if the operation is a dequeue.

\here{If space permits, we might want to add some examples in the following paragraphs that refer back to Figure \ref{implicit}.}

Each block in an internal \node\ $\var{v}$ has fields \eleft\ and \eright\ that store the indices of the block's last direct subblock in $\var{v}$'s left and right child.  
Thus, the direct subblocks of $\var{v}.\fld{blocks}[b]$ are
\begin{eqnarray}\label{defsubblock}
\var{v}.\fld{left.blocks}[\var{v}.\fld{blocks}[b-1].\fld{end\sub{left}}+1..\var{v}.\fld{blocks}[b].\fld{end\sub{left}}] \mbox{ and}\nonumber\\
\var{v}.\fld{right.blocks}[\var{v}.\fld{blocks}[b-1].\fld{end\sub{right}}+1..\var{v}.\fld{blocks}[b].\fld{end\sub{right}}].
\end{eqnarray}
The \eleft\ and \eright\ fields allow us to navigate to a block's direct subblocks.
Blocks also store some prefix sums:
%the block in 
$\var{v}.\fld{blocks}[b]$ has two fields \fld{sum\sub{enq}} and \fld{sum\sub{deq}}
that store the total numbers of enqueues and dequeues in $\var{v}.\fld{blocks}[1..\var{b}]$.
We use these to search for a particular operation.
% pinpoint the  location of an operation among the subblocks of a given block.
For example, consider finding the $r$th enqueue $E_r$ in the linearization.
A binary search for $r$ on \fld{sum\sub{enq}} fields of the root's blocks 
finds the block  containing $E_r$.
If we know a block $B$ in a \node\ $\var{v}$ contains $E_r$,
we can use the \fld{sum\sub{enq}} field again to determine which child of $\var{v}$ contains $E_r$
and then do a binary search
among the direct subblocks of $B$ in that child.
Thus, we work our way down the tree until we find the leaf block that  stores 
$E_r$ explicitly.
%Now, suppose we want to find the $r$th enqueue in the linearization ordering for task \ref{findenqueue}.Let $B_1, B_2, \ldots, B_k$ be the blocks in the root.
%First, we need prefix sums of the number of enqueues in $E(B_1)\cdot E(B_2)\cdots E(B_i)$
%so that we can do a binary search for the block $B_e$ that contains the $r$th enqueue.
%This prefix sum also allows us to know the rank $r'$ within $E(B_e)$ of the $r$th enqueue.
%Once we have $r'$, we need the number of enqueues that $B_e$ received from its left child
%to determine whether the enqueue came from the left or right child of the root.
%Suppose the enqueue came from the right child $v_r$.
%Then, we know that the index of the block in $v_r$ that contains the enqueue
%is between $B_{e-1}.\eright + 1$ and $B_e$.\eright.
%We can again do a binary search within this range.
%For this, we can again use the prefix sums of the number of enqueues in any prefix of the array $v_r.blocks$.
%We can then continue in this way down the tree until reaching a leaf where the enqueue is stored explicitly.
We shall show that the binary search in the root can be done in $O(\log p + \log q)$ steps,
and the binary search within each other \node\ along the path to a leaf takes $O(\log p)$ steps,
for a total of $O(\log^2 p + \log q)$ steps for task \ref{findenqueue}.
%All information needed for this search process can be derived from the 
%\eleft, \eright\ and \fld{sum\sub{enq}} fields.

A block is called the \emph{superblock} of all of its direct subblocks.
To facilitate task \ref{findinroot}, each block $B$ has a field \fld{super} that contains
the (approximate) index of its superblock in the parent \node's \fld{blocks} array (it may differ from the true index by 1).
%We shall ensure that the value of $B.\fld{super}$ differs from the true index of $B$'s superblock by at most 1.
This allows a process to determine the true location of the superblock by checking the \eleft\ or \eright\ values of just two \blocks\ in the parent \node.
Thus, starting from an operation in a leaf's block, one can use these indices to track the 
operation all the way up the path to the root, and determine the operation's location in a root block
in $O(\log p)$ time.

Now consider task \ref{findrank}.
To determine whether the queue is empty when a dequeue occurs,
each block in the root has a \fld{size} field storing the number of elements
in the queue after all operations in the linearization up to that block (inclusive) 
have been done.
We can  determine which dequeues in a block $B_d$ in the root are null dequeues using
$B_{d-1}.\fld{size}$, which is the size of the queue just before $B_d$'s operations, and the number of enqueues and dequeues in $B_d$.
Moreover, the total number of non-null dequeues in blocks $B_1, \ldots, B_{d-1}$ 
is $B_{d-1}.\fld{sum\sub{enq}}-B_{d-1}.\fld{size}$.
We can use this information to determine the
rank of a non-null dequeue in $B_d$ among all non-null dequeues in the linearization, which is the rank  (among all enqueues) of the enqueue
whose value the dequeue should return.

Having defined the fields required for tasks \ref{findinroot}, \ref{findrank} and \ref{findenqueue},
we can easily see how to construct a new block $B$ during a \op{Refresh} in $O(1)$ time.
A \op{Refresh} on \node\ $\var{v}$ reads the values $h_{\ell}$ and $h_{r}$ of the \fld{head} fields of $\var{v}$'s children and stores 
$h_{\ell}-1$ and $h_{r}-1$ in $B.\eleft$ and $B.\eright$.
Then, we can compute $B.\fld{sum\sub{enq}}$ as $\var{v}.\fld{left}.\fld{blocks}[B.\eleft].\fld{sum\sub{enq}} + \var{v}.\fld{right}.\fld{blocks}[B.\eright].\fld{sum\sub{enq}}$.
For a block $B$ in the root, $B.\fld{size}$ can be computed using the \fld{size} field of the previous block $B'$ and
the number of enqueues and dequeues in $B$:
$B.\fld{size} = \max(0, B'.\fld{size} + (B.\fld{sum\sub{enq}}-B'.\fld{sum\sub{enq}}) - (B.\fld{sum\sub{deq}} - B'.\fld{sum\sub{deq}}))$.

The only remaining field is $B.\fld{super}$.  When the block 
$B$ is created for a \node\ $\var{v}$, we do not yet know where its
superblock will eventually be installed in $B$'s parent.
So, we leave $B.\fld{super}$ blank.  
Soon after $B$ is installed,
a call to \op{Advance} at line \ref{helpAdvance} or \ref{advance} by some process will
set \var{B}.\fld{super} to a value read from \var{par}.\fld{head}.
We shall show that this happens soon enough that $B.\fld{super}$ can differ from the true index of $B'$
by at most 1.
%After $B$ is installed 
%in $\var{v}.\fld{blocks}[\var{h}]$, processes cooperate to fill in $B.\fld{super}$ 
%when they attempt to advance $\var{v}.\fld{head}$ from $h$ to $h+1$,
%using a value they read from the \fld{head} field of $\var{v}$'s parent.
%As mentioned above, this might not be the exact index of $B$'s superblock, but we
%shall prove that it is close.

\subsection{Details of the Implementation}

We now discuss the queue implementation in more detail.  Pseudocode is provided in Figure \ref{pseudocode1} and \ref{pseudocode2}.
%We use $v.\var{blocks[i].num\sub{enq}}$ as shorthand for 
%$v.\var{blocks[i].sum\sub{enq} - blocks[i-1].sum\sub{enq}}$, that is, 
%the number of enqueues in the block.  (For $\var{i}=0$, $v.\var{blocks[0].num\sub{enq}} = 0$.)
%We use \var{num\sub{deq}} similarly.
%\here{Check if the num abbreviation is really needed in the code--how many times do we use it?}

\renewcommand{\algorithmiccomment}[1]{\hfill\eqparbox{COMMENTDOUBLE}{\com\ #1}}

\here{For consistency of notation, perhaps change $n$ to $B$ in code for Enqueue, Dequeue.  Maybe also new to B for CreateBlock, Refresh.}

\begin{figure*}
\begin{minipage}[t]{0.405\textwidth}
\begin{algorithmic}[1]
\setcounter{ALG@line}{0}

\Function{void}{Enqueue}{\typ{Object} \var{e}} 
    \State \hangbox{let \var{B} be a new \typ{Block} with \fld{element} \assign\ \var{e},\\
		$\fld{sum\sub{enq}} \assign\ \var{leaf.}\fld{blocks}[\var{leaf.}\head-1].\fld{sum\sub{enq}}+1$,\\
		$\fld{sum\sub{deq}} \assign\ \var{leaf.}\fld{blocks}[\var{leaf.}\head-1].\fld{sum\sub{deq}}$}\label{enqNew}
    \State \Call{Append}{\var{B}}
\EndFunction{Enqueue}

\spac

\Function{Object}{Dequeue()}{} 
    \State \hangbox{let \var{B} be a new \typ{Block} with \fld{element} \assign\ \nl,\\
	    $\fld{sum\sub{enq}} \assign\ \var{leaf.}\fld{blocks}[\var{leaf.}\head-1].\fld{sum\sub{enq}}$,\\
	    $\fld{sum\sub{deq}} \assign\ \var{leaf.}\fld{blocks}[\var{leaf.}\head-1].\fld{sum\sub{deq}}+1$}\label{deqNew}
    \State \Call{Append}{\var{B}}
    \State $\langle \var{b}, \var{i}\rangle$ \assign\ \Call{IndexDequeue}{\var{leaf}, $\var{leaf.}\head-1$, $1$}\label{invokeIndexDequeue}
    \State \Return{ \Call{FindResponse}{\var{b, i}}}\label{deqRest}
\EndFunction{Dequeue}

\spac

\Function{void}{Append}{\typ{Block} \var{B}} \com\ append block to leaf and propagate to root
    \State \var{leaf.}\fld{blocks}[\var{leaf.}\head] \assign\ \var{B}\label{appendLeaf}
    \State $\var{leaf.}\head\ \assign\ \var{leaf.}\head+1$ \label{appendEnd} 
    \State \Call{Propagate}{\var{leaf.}\fld{parent}} 
\EndFunction{Append}

\spac

\Function{void}{Propagate}{\typ{Node} \var{v}} \com\ propagate blocks from \var{v}'s children to root
    \If{\bf{not} \Call{Refresh}{\var{v}}} \label{firstRefresh}  \hfill \com\ double refresh
        \State \Call{Refresh}{\var{v}} \label{secondRefresh}
    \EndIf
    \If{$\var{v} \neq \var{root}$} \hfill \com\ recurse up tree
        \State \Call{Propagate}{\var{v}.\fld{parent}}
    \EndIf
\EndFunction{Propagate}

\spac

\Function{boolean}{Refresh}{\typ{Node} \var{v}} \com\ try to append a new block to \var{v}.\fld{blocks}
    \State \var{h} \assign\ \var{v}.\head \label{readHead}
    \ForEach{\fld{dir} {\keywordfont{in}} $\{\fld{left, right}\}$} \label{startHelpChild1}
        \State \var{childHead} \assign\ \var{v}.\fld{dir}.\head \label{readChildHead}
        \If{\var{v}.\fld{dir.blocks}[\var{childHead}] $\neq$ \nl} \label{ifHeadnotNull}
            \State \Call{Advance}{\var{v}.\fld{dir}, \var{childHead}} \label{helpAdvance}
        \EndIf
    \EndFor \label{endHelpChild1}
    \State \var{new} \assign\ \Call{CreateBlock}{\var{v, h}} \label{invokeCreateBlock}
    \If{\var{new = \nl}} \Return{\tr} \label{addOP} 
	\Else
	    \State \var{result} \assign\ \Call{CAS}{\var{v}.\fld{blocks}[\var{h}], \nl, \var{new}} \label{cas}
    	\State \Call{Advance}{\var{v, h}}\label{advance}
    	\State \Return{ \var{result}}
	\EndIf
\EndFunction{Refresh}

\spac


\Function{Block}{CreateBlock}{\typ{Node} \var{v}, \typ{int} \var{i}} 
    \State \linecomment{create new block for a \op{Refresh} to install in \var{v}.\fld{blocks}[\var{i}]}
    \State let \var{new} be a new \typ{Block} \label{initNewBlock}
    \State \var{new}.\eleft \assign\ $\var{v}.\fld{left}.\head - 1$\label{createEndLeft}
    \State \var{new}.\eright \assign\ $\var{v}.\fld{right}.\head - 1$\label{createEndRight}
	\State \hangbox{\var{new}.\fld{sum\sub{enq}} \assign\ \var{v}.\fld{left.blocks}[\var{new}.\eleft].\fld{sum\sub{enq}} + \\
			\hspace*{11mm}\var{v}.\fld{right.blocks}[\var{new}.\eright].\fld{sum\sub{enq}}}\label{createSumEnq}
	\State \hangbox{\var{new}.\fld{sum\sub{deq}} \assign\ \var{v}.\fld{left.blocks}[\var{new}.\eleft].\fld{sum\sub{deq}} + \\
			\hspace*{11mm}\var{v}.\fld{right.blocks}[\var{new}.\eright].\fld{sum\sub{deq}}}\label{createSumDeq}
    \State \var{num\sub{enq}} \assign\ $\var{new}.\fld{sum\sub{enq}} - \var{v}.\fld{blocks}[\var{i}-1].\fld{sum\sub{enq}}$\label{computeNumEnq}
    \State \var{num\sub{deq}} \assign\ $\var{new}.\fld{sum\sub{deq}} - \var{v}.\fld{blocks}[\var{i}-1].\fld{sum\sub{deq}}$
    \If{$\var{v} = \var{root}$}
        \State \hangbox{\var{new}.\fld{size} \assign\ max(0, $\var{v}.\fld{blocks}[\var{i}-1].\size\ + \var{num\sub{enq}} - \var{num\sub{deq}}$)}\label{computeLength}
    \EndIf
    \If{$\var{num\sub{enq}} + \var{num\sub{deq}} = 0$}\label{testEmpty}
        \State \Return \nl \hfill \com\ no blocks need to be propagated to \var{v}
    \Else
        \State \Return \var{new}
    \EndIf
\EndFunction{CreateBlock}

\end{algorithmic}
\end{minipage}
\begin{minipage}[t]{0.585\textwidth}

\begin{algorithmic}[1]
\setcounter{ALG@line}{50}
\Function{void}{Advance}{\typ{Node} \var{v}, \typ{int} \var{h}} \com\ set \var{v}.\fld{blocks}[\var{h}].\fld{super} and increment \var{v}.\fld{head} from \var{h} to $\var{h}+1$
    \If{$\var{v}\neq \var{root}$}
	    \State \var{h\sub{p}} \assign\ \var{v}.\fld{parent}.\head \label{readParentHead}
    	\State \Call{CAS}{\var{v}.\fld{blocks}[\var{h}].\fld{super}, \nl, \var{h\sub{p}}} \label{setSuper1}
	\EndIf
    \State \Call{CAS}{\var{v}.\head, \var{h}, \var{h}+1} \label{incrementHead}
\EndFunction{Advance}

\spac

\Function{$\langle\typ{int}, \typ{int}\rangle$}{IndexDequeue}{\typ{Node} \var{v}, \typ{int} \var{b}, \typ{int} \var{i}} \com\ return $\langle\var{b}', \var{i}'\rangle$ such that \var{i}th dequeue in
    \State \linecomment{$D(\var{v}.\fld{blocks}[\var{b}])$ is $(\var{i}')$th dequeue of $D(\var{root}.\fld{blocks}[\var{b}'])$}
    \State \linecomment{Precondition: $\var{v}.\fld{blocks}[\var{b}]$ is not \nl, was propagated to root, and contains at least}
    \State \linecomment{\var{i} dequeues}
    \If{$\var{v} = \var{root}$} \Return $\langle\var{b, i}\rangle$ \label{indexBaseCase}
    \Else
	    \State \fld{dir} \assign\ (\var{v}.\fld{parent.left} = \var{v} ? \fld{left} : \fld{right}) 
    	\State \var{sup} \assign\ \var{v}.\fld{blocks}[\var{b}].\fld{super}\label{idsup1}
	    \If{$\var{b} > \var{v}.\fld{parent.blocks}[\var{sup}].\fld{end\sub{dir}}$} \var{sup} \assign\ $\var{sup}+1$\label{supertest}\label{idsup2}
	    \EndIf\label{idsup3}
	    \State \linecomment{compute index \var{i} of dequeue in superblock}
	    \State \hangbox{\var{i} += $\var{v}.\fld{blocks}[\var{b}-1].\fld{sum\sub{deq}} -$ 
	    		$\var{v}.\fld{blocks}[\var{v}.\fld{parent.blocks}[\var{sup}-1].\edir].\fld{sum\sub{deq}}$}\label{computeISuperStart}
        \If{$\fld{dir} = \fld{right}$} 
        	\State \hangbox{\var{i} += $\var{v}.\fld{blocks}[\var{v}.\fld{parent.blocks}[\var{sup}].\eleft].\fld{sum\sub{deq}} - \mbox{ }$\\
					$\var{v}.\fld{blocks}[\var{v}.\fld{parent.blocks}[\var{sup}-1].\eleft].\fld{sum\sub{deq}}$}\label{considerLeftBeforeRight}
        \EndIf \label{computeISuperEnd}
        \State \Return\Call{IndexDequeue}{\var{v}.\fld{parent}, \var{sup}, \var{i}}
    \EndIf
\EndFunction{IndexDequeue}

\spac

\Function{element}{FindResponse}{\typ{int} \var{b}, \typ{int} \var{i}} \com\ find response to \var{i}th dequeue in $D(\var{root}.\fld{blocks}[\var{b}])$
    \State \linecomment{Precondition:  $1\leq i\leq |D(\var{root}.\fld{blocks}[\var{b}])|$}
    %  $i\geq 1$ and \var{root}.\fld{blocks}[\var{b}] is non-null and}
    %  \State \linecomment{contains at least \var{i} dequeues}
    \State \hangbox{\var{num\sub{enq}} \assign\ $\var{root}.\fld{blocks}[\var{b}].\fld{sum\sub{enq}} - \var{root}.\fld{blocks}[\var{b}-1].\fld{sum\sub{enq}}$}\label{FRNum}
    \If{$\var{root}.\fld{blocks}[\var{b}-1].\size + \var{num\sub{enq}} < \var{i}$}\label{checkEmpty}
        \State \Return \nl \hfill \com\ queue is empty when dequeue occurs\label{returnNull}
    \Else \ \linecomment{response is the \var{e}th enqueue in the root}
        \State \var{e} \assign\ \var{i} + \var{root}.\fld{blocks}[\var{b}-1].\fld{sum\sub{enq}} - 
			\var{root}.\fld{blocks}[\var{b}-1].\size\label{computeE}
		\State \linecomment{compute enqueue's block using binary search}
		\State find min $b_e\leq \var{b}$ with $\var{root}.\fld{blocks}[b_e].\fld{sum\sub{enq}} \geq \var{e}$\label{FRb}
		\State \linecomment{find rank of enqueue within its block}
		\State $i_e \assign\ \var{e} - \var{root}.\fld{blocks}[b_e-1].\fld{sum\sub{enq}}$\label{FRi}
        \State \Return \Call{GetEnqueue}{\var{root}, $b_e$, $i_e$}\label{findAnswer}
    \EndIf
\EndFunction{FindResponse}

\Function{element}{GetEnqueue}{\typ{Node} \var{v}, \typ{int} \var{b}, \typ{int} \var{i}} \Comment{returns argument of \var{i}th enqueue in $E(\var{v}.\fld{blocks}[\var{b}])$}
    \State \linecomment{Preconditions: $\var{i}\geq 1$ and \var{v}.\fld{blocks}[\var{b}] is non-\nl\ and contains at least \var{i} enqueues}
	
    \If{\var{v} is a leaf node} \Return \var{v}.\fld{blocks}[\var{b}].\fld{element} \label{getBaseCase}
    \Else 
        \State \var{sum\sub{left}} \assign\ \var{v}.\fld{left.blocks}[\var{v}.\fld{blocks}[\var{b}].\eleft].\fld{sum\sub{enq}} 
        \State \linecomment{\var{sum\sub{left}} is the number of enqueues in \var{v}.\fld{blocks}[1..$\var{b}$] from \var{v}'s left child}
        \State \var{prev\sub{left}} \assign\ \var{v}.\fld{left.blocks}[\var{v}.\fld{blocks}[$\var{b}-1$].\eleft].\fld{sum\sub{enq}} 
        \State \linecomment{\var{prev\sub{left}} is the number of enqueues in \var{v}.\fld{blocks}[1..$\var{b}-1$] from \var{v}'s left child}
        \State \var{prev\sub{right}} \assign\ \var{v}.\fld{right.blocks}[\var{v}.\fld{blocks}[$\var{b}-1$].\eright].\fld{sum\sub{enq}} 
        \State \linecomment{\var{prev\sub{right}} is the number of enqueues in \var{v}.\fld{blocks}[1..$\var{b}-1$] from \var{v}'s right child}
        \If{$\var{i} \leq \var{sum\sub{left}} - \var{prev\sub{left}}$} \label{leftOrRight} \Comment{required enqueue is in \var{v}.\fld{left}}
            \State \fld{dir} \assign\ \fld{left}
        \Else \Comment{required enqueue is in \var{v}.\fld{right}}
            \State \fld{dir} \assign\ \fld{right}
            \State $\var{i}\ \assign\ \var{i} - (\var{sum\sub{left}} - \var{prev\sub{left}})$
        \EndIf \label{endChooseDir}
        \State \linecomment{Use binary search to find enqueue's block in \var{v}.\fld{dir} and its rank within block}
        \State \hangbox{find minimum $\var{b}'$ in range [\var{v}.\fld{blocks}[$\var{b}-1$].\edir+1..\var{v}.\fld{blocks}[\var{b}].\edir] such that\\
        	 $\var{v}.\fld{dir.blocks}[\var{b}'].\fld{sum\sub{enq}} \geq \var{i} + \var{prev\sub{dir}}$\label{getChild}}
        \State $\var{i}'$ \assign\ $\var{i} - (\var{v}.\fld{dir.blocks}[\var{b}'-1].\fld{sum\sub{enq}} - \var{prev\sub{dir}})$\label{getChildIndex}
        \State \Return\Call{GetEnqueue}{\var{v}.\fld{dir}, $\var{b}'$, $\var{i}'$} \label{getRecurse}
    \EndIf
\EndFunction{GetEnqueue}

\end{algorithmic}
\end{minipage}
\vspace*{-3mm}
\caption{Queue implementation.\label{pseudocode1}}
\end{figure*}

An \opemph{Enqueue}(\var{e}) appends a \block\ to the process's leaf.
The block has $\fld{element}=\var{e}$ to indicate it represents an \op{Enqueue}(\var{e}) operation.
It suffices to propagate the operation to the root and
then use its position in the linearization for future \op{Dequeue}
operations.

A \opemph{Dequeue} also appends a \block\ to the process's leaf.
The block has $\fld{element}=\nl$ to indicate that it represents a \op{Dequeue} operation.
After propagating the operation to the root, it computes
its position in the root using
\op{IndexDequeue} and then computes its response by calling \op{FindResponse}. 

\opemph{Append}(\var{B}) first adds the block \var{B} to the invoking process's leaf.
The leaf's \fld{head} field stores the first empty slot in the leaf's \fld{blocks} array,
so the \op{Append} writes \var{B} there and increments \fld{head}.
Since \op{Append} writes only to the process's own leaf, there cannot be concurrent updates to a leaf.
\op{Append} then calls \op{Propagate} to ensure the operation represented by \var{B} is propagated to the root.

\opemph{Propagate}(\var{v}) guarantees that any blocks that are in \var{v}'s children when \op{Propagate} is invoked are propagated to the root.
It uses the double refresh idea described
above and invokes two \op{Refresh}es on \var{v} in Lines
\ref{firstRefresh} and \ref{secondRefresh}. 
If both fail to add a block to \var{v}, it means some other process has done a successful \op{Refresh}
that propagated blocks that were in \var{v}'s children prior to line \ref{firstRefresh} to \var{v}.
Then, \op{Propagate} recurses to \var{v}.\fld{parent} to continue propagating blocks up to the root.  

%\paragraph{\tt{Refresh()} and \tt{Advance()}}
A \opemph{Refresh} on node \var{v} creates a block representing the new blocks
in \var{v}'s
children and tries to append it to \var{v}.\fld{blocks}. 
Line \ref{readHead} reads \var{v}.\fld{head} into the local variable \var{h}.
Line \ref{invokeCreateBlock} creates the new block to install in \var{v}.\fld{blocks}[\var{h}].
If line \ref{invokeCreateBlock} returns \nl\ instead of a new block, there were no new blocks in \var{v}'s children to propagate to \var{v}, so \op{Refresh} can return true at line \ref{addOP} and terminate.
Otherwise, the CAS at line \ref{cas} tries to install the new block into \var{v}.\fld{blocks}[\var{h}].
Either this CAS succeeds or some other process has installed a  block in this location.
Either way, line \ref{advance} then calls \opemph{Advance} to advance \var{v}'s head index 
from \var{h} to $\var{h}+1$
and fill in the \fld{super} field of the most recently appended block.
The boolean value returned by \op{Refresh} indicates whether its CAS succeeded.
A \op{Refresh} may pause after a successful CAS before calling \op{Advance} at line \ref{advance},
so other processes help keep \fld{head} up to date by  calling \op{Advance}, 
either at line \ref{helpAdvance} during a \op{Refresh} on \var{v}'s parent or line \ref{advance} during a \op{Refresh} on~\var{v}.

\opemph{CreateBlock}(\var{v, i}) is used
by \op{Refresh} to construct a block to be installed in \var{v}.\fld{blocks}[\var{i}].
%The block \var{new} is created in Line \ref{initNewBlock}. 
The \eleft\ and \eright\ fields store the indices of the last blocks appended to \var{v}'s
children, obtained by reading the \fld{head} index in \var{v}'s children.
Since the \fld{sum\sub{enq}} field should store the number of enqueues in
\var{v}.\fld{blocks}[1..\var{i}] and these enqueues come from \var{v}.\fld{left.blocks}[1..\var{new}.\eleft] and \var{v}.\fld{blocks}[1..\var{new}.\eright], line \ref{createSumEnq} sets
\fld{sum\sub{enq}} to $\var{v}.\fld{left.blocks}[\var{new}.\eleft].\fld{sum\sub{enq}} + \var{v}.\fld{right}.\fld{blocks}[\var{new}.\eright].\fld{sum\sub{enq}}$.
Line \ref{computeNumEnq} sets \var{num\sub{enq}} to the number of enqueues in the new block by
subtracting from \var{new}.\fld{sum\sub{enq}} the number of enqueues  in \var{v}.\fld{blocks}[$1..\var{i}-1$].
The values of \var{new}.\fld{sum\sub{deq}} and \var{num\sub{deq}} are computed similarly.
Then, if \var{new}
is going to be installed in the root, line \ref{computeLength} computes the \fld{size} field, which
represents the number of elements in the queue after the operations in the block are performed.
Finally, if the new block contains no operations, \op{CreateBlock} returns \nl\ to indicate
 there is no need to install it.

Once a dequeue is appended to a block of the process's leaf and propagated to the root,
the \opemph{IndexDequeue} routine finds the dequeue's location in the root.
More precisely, \opa{IndexDequeue}{v, b, i}
computes the block in the root and the rank
within that block  of the \var{i}th dequeue of the block \var{B} stored in \var{v}.\fld{blocks}[\var{b}].
Lines \ref{idsup1}--\ref{idsup3} compute the location of $B$'s superblock in \var{v}'s parent, taking into account the fact that $B.\fld{super}$ may be off by one from the superblock's true index.
The arithmetic in lines \ref{computeISuperStart}--\ref{computeISuperEnd} compute the dequeue's 
rank within the superblock's sequence of dequeues, using  (\ref{defSeqs}).

%Thus, after \var{B} is installed, the call to \op{Advance} at line \ref{advance} sets the value of \var{B}.\fld{super} by reading \var{par}.\fld{head}.
%As mentioned earlier, other processes may help by calling \op{Advance} at line \ref{helpAdvance} or \ref{advance} to ensure that $\var{B}.\fld{super}$ is filled in soon after \var{B} is installed.
%We shall show {$B$.\fld{super}}  differs
%from the index of $B'$ by at most~1. 

To compute the response of the $i$th \op{Dequeue} in the $b$th block
of the root, \opemph{FindResponse}(\var{b, i}) determines at line \ref{checkEmpty} if the queue is empty.
If not, line \ref{computeE} computes the rank \var{e} of the
\op{Enqueue} whose argument is the \op{Dequeue}'s response. 
A binary search on the \fld{sum\sub{enq}} fields of \var{root}.\fld{blocks} finds the index $b_e$ of the block that contains 
the \var{e}th enqueue.
Since the enqueue is linearized before the dequeue, $b_e\leq b$.  To find the left end of the range for the binary search for $b_e$, we can first do a doubling search \cite{BY76}, comparing \var{e} to the \fld{sum\sub{enq}} fields at indices $b-1, b-2, b-4, b-8, \ldots$.
Then, \op{GetEnqueue} is used to trace down through the tree to find the required enqueue in a leaf.

\opemph{GetEnqueue}(\var{v, b, i}) returns the argument of the
$i$th enqueue in the $b$th block of \typ{Node} $\var{v}$. 
The range of subblocks of a block is determined using the \fld{end\sub{left}} and
\fld{end\sub{right}} fields of the block and its previous block. Then,
the exact subblock is found using binary search on the \fld{sum\sub{enq}}
field (line~\ref{getChild}). 
\here{explain a little more?}



