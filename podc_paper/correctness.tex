% !TEX root =  podc-submission.tex

\section{Proof of Correctness}

After proving some basic invariants in Section \ref{sec::basicProperties},
we show in Section \ref{sec::propagating} that performing a double refresh at each node
is sufficient to propagate an operation to the root.
In Section \ref{sec::helpingCorrect} we show that the \op{getEnqueue} and \op{indexDequeue}
routines correctly navigate through the \ordering\ tree.
Finally, we show that the implementation is linearizable in Section \ref{sec::linearizability}.

\subsection{Basic Properties}
\label{sec::basicProperties}

A \typ{Block} object's fields, except for \fld{super}, are immutable:  they are written only 
when the block is created at line \ref{enqNew} or \ref{deqNew} (for a leaf's block) or lines \ref{initNewBlock}--\ref{computeLength} (for an internal node's block).  
Moreover, only a \op{CAS} at line \ref{setSuper1} can modify the \fld{super} field 
(from \nl\ to a non-\nl\ value), so it remains fixed once a value is stored in it.
Similarly, only a \op{CAS} at line \ref{cas} can modify an element of a node's \fld{blocks} array 
(from \nl\ to a non-\nl\ value), so once a block is stored in a node, it remains there forever.
Only a \op{CAS} at line \ref{incrementHead} can update a node's \head\ field by incrementing it,
which implies the following.

\begin{observation} \label{nonDecreasingHead}
For each node \var{v},  \var{v}.\fld{head} is non-decreasing over time.
\end{observation}

\begin{lemma} \label{lem::headInc}
Let $R$ be an instance of \opa{Refresh}{v} whose call to \op{CreateBlock} returns a non-\nl\ block.  When $R$ terminates, \var{v}.\head\ is greater than the value $R$ reads from it at line \ref{readHead}.
\end{lemma}
\begin{proof}
After $R$'s \op{CAS} at line \ref{incrementHead}, \var{v}.\head\ is no longer equal to the value \var{h}
read at line \ref{readHead}.  The claim follows from \Cref{nonDecreasingHead}.
\end{proof}

Now we show $v.\fld{blocks}[v.\head]$ is either the last non-\nl\ block or the first \nl\ block in node $v$.

\begin{invariant}\label{lem::headPosition} 
For $0 \leq i < v.\head$, $v.\fld{blocks}[i]\neq\nl$.  For $i>v.\head$, $v.\fld{blocks}[i]=\nl$.
If $v\neq \var{root}$,  $v.\fld{blocks}[i].super \neq \nl$ for $0<i<v.head$.
\end{invariant}

\begin{proof}
Initially, $v.\head=1$, $v.\fld{blocks}[0]\neq\nl$  and $v.\fld{blocks}[i]=\nl$ for  $i>0$, so the claims~hold.

Assume the claim holds before a change to $v.\fld{blocks}$, which can be made only
by a successful \op{CAS} at line \ref{cas}.
The \op{CAS} changes $v.\fld{blocks}[h]$ from \nl\ to a non-\nl\ value.
Since $v.\fld{blocks}[h]$ is \nl\ before the CAS, $v.\head \leq h$ by the hypothesis.
Since $h$ was read from $v.\fld{blocks}[h]$ earlier at line \ref{readHead}, 
$v.\head \geq h$ by \Cref{lem::headPosition}.
So, $h=v.\head$ and a change to $v.\fld{blocks}[v.\head]$ preserves the invariant.

Now, assume the claim holds before a change to $v.\head$, which can only be an increment from $h$ to $h+1$
by a successful \op{CAS} at line \ref{incrementHead} of \op{Advance}.
For the first two claims, it suffices to show that $v.\fld{blocks}[head] \neq \nl$.
\nf{Advance} is called either at line \ref{helpAdvance} 
after testing that $v.\fld{blocks}[h]\neq\nl$ at line \ref{ifHeadnotNull},
or at line \ref{advance} after the \op{CAS} at line \ref{cas} ensures $v.\fld{blocks}[h]\neq\nl$.
For the third claim, observe that prior to incrementing $v.\head$ at line \ref{incrementHead},
the \op{CAS} at line \ref{setSuper1} ensures that $v.\fld{blocks}[i].super\neq \nl$.
\end{proof}

It follows that blocks accessed by the \op{Enqueue}, \op{Dequeue} and \op{CreateBlock} routines are non-\nl.

The following two lemmas show that no operation appears in more than one block of the root.
\begin{lemma} \label{lem::headProgress}
 If $b>0$ and $v.\fld{blocks}[b] \neq \nl$ then 
 $v.\fld{blocks}[b-1].\fld{end\sub{left}} \leq v.\fld{blocks}[b].\fld{end\sub{left}}$ and 
 $v.\fld{blocks}[b-1].\fld{end\sub{right}} \leq v.\fld{blocks}[b].\fld{end\sub{right}}$.
\end{lemma}
\begin{proof}
Let $B$ be the block in $v.\fld{blocks}[b]$.
Before creating $B$ at line \ref{invokeCreateBlock}, the \op{Refresh} that installed $B$
read $b$ from $v.\head$ at line \ref{readHead}.
At that time, $v.\fld{blocks}[b-1]$ contained a block $B'$, by \Cref{lem::headPosition}.
Thus, the \op{CreateBlock}($v,b-1$) that created $B'$ terminated before the \op{CreateBlock}($v,b$) that
created $B$ started.
It follows from \Cref{nonDecreasingHead} that the value that 
line \ref{createEndLeft} of \op{CreateBlock}($v,b-1$) stores in $B'.\fld{end\sub{left}}$   
is less than or equal to the value that line \ref{createEndLeft} of \op{CreateBlock}($v,b$) 
stores in $B.\fld{end\sub{left}}$.
Similarly, the values stored in $B'.\eright$ and $B.\eright$ at line \ref{createEndRight} 
%of these calls to \op{CreateBlock} 
satisfy the claim.
\end{proof}

\begin{lemma} \label{lem::subblocksDistinct}
If $B$ and $B'$ are two blocks in nodes at the same depth, their sets of subblocks are disjoint.
\end{lemma}
\begin{proof}
We prove the lemma by reverse induction on the depth.
If $B$ and $B'$ are in leaves, they have no subblocks, so the claim holds.
Assume the claim holds for nodes at depth $d+1$ and let $B$ and $B'$ be two blocks in nodes at depth $d$.
Consider the direct subblocks of $B$ and $B'$ defined by~(\ref{defsubblock}).
If $B$ and $B'$ are in different nodes at depth $d$, then their direct subblocks are disjoint.
If $B$ and $B'$ are in the same node, it follows from \Cref{lem::headProgress} that their direct subblocks are disjoint.
Either way, their direct subblocks (at depth $d+1$) are disjoint, so the claim follows from the induction hypothesis.
\end{proof}

It follows that each block has at most one superblock.
Moreover, we can now prove each operation is contained in at most one block of each node,
and hence appears at most once in the linearization~$L$.

\here{Might be able to get rid of this corollary and just cite previous lemma instead to save space}
\begin{corollary}\label{lem::noDuplicates}
For  $i\neq j$, $v.\fld{blocks}[i]$ and $v.\fld{blocks}[j]$ cannot both contain the same operation.
\end{corollary}
\begin{proof}
The operations contained in a block $B$ are those that appear in subblocks of $B$ in the leaves of the tree.
Since each process puts each of its operations in only one block of its own leaf, an operation 
cannot be in two different leaf blocks. 
By \Cref{lem::subblocksDistinct}, $v.\fld{blocks}[i]$ and $v.\fld{blocks}[j]$ have no subblocks in common, so the claim follows.
\end{proof}



%\begin{definition}
%$n\nf{.blocks[}i\nf{]}$ is \emph{established} if $n\nf{.head}>i$. An operation is \it{established} in node $n$ 
%if it is in an established block of $n$. $EST^t_n$ is the set of established operations in node $n$ at time $t$.
%\end{definition}
%
%Now we want to say that blocks of a node grow over time.
%\begin{observation}\label{lem::blocksOrder}
%  If  time $t<$ time $t^\prime$ ($t$ is before $t^\prime$), then $ops(n.blocks)$ at time $t$ is a subset of 
%$ops(n.blocks)$ at time $t^\prime$.
%\end{observation}
%\begin{proof}
%Blocks are only appended (not modified) with \nf{CAS} to $n\nf{.blocks[}n\nf{.head]}$, so the set of the blocks of a node after the \nf{CAS} contains the set of the blocks before the \nf{CAS}.
%\end{proof}

% \begin{corollary}\label{lem::establishedOrder}
%   If  time $t<$ time $t^\prime$, then $EST_n^t\subseteq EST_n^{t^\prime}$.
% \end{corollary}
% \begin{proof}
% From Observations \ref{nonDecreasingHead}, \ref{lem::blocksOrder}.  
% \end{proof}

The following shows that the values stored in \fld{sum\sub{enq}} and \fld{sum\sub{deq}} fields are accurate.
\here{If space is tight, could easily move next proof to appendix and just say it follows easily from the definition of subblocks and the values stored in the fields on lines \ref{enqNew}, \ref{deqNew}, \ref{createSumEnq}, \ref{createSumDeq}.}

\begin{invariant}\label{lem::sum}
If $B$ is a block stored in $v.\fld{blocks}[i]$,
$B.\fld{sum\sub{enq}} = | E(v.\fld{blocks}[0])\cdots E(v.\fld{blocks}[i]) |$ and
$B.\fld{sum\sub{deq}} = | D(v.\fld{blocks}[0])\cdots D(v.\fld{blocks}[i]) |$.
\end{invariant}
\begin{proof}
Initially, each \fld{blocks} array only contains an empty block $B_0$ in location 0.
By definition, $E(B_0)$ and $D(B_0)$ are empty sequences.
Moreover, $B_0.\fld{sum\sub{enq}} = B_0.\fld{sum\sub{deq}} = 0$, so the claim is true.

We show that each installation of a block $B$ into some location $v.\fld{blocks}[i]$ preserves the claim,
assuming the claim holds before this installation.  We consider two cases.

If $v$ is a leaf, $B$ was created at line \ref{enqNew} or \ref{deqNew}.
For line \ref{enqNew}, $B$ represents a single enqueue, so $|E(B)|=1$ and $|D(B)|=0$.
Since $B.\fld{sum\sub{enq}}$ is set to $v.\fld{blocks}[i-1]+1$ and
$B.\fld{sum\sub{deq}}$ is set to $v.\fld{blocks}[i-1]$, the claim follows from the hypothesis.
The proof for line \ref{deqNew}, where $B$ has a single dequeue, is similar.

Now suppose $v$ is an internal node. By the definition of subblocks in (\ref{defsubblock}) and \Cref{lem::headProgress}, the
subblocks of $v.\fld{blocks}[1..i]$ are $v.\fld{left.blocks}[1..B.\eleft]$ 
and $v.\fld{right.blocks}[1..B.\eright]$.
Thus, the enqueues in $E(v.\fld{blocks}[0])\cdots E(v.\fld{blocks}[i])$ are those in
$E(v.\fld{left.blocks}[0]) \cdots E(v.\fld{left.blocks}[B.\eleft])$ and
$E(v.\fld{left.blocks}[0]) \cdots E(v.\fld{left.blocks}[B.\eright])$.
By the hypothesis, the total number of these enqueues is $v.\fld{left.blocks}[B.\eleft].\fld{sum\sub{enq}} + v.\fld{right.blocks}[B.\eright].\fld{sum\sub{enq}}$, which is the value that line \ref{createSumEnq} stored in $B.\fld{sum\sub{enq}}$ when $B$ was created.
The proof for \fld{sum\sub{deq}} (stored on line~\ref{createSumDeq}) is similar.
\end{proof}

This allows us to prove that every block a Refresh installs contains at least one operation.

\begin{corollary}\label{blockNotEmpty}
If a block $B$ is in $v.\fld{blocks}[i]$ where $i>0$, then $E(B)$ and $D(B)$ are not both empty.
\end{corollary}
\begin{proof}
The \op{Refresh} that installed $B$ got $B$ as the response to its call to \op{CreateBlock} on line \ref{invokeCreateBlock}.
Thus, at line \ref{testEmpty} $\var{num\sub{enq}}+\var{num\sub{deq}}\neq 0$.
By \Cref{lem::sum}, $\var{num\sub{enq}} = |E(B)|$ and $\var{num\sub{deq}} = |D(B)|$,
so these sequences cannot both be empty.
\end{proof}



\subsection{Propagating Operations to the Root}
\label{sec::propagating}

Next, we show two \op{Refresh}es suffice to propagate operations from a child to its parent.
We say that node $v$ \emph{contains} an operation if some block in $v.\fld{blocks}$ contains the operation.
\here{move this defn earlier and just recall it here?}
Once a block is added to a node, it remains there forever.  Thus, if $v$ contains an operation at some time, it contains the operation at all later times too.

\begin{lemma}\label{successfulRefresh}
Let $R$ be a call to \op{Refresh}($v$) that performs a successful \op{CAS} on line \ref{cas} (or terminates at line \ref{addOP}).
In the configuration after that CAS (or termination, respectively), $v$ contains all operations that $v$'s children contained 
when $R$ executed line~\ref{readHead}.
\end{lemma}
\begin{proof}
Suppose $v$'s child (without loss of generality, $v.\fld{left}$) contained an operation $op$ 
when $R$ executed line \ref{readHead}.
Let $i$ be the index such that the block $B=v.\fld{left.blocks}[i]$ contains $op$.
By \Cref{nonDecreasingHead} and \Cref{lem::headProgress}, the value of $childHead$ that $R$ reads from
$v.\fld{left.head}$ in line \ref{readChildHead} is at least $i$.
If it is equal to $i$, $R$ calls \op{Advance} at line \ref{helpAdvance}, which ensures that 
$v.\fld{left.head} > i$.
Then, $R$ calls \op{CreateBlock}($v,h$) in line \ref{invokeCreateBlock}, where $h$ is the value $R$ reads at line \ref{readHead}.
\op{CreateBlock} reads a value greater than $i$ from $v.\fld{left.head}$ at line \ref{createEndLeft}.
Thus, $new.\eleft \geq i$.  We consider two cases.

Suppose $R$'s call to \op{CreateBlock} returns the new block $B'$ and $R$'s \op{CAS} at line \ref{cas} 
installs $B'$ in $v.\fld{blocks}$.
Then, $B$ is a subblock of some block in $v$, since  $B'.\eleft$ is greater than or equal to $B$'s index
in $v.\fld{left.blocks}$.
Hence $v$ contains $op$, as required.

\Eric{I think this case was missing from the proof in the thesis}
Now suppose $R$'s call to \op{CreateBlock} returns \nl, causing $R$ to terminate at line \ref{addOP}.
Intuitively, since there are no operations in $v$'s children to promote, $op$ is already in $v$.
We formalize this intuition.
The value computed at line \ref{createSumEnq} is
\begin{eqnarray*}
\var{num\sub{enq}} 
&=& v.\fld{left.blocks}[new.\eleft].\fld{sum\sub{enq}} + v.\fld{right.blocks}[new.\eright].\fld{sum\sub{enq}} - v.\fld{blocks}[h-1].\fld{sum\sub{enq}} \\
&=& v.\fld{left.blocks}[new.\eleft].\fld{sum\sub{enq}} + v.\fld{right.blocks}[new.\eright].\fld{sum\sub{enq}} \\
&&\mbox{ }- v.\fld{left.blocks}[v.\fld{blocks}[h-1].\eleft].\fld{sum\sub{enq}} - v.\fld{right.blocks}[v.\fld{blocks}[h-1].\eright].\fld{sum\sub{enq}}
\end{eqnarray*}
It follows from \Cref{lem::sum} that $num\sub{enq}$ is the total number of enqueues contained in the blocks
$v.\fld{left.blocks}[v.\fld{blocks}[h-1].\eleft+1..new.\eleft]$ and
$v.\fld{right.blocks}[v.\fld{blocks}[h-1].\eright+1..new.\eright]$.
Similarly, $num\sub{deq}$ is the total number of dequeues contained in these blocks.
Since $num\sub{enq}+num\sub{deq}=0$ at line \ref{testEmpty},
these blocks contain no operations.
By \Cref{blockNotEmpty}, this means the ranges of blocks are empty, so that $v.\fld{blocks}[h-1].\eleft \geq \var{new}.\eleft \geq i$.
Hence, $B$ is already a subblock of some block in $v$, so $v$ contains $op$.
\end{proof}

We now show that if a process fails its \op{Refresh}($v$) twice, some other process must have succeeded.

\begin{lemma}\label{lem::doubleRefresh}
Consider two consecutive terminating calls $R_1$, $R_2$ to \op{Refresh}($v$) by the same process.
When $R_2$ terminates, $v$ contains all operations that $v$'s children contained when $R_1$ was invoked.
\end{lemma}
\begin{proof}
If either $R_1$ or $R_2$ performs a successful \op{CAS} at line \ref{cas} or terminates at line \ref{addOP}, the claim follows
from \Cref{successfulRefresh}.
So suppose both $R_1$ and $R_2$ perform a failed \op{CAS} at line \ref{cas}.
Let $h_1$ and $h_2$ be the values $R_1$ and $R_2$ read from $v.\head$ at line \ref{readHead}.
By \Cref{lem::headInc}, $h_2>h_1$.
By \Cref{lem::headProgress}, $v.blocks[h_2]=\nl$ when $R_1$ executes line \ref{readHead}.
Since $R_2$ fails its \op{CAS} on $v.blocks[h_2]$, some other \op{Refresh} $R_3$ must have done
a successful \op{CAS} on $v.blocks[h_2]$ before $R_2$'s \op{CAS}.
$R_3$ must have executed line \ref{readHead} after $R_1$, since $R_3$ read the value $h_2$ from $v.\head$ and the value of $v.\head$ is non-decreasing, by \Cref{nonDecreasingHead}.
Thus, all operations contained in $v$'s children when $R_1$ begins
are also contained in $v$'s children when $R_3$ later executes line \ref{readHead}.
By \Cref{successfulRefresh}, these operations are contained in $v$ when $R_3$ performs its successful \op{CAS},
which is before $R_2$ terminates.
\end{proof}

\begin{lemma} \label{lem::appendExactlyOnce}
When an \op{Append}($B$) terminates, $B$'s operation is contained in exactly one block in each node along the path from the process's leaf to the root.
\end{lemma}
\begin{proof}
\op{Append} adds $B$ to the process's leaf and calls \op{Propagate}, which
does a double \op{Refresh}~on each internal node on the path $P$ from the leaf to the root.
By \Cref{lem::doubleRefresh}, this ensures a block in each node on $P$ contains $B$'s operation.
There is at most one such block in each node, by \Cref{lem::noDuplicates}.
\end{proof}

\subsection{Correctness of Helping Routines}
\label{sec::helpingCorrect}

We first show that \op{getEnqueue} works correctly.

\begin{lemma}\label{lem::get}
If $1\leq i\leq |E(v.\fld{blocks}[b])|$ then \op{getEnqueue}($v,b,i$) returns the argument of the $i$th enqueue in $E(v.\fld{blocks}[b])$.
\end{lemma}
\begin{proof}
We prove the claim by induction on the height of node $v$.
If $v$ is a leaf, the hypothesis implies that $i=1$ and the block $v.\fld{blocks}[b]$ represents 
an enqueue whose argument is stored in $v.\fld{blocks}[b].\fld{element}$.
\op{GetEnqueue} returns the argument of this enqueue at line \ref{getBaseCase}.

Assuming the claim holds for $v$'s children, we prove it for $v$.
Let $B$ be $v.\fld{blocks}[b]$.
By (\ref{defSeqs}),
$E(B)$ is obtained by concatenating the enqueue sequences of the direct subblocks
of $B$, which are listed in (\ref{defsubblock}).
By \Cref{lem::sum}, $\var{sum\sub{left}}-\var{prev\sub{left}}$ is the number
of enqueues in $E(B)$ that come from $B$'s subblocks in $v$'s left child.
Thus, $dir$ is set to the direction for the child of $v$ that contains the required enqueue.
Moreover, when line \ref{endChooseDir} is reached, $i$ is the position of the required enqueue within the portion $E'$ of $E(B)$ that comes from that child.
Thus,  line \ref{getChild} finds the index $b'$ of the subblock $B'$ containing the required enqueue.
By \Cref{lem::sum}, $v.\var{dir.blocks}[b'-1].\var{sum\sub{enq}} - \var{prev\sub{dir}}$ is the number of 
enqueues in $E'$ before the enqueues of block $B'$, so
the value $i'$ computed on line \ref{getChildIndex} is the position of the required enqueue within $E(B')$.
Thus, the recursive call on line \ref{getRecurse} satisfies its precondition, and 
returns the required result, by the induction hypothesis.
\end{proof}

Next, we prove the \fld{super} field of a block is within one of the true index of the block's superblock.

\begin{lemma}\label{superRelation}
Let $B=v.\var{blocks}[b]$.
  If $v.\fld{parent.blocks}[s]$ is the superblock of $B$ then $s-1\leq B.\fld{super}\leq s$.
\end{lemma}
\begin{proof}
We first show $B.\fld{super}\leq s$.
Let $R_s$ be the instance of \op{Refresh}($v.\fld{parent}$) that installs the superblock of $B$ 
in $v.\fld{parent.blocks}[s]$.
By the definition of subblocks (\ref{defsubblock}), $R_s$'s read $r$ of $v.head$ at line \ref{createEndLeft} or \ref{createEndRight} obtains a value greater than $b$.
By \Cref{lem::headPosition}, $B.\fld{super} \neq \nl$ when $r$ occurs, which means
that $B.\fld{super}$ was set (by line~\ref{setSuper1}) to a value read from $v.\fld{parent}.\head$ before $r$.
When $r$ occurs, $v.\fld{parent.blocks}[s] = \nl$, since the later \op{CAS} by $R_s$ at line
\ref{cas} succeeds.
So, by \Cref{lem::headPosition}, $v.\fld{parent}.\head \leq s$ when $r$ occurs.
Since the value stored $B.\fld{super}$ was read from $v.\fld{parent.head}$ before $r$ and the \head\ field is non-decreasing by \Cref{nonDecreasingHead}, it follows that $B.super\leq s$.

Next, we show that $B.\fld{super}\geq s-1$.
The value stored in $B.\fld{super}$ at line \ref{setSuper1} is read from $v.\fld{parent}.\head$ at line \ref{readParentHead} and \head\ values are always at least 1, so $B.\fld{super} \geq 1$.
So, if $s\leq 2$, the claim is trivial.  Assume $s>2$ for the remainder of the proof.
By \Cref{lem::headProgress}, $v.\fld{parent.blocks}[s-1]\neq \nl$.  Let $R_{s-1}$ be the call to
$\op{Refresh}(v.\fld{parent})$ that installed the block in $v.\fld{parent.blocks}[s-1]$.
Let $r'$ be the step when $R_{s-1}$ reads $s-1$ in $v.\fld{parent}.\head$ at line \ref{readHead}.
This read $r'$ must be before $B$ is installed in $v$;
otherwise, \Cref{successfulRefresh} would imply that $B$ is a subblock of one of 
$v.\fld{parent.blocks}[1..s-1]$, contrary to the hypothesis.
Now, consider the call to \op{Advance}($v, b$) that writes $B.\fld{super}$.
It is invoked either 
at line \ref{helpAdvance} after seeing $v.\fld{blocks}[h]\neq \nl$ at line \ref{ifHeadnotNull}
or at line \ref{advance} after ensuring $v.\fld{blocks}[h]\neq \nl$ at line~\ref{cas}.
Either way, the \op{Advance} is invoked after $B$ is installed, and therefore after $r'$.
By \Cref{nonDecreasingHead}, $v.\fld{parent}.\head$ is non-decreasing, so 
the value this \op{Advance} reads in $v.\fld{parent}.\head$ and
writes in $B.\fld{super}$ is greater than or equal to the value $s-1$ that $r'$ reads in $v.\fld{parent}.\head$.
\end{proof}

%The reader may wonder when the case $b\nf{.super}=s$ happens. This can happen when $
%\nf{$n$.parent.blocks[$B$.super]}=\nf{null}$ when $B$\nf{.super} is written and $R_p$ puts its created block 
%into \nf{$n$.parent.blocks[$B$\nf{.super}]} afterwards.

We prove \op{IndexDequeue}'s correctness using \Cref{superRelation} on each step of the \op{IndexDequeue}.
\here{mention somewhere why precondition of IndexDequeue is true; also check that block of parent indexed by sup in that routine is non-null}

\begin{lemma}\label{lem::indexDequeue}
If $v.\fld{blocks}[b]$ has been propagated to the root and $1\leq i\leq |D(v.\fld{blocks}[b])|$, 
 then \op{IndexDequeue}($v, b, i$) returns $\langle b',i' \rangle$ such that the \var{i}th dequeue in $D(\var{v}.\fld{blocks}[\var{b}])$ is the $(i')$th dequeue of $D(\var{root}.\fld{blocks}[b'])$.
\end{lemma}
\begin{proof}
We prove the claim by induction on the depth of node $v$. The base case where $v$ is the root is trivial (see Line \ref{indexBaseCase}).
Assuming the claim holds for $v$'s parent, we prove it for $v$.
Let $B=v.\fld{blocks}[b]$ and $B'$ be the superblock of $B$.
\op{IndexDequeue}($v, b, i$) first computes the index $sup$ of $B'$ in $v.\fld{parent}$.
By \Cref{superRelation}, this index is either $B.super$ or $B.super+1$.
The correct index is determined by testing on line \ref{supertest} whether $B$ is not a subblock of $v.\fld{parent.blocks}[B.\fld{super}]$.

Next, the position of the required dequeue in $D(B')$ is computed in 
lines \ref{computeISuperStart}--\ref{computeISuperEnd}. 
We first add the number of dequeues in the subblocks of $B'$ in $v$ that preced $B$ on line \ref{computeISuperStart}.
If $v$ is the right child of its parent, then all of the subblocks of $B'$ from $v$'s left sibling
also precede the required dequeue, so we add the number of dequeues in those subblocks in line \ref{considerLeftBeforeRight}.

Finally, \op{IndexDequeue} is called recursively on $v$'s parent.
Since $B$ has been propagated to the root, so has its superblock $B'$.
Thus, all preconditions of the recursive call are met.
By the induction hypothesis, the recursive call returns the location of the required dequeue in the root.\end{proof}

\subsection{Linearizability}
\label{sec::linearizability}

Consider any finite execution.  We must show that the linearization ordering $L$ defined in 
(\ref{linearization}) is a legal permutation of a subset of the operations in 
the execution, i.e., that it includes all operations that terminate and 
if one operation $op_1$ terminates before another operation $op_2$ begins, then $op_2$ does appear
before $op_1$ in $L$.  Then we must show that each dequeue that terminates returns the 
same response as it would in the sequential execution described by $L$.

\begin{lemma} \label{linearSat}
$L$ is a legal linearization ordering.
\end{lemma}
\begin{proof}
By \Cref{lem::noDuplicates}, $L$ is a permutation of a subset of the operations in the execution.
Since each operation creates a block $B$ containing the operation and then calls \op{Append}($B$), 
the operation is propagated to the root before the operation terminates, by \Cref{lem::appendExactlyOnce},
so it appears in $L$.
Also, if $op_{1}$ terminates before $op_{2}$ begins, then $op_{1}$ 
is propagated to the root before $op_2$ begins, so $op_2$ appears before $op_2$ in $L$.
\end{proof}

In the next lemma, we show that the \fld{size} field in each block in the root is computed correctly.
\begin{lemma}\label{sizeCorrectness}
If the operations of $\var{root}.\fld{blocks}[0..b]$ are applied sequentially in the order of~$L$ on an initially empty queue, the resulting queue has $\var{root}.\fld{blocks}[b].\size$ elements.  
\end{lemma}
\begin{proof}
We prove the claim by induction on $b$. 
The base case when ${b=0}$ is trivial, since the queue is initially empty and 
$\var{root}.\fld{blocks}[0]$ contains an empty block whose \size\ field is $0$. 
Assuming the claim holds for $b-1$, we prove it for $b$.
The \fld{size} field of the block $B$ installed in $\var{root}.\fld{blocks}[b]$ is computed
at line \ref{computeLength} of a call to \op{CreateBlock}(\var{root, b}).
By the induction hypothesis, $root.\fld{blocks}[b-1].\size$ gives the size of the queue before the operations
of block $B$ are performed.
By \Cref{lem::sum}, the values of \var{num\sub{enq}} and \var{num\sub{deq}}
are the number of enqueues and dequeues contained in $B$.
Hence, the size of the queue after the operations of $B$ are performed (with enqueues before dequeues as specified by $L$)
is $\max(0, \var{root}.\fld{blocks}[b-1].\size + \var{num\sub{enq}} - \var{num\sub{deq}})$.
\end{proof}

Next, we show operations return the same response as they would in the sequential execution $L$.

\begin{lemma}\label{linearCorrect}
Each terminating dequeue returns the response it would in the sequential execution $L$.
\end{lemma}
\begin{proof}
If a dequeue $D$ terminates, it is contained in some block in the root, by \Cref{lem::appendExactlyOnce}.
By \Cref{lem::indexDequeue}, $D$'s call to \op{IndexDequeue} on line \ref{invokeIndexDequeue}
returns a pair $\langle b,i\rangle$ such that $D$ is the $i$th dequeue in the block 
$B=\var{root}.\fld{blocks}[b]$.
$D$ then calls \op{FindResponse}($b,i$) on line \ref{deqRest}.
By \Cref{sizeCorrectness}, the queue contains $\var{root}.\fld{blocks}[b-1].\fld{size}$ elements
after the operations in $\var{root}.\fld{blocks}[1..b-1]$ are performed sequentially 
in the order given by $L$.
By \Cref{lem::sum}, the value of \var{num\sub{enq}} computed on line \ref{FRNum}
is the number of enqueues in $B$.
Since the enqueues in block $B$ precede the dequeues,
the queue is empty when the $i$th dequeue of $B$ occurs if 
$\var{root}.\fld{blocks}[b-1].\fld{size} + \var{num\sub{enq}} < i$.
So $D$ returns \nl\ on line \ref{returnNull} if and only if it would do so in the sequential
execution $L$.
Otherwise, the size of the queue after doing the operations in $\var{root}.\fld{blocks}[0..b-1]$
in the sequential execution $L$ is $\var{root}.\fld{blocks}[b-1].\fld{sum\sub{enq}}$ minus
the number of non-\nl\ dequeues in that prefix of $L$.
Hence, line \ref{computeE} sets $e$ to the rank of $D$ among all the non-\nl\ dequeues in $L$.
Thus, in the sequential execution~$L$, $D$ returns the value enqueued by the $e$th enqueue in $L$.
By \Cref{lem::sum}, this enqueue is the $i_e$th enqueue 
in $E(\var{root}.\fld{blocks}[b_e])$, where
$b_e$ and $i_e$ are the values $D$ computes on line \ref{FRb} and \ref{FRi}.
By \Cref{lem::get}, the call to \op{GetEnqueue} returns the argument of the required enqueue.
\end{proof}

Combining \Cref{linearSat} and \Cref{linearCorrect} provides our main result.

\begin{theorem}
The queue implementation is linearizable.
\end{theorem}


\here{Should there be a lemma somewhere in this section that explicitly shows that each
object we dereference is non-null?}