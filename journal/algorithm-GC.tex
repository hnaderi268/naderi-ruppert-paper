% !TEX root =  queue.tex

\subsection{Detailed Description}
\label{reducing-details}


\renewcommand{\algorithmiccomment}[1]{\hfill\eqparbox{COMMENTSINGLEAPP}{\com\ #1}}

To avoid confusion, we use nodes to refer to the nodes of the ordering tree, and blocks to refer
to the nodes of a RBT (since the RBT stores blocks).
The space-bounded implementation uses two  shared arrays:
the \var{leaf} array allows processes to access one another's leaves to perform helping, and
the \var{last} array is used to determine which blocks are safe to discard.

The \fld{blocks} field of each node in the \ordering\ tree is implemented as a pointer to the root of a RBT of \typ{Blocks} rather than an infinite array.  
Each RBT is initialized with an empty block with index~0.
Any access to an entry of the \fld{blocks} array is replaced by a search in the RBT.
The node's \head\ field, which previously gave the next position to insert into the \fld{blocks} array is no
longer needed; we can instead simply find the maximum \fld{index} of any block in the RBT.
To facilitate this, \op{MaxBlock} is a query operation on the RBT that 
returns the block with the maximum \fld{index}.
We can store, in the root of the RBT, a pointer to the maximum block so that \op{MaxBlock}
can be done in constant time, without affecting the time of other RBT operations.
Similarly, a \op{MinBlock} query finds the block with the minimum \fld{index} in a RBT.

Blocks no longer require the \fld{super} field.  It was used to quickly find a block's 
superblock in the parent node's \fld{blocks} array, but this can now be done efficiently 
by searching the parent's \fld{blocks} RBT instead.
Each \typ{Block} has an additional field.
\begin{itemize}
\item \typ{int} \var{index} 
\end{itemize}
that represents the position this block would have in the \fld{blocks} array.
To facilitate helping, each \typ{Block} in a leaf has one more additional field 
\begin{itemize}
\item \typ{Object} \var{response}
\end{itemize}
which is used only for blocks that store a dequeue operation and stores the response of the dequeue in the block.


Pseudocode for the space-bounded implementation appears in Figures \ref{code1GC} to \ref{code3GC}.
New or modified code appears in blue.
The \op{Propagate} and \op{GetEnqueue} routines are unchanged.
A few lines have been added to \op{FindResponse} to update the \var{last} array
to ensure that it stores the value described above.
Minor modifications have also
been made to \op{Enqueue}, \op{Dequeue}, \op{CreateBlock}, \op{Refresh} and \op{Append}
to accommodate  the switch from an array of blocks to a RBT of blocks (and the corresponding disappearance
of the \head\ field).
In addition, the second half of the \op{Dequeue} routine is now in a
separate routine called \op{CompleteDeq} so that it can also be used by other processes
helping to complete the operation.
The \op{Refresh} routine no longer needs to set the \fld{super} field of blocks since
that field has been removed.
The \op{IndexDequeue} routine, which must trace the location of a dequeue along
a path from its leaf to the root has a minor modification to search the \fld{blocks} RBT at 
each level instead of using the \fld{super} field.


\renewcommand{\algorithmiccomment}[1]{\hfill\eqparbox{COMMENTDOUBLE}{\com\ #1}}

\begin{figure*}
\begin{algorithmic}[1]
\setcounter{ALG@line}{200}
\State \linecomment Shared variables
\State \typ{Node} root \Comment{root of ordering tree}%
\gc{
\State \typ{Node[]} \var{leaf}$[1..p]$\Comment{\var{leaf}$[k]$ is the leaf assigned to process $k$}
\State \typ{int[]} \var{last}$[1..p]$\Comment{\var{last}$[k]$ is max \fld{index} of a root block that process $k$ saw}
\State \mbox{ }\Comment{contains either a dequeued enqueue or a null dequeue}%
}

\spac

\Function{void}{Enqueue}{\typ{Object} \var{e}} 
    
    \gc{\State \var{h} \assign\ \Call{MaxBlock}{\var{leaf}[\var{id}].\fld{blocks}}.\fld{index}+1}
	\State \hangbox{let \var{B} be a new \typ{Block} with fields \fld{element} \assign\ \var{e},
		\fld{sum\sub{enq}}\ \assign\ $\var{leaf}[\var{id}].\fld{blocks}[\gc{h-1}].\fld{sum\sub{enq}}+1,$\\
		\fld{sum\sub{deq}}\ \assign\ $\var{leaf}[\var{id}].\fld{blocks}[\gc{h-1}].\fld{sum\sub{deq}}$,
		\gc{\fld{index}\ \assign $h$}}\label{enqNewGC}
    \State \Call{Append}{\var{B}}\Comment{\rev{append \op{Enqueue} to \var{leaf}[\var{id}].\fld{blocks}, propagate to root}}
\EndFunction{Enqueue}

\spac

\Function{Object}{Dequeue()}{} 
    \State \gc{\var{h} \assign\ \Call{MaxBlock}{\var{leaf}[\var{id}].\fld{blocks}}.\fld{index}+1}
    \State \hangbox{let \var{B} be a new \typ{Block} with fields \fld{element} \assign\ \nl,
	    \fld{sum\sub{enq}}\ \assign\ $\var{leaf}[\var{id}].\fld{blocks}[\gc{h-1}].\fld{sum\sub{enq}},$\\
	    \fld{sum\sub{deq}}\ \assign\ $\var{leaf}[\var{id}].\fld{blocks}[\gc{h-1}].\fld{sum\sub{deq}}+1$,
	    \gc{\fld{index} \assign $h$}}\label{deqNewGC}
    \State \Call{Append}{\var{B}}\Comment{\rev{append \op{Dequeue} to \var{leaf}[\var{id}].\fld{blocks}, propagate to root}}%
    \gc{\State \Return \Call{CompleteDeq}{\var{leaf}[\var{id}], $h$}}\Comment{\rev{compute response \op{Dequeue} should return}}
\EndFunction{Dequeue}

\spac

\gc{
\Function{Object}{CompleteDeq}{\typ{Node} \var{leaf}, \typ{int} $h$} \Comment{finish propagated dequeue in $\var{leaf}.\fld{blocks}[h]$}
    \State $\langle \var{b}, \var{i}\rangle$ \assign\ $\Call{IndexDequeue}{\var{leaf}, h, 1}$\Comment{\rev{find where \op{Dequeue} is in \var{root}.\fld{blocks}}}
    \State \Return $\Call{FindResponse}{b, i}$\Comment{\rev{find response \op{Dequeue} should return}}
\EndFunction{CompleteDeq}
}

\spac

\Function{void}{Append}{\typ{Block} \var{B}}  \Comment{append block to leaf and propagate to root}%
    \gc{\State \var{leaf}[\var{id}].\fld{blocks} \assign\ \Call{AddBlock}{\var{leaf}[\var{id}], \var{leaf}[\var{id}].\fld{blocks}, \var{B}}\label{appendLeafGC}}
    \State \Call{Propagate}{\var{leaf}[\var{id}].\fld{parent}} 
\EndFunction{Append}

\spac

\Function{void}{Propagate}{\typ{Node} \var{v}} \Comment{propagate blocks from \var{v}'s children to root}
    \If{{\keywordfont not} \Call{Refresh}{\var{v}}} \label{firstRefreshGC}  \hfill \Comment{double refresh: \rev{if first fails, try again}}
        \State \Call{Refresh}{\var{v}} \label{secondRefreshGC}
    \EndIf
    \If{$\var{v} \neq \var{root}$} \hfill \Comment{recurse up tree}
        \State \Call{Propagate}{\var{v}.\fld{parent}}
    \EndIf
\EndFunction{Propagate}

\spac

\Function{boolean}{Refresh}{\typ{Node} \var{v}} \Comment{try to append a new block $B$ to $\var{v}.\fld{blocks}$}%
	\gc{
	\State $T$ \assign\ $\var{v}.\fld{blocks}$\label{refreshReadTGC}
    \State $\var{h}$ \assign\ $\Call{MaxBlock}{T}.\fld{index}+1$ \label{refreshReadMax}
    }
    \State $\var{B}$ \assign\ $\Call{CreateBlock}{\var{v, h}}$\Comment{\rev{create new block to append to \var{v}.\fld{blocks}}}\label{invokeCreateBlockGC}
    \If{\var{B = \nl}} \Return{\tr}\Comment{\rev{if no new operations to propagate to \var{v}, do nothing}}\label{addOPGC}
    \Else  
    	\gc{
	    \State $T'$ \assign\ $\Call{AddBlock}{v, T, \var{B}}$\Comment{\rev{$T'$ is a copy of $T$ with $B$ added (and GC if needed)}}\label{refreshABGC}
		\State \Return \Call{CAS}{\var{v}.\fld{blocks}, $T$, $T'$}\Comment{\rev{try installing $T'$ as \var{v}'s \fld{blocks} tree}}\label{casGC}%
		}
    \EndIf
\EndFunction{Refresh}

\spac

\gc{
\Function{RBT}{AddBlock}{\typ{Node} \var{v}, \typ{RBT} \var{T}, \typ{Block} \var{B}} \Comment{do GC if necessary; add block $\var{B}\neq\nl$ to \var{T}}
    \If{\var{B}.\fld{index} is a multiple of $G$} \Comment{do garbage collection}\label{GCstart}
        \State $s$ \assign\ \Call{SplitBlock}{\var{v}}.\var{index}\Comment{\rev{find lowest index of block to keep}}\label{callSplitBlock}
        \State \Call{Help}{}\Comment{\rev{help complete pending \op{Dequeues}}}\label{invokeHelp}
        \State $T'$ \assign\ \Call{Split}{$T,s$} \Comment{\op{Split} removes blocks with $\fld{index} < s$}\label{splitGC}
        \State \Return \Call{Insert}{$T',B$}\Comment{\rev{add $B$ to the tree}}\label{GCend}
	\Else\ \Return \Call{Insert}{$T, B$}\Comment{\rev{if GC not needed, just add $B$ to the tree}}\label{noGC-insert}
    \EndIf
\EndFunction{AddBlock}
}

\end{algorithmic}
\caption{Bounded-space queue implementation's main routines for process number $id$.\label{code1GC}}
\end{figure*}

\begin{figure*}
\begin{algorithmic}[1]
\setcounter{ALG@line}{251}
\gc{
\Function{Block}{SplitBlock}{\typ{Node} v} \Comment{\rev{find lowest index of block GC should keep in \var{v}}}
	\If{$\var{v}=\var{root}$} \Comment{return block before the last block containing a}
		\State $m$ \assign\ 0\Comment{dequeued enqueue or null dequeue}
		\For{$k$ \assign\ $1..p$} $m$ \assign\ $\max(m, \var{v}.\fld{last}[k])$
		\EndFor
		\State $B$ \assign\ $\var{root}.\fld{blocks}[m-1]$\label{SBlookup1}
	\Else \Comment{\rev{return last sublock of first block kept in parent}}
		\State $B_p$ \assign\ $\Call{SplitBlock}{\var{v}.\fld{parent}}$\label{SBrecurse}
		\State \var{dir} \assign\ ($\var{v}=\var{v}.\fld{parent.left}$ ? \fld{left} : \fld{right})
		\State $B$ \assign\ $\var{v}.\fld{blocks}[B_p.\edir]$\label{SBlookup2}
	\EndIf
	\State \Return ($B = \nl$ ? $\Call{MinBlock}{\var{v}.\var{blocks}}$ : $B$) \Comment{If $B$ was discarded, keep all blocks}\label{substitute}
\EndFunction{SplitBlock}
}


\spac

\gc{
\Function{void}{Help}{} \Comment{help pending \op{Dequeues}}
    
    \For{$\ell$ in $\var{leaf}[1..k]$}
        \State $\var{B}$ \assign\ $\Call{MaxBlock}{\ell.\fld{blocks}}$
    	%\State \Call{Propagate}{$\ell.\fld{parent}$}   <--- don't do this because recursive helping could be infinite loop
        \If{$\var{B}.\fld{element} = \nl$ and $\var{B}.\fld{index}>0$ and \Call{Propagated}{$\ell, \var{B}.\fld{index}$}} 
            \State \linecomment{operation is a propagated dequeue}
            \State $\var{B}.\fld{response}$ \assign\ $\Call{CompleteDeq}{\ell, \var{B}.\fld{index}}$ \Comment{\rev{compute response and store it in leaf block}}
        \EndIf
    \EndFor
\EndFunction{Help}
}
\spac

\gc{
\Function{boolean}{Propagated}{\typ{Node} \var{v}, \typ{int} \var{b}}  \Comment{check if $\var{v}.\fld{blocks}[b]$ has  propagated to \var{root}}
    \State \linecomment{Precondition:  $\var{v}.\fld{blocks}[b]$ exists}
    \If {$\var{v} = \var{root}$} \Return{true}
    \Else
    	\State $T$ \assign\ $\var{v}.\fld{parent.blocks}$
		\State \fld{dir} \assign\ (\var{v}.\fld{parent.left} = \var{v} ? \fld{left} : \fld{right}) 
		\If{$\Call{MaxBlock}{T}.\edir < b$} \Return \fa\ \Comment{\rev{\var{v}.\fld{blocks}[\var{b}] has not propagated to \var{v}.\fld{parent}}}
		\Else \Comment{\rev{check if superblock of \var{v}.\fld{blocks}[\var{b}] propagated to root}}
			\State $B_p$ \assign\ minimum index block in $T$ with $\edir \geq b$\label{searchsuper3}
			\State \Return \Call{Propagated}{\var{v}.\fld{parent}, $B_p$.\fld{index}}
		\EndIf
	\EndIf
\EndFunction{Propagated}
}

\spac

\Function{Block}{CreateBlock}{\typ{Node} \var{v}, \typ{int} \var{i}} \linecomment{create new block $B$ to install in \var{v}.\fld{blocks}[\var{i}]}
    \State \hangbox{let \var{B} be a new \typ{Block} with fields\label{initNewBlockGC}
    \gc{
    	\eleft\ \assign\ $\Call{MaxBlock}{\var{v}.\fld{left}.\fld{blocks}}.\fld{index}$\label{createEndLeftGC},\\
    \eright\ \assign\ $\Call{MaxBlock}{\var{v}.\fld{right}.\fld{blocks}}.\fld{index}$,\label{createEndRightGC}
    \fld{index} \assign\ $i$\label{createIndexGC}
    }\\
	\fld{sum\sub{enq}} \assign\ \var{v}.\fld{left.blocks}[\var{B}.\eleft].\fld{sum\sub{enq}} + 
			\var{v}.\fld{right.blocks}[\var{B}.\eright].\fld{sum\sub{enq}}\\
	\fld{sum\sub{deq}} \assign\ \var{v}.\fld{left.blocks}[\var{B}.\eleft].\fld{sum\sub{deq}} +
			\var{v}.\fld{right.blocks}[\var{B}.\eright].\fld{sum\sub{deq}}}
    \State \var{num\sub{enq}} \assign\ $\var{B}.\fld{sum\sub{enq}} - \var{v}.\fld{blocks}[\var{i}-1].\fld{sum\sub{enq}}$\label{computeNumEnqGC}
    \State \var{num\sub{deq}} \assign\ $\var{B}.\fld{sum\sub{deq}} - \var{v}.\fld{blocks}[\var{i}-1].\fld{sum\sub{deq}}$
    \If{$\var{num\sub{enq}} + \var{num\sub{deq}} = 0$} \Return \nl\ \Comment{no blocks need to be propagated to \var{v}}\label{testEmptyGC}\label{CBnullGC}
    \ElsIf{$\var{v} = \var{root}$} \var{B}.\fld{size} \assign\ max(0, $\var{v}.\fld{blocks}[\var{i}-1].\size\ + 
        	\var{num\sub{enq}} - \var{num\sub{deq}}$)\label{computeLengthGC}
    \EndIf
    \State\Return \var{B}
\EndFunction{CreateBlock}

\end{algorithmic}
\caption{Bounded-space queue implementation's routines for GC and creating new \typ{Block}.\label{code2GC}}
\end{figure*}

\begin{figure*}
\begin{algorithmic}[1]
\setcounter{ALG@line}{296}
\Function{$\langle\typ{int}, \typ{int}\rangle$}{IndexDequeue}{\typ{Node} \var{v}, \typ{int} \var{b}, \typ{int} \var{i}}
    \State \linecomment{return $\langle b', i' \rangle$ such that \var{i}th dequeue in $D(\var{v}.\fld{blocks}[\var{b}])$ is ($i'$)th dequeue of $D(\var{root}.\fld{blocks}[b'])$}
    \State \linecomment{Precondition: \var{v}.\fld{blocks}[\var{b}] exists and has propagated to root and $|D(\var{v}.\fld{blocks}[\var{b}])|\geq \var{i}$}
    \If{$\var{v} = \var{root}$} \Return $\langle\var{b, i}\rangle$ \label{indexBaseCaseGC}
    \Else \Comment{First, find the superblock $B_p$ of $v.blocks[b]$}
	    \State \fld{dir} \assign\ (\var{v}.\fld{parent.left} = \var{v} ? \fld{left} : \fld{right}) 
    	\gc{
		\State $T$ \assign\ $\var{v}.\fld{parent.blocks}$
		\State $B_p$ \assign\ min block in $T$ with $\edir \geq b$\label{searchsuper1}
		\State $B_p'$ \assign\ max block in $T$ with $\edir < b$ \Comment{predecessor of $B_p$}\label{searchsuper2}}
	    \State \linecomment{compute index \var{i} of dequeue in superblock $B_p$}
	    \State \linecomment{\rev{add \# dequeues in blocks of $v$ before $v.\fld{blocks}[b]$ within superblock $B_p$}}
	    \State \hangbox{\var{i} += $\var{v}.\fld{blocks}[\var{b}-1].\fld{sum\sub{deq}} -
	    		\var{v}.\fld{blocks}[\gc{B_p'}.\edir].\fld{sum\sub{deq}}$}
        \If{$\fld{dir} = \fld{right}$}\Comment{\rev{add \# dequeues in $B_p$ from left sibling}}
        	\State \hangbox{\var{i} += $\var{v}.\fld{blocks}[\gc{B_p}.\eleft].\fld{sum\sub{deq}} - \mbox{ }
					\var{v}.\fld{blocks}[\gc{B_p'}.\eleft].\fld{sum\sub{deq}}$}\label{considerLeftBeforeRightGC}
        \EndIf \label{computeISuperEndGC}
        \State \Return\Call{IndexDequeue}{\var{v}.\fld{parent}, $\gc{B_p.\fld{index}}$, \var{i}} \Comment{\rev{recurse up the tree}}
    \EndIf
\EndFunction{IndexDequeue}

\spac 


\Function{element}{FindResponse}{\typ{int} \var{b}, \typ{int} \var{i}} \linecomment{find response to \var{i}th dequeue in $D(\var{root}.\fld{blocks}[\var{b}])$}
    \State \linecomment{Precondition:  $1\leq i\leq |D(\var{root}.\fld{blocks}[\var{b}])|$}
    \State \var{num\sub{enq}} \assign\ $\var{root}.\fld{blocks}[\var{b}].\fld{sum\sub{enq}} -
    		\var{root}.\fld{blocks}[\var{b}-1].\fld{sum\sub{enq}}$\label{FRnumenqGC}
    \If{$\var{root}.\fld{blocks}[\var{b}-1].\size + \var{num\sub{enq}} < \var{i}$}\label{checkEmptyGC} \Comment{queue is empty when dequeue occurs}%
        \gc{\If{$b > \var{last}[\var{id}]$} $\var{last}[\var{id}]$\ \assign\ $b$
        \EndIf
        }
        \State \Return \nl \hfill \label{returnNullGC}
    \Else \ \Comment{response is the \var{e}th enqueue in the root}
        \State $\var{e}$\ \assign\ $\var{i} + \var{root}.\fld{blocks}[\var{b}-1].\fld{sum\sub{enq}} - 
			\var{root}.\fld{blocks}[\var{b}-1].\size$\label{computeEGC}
		 
		\State use \gc{BST search in \var{root}.\fld{blocks} to find minimum index $b_e$ of a \typ{Block} with $\fld{sum\sub{enq}} \geq \var{e}$}\label{FRsearchGC} 
		\State $i_e$ \assign\ $\var{e} - \var{root}.\fld{blocks}[b_e-1].\fld{sum\sub{enq}}$ \Comment{find rank of enqueue within its block}%
		\gc{
        \State \var{res} \assign\ \Call{GetEnqueue}{\var{root}, $b_e$, $i_e$}\Comment{\rev{find argument of that enqueue}}\label{findAnswerGC}
        \If {$b_e > \var{last}[\var{id}]$} $\var{last}[\var{id}]$ \assign\ $b_e$
        \EndIf
        \State \Return \var{res}
		}
    \EndIf
\EndFunction{FindResponse}

\spac

\Function{element}{GetEnqueue}{\typ{Node} \var{v}, \typ{int} \var{b}, \typ{int} \var{i}} \Comment{returns argument of \var{i}th enqueue in $E(\var{v}.\fld{blocks}[\var{b}])$}
    \State \linecomment{Preconditions: $\var{i}\geq 1$ and \var{v}.\fld{blocks}[\var{b}] exists and contains at least \var{i} enqueues}
    \If{\var{v} is a leaf node} \Return \var{v}.\fld{blocks}[\var{b}].\fld{element} \label{getBaseCaseGC}
    \Else 
        \State \var{sum\sub{left}} \assign\ \var{v}.\fld{left.blocks}[\var{v}.\fld{blocks}[\var{b}].\eleft].\fld{sum\sub{enq}} \Comment{\#\ enqueues in \var{v}.\fld{blocks}[1..$\var{b}$] from \var{v}.\fld{left}}
        \State \var{prev\sub{left}} \assign\ \var{v}.\fld{left.blocks}[\var{v}.\fld{blocks}[$\var{b}-1$].\eleft].\fld{sum\sub{enq}} \Comment{\#\ enqueues in \var{v}.\fld{blocks}[1..$\var{b}-1$] from \var{v}.\fld{left}}
        \State \var{prev\sub{right}} \assign\ \var{v}.\fld{right.blocks}[\var{v}.\fld{blocks}[$\var{b}-1$].\eright].\fld{sum\sub{enq}} \Comment{\#\ enqueues in \var{v}.\fld{blocks}[1..$\var{b}-1$] from \var{v}.\fld{right}}
        \If{$\var{i} \leq \var{sum\sub{left}} - \var{prev\sub{left}}$} \fld{dir} \assign\ \fld{left}\Comment{required enqueue is in \var{v}.\fld{left}}\label{leftOrRightGC} 
        \Else \Comment{required enqueue is in \var{v}.\fld{right}}
            \State \fld{dir} \assign\ \fld{right}
            \State $\var{i}\ \assign\ \var{i} - (\var{sum\sub{left}} - \var{prev\sub{left}})$\Comment{\rev{deduct \# enqueues in $v.\fld{blocks}[b]$ that came from $v.\fld{left}$}}
        \EndIf
        \State \linecomment{find enqueue's block in \var{v}.\fld{dir.blocks} and its rank within block}
        \State \gc{use BST search in \var{v}.\fld{dir.blocks} to find minimum index $\var{b}'$ of a \typ{Block} with $\fld{sum\sub{enq}} \geq \var{i} + \var{prev\sub{dir}}$}\label{getChildGC}
        \State $\var{i}'$ \assign\ $\var{i} - (\var{v}.\fld{dir.blocks}[\var{b}'-1].\fld{sum\sub{enq}} - \var{prev\sub{dir}})$
        \State \Return\Call{GetEnqueue}{\var{v}.\fld{dir}, $\var{b}'$, $\var{i}'$}
    \EndIf
\EndFunction{GetEnqueue}

\end{algorithmic}
\caption{Bounded-space queue implementation's  routines to compute responses to operations.\label{code3GC}}
\end{figure*}




%\begin{algorithm}
%\caption{\typ{Node}}
%\begin{algorithmic}[1]
%\setcounter{ALG@line}{25}
%
%\Statex $\leadsto$ \textsf{Precondition: \var{blocks[start..end]} contains a block with \fld{sum\sub{enq}} greater than or equal to \var{x}}
%\Statex \com\ \textmd{Does a binary search for~the value \var{x} of \fld{sum\sub{enq}} field and returns the index of the leftmost block in\\
%\com\ \var{blocks[start..end]} whose \fld{sum\sub{enq}} is $\geq$ \var{x}}.
%\Function{int}{BinarySearch}{\typ{int} x, \typ{int} start, \typ{int} end}
% \State \Return $\min\{j: \fld{blocks}[j].\fld{sum\sub{enq}} \geq x\}}
%\While{\nf{start<end}}
%\State \typ{int} \var{mid} \assign\ floor((start+end)/2)
%\If{\nf{blocks[mid].sum\sub{enq}<x}}
%\State \nf{start \assign\ mid+1}
%\Else
%\State \nf{end \assign\ mid}
%\EndIf
%\EndWhile
%\State\Return \nf{start}
%\EndFunction{BinarySearch}
%
%\end{algorithmic}
%\end{algorithm}

%\begin{algorithm}
%\caption{\typ{Root}}
%\begin{algorithmic}[1]
%\setcounter{ALG@line}{36}
%\Statex
%\Statex $\leadsto$ \textsf{Precondition: \var{root.blocks[end].sum\sub{enq} $\geq$ \var{e}}}
%\Statex \com\ \textmd{Returns \var{<b,i>} such that $E_\nf{e}(\nf{root})$ is $E_\nf{i}(\nf{root},\nf{b})$, i.e., the \nf{e}th \nf{Enqueue} in the \nf{root} is the \nf{i}th \nf{Enqueue} within \\
%\com\ the \nf{b}th block in the \nf{root}.}
%
%\Function{<int, int>}{DoublingSearch}{\typ{int} e, \typ{int} end}
%\State \var{start \assign\ end-1} \label{dsearchStart}
%\While{\var{root.blocks[start].sum\sub{enq}}$\geq$\var{e}}
%\State \var{start \assign\ max(start-(end-start), 0)} \label{doubling}
%\EndWhile \label{dsearchEnd}
%\State \var{b \assign\ root.BinarySearch(e, start, end)} \label{dsearchBinarySearch}
%\State \var{i \assign\ e- root.blocks[b-1].sum\sub{enq}} \label{DSearchComputei}
%\State\Return \var{<b,i>}
%\EndFunction{DoublingSearch}
%\end{algorithmic}
%\end{algorithm}

The new routines, \op{AddBlock}, \op{SplitBlock}, \op{Help} and \op{Propagated} are
used to implement the garbage collection (GC) phase.
When a \op{Refresh} or \op{Append} wants to add a new block to a node's \fld{blocks} RBT,
it calls the new \opemph{AddBlock} routine.
Before attempting to add the block to a node's RBT, \op{AddBlock} triggers a GC phase on the RBT if 
the new block's \fld{index} is a multiple of the constant $G$, which we choose to be $p^2\ceil{\log p}$.
This ensures that obsolete blocks are removed from the RBT once every $G$ times a new block is added to it.
The GC phase uses \op{SplitBlock} to determine the index $s$ of the oldest block to keep,
calls \op{Help} to help all pending dequeues that have been propagated to the root
(to ensure that all blocks before $s$ can safely be discarded), and then
uses the standard RBT \op{Split} routine \cite{Tar83}
to remove all blocks with \fld{index} less than $s$.
Then \op{AddBlock}
inserts the new block at line \ref{GCend} or \ref{noGC-insert}
and returns the root of the resulting RBT.
The \op{Append} or \op{Refresh} then stores this root into the node's \fld{blocks} field at line 
 \ref{appendLeafGC} or~\ref{casGC}.

To determine the oldest block in a node \var{v} to keep, the \opemph{SplitBlock} routine  first
uses the \var{last} array to
find the most recent block $B_{root}$ in the root that contains either an enqueue that has been dequeued
or a null dequeue.  By the FIFO property of queues, all enqueues in blocks before $B_{root}$ are either
dequeued or will be dequeued by a dequeue that is currently in progress.  Once those pending dequeues have been
helped to complete by line \ref{invokeHelp},
it is safe to discard any blocks in the root older than $B_{root}$, as well
as their subblocks.\footnote{If we used the more conservative approach of discarding blocks whose indices 
are smaller than the \emph{minimum} entry of \var{last} instead of the maximum, helping would be unnecessary, but then one slow process could prevent GC from discarding any blocks, so the space would not be bounded.}
The \op{SplitBlock} uses the \eleft\ and \eright\ fields to find the last block in \var{v} that is a subblock
of $B_{root}$ (or any older block in the root, in case $B_{root}$ has no subblocks in \var{v}).
While \op{SplitBlock} is in progress, it is possible that some block that it needs in a node $\var{v}'$
along the path from \var{v} to the root is discarded by another GC phase.
In this case, \op{SplitBlock} uses the last subblock in \var{v} of the oldest block in $\var{v}'$ instead
(since a GC phase on $\var{v}'$ determined that all blocks older than that are safe to discard anyway).

The \opemph{Help} routine is fairly straightforward:  it loops through all leaves and
helps the dequeue that is in progress there if it has already been propagated to the root.
The \opemph{Propagated} function is used to determine whether the dequeue has propagated to the root.

In the code, we use $\var{v}.\fld{blocks}[i]$ to refer to the block in the RBT stored in $\var{v}.\fld{blocks}$
with index~$i$.
A search for this block may sometimes not find it, if it has already been discarded
by another process's GC phase.
As mentioned above, if this happens to an enqueue operation,
the enqueue can simply terminate because the fact that the block is gone means that another
process has helped the enqueue reach the root of the ordering tree.
Similarly, if  a dequeue operation performs a failed search on a RBT, the dequeue can return the value 
written in the \fld{response} field of the 
leaf block that represents the dequeue and terminate, since some other process
will have written the \fld{response} there before discarding the needed block.
We do not explicitly write this early termination in the pseudocode every time we do a lookup in an RBT.
There is one exception to this rule:  if an RBT lookup for block $B$ returns \nl\
on line \ref{SBlookup1} or \ref{SBlookup2} of
\op{SplitBlock} because the required block has been discarded, 
we continue doing GC, since we do not want a GC phase on one node
to be prevented from cleaning up its RBT because a GC phase on a different node threw away some blocks
that were needed.
Line \ref{substitute} says what to do in this case.
