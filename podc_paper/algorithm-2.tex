% !TEX root =  podc-submission.tex

\section{Detailed Description of the Bounded-Space Implementation}
\label{reducing-details}

Here, we give more details of the bounded-space construction sketched in Section \ref{reducing}.
We first describe the modifications to the implementation.
It uses two additional shared arrays.
\begin{itemize}
\item \typ{Node[]} \var{leaf}$[1..p]$ \Comment{\var{leaf}$[k]$ is the leaf assigned to process $k$}
\item \typ{int[]} \var{last}$[1..p]$ \Comment{\var{last}$[k]$ indexes last block in $root$ that process $k$ has dequeued from}
\end{itemize}
The \fld{blocks} field is implemented as a pointer to the root of a RBT of \typ{Blocks} rather than an infinite array.
Each \typ{Block} has an additional field.
\begin{itemize}
\item \typ{int} \var{index} \Comment{position this block would have in the \fld{blocks} array}
\end{itemize}
To facilitate helping, each \typ{Block} in a leaf has one more additional field.
\begin{itemize}
\item \typ{Object} \var{response} \Comment{response of the operation in the block, if it is a dequeue}
\end{itemize}

\renewcommand{\algorithmiccomment}[1]{\hfill\eqparbox{COMMENTDOUBLE}{\com\ #1}}

\begin{figure}
\begin{minipage}[t]{0.465\textwidth}
\begin{algorithmic}[1]
\setcounter{ALG@line}{0}

\Function{void}{Enqueue}{\typ{Object} \var{e}} 
    \State \hangbox{let \var{n} be a new \typ{Block} with \fld{element} \assign\ \var{e},\\
		\fld{sum\sub{enq}} \assign\ \var{leaf}.\fld{blocks}[\var{leaf}.\head-1].\fld{sum\sub{enq}}+1,\\
		\fld{sum\sub{deq}} \assign\ \var{leaf}.\fld{blocks}[\var{leaf}.\head-1].\fld{sum\sub{deq}}}
    \State \Call{Append}{\var{n}}
\EndFunction{Enqueue}

\spac

\Function{Object}{Dequeue()}{} 
    \State \hangbox{let \var{n} be a new \typ{Block} with \fld{element} \assign\ \nl,\\
	    \fld{sum\sub{enq}} \assign\ \var{leaf}.\fld{blocks}[\var{leaf}.\head-1].\fld{sum\sub{enq}},\\
	    \fld{sum\sub{deq}} \assign\ \var{leaf}.\fld{blocks}[\var{leaf}.\head-1].\fld{sum\sub{deq}}+1}
    \State \Call{Append}{\var{n}}
    \State $\langle \var{b}, \var{i}\rangle$ \assign\ \Call{IndexDequeue}{\var{leaf}, \var{leaf}.\head, 1}
    \State \Return{ \Call{FindResponse}{\var{b, i}}}
    \label{deqRest}
\EndFunction{Dequeue}

\spac

\Function{void}{Append}{\typ{Block} \var{B}} 
    \State \linecomment{append block to leaf and propagate to root}
    \State \var{leaf}.\fld{blocks}[\var{leaf}.\head] \assign\ \var{B}\label{appendLeaf}
    \gc{\State \var{leaf}.\fld{blocks}.\Call{TryAppend}{\var{B}, \var{leaf}.\head}\label{appendLeafGC}}
    \State \var{leaf}.\head\ \assign\ \var{leaf}.\head+1 \label{appendEnd} 
    \State \Call{Propagate}{\var{leaf}.\fld{parent}} 
\EndFunction{Append}

\spac

\Function{void}{Propagate}{\typ{Node} \var{v}}
    \State \linecomment{propagate blocks from \var{v}'s children to root}
    \If{\bf{not} \Call{Refresh}{\var{v}}} \label{firstRefresh}  \hfill \com\ double refresh
        \State \Call{Refresh}{\var{v}} \label{secondRefresh}
    \EndIf
    \If{\var{v} \bf{is not} \var{root}} \hfill \com\ recurse up tree
        \State \Call{Propagate}{\var{v}.\fld{parent}}
    \EndIf
\EndFunction{Propagate}

\spac

\Function{boolean}{Refresh}{\typ{Node} \var{v}}
    \State \linecomment{try to append a new block to \var{v}.\fld{blocks}}
    \State \var{h} \assign\ \var{v}.\head \label{readHead}

\gc{    
	\If{ \var{v} is root and $\var{h} \mbox{ mod } p^2=0$}
	\State \var{max} \assign 0
	\For {$k \assign 1..p$}
		\State $\var{max} \assign \max(\var{max}, last[k])$
	\EndFor 
	\State \Call{Help()}{}
	\State \Call{FreeMemory}{$\var{root}, \var{max}-1$}
	\EndIf
}

    \ForEach{\fld{dir} {\keywordfont{in}} \fld{\{left, right\}}} \label{startHelpChild1}
        \State \var{childHead} \assign\ \var{v}.\fld{dir}.\head \label{readChildHead}
        \If{\var{v}.\fld{dir.blocks}[\var{childHead}] $\neq$ \nl} \label{ifHeadnotNull}
            \State \Call{Advance}{\var{v}.\fld{dir}, \var{childHead}} \label{helpAdvance}
        \EndIf
    \EndFor \label{endHelpChild1}
    \State \var{new} \assign\ \Call{CreateBlock}{\var{v, h}} \label{invokeCreateBlock}
    \If{\var{new = \nl}} \Return{\tr} \label{addOP} 
    \EndIf
    \State \var{result} \assign\ \Call{CAS}{\var{v}.\fld{blocks}[\var{h}], \nl, \var{new}} \label{cas}
    \gc{\State \var{result} \assign\ \var{v}.\fld{blocks}.\Call{TryAppend}{\var{new}, [\var{h}]} \label{GCcas}}
    \State \var{\Call{Advance}{\var{v, h}}}\label{advance}
    \State \Return{ \var{result}}
\EndFunction{Refresh}

\spac

\Function{void}{Advance}{\typ{Node} \var{v}, \typ{int} \var{h}} 
    \State \linecomment{set \var{v}.\fld{blocks}[\var{h}].\fld{super} and }
    \State \linecomment{increment \var{v}.\fld{head} from \var{h} to \var{h}+1}
    \State \var{h\sub{p}} \assign\ \var{v}.\fld{parent}.\head \label{readParentHead}
    \State \Call{CAS}{\var{v}.\fld{blocks}[\var{h}].\fld{super}, \nl, \var{h\sub{p}}} \label{setSuper1}
    \State \Call{CAS}{\var{v}.\head, \var{h}, \var{h}+1} \label{incrementHead}
\EndFunction{Advance}

\end{algorithmic}
\end{minipage}
\begin{minipage}[t]{0.529\textwidth}

\begin{algorithmic}[1]
\setcounter{ALG@line}{48}

\Function{Block}{CreateBlock}{\typ{Node} \var{v}, \typ{int} \var{i}} 
    \State\linecomment{create new block for a \op{Refresh} to install in \var{v}.\fld{blocks}[\var{i}]}
    \State let \var{new} be a new \typ{Block} \label{initNewBlock}
    \State \var{new}.\eleft \assign\ $\var{v}.\fld{left}.\head - 1$\label{createEndLeft}
    \State \var{new}.\eright \assign\ $\var{v}.\fld{right}.\head - 1$
	\State \hangbox{\var{new}.\fld{sum\sub{enq}} \assign\ \var{v}.\fld{left.blocks}[\var{new}.\eleft].\fld{sum\sub{enq}} + \\
			\var{v}.\fld{right.blocks}[\var{new}.\eright].\fld{sum\sub{enq}}}
	\State \hangbox{\var{new}.\fld{sum\sub{deq}} \assign\ \var{v}.\fld{left.blocks}[\var{new}.\eleft].\fld{sum\sub{deq}} + \\
			\var{v}.\fld{right.blocks}[\var{new}.\eright].\fld{sum\sub{deq}}}
    \State \var{num\sub{enq}} \assign\ $\var{new}.\fld{sum\sub{enq}} - \var{v}.\fld{blocks}[\var{i}-1].\fld{sum\sub{enq}}$\label{computeNumEnq}
    \State \var{num\sub{deq}} \assign\ $\var{new}.\fld{sum\sub{deq}} - \var{v}.\fld{blocks}[\var{i}-1].\fld{sum\sub{deq}}$
    \If{$\var{v} = \var{root}$}
        \State \hangbox{\var{new}.\fld{size} \assign\ max(0, $\var{v}.\fld{blocks}[\var{i}-1].\size\ + \var{num\sub{enq}} - \var{num\sub{deq}}$)}\label{computeLength}
    \EndIf
    \If{$\var{num\sub{enq}} + \var{num\sub{deq}} = 0$}
        \State \Return \nl \hfill \com\ no blocks need to be propagated to \var{v}
    \Else
        \State \Return \var{new}
    \EndIf
\EndFunction{CreateBlock}

\spac

\Function{$\langle\typ{int}, \typ{int}\rangle$}{IndexDequeue}{\typ{Node} \var{v}, \typ{int} \var{b}, \typ{int} \var{i}}
    \State \linecomment{return $\langle\var{x, y}\rangle$ such that \var{i}th dequeue in $D(\var{v}.\fld{blocks}[\var{b}])$}
    \State \linecomment{is \var{y}th dequeue of $D(\var{root}.\fld{blocks}[\var{x}])$}
    \State \linecomment{Precondition: \var{v}.\fld{blocks}[\var{b}] has been propagated to root}
    \State \linecomment{and $D(\var{v}.\fld{blocks}[\var{b}])$ has at least \var{i} dequeues}
    \If{$\var{v} = \var{root}$}
        \State\Return $\langle\var{b, i}\rangle$ \label{indexBaseCase}
    \Else
	    \State \fld{dir} \assign\ (\var{v}.\fld{parent.left} = \var{v} ? \fld{left} : \fld{right}) 
    	\State \var{sup} \assign\ \var{v}.\fld{blocks}[\var{b}].\fld{super}
	    \If{$\var{b} > \var{v}.\fld{parent.blocks}[\var{sup}].\fld{end\sub{dir}}$}
	        \State{\var{sup} \assign\ $\var{sup}+1$}
	    \EndIf
	    \State \linecomment{compute index \var{i} of dequeue in superblock}
	    \State \hangbox{\var{i} += $\var{v}.\fld{blocks}[\var{b}-1].\fld{sum\sub{deq}} -$ \\
	    		$\var{v}.\fld{blocks}[\var{v}.\fld{parent.blocks}[\var{sup}-1].\edir].\fld{sum\sub{deq}}$}
        \If{$\fld{dir} = \fld{right}$} 
        	\State \hangbox{\var{i} += $\var{v}.\fld{blocks}[\var{v}.\fld{parent.blocks}[\var{sup}].\eleft].\fld{sum\sub{deq}} - \mbox{ }$\\
					$\var{v}.\fld{blocks}[\var{v}.\fld{parent.blocks}[\var{sup}-1].\eleft].\fld{sum\sub{deq}}$}\label{considerLeftBeforeRight}
        \EndIf \label{computeISuperEnd}
        \State \Return\Call{IndexDequeue}{\var{v}.\fld{parent}, \var{sup}, \var{i}}
    \EndIf
\EndFunction{IndexDequeue}

\spac

\Function{element}{FindResponse}{\typ{int} \var{b}, \typ{int} \var{i}}
    \State \linecomment{find response to \var{i}th dequeue in \var{root}.\fld{blocks}[\var{b}]}
    \State \hangbox{\var{num\sub{enq}} \assign\ $\var{root}.\fld{blocks}[\var{b}].\fld{sum\sub{enq}} - \mbox{ }$\\
    		$\var{root}.\fld{blocks}[\var{b}-1].\fld{sum\sub{enq}}$}
    \If{$\var{root}.\fld{blocks}[\var{b}-1].\size + \var{num\sub{enq}} < \var{i}$}\label{checkEmpty}
    \gc{\State \var{last[this process id]} \assign\ \var{b}}
        \State \Return \nl \hfill \com\ queue is empty when dequeue occurs\label{returnNull}
            \EndIf
\State \linecomment{response is the \var{e}th enqueue in the root}
        \State \var{e} \assign\ \var{i} + \var{root}.\fld{blocks}[\var{b}-1].\fld{sum\sub{enq}} - 
			\var{root}.\fld{blocks}[\var{b}-1].\size\label{computeE}
		\State \linecomment{compute enqueue's block using binary search}
		\State find min $\var{b}'\leq \var{b}$ with $\var{root}.\fld{blocks}[\var{b}'].\fld{sum\sub{enq}} \geq \var{e}$
		\State \linecomment{find rank of enqueue within its block}
		\State $\var{i}' \assign\ \var{e} - \var{root}.\fld{blocks}[\var{b}'-1].\fld{sum\sub{enq}}$
        \State \var{e} \assign\ \Call{GetEnqueue}{\var{root}, $\var{b}'$, $\var{i}'$}\label{findAnswer}
        \gc{\State \var{last[this process id]} \assign\ $\var{b}'$}
        \State \Return{\var{e}}

\EndFunction{FindResponse}


\end{algorithmic}
\end{minipage}
\caption{Queue implementation.\label{pseudocode1}}
\end{figure}



%\begin{algorithm}
%\caption{\tt{\sl{Node}}}
%\begin{algorithmic}[1]
%\setcounter{ALG@line}{25}
%
%\Statex $\leadsto$ \textsf{Precondition: \tt{blocks[start..end]} contains a block with \tt{sum\sub{enq}} greater than or equal to \tt{x}}
%\Statex \com\ \textmd{Does a binary search for~the value \tt{x} of \tt{sum\sub{enq}} field and returns the index of the leftmost block in\\
%\com\ \tt{blocks[start..end]} whose \tt{sum\sub{enq}} is $\geq$ \tt{x}}.
%\Function{int}{BinarySearch}{\sl{int} x, \sl{int} start, \sl{int} end}
% \State \Return \tt{min\{j: blocks[j].sum\sub{enq}$\geq$x\}}
%\While{\nf{start<end}}
%\State \tt{\sl{int}} \tt{mid \assign\ floor((start+end)/2)}
%\If{\nf{blocks[mid].sum\sub{enq}<x}}
%\State \nf{start \assign\ mid+1}
%\Else
%\State \nf{end \assign\ mid}
%\EndIf
%\EndWhile
%\State\Return \nf{start}
%\EndFunction{BinarySearch}
%
%\end{algorithmic}
%\end{algorithm}

%\begin{algorithm}
%\caption{\tt{\sl{Root}}}
%\begin{algorithmic}[1]
%\setcounter{ALG@line}{36}
%\Statex
%\Statex $\leadsto$ \textsf{Precondition: \tt{root.blocks[end].sum\sub{enq} $\geq$ \tt{e}}}
%\Statex \com\ \textmd{Returns \tt{<b,i>} such that $E_\nf{e}(\nf{root})$ is $E_\nf{i}(\nf{root},\nf{b})$, i.e., the \nf{e}th \nf{Enqueue} in the \nf{root} is the \nf{i}th \nf{Enqueue} within \\
%\com\ the \nf{b}th block in the \nf{root}.}
%
%\Function{<int, int>}{DoublingSearch}{\sl{int} e, \sl{int} end}
%\State \tt{start \assign\ end-1} \label{dsearchStart}
%\While{\tt{root.blocks[start].sum\sub{enq}}$\geq$\tt{e}}
%\State \tt{start \assign\ max(start-(end-start), 0)} \label{doubling}
%\EndWhile \label{dsearchEnd}
%\State \tt{b \assign\ root.BinarySearch(e, start, end)} \label{dsearchBinarySearch}
%\State \tt{i \assign\ e- root.blocks[b-1].sum\sub{enq}} \label{DSearchComputei}
%\State\Return \tt{<b,i>}
%\EndFunction{DoublingSearch}
%\end{algorithmic}
%\end{algorithm}


\begin{figure}
\begin{algorithmic}[1]
\setcounter{ALG@line}{101}

\Function{element}{GetEnqueue}{\typ{Node} \var{v}, \typ{int} \var{b}, \typ{int} \var{i}} \Comment{returns argument of \var{i}th enqueue in $E(\var{v}.\fld{blocks}[\var{b}])$}
    \State \linecomment{Precondition: $\var{i}\geq 1$ and $E(\var{v}.\fld{blocks}[\var{b}])$ contains at least \var{i} enqueues}

    \If{\var{v} is a leaf node}
        \State\Return \var{v}.\fld{blocks}[\var{b}].\fld{element} \label{getBaseCase}
    \Else 
        \State \var{sum\sub{left}} \assign\ \var{v}.\fld{left.blocks}[\var{v}.\fld{blocks}[\var{b}].\eleft].\fld{sum\sub{enq}} \Comment{\#\ of enqueues in \var{v}.\fld{blocks}[1..$\var{b}$] from left child}
        \State \var{prev\sub{left}} \assign\ \var{v}.\fld{left.blocks}[\var{v}.\fld{blocks}[$\var{b}-1$].\eleft].\fld{sum\sub{enq}} \Comment{\#\ of enqueues in \var{v}.\fld{blocks}[1..$\var{b}-1$] from left child}
        \State \var{prev\sub{right}} \assign\ \var{v}.\fld{right.blocks}[\var{v}.\fld{blocks}[$\var{b}-1$].\eright].\fld{sum\sub{enq}} \Comment{\#\ of enqueues in \var{v}.\fld{blocks}[1..$\var{b}-1$] from right child}
        \If{$\var{i} \leq \var{sum\sub{left}} - \var{prev\sub{left}}$} \label{leftOrRight} \cmt{required enqueue is in \var{v}.\fld{left}}
            \State \fld{dir} \assign\ \fld{left}
        \Else
            \State \fld{dir} \assign\ \fld{right}
            \State $\var{i}\ \assign\ \var{i} - (\var{sum\sub{left}} - \var{prev\sub{left}})$
        \EndIf
        \State \linecomment{Use binary search to find enqueue's block in \var{v}.\fld{dir.blocks} and its rank within block}
        \State find minimum $\var{b}'$ in range [\var{v}.\fld{blocks}[\var{b}-1].\edir+1..\var{v}.\fld{blocks}[\var{b}].\edir] s.t. $\var{v}.\fld{dir.blocks}[\var{b}'].\fld{sum\sub{enq}} \geq \var{i} + \var{prev\sub{dir}}$\label{getChild}
        \State $\var{i}'$ \assign\ $\var{i} - (\var{v}.\fld{dir.blocks}[\var{b}'-1].\fld{sum\sub{enq}} - \var{prev\sub{dir}})$
        \State \Return\Call{GetEnqueue}{\var{v}.\fld{dir}, $\var{b}'$, $\var{i}'$} 
    \EndIf
\EndFunction{GetEnqueue}

\gc{
\Function{void}{Help}{}
\State \bf{for} \var{i} \assign\ 1 to $p$
\State h=leafOfProcess[\var{i}]
\If{leafOfProcess[\var{i}].blocks[h].num\sub{deq}==1 and leafOfProcess[\var{i}].IndexDequeue(h,1)!=null}
\State <b, i> \assign\ \Call{IndexDequeue}{leafOfProcess[\var{i}], h, 1}
\State output \assign\ \Call{FindResponse}{b, i} 
\State leafOfProcess[\var{i}].blocks[h].response \assign\ output
\EndIf
\State \bf{end for}
\EndFunction{Help}

\Function{}{FreeMemory}{\typ{Node} \var{n}, \typ{int} \var{i}}
\State Split \var{n}.blocks.split over \var{i} and assign the grater tree to \var{n}.blocks with \tt{CAS}.
\If{\var{n} is not a leaf}
\State \Call{FreeMemory}{\var{n}.left, n.blocks[i].end\sub{left}}
\State \Call{FreeMemory}{\var{n}.left, n.blocks[i].end\sub{right}}
\EndIf
\EndFunction{FreeMemory}

}
\end{algorithmic}
\caption{\label{pseudocode2}Queue implementation, continued. \here{Would this routine be simpler to write if i parameter was the rank within whole node rather than within the block?}}
\end{figure}

If we garbage collect the unnecessary blocks in the nodes every $p^2$ block appended to the root, the garbage collection cost is amortized over $p^2$ blocks which is $O(p^3)$ and $\Omega(p^2)$ operations.

In our design a process attempts to collect the garbage, when it is going to to append a block in the $kp^2$ position in \nf{root.blocks} (see Line \ref{}). Every $p^2$  block appended to the root, one \nf{GarbageCollect} terminates because \nf{root.head} cannot advance until a \nf{GarbageCollect} garbage collect is done (see Lines ).

%\begin{lemma}
%Size of \nf{root.blocks} after \nf{GarbageCollect} is $O(p^2+q)$.
%\end{lemma}
%
%\begin{lemma}
%Total number of the blocks in the tree is $O(p^3+pq)$.
%\end{lemma}

\nf{Enqueue} operation $e$ can be removed from the tree after termination of \nf{Dequeue} $d$ where $Resp(d)=e$. It is safe to remove \nf{Dequeue} $d$ from the tree after the $d$ terminates. We can remove a block after the conditions told are satisfied for all of its \nf{Enqueue}s and \nf{Dequeue}s. A \nf{Dequeue} may go to sleep for a long time and prevent a \nf{block} to be removed (the block it is in or the block its response is in). In that situation other processes can help the \nf{Dequeue} by computing its response and writing it down somewhere. After writing down the response of th \nf{Dequeue} can be removed, since if the process wanted the response it can read from the helped response.

\begin{definition}
A block  is \it{finished} if all of its \nf{Enqueue}s have been computed to be the response of some \nf{Dequeue} and all of its \nf{Dequeues} responses have been computed.  
\end{definition}

\begin{corollary}
If a block is finished, then all of its subblocks are also finished.  
\end{corollary}

\begin{lemma}
It is safe to remove a finished block from the non-leaf nodes.
\end{lemma}

If the $i$th \nf{Enqueue} gets dequeued in a FIFO queue, it means the first $i-1$ \nf{Enqueue} operations have been already dequeued. This gives us the idea that if a block is finished then all the blocks before it are also finished. If an operation in a block goes to sleep for a long time then other processes help the operation so the block gets finished. There are less than $p$ idle operations, we can help them before garbage collection and then remove all the finished blocks safely in an amortized poly-log time.

\begin{lemma}
    If all \nf{Dequeue} operations in the root are helped, then all the blocks before a finished block in the root are also finished.
\end{lemma}

\begin{lemma}
  Blocks before the most recent block that has been dequeued(computed to be dequeue) from are finished. If the most current \nf{Dequeue} returned(computed) \nf{null} then all the blocks before the block containing the \nf{Dequeue} are finished.
\end{lemma}

The idea above leads us to a poly-log data structure that supports throwing away all the blocks with keys smaller than an index. Red-black trees do this for us. \nf{Get(i)}, \nf{Append()} and \nf{Split(i)} are logarithmic in block trees.
We can create a shared red-black tree just creating a new path for the operation and then using \nf{CAS} to change the root of the tree. See [this] for more.

\begin{observation}
PBRT supports poly-log operations ....
\end{observation}

\begin{lemma}
    If we replace the arrays we used to implement \nf{blocks} with red-black trees the amortized complexity of the algorithm would be $PolyLog(p,q)$. And also the algorithm is correct.
\end{lemma}

We can help a \nf{Dequeue} by computing its response and writing it down. If the process in future failed to execute, it can read the helped value written down.

\begin{lemma}
The \nf{response} written is correct.
\end{lemma}

But how can we know which blocks in each node are finished or not? 

\begin{observation}
Every $p^2$ block appended to the root, \nf{FreeMemory} is invoked.  
\end{observation}

To know the last block dequeied from we can implement a shared array among processes which they write the last root block they have dequeued from. 

\begin{lemma}
    $\nf{Max(Last)} - \text{index of the last finished block}$ in the node $n$ is $O(p)$.
\end{lemma}

\begin{lemma}
    After \nf{FreeMemory}, the space taken by each node of the tree is $O(p+q)$.
\end{lemma}

\begin{corollary}
The space taken by each node of the tree is $O(p^2+q)$. The total space in the tree is $PolyLog(p+q)$.
\end{corollary}

\begin{lemma}
  The amortized step per process for the algorithm with garbage collection is $PolyLog(p+q)$.
\end{lemma}

\begin{lemma}
  Algorithm is wait-free and linearizable.
\end{lemma}

