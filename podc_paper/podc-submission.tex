%%
%% This is file generated with the docstrip utility from the source file: 
%%
%% samples.dtx  (with options: `acmsmall-conf')
%% With some modifications for PODC 2023.
%% For the copyright see the source file.
%%
%%
%% Commands for TeXCount
%TC:macro \cite [option:text,text]
%TC:macro \citep [option:text,text]
%TC:macro \citet [option:text,text]
%TC:envir table 0 1
%TC:envir table* 0 1
%TC:envir tabular [ignore] word
%TC:envir displaymath 0 word
%TC:envir math 0 word
%TC:envir comment 0 0
%%
%%
%% The first command in your LaTeX source must be the \documentclass
%% command.
%%
%% For submission and review of your manuscript please change the
%% command to \documentclass[manuscript, screen, review]{acmart}.
%%
%% When submitting camera ready or to TAPS, please change the command
%% to \documentclass[sigconf]{acmart} or whichever template is required
%% for your publication.
%%
%%
\documentclass[acmsmall,nonacm,anonymous]{acmart}

\usepackage{tikz-qtree}
\usepackage{algorithm}
\usepackage{algpseudocode}
\makeatletter
\renewcommand{\ALG@beginalgorithmic}{\footnotesize}
\makeatother
\usepackage{graphicx}
\usepackage{subcaption}
\usepackage{hyperref}
\usepackage{amsmath}
\usepackage{relsize}
\usepackage{enumitem}
\usepackage{bold-extra}
\usepackage{showkeys}
\renewcommand*\contentsname{Table of Contents}

\usepackage{multicol}
\setlength\columnsep{24pt}

\def\draft{0}  % set to 1 for draft (include comments, color edits), 0 for clean final copy (omit comments, edits in black)
\if\draft 1
\newcommand{\mycomment}[3]{{\color{#2}{[\bf{#1: #3}]}}}
\newcommand{\myedit}[2]{{\color{#1}{#2}}\normalcolor}
\newcommand{\rebuttaledit}[2]{{\color{#1}{#2}}\normalcolor}
\newcommand{\here}[1]{\bf{[[[#1]]]}}
%\usepackage{showkeys}
\else
\newcommand{\mycomment}[3]{}
\newcommand{\rebuttaledit}[2]{#2}
\newcommand{\myedit}[2]{#2}
\newcommand{\here}[1]{}  % \here's include comments to ourselves that readers should not see.
\fi

\newcommand{\Eric}[1]{\mycomment{Eric}{magenta}{#1}}
\newcommand{\Hossein}[1]{\mycomment{Hossein}{blue}{#1}}
\newcommand{\er}[1]{\myedit{magenta}{#1}}

\algnewcommand\algorithmicforeach{\bf{for each}}
\algdef{S}[FOR]{ForEach}[1]{\algorithmicforeach\ #1\ \algorithmicdo}

\algdef{S}[FUNCTION]{Function}
   [3]{{\tt{\sl{#1}}} {\tt{#2}}\ifthenelse{\equal{#3}{}}{}{\tt{(#3)}}}
  
\algdef{E}[FUNCTION]{EndFunction}
   [1]{\algorithmicend\ \tt{{#1}}}

\algrenewcommand\Call[2]{\tt{{#1}\ifthenelse{\equal{#2}{}}{}{(#2)}}}
\usepackage{eqparbox}
\renewcommand{\algorithmiccomment}[1]{\hfill\eqparbox{COMMENT}{\com\ #1}}

\newcommand\keywordfont{\sffamily\bfseries}
\algrenewcommand\algorithmicend{{\keywordfont end}}
\algrenewcommand\algorithmicfor{{\keywordfont for}}
\algrenewcommand\algorithmicforeach{{\keywordfont for each}}
\algrenewcommand\algorithmicdo{{\keywordfont do}}
\algrenewcommand\algorithmicuntil{{\keywordfont until}}
\algrenewcommand\algorithmicfunction{{\keywordfont function}}
\algrenewcommand\algorithmicif{{\keywordfont if}}
\algrenewcommand\algorithmicthen{{\keywordfont then}}
\algrenewcommand\algorithmicelse{{\keywordfont else}}
\algrenewcommand\algorithmicreturn{{\keywordfont return}}

\newcommand{\eleft}{\var{end\sub{left}}}
\newcommand{\eright}{\var{end\sub{right}}}
\newcommand{\edir}{\var{end\sub{dir}}}


\renewcommand\thealgorithm{}
\newcommand{\setalglineno}[1]{%
  \setcounter{ALC@line}{\numexpr#1-1}}

\newcommand{\sub}[1]{\textsubscript{#1}}
\renewcommand{\tt}[1]{\texttt{#1}}
\renewcommand{\sl}[1]{\textsl{#1}}
\renewcommand{\it}[1]{\textit{#1}}
\renewcommand{\sc}[1]{\textsc{#1}}
\renewcommand{\bf}[1]{\textbf{#1}}
\newcommand{\nf}[1]{{\normalfont{\texttt{#1}}}}
\newcommand{\cmt}[1]{\Comment{#1}}
\newcommand{\head}{head}
\newcommand{\size}{size }

\newcommand{\ceil}[1]{\lceil #1 \rceil}
\newcommand{\nil}{nil}
\newcommand{\opa}[2]{\nf{#1(#2)}}
\newcommand{\op}[1]{\nf{#1}}
\newcommand{\var}[1]{\nf{#1}}
\newcommand{\com}{$\triangleright$}

\usepackage{amsmath,amsthm}
%\newtheorem{theorem}{Theorem}
%\newtheorem{lemma}[theorem]{Lemma}
%\newtheorem{corollary}[theorem]{Corollary}
\newtheorem{observation}{Observation}
%\theoremstyle{definition}
%\newtheorem{definition}[theorem]{Definition}
\newtheorem{invariant}{Invariant}
%\newtheorem{proposition}[theorem]{Proposition}

%%
%% \BibTeX command to typeset BibTeX logo in the docs
\AtBeginDocument{%
  \providecommand\BibTeX{{%
    Bib\TeX}}}

%% Rights management information.  This information is sent to you
%% when you complete the rights form.  These commands have SAMPLE
%% values in them; it is your responsibility as an author to replace
%% the commands and values with those provided to you when you
%% complete the rights form.
\setcopyright{none}  %FIX FOR CAMERA READY
%\copyrightyear{2023}
%\acmYear{2023}
\acmDOI{XXXXXXX.XXXXXXX}
\fancyfoot{} % REMOVE THIS LINE FOR CAMERA READY

%% These commands are for a PROCEEDINGS abstract or paper.
\acmConference[PODC 2023]{ACM Symposium on Principles of Distributed Computing}{June 19--23,
  2023}{Orlando, FL}
%%
%%  Uncomment \acmBooktitle if the title of the proceedings is different
%%  from ``Proceedings of ...''!
%%
%%\acmBooktitle{Woodstock '18: ACM Symposium on Neural Gaze Detection,
%%  June 03--05, 2018, Woodstock, NY}
%\acmPrice{15.00}
%\acmISBN{978-1-4503-XXXX-X/18/06}


%%
%% Submission ID.
%% Use this when submitting an article to a sponsored event. You'll
%% receive a unique submission ID from the organizers
%% of the event, and this ID should be used as the parameter to this command.
%%\acmSubmissionID{123-A56-BU3}

%%
%% For managing citations, it is recommended to use bibliography
%% files in BibTeX format.
%%
%% You can then either use BibTeX with the ACM-Reference-Format style,
%% or BibLaTeX with the acmnumeric or acmauthoryear sytles, that include
%% support for advanced citation of software artefact from the
%% biblatex-software package, also separately available on CTAN.
%%
%% Look at the sample-*-biblatex.tex files for templates showcasing
%% the biblatex styles.
%%

%%
%% The majority of ACM publications use numbered citations and
%% references.  The command \citestyle{authoryear} switches to the
%% "author year" style.
%%

%%
%% end of the preamble, start of the body of the document source.
\begin{document}

\title{A Wait-free Queue with Polylogarithmic Step Complexity}


%%
%% The "author" command and its associated commands are used to define
%% the authors and their affiliations.
%% Of note is the shared affiliation of the first two authors, and the
%% "authornote" and "authornotemark" commands
%% used to denote shared contribution to the research.
\author{Eric Ruppert}
\affiliation{%
  \institution{York University}
  \streetaddress{P.O. Box 1212}
  \city{Toronto}
  \state{Ontario}
  \country{Canada}
  \postcode{43017-6221}
}
\email{ruppert@cse.yorku.ca}
\orcid{1234-5678-9012}

\author{Hossein Naderibeni}
\email{hnaderi268@gmail.com}
\affiliation{%
  \institution{York University}
  \streetaddress{P.O. Box 1212}
  \city{Toronto}
  \state{Ontario}
  \country{Canada}
  \postcode{43017-6221}
}

\settopmatter{printacmref=false}  % REMOVE THIS LINE FOR CAMERA READY

\begin{abstract}
We introduce a novel linearizable wait-free queue implementation using single-word 
CAS instructions.
Previous lock-free queues all take $\Omega(p)$ steps per operation in the worst case, 
where $p$ is the number of processes that can access the queue.
We achieve $O(\log^2 p +\log q)$ steps per operation, where $q$ is the size of the queue.
\here{mention space usage; update time bound if space-efficient version takes more time per op}
\end{abstract}

%%
%% The code below is generated by the tool at http://dl.acm.org/ccs.cfm.
%% Please copy and paste the code instead of the example below.
%%

%%
%% Keywords. The author(s) should pick words that accurately describe
%% the work being presented. Separate the keywords with commas.
\keywords{concurrent data structures, wait-free queues}

%%
%% This command processes the author and affiliation and title
%% information and builds the first part of the formatted document.
\maketitle

% !TEX root =  podc-submission.tex

\section{Introduction}

There has been a great deal of research in the past several decades on the design of linearizable, lock-free queues.
Besides being a fundamental data structure, queues are used in
significant concurrent applications, including OS kernels \cite{MP91}, memory management (e.g., \cite{BBFRSW21} \here{try to find an older, more canonical reference}),
packet processing \cite{DPDK},
and sharing resources or tasks.
The lock-free MS-queue of Michael and Scott \cite{MS98} is a classic shared queue implementation.
It uses a singly-linked list with pointers to the front and back nodes.
To dequeue or enqueue an element, the front or back pointer is updated by a 
compare-and-swap (CAS) instruction.
If this CAS fails, the operation must retry.
In the worst case, this means that each successful CAS may cause all other processes to
fail and retry, leading to an amortized step complexity of $\Omega(p)$ per operation.
(To measure amortized step complexity of a lock-free implementation, we consider all possible finite executions
and divide the number of steps in the execution by the number of operations initated in the execution.)
Numerous papers have suggested modifications to the MS-queue \cite{DBLP:conf/opodis/HoffmanSS07,DBLP:conf/podc/KoganH14,DBLP:conf/ppopp/KoganP11,DBLP:journals/dc/Ladan-MozesS08,MKLLP22,DBLP:conf/spaa/MoirNSS05,RC17}, but 
all still have $\Omega(p)$ amortized step complexity as a result of
contention on the front and back of the queue.
Morrison and Afek \cite{DBLP:conf/ppopp/MorrisonA13} called this the \emph{CAS retry problem}.
The same problem occurs in array-based implementations of queues \cite{DBLP:conf/iceccs/ColvinG05,DBLP:conf/icdcn/Shafiei09,DBLP:conf/spaa/TsigasZ01,DBLP:conf/opodis/GidenstamST10}.
Solutions that tried to sidestep this problem using fetch\&increment \cite{DBLP:conf/ppopp/MorrisonA13,DBLP:conf/ppopp/YangM16,Nik19,10.1145/3490148.3538572}
rely on slower mechanisms to handle worst-case executions and still have $\Omega(p)$ step complexity.

Many concurrent data structures that keep track of a set of elements also have an $\Omega(p)$ term in their step complexity, as observed by Ruppert \cite{Rup16}.
For example, lock-free lists \cite{FR04,Sha15}, stacks \cite{Tre86} and search trees \cite{EFHR14} 
have an $\Omega(c)$ term in their step complexity, where $c$ represents contention,
the number of processes that access the data structure concurrently, which can be $p$ in the worst case.
Attiya and Fouren \cite{DBLP:conf/opodis/AttiyaF17} proved 
that amortized $\Omega(c)$ steps per operation are indeed necessary
for any CAS-based implementation of a lock-free bag data structure, which provide operations
to insert an element or remove an arbitrary element (chosen non-deterministically).
Since a queue trivially implements a bag, this lower bound also holds for queues.
At first glance, this would seem to settle the question of the step complexity of lock-free queues.
However, the lower bound leaves a loophole:  it holds only if $c$ is $O(\log\log p)$.
Thus, the lower bound could be stated more precisely as an amortized bound of $\Omega(\min(c,\log\log p))$ steps per operation.

We exploit this loophole.  We show that it is, in fact, possible to implement a linearizable, wait-free queue
whose step complexity does not depend linearly on $p$.
Our implementation is the first with step complexity that is polylogarithmic in $p$ and in $q$, the number of elements in the queue.
For ease of presentation, we first give an unbounded-space construction that uses
$O(\log^2 p + \log q)$ steps per operation.
Then, we show how to modify the implementation to use 
\here{$O(?)$ space} and \here{$O(?)$ steps per operation}.
Both implementations use single-word CAS and reasonably-sized words.
Moreover, they use $O(\log p)$ CAS instructions per operation in the worst case, whereas previous
lock-free queues use 
%an unbounded number of CAS instructions in the worst case and 
$\Omega(p)$ CAS instructions, even in an amortized sense.
For the space-bounded implementation, we unlink unneeded memory from our data structure.
We do not address the orthogonal problem of reclaiming unlinked memory; we assume a safe
garbage collector, such as the highly optimized one that Java provides.

Our queue uses a tournament tree where each process has its own leaf.\here{Is "tournament tree" the right word here?  Maybe call it an ordering tree?}
Each process propagates its operations along the path from its leaf up to the root.
Operations in the root are ordered, 
and this order is used to linearize the operations and compute their responses.
\here{Either here or in related work section, talk about previous usage of tournament tree and how ours differs from it}
Helping  ensures wait-freedom and avoids the CAS retry problem:  a process propagating its operation 
to a node $v$ helps propagate all operations from both of $v$'s children.
Explicitly storing all operations in the nodes would be too costly.
Instead, we use a novel implicit representation of sets
of operations that allows us to quickly merge two sets from the children of a node,
and quickly access any  operation in a~set.
\here{Maybe say a little more about techniques used in the implementation, if space permits}

% !TEX root =  podc-submission.tex

\section{Related Work}

Michael and Scott \cite{MS98} describe the early history of concurrent queue algorithms
in the paper that described their MS-queue, which
is a lock-free queue that has stood the test of time.
A version of it is
included in the standard Java Concurrency Package.  %java.util.concurrent.ConcurrentLinkedQueue
However, as mentioned above, it suffers from the CAS retry problem
which means it is lock-free but not wait-free and has an amortized step complexity
of $\Theta(p)$ per operation.

A number of papers have described ways to reduce contention at the front
and back of the MS-queue.
Moir et al.~\cite{DBLP:conf/spaa/MoirNSS05} 
added an elimination array that allows an enqueue to pass its enqueued value directly
to a concurrent dequeue when the queue is empty or when concurrent operations 
can be linearized to empty the queue.
However, when there are $p$ concurrent enqueues (and no dequeues), the CAS retry problem
is still present.
The baskets queue of
Hoffman, Shalev, and Shavit~\cite{DBLP:conf/opodis/HoffmanSS07} 
is another approach to avoid contention:  concurrent enqueues are grouped into baskets.
If an enqueue fails its CAS, it is put in the basket with the operation that succeeded,
and operations within a basket can be ordered arbitrarily.
However, if there are $p$ concurrent enqueues in the same basket
the CAS retry problem occurs when they order themselves into a list using CAS instructions.
Both modifications thus still have $\Omega(p)$ amortized step complexity.


- MKLL18 batches several of a process's operations together and apply them as a group to the queue but this assumes processes can wait to do a batch all at once.  In worst case batches of size 1 reduces to MS98
- RC17 sketched how to add a helping mechanism to ensure wait-freedom (increases amortized time to Omega($P^2$) for enqueue)
- Kogan Petrank PPoPP 2011 add helping to MS98 queue to make it wait-free; only makes complexity worse

LS08 modified MS98; switched to doubly-linked list to reduce number of successful CASes needed for an op from 2 to 1 in optimistic approach, but could get inconsistent pointers that can be fixed when needed.  Fixing the list, though rare in practice, is expensive, so Omega(nP) amortized time per op.

-----

Ladan-Mozes and Shavit~\cite{DBLP:journals/dc/Ladan-MozesS08}
presented an optimistic approach to implement a queue. MS-queue uses
two \nf{CAS}es to do an enqueue: one to change the tail to the new
node and another one to change the next pointer of the previous node
to the new node. They use a doubly-linked list to do fewer
\texttt{CAS} operations in an \nf{Enqueue} than MS-queue. As in
previous algorithms, the worst case happens in the case where the
contention is high: when $p$ concurrent enqueues happen, their nodes
have to be appended to the linked list one by one. The amortized
complexity is still $\Omega(p)$ \texttt{CAS}es. 

Hendler et al.~\cite{DBLP:conf/spaa/HendlerIST10} proposed a new
paradigm called flat combining. The key idea behind flat combining is
to allow a combiner who has acquired the global lock on the data
structure to learn about the requests of threads on the queue, combine
them and apply the combined results on the data structure. Their queue
is linearizable but not lock-free and they present experiments that
show their algorithm performs well in some situations. 

Gidenstam, Sundell, and Tsigas~\cite{DBLP:conf/opodis/GidenstamST10}
introduced a new algorithm using a linked list of arrays. The queue is
stored in a shared array where head and tail pointers point to the
current elements in the queue. When the array is full, an empty array
is linked to the array and tail pointers are updated. A global head
points to the array containing the first element in the queue, and
each process has a local head index that points to the first element
in that array. Global tail and local tail pointers are similar. A
process updates the position of the pointers after it does an
operation. One process might go to sleep before setting the pointers,
so the pointers might be behind their real places. They mention how to
scan the arrays to update pointers while doing an operation. A process
writes an element in the location head by a \nf{CAS} instruction, so
if $p$ processes try to enqueue simultaneously, the amortized step
(and \nf{CAS}) complexity remains $\Omega(p)$. Their design is
lock-free but not wait-free. 

Kogan and Petrank~\cite{DBLP:conf/ppopp/KoganP11} introduced wait-free
queues based on the MS-queue and use Herlihy's helping
technique~\cite{10.1145/114005.102808} to achieve wait-freedom. Their
amortized step complexity is $\Omega(p)$ because of the helping
mechanism. 

Milman-Sela et al.~\cite{MKLLP22} designed a lock-free
queue supporting futures. In their queue, operations  return future
objects instead of responses. Later when the response is needed, it
can be evaluated from the future object. They also define a weaker
linearizability condition such that each operation can be linearized
between its invocation and when its future is evaluated. Their idea of
batching allows a sequence of operations to be submitted as a batch
for later execution on the MS-queue. They use some properties of the
queue size before and after a batch, similar to a part of our
work. Their queue is not wait-free: in fact, if the batch sizes are 1,
then the queue is like MS-queue. 

Nikolaev and Ravindran~\cite{10.1145/3490148.3538572} present a
wait-free queue that uses the fast-path slow-path methodology
introduced by Kogan and Petrank~\cite{10.1145/2370036.2145835}. Their
work is based on a circular queue using bounded memory. When a process
wishes to do an enqueue or a dequeue, it starts two paths. The fast
path  ensures good performance while the slow path ensures
termination. They show that these two paths do not affect each other
and the queue remains consistent. If a process makes no progress,
other processes help its slow path to finish. The helping phase
suffers from the CAS Retry Problem because processes compete in a
\nf{CAS} loop to decide which succeeds to help. Because of this, the
amortized complexity cannot be better than $\Omega(p)$. 

The CAS Retry Problem is not limited to list-based queues; array-based
queues also share
it~\cite{DBLP:conf/iceccs/ColvinG05,DBLP:conf/icdcn/Shafiei09,DBLP:conf/spaa/TsigasZ01}.
Our motivation is to overcome this problem and present a wait-free
sublinear queue. 

\here{Check:
Turek Shasha Prakash locking without blocking (PODS92).
Prakash Lee Johnson non blocking algorithms for concurrent data structures Tech rept 91-002 U of Florida 1991.
}

\paragraph{Other Primitives and Restricted Queues}

David introduced the first sublinear-time queue
\cite{DBLP:conf/wdag/David04}, but it works only for a single enqueuer.
It uses fetch\&increment and swap primitive instructions and takes constant time per operation, but
uses unbounded memory.  A modification to use bounded space
is described, but it increases the time per operation to $\Omega(p)$.

Jayanti and Petrovic introduced a wait-free poly-logarithmic
queue~\cite{DBLP:conf/fsttcs/JayantiP05}, but it only works for a single dequeuer. 
Our implementation uses their idea of having
a tournament tree among processes to agree on the linearization of
operations.

\here{new result I found:}
Khanchandani and Wattenhofer \cite{KW18} gave a wait-free queue implementation
with $O(\sqrt{p})$ complexity using non-standard synchronization primitives
called half-increment and half-max, which can be viewed as particular kinds of
double-word read-modify-write operations.
They use this as evidence that their primitives can be more efficient than CAS
since previous CAS-based queues all required $\Omega(p)$ step complexity.
Our new implementation counters this argument.


\subsection{Universal Constructions and Other Poly-log Time Data Structures}
A \textit{universal construction} is an algorithm that can implement a
shared version of any given sequential object. The first universal
construction was introduced by
Herlihy~\cite{10.1145/114005.102808}. We can implement a concurrent
queue using a universal construction. Jayanti proved an $\Omega(\log
p)$ lower bound on the worst-case shared-access time complexity of
$p$-process universal
constructions~\cite{DBLP:conf/podc/Jayanti98a}. He also mentions that
the universal construction by Afek, Dauber, and
Touitou~\cite{DBLP:conf/stoc/AfekDT95} can be modified to $O(\log p)$
worst-case step complexity, using atomic access to $\Omega(p \log
p)$-bit words. Chandra, Jayanti and Tan introduced a semi-universal
construction that achieves $\textsc{O}(\log^2 p)$ shared
accesses~\cite{DBLP:conf/podc/ChandraJT98}. However, their algorithm
cannot be used to create a queue. We mention a non-practical universal
construction with a poly-log number of \nf{CAS} instructions in the
last paragraph of page 13. 

Ellen and Woelfel introduced an implementation of a Fetch\&Inc object
with step complexity of $O(\log p)$ using $O(\log n$)-bit
\texttt{LL/SC} objects, where $n$ is the number of
operations~\cite{10.1007/978-3-642-41527-2_20}. Their idea to achieve
logarithmic complexity is to use a tree storing the Fetch\&Inc
operations invoked by processes. When a process wants to do a
Fetch\&Inc it adds its Fetch\&Inc to the tree and returns the number
of elements in the tree. There are some similarities between designing
a queue and a Fetch\&Inc object. A Fetch\&Inc object can be
constructed from a queue. The algorithm by Ellen and Woelfel is
interesting because of the similarities between Fetch\&Inc objects and
queues. Also, it is one of the few wait-free data structures achieving
poly-logarithmic complexity. 


% !TEX root =  podc-submission.tex

\section{Queue Implementation} \label{DescriptQ}

\subsection{Overview}
Our \emph{\ordering\ tree} data structure is used to agree on a total ordering of the operations performed on the queue.
It is a static binary tree of height $\ceil{\log_2 p}$ with one leaf 
for each process. 
Each tree node  stores an array of \emph{blocks}, where each block represents a 
sequence of enqueues and a sequence of dequeues.
See Figure \ref{orderingtree} for an example.
In this section, we use an infinite array of blocks in each node.
Section \ref{reducing} describes how to replace the infinite array by a representation that uses bounded space.

To perform an operation on the queue, a process $P$ appends a new block containing that  
operation to the \fld{blocks} array in $P$'s leaf.
Then, $P$ attempts to propagate the operation to each node along the path from that leaf to the root of the tree.
We shall define a total order on all operations that have been propagated to the root, which 
will serve as the linearization ordering of the operations.

\Eric{Reworded following parag after Hossein's comments of Jan 4}
To propagate operations from a node $\var{v}$'s children to $\var{v}$, $P$ first observes
the blocks in both of $\var{v}$'s children that are not already in $\var{v}$,
creates a new block by combining information from those blocks, and attempts to append this 
new block to $\var{v}$'s \fld{blocks} array using a \op{CAS}.
Following \cite{DBLP:conf/fsttcs/JayantiP05}, we call this a 3-step sequence a
\op{Refresh} on $\var{v}$. %(see Figure \ref{fig::propagstep}).
A \op{Refresh}'s \op{CAS} may fail if there is another concurrent \op{Refresh} on~$\var{v}$.
However, since a successful \op{Refresh} propagates multiple pending operations 
from $\var{v}$'s children to $\var{v}$,
we can prove that if two \op{Refresh}es by $P$ on $\var{v}$ fail,
then $P$'s operation has been propagated to $\var{v}$ by some other process, so $P$ can continue 
onwards towards the~root.

Now suppose $P$'s operation has been propagated all the way to the root.
If $P$'s operation is an enqueue, it has obtained a place in the linearization ordering and can terminate.
If $P$'s operation is a dequeue, $P$ must use information in the tree to compute the value that the
dequeue must return.  To do this, $P$ first determines which block in the root contains its dequeue
(since the dequeue may have been propagated to the root by some other process).
$P$ does this by finding the dequeue's location in each node along the path from the leaf to the root.
Then, $P$ determines whether the queue is empty when its dequeue is linearized. 
If so, it returns \nl\ and we call it a \emph{null dequeue}.
If not, $P$ computes the rank $r$ of its dequeue among all non-null dequeues in the linearization ordering.  (We say that the $r$th element in a sequence has \emph{rank} $r$ within that sequence.)
$P$ then returns the value of the $r$th enqueue in the linearization.

We must choose what to store in each block so that the following tasks can be done efficiently.
\begin{enumerate}[label={(T\arabic*)}]
\item
\label{construct}
Construct a block for node $\var{v}$ that represents the operations in consecutive blocks in $\var{v}$'s children, as required for a \op{Refresh}.
\item
\label{findinroot}
Given a dequeue in a leaf that has been propagated to the root, find that operation's position in the root's \fld{blocks} array.
\item
\label{findrank}
Given a dequeue's position in the root, decide if it is a null dequeue (i.e., if the queue is empty when it is linearized)
or determine the rank $r$ of the enqueue whose value it returns.
\item
\label{findenqueue}
Find the $r$th enqueue in the linearization ordering.
\end{enumerate}
Since these tasks depend on the linearization ordering, we describe that ordering next.

\begin{figure*}[t]
\input{tournamentTree.pdf_t}
\caption{An example \ordering\ tree with four processes. 
We show explicitly the enqueue sequence and dequeue sequence represented by each block in the \fld{blocks} arrays of the seven nodes.  The leftmost element of each \fld{blocks} array is a dummy block.
Arrows represent the indices stored in \eleft\ and \eright\ fields of blocks (as described in Section \ref{sec:fields}).
The fourth process's Deq\sub{6} is still propagating towards the root.
The linearization order for this tree is
Enq(a) Enq(e) Deq\sub{2} $\mid$ Enq(b) Deq\sub{4} Deq\sub{5} $\mid$ Enq(d) Enq(f) Enq(h) Deq\sub{1} $\mid$ Enq(c) Deq\sub{3} $\mid$ Enq(g), where vertical bars indicate boundaries of blocks in the root.\label{orderingtree}}
\medskip
\input{implicit.pdf_t}
\caption{\label{implicit}The actual, implicit representation of the tree shown in Figure \ref{orderingtree}.
The leaf blocks simply show the \fld{element} field.
Internal blocks show the \fld{sum\sub{enq}} and \fld{sum\sub{deq}} fields,
and \eleft\ and \eright\ fields are shown using arrows as in Figure \ref{orderingtree}.
Root blocks also have the additional \fld{size} field.
The \fld{super} field is not shown.}
\end{figure*}

\subsection{Linearization Ordering}

Performing a double \op{Refresh} at each node along the path from the leaf to the root ensures 
a block containing the operation is appended to the root before the operation completes.
So, if an operation $op_1$ terminates before another operation $op_2$ begins, 
$op_1$ will be in an earlier block than $op_2$ in the root's blocks array.
Thus, we linearize operations according to the block they belong to in the root's array.
We can choose how to order operations in the same block, since they must be concurrent.

Each block in a leaf represents one operation.
Each block $B$ in an internal node $\var{v}$ results from merging
several consecutive blocks from each of $\var{v}$'s children.
The merged blocks in $\var{v}$'s children are called the \emph{direct subblocks} of $B$.
A block $B'$ is a \emph{subblock} of $B$ if it is a direct subblock of $B$
or a subblock of a direct subblock of $B$.
A block $B$ represents the set of operations in all of $B$'s subblocks in leaves of the tree.
The operations propagated by a \op{Refresh} are all pending when the \op{Refresh} occurs,
so there is at most one operation per process.
Hence, a block represents at most $p$ operations in total.  
Moreover, we never append empty blocks, so 
each block represents at least one operation and it follows that a block can have at most $p$ direct subblocks.

As mentioned above, we are free to order operations within a block however we like.
We order the enqueues and dequeues separately, and put the 
operations propagated from the left child before the operations from the right child.
More formally, we inductively define sequences $E(B)$ and $D(B)$ of the enqueues and dequeues
represented by a block $B$.
If $B$ is a block in a leaf representing an enqueue operation, its enqueue sequence $E(B)$ is that operation
and its dequeue sequence $D(B)$ is empty.  If $B$ is a block in a leaf representing a dequeue, $D(B)$ is that single operation and $E(B)$ is empty.
If $B$ is a block in an internal node $\var{v}$ with direct subblocks $B^L_1, \ldots, B^L_\ell$ from 
$\var{v}$'s left child
and $B^R_1,\ldots,B^R_r$ from $\var{v}$'s right child, then $B$'s operation sequences are defined by the concatenations 
\begin{eqnarray}
E(B) &=& E(B^L_1)\cdots E(B^L_\ell)\cdot E(B^R_1) \cdots E(B^R_r) \mbox{ and }\nonumber\\
D(B) &=& D(B^L_1)\cdots D(B^L_\ell)\cdot D(B^R_1) \cdots D(B^R_r)\label{defSeqs}
\end{eqnarray}
We say the block $B$ \emph{contains} the operations in $E(B)$ and $D(B)$.

When linearizing the operations propagated to the root, we must
choose how to order operations within a block.  
We choose to put
each block's enqueues before its dequeues.
Thus, if the root's blocks array contains blocks $B_1, \ldots, B_k$, the 
linearization ordering~is 
\begin{equation}
L=E(B_1)\cdot D(B_1) \cdot E(B_2) \cdot D(B_2) \cdots E(B_k) \cdot D(B_k).
\label{linearization}
\end{equation}

\subsection{Designing a Block Representation to Solve Tasks \ref{construct} to \ref{findenqueue}}
\label{sec:fields}

\renewcommand{\algorithmiccomment}[1]{\hfill\eqparbox{COMMENTSINGLE}{\com\ #1}}
\begin{figure}
\begin{algorithmic}
\setcounter{ALG@line}{1}
\Statex \linecomment{Shared variable}
\begin{itemize}
\item \typ{Node} \var{root} \Comment{root of binary tree of \tt{Node}s with one leaf per process}
\end{itemize}

\Statex \linecomment{Thread-local variable}
\begin{itemize}
\item \typ{Node} \var{leaf} \Comment{process's leaf in the tree}
\end{itemize}

\Statex $\blacktriangleright$ \typ{Node}
\begin{itemize}
\item \typ{Node} \fld{left}, \fld{right}, \fld{parent} \Comment{tree pointers initialized  when creating the tree}
\item \typ{Block}[0..$\infty$] \fld{blocks} \Comment{blocks that have been propagated to this node;}\linebreak
	\mbox{ }\Comment{\var{blocks}[0] is empty block whose integer fields are 0}
\item \typ{int} \head \Comment{position to append next \block\ to \fld{blocks}, initially 1}
\end{itemize}

\Statex $\blacktriangleright$ \typ{Block} 

\begin{itemize}
  	\item \typ{int} \fld{sum\sub{enq}, sum\sub{deq}}
  		\Comment{number of enqueues, dequeues in \fld{blocks} array}\linebreak
		\mbox{ }\Comment{up to this block (inclusive)}
  	\item \typ{int} \fld{super}
  		\Comment{approximate index of superblock in \fld{parent.blocks}}
	\item[\com] Blocks in internal \nodes\ have the following additional fields
	\begin{itemize}[leftmargin=3mm]
		\item[$\bullet$] \typ{int} \eleft, \eright
  		\Comment{index of last direct subblock in the left and right child}
	\end{itemize}
  	\item[\com] Blocks in leaf \nodes\ have the following additional field
  	\begin{itemize}[leftmargin=3mm]
		\item[$\bullet$] \typ{Object} \fld{element}
  		\Comment{\var{x} for \opa{Enqueue}{x} operation; otherwise \nl}
	\end{itemize}
	\item[\com] Blocks in the root \node\ have the following additional field
	\begin{itemize}[leftmargin=3mm]
		\item[$\bullet$] \typ{int} \size%
  		\Comment{size of queue after performing all operations up }\linebreak
		\mbox{ }\Comment{to the end of this block}
	\end{itemize}
\end{itemize}

%\Statex {\com\  Blocks in internal nodes have the following additional fields}
%\Statex $\blacktriangleright$ \tt{\sl{InternalBlock} extends \sl{Block}} \sf{\com\ the following additional fields are used only for blocks in internal nodes}
%\begin{itemize}
%	\item \tt{int} \eleft, \eright
%  		\Comment{index of last direct subblock in the left and right child}
%\end{itemize}

%\Statex {\com\ Blocks in leaf nodes have the following additional field}
%\Statex $\blacktriangleright$ \tt{\sl{LeafBlock} extends \sl{Block}} \sf{\com\ the following additional field is used only for blocks in leaves}
%\begin{itemize}
%  \item \tt{\sl{Object} element}
%  \Comment{if the block's operation is \tt{enqueue(x)} then \tt{element=x}, otherwise \tt{element=null}.}
%\end{itemize}

%\Statex {\com\ Blocks in the root node have the following additional field}
%\Statex $\blacktriangleright$ \tt{\sl{RootBlock} extends \sl{InternalBlock}} \sf{\com\ the following additional field is used only for blocks in the root}
%\begin{itemize}
%  \item \tt{\sl{int} \size}%
%  \Comment{size of the queue after performing all operations up to the end of this block}
%\end{itemize}

\end{algorithmic}
\caption{Objects used in the \ordering\ tree data structure.\label{object-fields}\Eric{Fix indentation}}
\end{figure}

Each \node\ of the \ordering\ tree has an infinite array called \fld{blocks}.
To simplify the code, \fld{blocks}[0] is initialized with an empty \block\ $B_0$, 
where $E(B_0)$ and $D(B_0)$ are empty sequences.
Each \node's \fld{head} index  stores the position in the \fld{blocks} array to be used
for the next attempt to append a \block.

If a block contained an explicit representation of its sequences of enqueues and dequeues,
it would take $\Omega(p)$ time to construct a block, which would be too slow for task \ref{construct}.
Instead, the block stores an implicit representation of the sequences.
We now explain how we designed the fields for this implicit representation. 
Refer to Figure \ref{implicit} for an example showing how the tree in Figure \ref{orderingtree} is actually represented, and Figure \ref{object-fields} for the definitions of the fields of \blocks\ and \nodes.

%The information stored in a \block\ depends on whether it is in an internal node or a leaf.
A block in a leaf represents a single enqueue or dequeue.  The block's \fld{element} field stores the value
enqueued if the operation is an enqueue, or \nl\ if the operation is a dequeue.

\here{If space permits, we might want to add some examples in the following paragraphs that refer back to Figure \ref{implicit}.}

Each block in an internal \node\ $\var{v}$ has fields \eleft\ and \eright\ that store the indices of the block's last direct subblock in $\var{v}$'s left and right child.  
Thus, the direct subblocks of $\var{v}.\fld{blocks}[b]$ are
\begin{eqnarray}\label{defsubblock}
\var{v}.\fld{left.blocks}[\var{v}.\fld{blocks}[b\!-\!1].\fld{end\sub{left}}\!+\!1..\var{v}.\fld{blocks}[b].\fld{end\sub{left}}] \mbox{ and}\nonumber\\
\var{v}.\fld{right.blocks}[\var{v}.\fld{blocks}[b\!-\!1].\fld{end\sub{right}}\!+\!1..\var{v}.\fld{blocks}[b].\fld{end\sub{right}}].\!
\end{eqnarray}
The \eleft\ and \eright\ fields allow us to navigate to a block's direct subblocks.
Blocks also store some prefix sums:
%the block in 
$\var{v}.\fld{blocks}[b]$ has two fields \fld{sum\sub{enq}} and \fld{sum\sub{deq}}
that store the total numbers of enqueues and dequeues in $\var{v}.\fld{blocks}[1..\var{b}]$.
We use these to search for a particular operation.
% pinpoint the  location of an operation among the subblocks of a given block.
For example, consider finding the $r$th enqueue $E_r$ in the linearization.
A binary search for $r$ on \fld{sum\sub{enq}} fields of the root's blocks 
finds the block  containing $E_r$.
If we know a block $B$ in a \node\ $\var{v}$ contains $E_r$,
we can use the \fld{sum\sub{enq}} field again to determine which child of $\var{v}$ contains $E_r$
and then do a binary search
among the direct subblocks of $B$ in that child.
Thus, we work our way down the tree until we find the leaf block that  stores 
$E_r$ explicitly.
%Now, suppose we want to find the $r$th enqueue in the linearization ordering for task \ref{findenqueue}.Let $B_1, B_2, \ldots, B_k$ be the blocks in the root.
%First, we need prefix sums of the number of enqueues in $E(B_1)\cdot E(B_2)\cdots E(B_i)$
%so that we can do a binary search for the block $B_e$ that contains the $r$th enqueue.
%This prefix sum also allows us to know the rank $r'$ within $E(B_e)$ of the $r$th enqueue.
%Once we have $r'$, we need the number of enqueues that $B_e$ received from its left child
%to determine whether the enqueue came from the left or right child of the root.
%Suppose the enqueue came from the right child $v_r$.
%Then, we know that the index of the block in $v_r$ that contains the enqueue
%is between $B_{e-1}.\eright + 1$ and $B_e$.\eright.
%We can again do a binary search within this range.
%For this, we can again use the prefix sums of the number of enqueues in any prefix of the array $v_r.blocks$.
%We can then continue in this way down the tree until reaching a leaf where the enqueue is stored explicitly.
We shall show that the binary search in the root can be done in $O(\log p + \log q)$ steps,
and the binary search within each other \node\ along the path to a leaf takes $O(\log p)$ steps,
for a total of $O(\log^2 p + \log q)$ steps for task \ref{findenqueue}.
%All information needed for this search process can be derived from the 
%\eleft, \eright\ and \fld{sum\sub{enq}} fields.

A block is called the \emph{superblock} of all of its direct subblocks.
To facilitate task \ref{findinroot}, each block $B$ has a field \fld{super} that contains
the (approximate) index of its superblock in the parent \node's \fld{blocks} array (it may differ from the true index by 1).
%We shall ensure that the value of $B.\fld{super}$ differs from the true index of $B$'s superblock by at most 1.
This allows a process to determine the true location of the superblock by checking the \eleft\ or \eright\ values of just two \blocks\ in the parent \node.
Thus, starting from an operation in a leaf's block, one can use these indices to track the 
operation  up the path to the root, and determine the operation's location in a root block
in $O(\log p)$ time.

Now consider task \ref{findrank}.
To determine whether the queue is empty when a dequeue occurs,
each block in the root has a \fld{size} field storing the number of elements
in the queue after all operations in the linearization up to that block (inclusive) 
have been done.
We can  determine which dequeues in a block $B_d$ in the root are null dequeues using
$B_{d-1}.\fld{size}$, which is the size of the queue just before $B_d$'s operations, and the number of enqueues and dequeues in $B_d$.
Moreover, the total number of non-null dequeues in blocks $B_1, \ldots, B_{d-1}$ 
is $B_{d-1}.\fld{sum\sub{enq}}-B_{d-1}.\fld{size}$.
We can use this information to determine the
rank of a non-null dequeue in $B_d$ among all non-null dequeues in the linearization, which is the rank  (among all enqueues) of the enqueue
whose value the dequeue should return.

Having defined the fields required for tasks \ref{findinroot}, \ref{findrank} and \ref{findenqueue},
we can easily see how to construct a new block $B$ during a \op{Refresh} in $O(1)$ time.
A \op{Refresh} on \node\ $\var{v}$ reads the values $h_{\ell}$ and~$h_{r}$ of the \fld{head} fields of $\var{v}$'s children and stores 
$h_{\ell}-1$ and $h_{r}-1$ in $B.\eleft$ and $B.\eright$.
Then, we can compute 
\begin{eqnarray*}
B.\fld{sum\sub{enq}}&=&\var{v}.\fld{left}.\fld{blocks}[B.\eleft].\fld{sum\sub{enq}} \\
&&+ \var{v}.\fld{right}.\fld{blocks}[B.\eright].\fld{sum\sub{enq}}.
\end{eqnarray*}
For a block $B$ in the root, $B.\fld{size}$ is computed using the \fld{size} field of the previous block $B'$ and
the number of enqueues and dequeues in~$B$:
\begin{eqnarray*}
B.\fld{size} &=& \max(0, B'.\fld{size} + (B.\fld{sum\sub{enq}}-B'.\fld{sum\sub{enq}})\\ 
			&& \hspace*{19mm} - (B.\fld{sum\sub{deq}} - B'.\fld{sum\sub{deq}})).	
\end{eqnarray*}

The only remaining field is $B.\fld{super}$.  When the block 
$B$ is created for a \node\ $\var{v}$, we do not yet know where its
superblock will eventually be installed in \var{v}'s parent.
So, we leave $B.\fld{super}$ blank.  
Soon after $B$ is installed,
%a call to \op{Advance} at line \ref{helpAdvance} or \ref{advance} by 
some process will
set \var{B}.\fld{super} to a value read from the \fld{head} field of \var{v}'s parent.
We shall show that this happens soon enough that $B.\fld{super}$ can differ from the true index of $B'$
by at most 1.
%After $B$ is installed 
%in $\var{v}.\fld{blocks}[\var{h}]$, processes cooperate to fill in $B.\fld{super}$ 
%when they attempt to advance $\var{v}.\fld{head}$ from $h$ to $h+1$,
%using a value they read from the \fld{head} field of $\var{v}$'s parent.
%As mentioned above, this might not be the exact index of $B$'s superblock, but we
%shall prove that it is close.

\subsection{Details of the Implementation}

We now discuss the queue implementation in more detail.  Pseudocode is provided in Figure \ref{pseudocode1}.
%We use $v.\var{blocks[i].num\sub{enq}}$ as shorthand for 
%$v.\var{blocks[i].sum\sub{enq} - blocks[i-1].sum\sub{enq}}$, that is, 
%the number of enqueues in the block.  (For $\var{i}=0$, $v.\var{blocks[0].num\sub{enq}} = 0$.)
%We use \var{num\sub{deq}} similarly.
%\here{Check if the num abbreviation is really needed in the code--how many times do we use it?}

\renewcommand{\algorithmiccomment}[1]{\hfill\eqparbox{COMMENTDOUBLE}{\com\ #1}}

\here{For consistency of notation, perhaps change $n$ to $B$ in code for Enqueue, Dequeue.  Maybe also new to B for CreateBlock, Refresh.}

\begin{figure*}
\begin{minipage}[t]{0.405\textwidth}
\begin{algorithmic}[1]
\setcounter{ALG@line}{0}

\Function{void}{Enqueue}{\typ{Object} \var{e}} 
    \State \hangbox{let \var{B} be a new \typ{Block} with \fld{element} \assign\ \var{e},\\
		$\fld{sum\sub{enq}} \assign\ \var{leaf.}\fld{blocks}[\var{leaf.}\head-1].\fld{sum\sub{enq}}+1$,\\
		$\fld{sum\sub{deq}} \assign\ \var{leaf.}\fld{blocks}[\var{leaf.}\head-1].\fld{sum\sub{deq}}$}\label{enqNew}
    \State \Call{Append}{\var{B}}
\EndFunction{Enqueue}

\spac

\Function{Object}{Dequeue()}{} 
    \State \hangbox{let \var{B} be a new \typ{Block} with \fld{element} \assign\ \nl,\\
	    $\fld{sum\sub{enq}} \assign\ \var{leaf.}\fld{blocks}[\var{leaf.}\head-1].\fld{sum\sub{enq}}$,\\
	    $\fld{sum\sub{deq}} \assign\ \var{leaf.}\fld{blocks}[\var{leaf.}\head-1].\fld{sum\sub{deq}}+1$}\label{deqNew}
    \State \Call{Append}{\var{B}}
    \State $\langle \var{b}, \var{i}\rangle$ \assign\ \Call{IndexDequeue}{\var{leaf}, $\var{leaf.}\head-1$, $1$}\label{invokeIndexDequeue}
    \State \Return{ \Call{FindResponse}{\var{b, i}}}\label{deqRest}
\EndFunction{Dequeue}

\spac

\Function{void}{Append}{\typ{Block} \var{B}} \com\ append block to leaf and propagate to root
    \State \var{leaf.}\fld{blocks}[\var{leaf.}\head] \assign\ \var{B}\label{appendLeaf}
    \State $\var{leaf.}\head\ \assign\ \var{leaf.}\head+1$ \label{appendEnd} 
    \State \Call{Propagate}{\var{leaf.}\fld{parent}} 
\EndFunction{Append}

\spac

\Function{void}{Propagate}{\typ{Node} \var{v}} \com\ propagate blocks from \var{v}'s children to root
    \If{\bf{not} \Call{Refresh}{\var{v}}} \label{firstRefresh}  \hfill \com\ double refresh
        \State \Call{Refresh}{\var{v}} \label{secondRefresh}
    \EndIf
    \If{$\var{v} \neq \var{root}$} \hfill \com\ recurse up tree
        \State \Call{Propagate}{\var{v}.\fld{parent}}
    \EndIf
\EndFunction{Propagate}

\spac

\Function{boolean}{Refresh}{\typ{Node} \var{v}} \com\ try to append a new block to \var{v}.\fld{blocks}
    \State \var{h} \assign\ \var{v}.\head \label{readHead}
    \ForEach{\fld{dir} {\keywordfont{in}} $\{\fld{left, right}\}$} \label{startHelpChild1}
        \State \var{childHead} \assign\ \var{v}.\fld{dir}.\head \label{readChildHead}
        \If{\var{v}.\fld{dir.blocks}[\var{childHead}] $\neq$ \nl} \label{ifHeadnotNull}
            \State \Call{Advance}{\var{v}.\fld{dir}, \var{childHead}} \label{helpAdvance}
        \EndIf
    \EndFor \label{endHelpChild1}
    \State \var{new} \assign\ \Call{CreateBlock}{\var{v, h}} \label{invokeCreateBlock}
    \If{\var{new = \nl}} \Return{\tr} \label{addOP} 
	\Else
	    \State \var{result} \assign\ \Call{CAS}{\var{v}.\fld{blocks}[\var{h}], \nl, \var{new}} \label{cas}
    	\State \Call{Advance}{\var{v, h}}\label{advance}
    	\State \Return{ \var{result}}
	\EndIf
\EndFunction{Refresh}

\spac


\Function{Block}{CreateBlock}{\typ{Node} \var{v}, \typ{int} \var{i}} 
    \State \linecomment{create new block for a \op{Refresh} to install in \var{v}.\fld{blocks}[\var{i}]}
    \State let \var{new} be a new \typ{Block} \label{initNewBlock}
    \State \var{new}.\eleft \assign\ $\var{v}.\fld{left}.\head - 1$\label{createEndLeft}
    \State \var{new}.\eright \assign\ $\var{v}.\fld{right}.\head - 1$\label{createEndRight}
	\State \hangbox{\var{new}.\fld{sum\sub{enq}} \assign\ \var{v}.\fld{left.blocks}[\var{new}.\eleft].\fld{sum\sub{enq}} + \\
			\hspace*{11mm}\var{v}.\fld{right.blocks}[\var{new}.\eright].\fld{sum\sub{enq}}}\label{createSumEnq}
	\State \hangbox{\var{new}.\fld{sum\sub{deq}} \assign\ \var{v}.\fld{left.blocks}[\var{new}.\eleft].\fld{sum\sub{deq}} + \\
			\hspace*{11mm}\var{v}.\fld{right.blocks}[\var{new}.\eright].\fld{sum\sub{deq}}}\label{createSumDeq}
    \State \var{num\sub{enq}} \assign\ $\var{new}.\fld{sum\sub{enq}} - \var{v}.\fld{blocks}[\var{i}-1].\fld{sum\sub{enq}}$\label{computeNumEnq}
    \State \var{num\sub{deq}} \assign\ $\var{new}.\fld{sum\sub{deq}} - \var{v}.\fld{blocks}[\var{i}-1].\fld{sum\sub{deq}}$
    \If{$\var{v} = \var{root}$}
        \State \hangbox{\var{new}.\fld{size} \assign\ max(0, $\var{v}.\fld{blocks}[\var{i}-1].\size\ + \var{num\sub{enq}} - \var{num\sub{deq}}$)}\label{computeLength}
    \EndIf
    \If{$\var{num\sub{enq}} + \var{num\sub{deq}} = 0$}\label{testEmpty}
        \State \Return \nl \hfill \com\ no blocks need to be propagated to \var{v}
    \Else
        \State \Return \var{new}
    \EndIf
\EndFunction{CreateBlock}

\end{algorithmic}
\end{minipage}
\begin{minipage}[t]{0.585\textwidth}

\begin{algorithmic}[1]
\setcounter{ALG@line}{57}
\Function{void}{Advance}{\typ{Node} \var{v}, \typ{int} \var{h}} \com\ set \var{v}.\fld{blocks}[\var{h}].\fld{super} and increment \var{v}.\fld{head} from \var{h} to $\var{h}+1$
    \If{$\var{v}\neq \var{root}$}
	    \State \var{h\sub{p}} \assign\ \var{v}.\fld{parent}.\head \label{readParentHead}
    	\State \Call{CAS}{\var{v}.\fld{blocks}[\var{h}].\fld{super}, \nl, \var{h\sub{p}}} \label{setSuper1}
	\EndIf
    \State \Call{CAS}{\var{v}.\head, \var{h}, \var{h}+1} \label{incrementHead}
\EndFunction{Advance}

\spac

\Function{$\langle\typ{int}, \typ{int}\rangle$}{IndexDequeue}{\typ{Node} \var{v}, \typ{int} \var{b}, \typ{int} \var{i}} \com\ return $\langle\var{b}', \var{i}'\rangle$ such that \var{i}th dequeue in
    \State \linecomment{$D(\var{v}.\fld{blocks}[\var{b}])$ is $(\var{i}')$th dequeue of $D(\var{root}.\fld{blocks}[\var{b}'])$}
    \State \linecomment{Precondition: $\var{v}.\fld{blocks}[\var{b}]$ is not \nl, was propagated to root, and contains at least}
    \State \linecomment{\var{i} dequeues}
    \If{$\var{v} = \var{root}$} \Return $\langle\var{b, i}\rangle$ \label{indexBaseCase}
    \Else
	    \State \fld{dir} \assign\ (\var{v}.\fld{parent.left} = \var{v} ? \fld{left} : \fld{right}) 
    	\State \var{sup} \assign\ \var{v}.\fld{blocks}[\var{b}].\fld{super}\label{idsup1}
	    \If{$\var{b} > \var{v}.\fld{parent.blocks}[\var{sup}].\fld{end\sub{dir}}$} \var{sup} \assign\ $\var{sup}+1$\label{supertest}\label{idsup2}
	    \EndIf\label{idsup3}
	    \State \linecomment{compute index \var{i} of dequeue in superblock}
	    \State \hangbox{\var{i} += $\var{v}.\fld{blocks}[\var{b}-1].\fld{sum\sub{deq}} -$ 
	    		$\var{v}.\fld{blocks}[\var{v}.\fld{parent.blocks}[\var{sup}-1].\edir].\fld{sum\sub{deq}}$}\label{computeISuperStart}
        \If{$\fld{dir} = \fld{right}$} 
        	\State \hangbox{\var{i} += $\var{v}.\fld{blocks}[\var{v}.\fld{parent.blocks}[\var{sup}].\eleft].\fld{sum\sub{deq}} - \mbox{ }$\\
					$\var{v}.\fld{blocks}[\var{v}.\fld{parent.blocks}[\var{sup}-1].\eleft].\fld{sum\sub{deq}}$}\label{considerLeftBeforeRight}
        \EndIf \label{computeISuperEnd}
        \State \Return\Call{IndexDequeue}{\var{v}.\fld{parent}, \var{sup}, \var{i}}
    \EndIf
\EndFunction{IndexDequeue}

\spac

\Function{element}{FindResponse}{\typ{int} \var{b}, \typ{int} \var{i}} \com\ find response to \var{i}th dequeue in $D(\var{root}.\fld{blocks}[\var{b}])$
    \State \linecomment{Precondition:  $1\leq i\leq |D(\var{root}.\fld{blocks}[\var{b}])|$}
    %  $i\geq 1$ and \var{root}.\fld{blocks}[\var{b}] is non-null and}
    %  \State \linecomment{contains at least \var{i} dequeues}
    \State \hangbox{\var{num\sub{enq}} \assign\ $\var{root}.\fld{blocks}[\var{b}].\fld{sum\sub{enq}} - \var{root}.\fld{blocks}[\var{b}-1].\fld{sum\sub{enq}}$}\label{FRNum}
    \If{$\var{root}.\fld{blocks}[\var{b}-1].\size + \var{num\sub{enq}} < \var{i}$}\label{checkEmpty}
        \State \Return \nl \hfill \com\ queue is empty when dequeue occurs\label{returnNull}
    \Else \ \linecomment{response is the \var{e}th enqueue in the root}
        \State \var{e} \assign\ \var{i} + \var{root}.\fld{blocks}[\var{b}-1].\fld{sum\sub{enq}} - 
			\var{root}.\fld{blocks}[\var{b}-1].\size\label{computeE}
		\State \linecomment{compute enqueue's block using binary search}
		\State find min $b_e\leq \var{b}$ with $\var{root}.\fld{blocks}[b_e].\fld{sum\sub{enq}} \geq \var{e}$\label{FRb}
		\State \linecomment{find rank of enqueue within its block}
		\State $i_e \assign\ \var{e} - \var{root}.\fld{blocks}[b_e-1].\fld{sum\sub{enq}}$\label{FRi}
        \State \Return \Call{GetEnqueue}{\var{root}, $b_e$, $i_e$}\label{findAnswer}
    \EndIf
\EndFunction{FindResponse}

\spac 

\Function{element}{GetEnqueue}{\typ{Node} \var{v}, \typ{int} \var{b}, \typ{int} \var{i}} \Comment{returns argument of \var{i}th enqueue in $E(\var{v}.\fld{blocks}[\var{b}])$}
    \State \linecomment{Preconditions: $\var{i}\geq 1$ and \var{v}.\fld{blocks}[\var{b}] is non-\nl\ and contains at least \var{i} enqueues}
	
    \If{\var{v} is a leaf node} \Return \var{v}.\fld{blocks}[\var{b}].\fld{element} \label{getBaseCase}
    \Else 
        \State \var{sum\sub{left}} \assign\ \var{v}.\fld{left.blocks}[\var{v}.\fld{blocks}[\var{b}].\eleft].\fld{sum\sub{enq}} 
        \State \linecomment{\var{sum\sub{left}} is the number of enqueues in \var{v}.\fld{blocks}[1..$\var{b}$] from \var{v}'s left child}
        \State \var{prev\sub{left}} \assign\ \var{v}.\fld{left.blocks}[\var{v}.\fld{blocks}[$\var{b}-1$].\eleft].\fld{sum\sub{enq}} 
        \State \linecomment{\var{prev\sub{left}} is the number of enqueues in \var{v}.\fld{blocks}[1..$\var{b}-1$] from \var{v}'s left child}
        \State \var{prev\sub{right}} \assign\ \var{v}.\fld{right.blocks}[\var{v}.\fld{blocks}[$\var{b}-1$].\eright].\fld{sum\sub{enq}} 
        \State \linecomment{\var{prev\sub{right}} is the number of enqueues in \var{v}.\fld{blocks}[1..$\var{b}-1$] from \var{v}'s right child}
        \If{$\var{i} \leq \var{sum\sub{left}} - \var{prev\sub{left}}$} \label{leftOrRight} \Comment{required enqueue is in \var{v}.\fld{left}}
            \State \fld{dir} \assign\ \fld{left}
        \Else \Comment{required enqueue is in \var{v}.\fld{right}}
            \State \fld{dir} \assign\ \fld{right}
            \State $\var{i}\ \assign\ \var{i} - (\var{sum\sub{left}} - \var{prev\sub{left}})$
        \EndIf \label{endChooseDir}
        \State \linecomment{Use binary search to find enqueue's block in \var{v}.\fld{dir} and its rank within block}
        \State \hangbox{find minimum $\var{b}'$ in range [\var{v}.\fld{blocks}[$\var{b}-1$].\edir+1..\var{v}.\fld{blocks}[\var{b}].\edir] such that\\
        	 $\var{v}.\fld{dir.blocks}[\var{b}'].\fld{sum\sub{enq}} \geq \var{i} + \var{prev\sub{dir}}$\label{getChild}}
        \State $\var{i}'$ \assign\ $\var{i} - (\var{v}.\fld{dir.blocks}[\var{b}'-1].\fld{sum\sub{enq}} - \var{prev\sub{dir}})$\label{getChildIndex}
        \State \Return\Call{GetEnqueue}{\var{v}.\fld{dir}, $\var{b}'$, $\var{i}'$} \label{getRecurse}
    \EndIf
\EndFunction{GetEnqueue}

\end{algorithmic}
\end{minipage}
\vspace*{-3mm}
\caption{Queue implementation.\label{pseudocode1}}
\end{figure*}

An \opemph{Enqueue}(\var{e}) appends a \block\ to the process's leaf.
The block has $\fld{element}=\var{e}$ to indicate it represents an \op{Enqueue}(\var{e}) operation.
It suffices to propagate the operation to the root and
then use its position in the linearization for future \op{Dequeue}
operations.

A \opemph{Dequeue} also appends a \block\ to the process's leaf.
The block has $\fld{element}=\nl$ to indicate that it represents a \op{Dequeue} operation.
After propagating the operation to the root, it computes
its position in the root using
\op{IndexDequeue} and then computes its response by calling \op{FindResponse}. 

\opemph{Append}(\var{B}) first adds the block \var{B} to the invoking process's leaf.
The leaf's \fld{head} field stores the first empty slot in the leaf's \fld{blocks} array,
so the \op{Append} writes \var{B} there and increments \fld{head}.
Since \op{Append} writes only to the process's own leaf, there cannot be concurrent updates to a leaf.
\op{Append} then calls \op{Propagate} to ensure the operation represented by \var{B} is propagated to the root.

\opemph{Propagate}(\var{v}) guarantees that any blocks that are in \var{v}'s children when \op{Propagate} is invoked are propagated to the root.
It uses the double \op{Refresh} idea described
above and invokes two \op{Refresh}es on \var{v} in Lines
\ref{firstRefresh} and \ref{secondRefresh}. 
If both fail to add a block to \var{v}, it means some other process has done a successful \op{Refresh}
that propagated blocks that were in \var{v}'s children prior to line \ref{firstRefresh} to \var{v}.
Then, \op{Propagate} recurses to \var{v}.\fld{parent} to continue propagating blocks up to the root.  

%\paragraph{\tt{Refresh()} and \tt{Advance()}}
A \opemph{Refresh} on node \var{v} creates a block representing the new blocks
in \var{v}'s
children and tries to append it to \var{v}.\fld{blocks}. 
Line \ref{readHead} reads \var{v}.\fld{head} into the local variable \var{h}.
Line \ref{invokeCreateBlock} creates the new block to install in \var{v}.\fld{blocks}[\var{h}].
If line \ref{invokeCreateBlock} returns \nl\ instead of a new block, there were no new blocks in \var{v}'s children to propagate to \var{v}, so \op{Refresh} can return true at line \ref{addOP} and terminate.
Otherwise, the CAS at line \ref{cas} tries to install the new block into \var{v}.\fld{blocks}[\var{h}].
Either this CAS succeeds or some other process has installed a  block in this location.
Either way, line \ref{advance} then calls \opemph{Advance} to advance \var{v}'s head index 
from \var{h} to $\var{h}+1$
and fill in the \fld{super} field of the most recently appended block.
The boolean value returned by \op{Refresh} indicates whether its CAS succeeded.
A \op{Refresh} may pause after a successful CAS before calling \op{Advance} at line \ref{advance},
so other processes help keep \fld{head} up to date by  calling \op{Advance}, 
either at line \ref{helpAdvance} during a \op{Refresh} on \var{v}'s parent or line \ref{advance} during a \op{Refresh} on~\var{v}.

\opemph{CreateBlock}(\var{v, i}) is used
by \op{Refresh} to construct a block to be installed in \var{v}.\fld{blocks}[\var{i}].
%The block \var{new} is created in Line \ref{initNewBlock}. 
The \eleft\ and \eright\ fields store the indices of the last blocks appended to \var{v}'s
children, obtained by reading the \fld{head} index in \var{v}'s children.
Since the \fld{sum\sub{enq}} field should store the number of enqueues in
\var{v}.\fld{blocks}[1..\var{i}] and these enqueues come from \var{v}.\fld{left.blocks}[1..\var{new}.\eleft] and \var{v}.\fld{blocks}[1..\var{new}.\eright], line \ref{createSumEnq} sets
\fld{sum\sub{enq}} to the sum of $\var{v}.\fld{left.blocks}[\var{new}.\eleft].\fld{sum\sub{enq}}$ and $\var{v}.\fld{right}.\fld{blocks}[\var{new}.\eright].\fld{sum\sub{enq}}$.
Line \ref{computeNumEnq} sets \var{num\sub{enq}} to the number of enqueues in the new block by
subtracting  the number of enqueues  in \var{v}.\fld{blocks}[$1..\var{i}-1$] from \var{new}.\fld{sum\sub{enq}}.
The values of \var{new}.\fld{sum\sub{deq}} and \var{num\sub{deq}} are computed similarly.
Then, if \var{new}
is going to be installed in the root, line \ref{computeLength} computes the \fld{size} field, which
represents the number of elements in the queue after the operations in the block are performed.
Finally, if the new block contains no operations, \op{CreateBlock} returns \nl\ to indicate
 there is no need to install it.

Once a dequeue is appended to a block of the process's leaf and propagated to the root,
the \opemph{IndexDequeue} routine finds the dequeue's location in the root.
More precisely, \opa{IndexDequeue}{v, b, i}
computes the block in the root and the rank
within that block  of the \var{i}th dequeue of the block \var{B} stored in \var{v}.\fld{blocks}[\var{b}].
Lines \ref{idsup1}--\ref{idsup3} compute the location of $B$'s superblock in \var{v}'s parent, taking into account the fact that $B.\fld{super}$ may differ from the superblock's true index by one.
The arithmetic in lines \ref{computeISuperStart}--\ref{computeISuperEnd} compute the dequeue's 
rank within the superblock's sequence of dequeues, using  (\ref{defSeqs}).

%Thus, after \var{B} is installed, the call to \op{Advance} at line \ref{advance} sets the value of \var{B}.\fld{super} by reading \var{par}.\fld{head}.
%As mentioned earlier, other processes may help by calling \op{Advance} at line \ref{helpAdvance} or \ref{advance} to ensure that $\var{B}.\fld{super}$ is filled in soon after \var{B} is installed.
%We shall show {$B$.\fld{super}}  differs
%from the index of $B'$ by at most~1. 

To compute the response of the $i$th \op{Dequeue} in the $b$th block
of the root, \opemph{FindResponse}(\var{b, i}) determines at line \ref{checkEmpty} if the queue is empty.
If not, line \ref{computeE} computes the rank \var{e} of the
\op{Enqueue} whose argument is the \op{Dequeue}'s response. 
A binary search on the \fld{sum\sub{enq}} fields of \var{root}.\fld{blocks} finds the index $b_e$ of the block that contains 
the \var{e}th enqueue.
Since the enqueue is linearized before the dequeue, $b_e\leq b$.  To find the left end of the range for the binary search for $b_e$, we can first do a doubling search \cite{BY76}, comparing \var{e} to the \fld{sum\sub{enq}} fields at indices $b-1, b-2, b-4, b-8, \ldots$.
Then, \op{GetEnqueue} traces down through the tree to find the required enqueue in a leaf.

\opemph{GetEnqueue}(\var{v, b, i}) returns the argument of the
$i$th enqueue in the $b$th block $B$ of \typ{Node} $\var{v}$. 
It recursively finds the location of the enqueue in each node along the path from $v$ to a leaf, which stores the argument explicitly.
\op{GetEnqueue} first determines which child of \var{v} contains the enqueue, and
then finds the range of blocks within that child that are subblocks of $B$ using information stored
in $B$ and the block that precedes $B$ in $v$.
\op{GetEnqueue} finds the exact subblock containing the enqueue using a binary search on the \fld{sum\sub{enq}}
field (line~\ref{getChild}) and proceeds recursively down the tree.






\subsection{Pseudocode} \label{algQ}
We present our algorithm in pseudocode. page~22 contains the description of the fields in the tree nodes and the blocks. The value of any uninitialized field is \nf{null}. page~23 contains major routines and the rest of this section consists of the auxiliary routines. The abbreviations below are used in the pseudocode and the proof of correctness.
\begin{itemize}
 \item \tt{blocks[$b$].sum\sub{x}=blocks[$b$].sum\sub{x-left}+blocks[$b$].sum\sub{x-right}} (for internal blocks where $b\geq 0$ and \tt{x} $\in$ \tt{\{enq, deq\}})
%  \item \tt{blocks[b].sum=blocks[b].sum\sub{enq}+blocks[b].sum\sub{deq}}  \tt{ (for b$\geq$0})
  \item \tt{blocks[$b$].num\sub{x}=blocks[$b$].sum\sub{x}-blocks[$b-1$].sum\sub{x}}  (for all blocks where $b>0$ and \tt{x} $\in$ \tt{\{enq, deq, enq-left, enq-right, deq-left, deq-right\}})
\end{itemize}




\begin{algorithm}
\caption{\tt{\sl{Queue}}}
\begin{algorithmic}[1]
\setcounter{ALG@line}{0}


\Function{void}{Enqueue}{\sl{Object} e} \cmt{Creates a \tt{block} with element \tt{e} and adds it to the tree.}
\State \tt{block newBlock= \Call{new}{\sl{LeafBlock}}}
\State \tt{newBlock.element= e}
\State \tt{newBlock.sum\sub{enq}= leaf.blocks[leaf.\head].sum\sub{enq}+1}
\State \tt{newBlock.sum\sub{deq}= leaf.blocks[leaf.\head].sum\sub{deq}}
\State \tt{leaf.}\Call{Append}{newBlock}
\EndFunction{Enqueue}

\Statex

\Statex \com\ Creates a block with \nf{null} value element, appends it to the tree and returns its response.
\Function{Object}{Dequeue()}{} 
\State \tt{block newBlock= \Call{new}{\sl{LeafBlock}}} 
\State \tt{newBlock.element= null}
\State \tt{newBlock.sum\sub{enq}= leaf.blocks[leaf.\head].sum\sub{enq}}
\State \tt{newBlock.sum\sub{deq}= leaf.blocks[leaf.\head].sum\sub{deq}+1}
\State \tt{leaf.}\Call{Append}{newBlock}
\State \tt{<b, i>=} \Call{IndexDequeue}{leaf.\head, 1}
\State \tt{output=} \Call{FindResponse}{b, i} 
\label{deqRest}
\State \Return{\tt{output}}
\EndFunction{Dequeue}

\Statex

\Statex \com\ Returns the response to $D_i(root,b)$, the \nf{i}th \nf{Dequeue} in \nf{root.blocks[b]}.
\Function{element}{FindResponse}{\sl{int} b, \sl{int} i}
\If{\tt{ root.blocks[b-1].\size}\tt{ + root.blocks[b].num\sub{enq} - i $<$ 0}} \label{checkEmpty}\cmt{Check if the queue is empty.}
\State \Return \tt{null} \label{returnNull}
\Else \cmt{The response is $E_e(root)$, the \nf{e}th \nf{Enqueue} in the root.}
\State \tt{e= i + (root.blocks[b-1].sum\sub{enq}-root.blocks[b-1].size)} \label{computeE}
\State \Return \tt{root.GetEnqueue(root.\Call{DoublingSearch}{e, b})}\label{findAnswer}
\EndIf
\EndFunction{FindResponse}

\end{algorithmic}
\end{algorithm}



\begin{algorithm}
\caption{\tt{\sl{Node}}}
\begin{algorithmic}[1]
\setcounter{ALG@line}{25}

\Statex $\leadsto$ \textsf{Precondition: \tt{blocks[start..end]} contains a block with \tt{sum\sub{enq}} greater than or equal to \tt{x}}
\Statex \com\ \textmd{Does a binary search for~the value \tt{x} of \tt{sum\sub{enq}} field and returns the index of the leftmost block in\\
\com\ \tt{blocks[start..end]} whose \tt{sum\sub{enq}} is $\geq$ \tt{x}}.
\Function{int}{BinarySearch}{\sl{int} x, \sl{int} start, \sl{int} end}
% \State \Return \tt{min\{j: blocks[j].sum\sub{enq}$\geq$x\}}
\While{\nf{start<end}}
\State \tt{\sl{int}} \tt{mid= floor((start+end)/2)}
\If{\nf{blocks[mid].sum\sub{enq}<x}}
\State \nf{start= mid+1}
\Else
\State \nf{end= mid}
\EndIf
\EndWhile
\State\Return \nf{start}
\EndFunction{BinarySearch}

\end{algorithmic}
\end{algorithm}

\begin{algorithm}
\caption{\tt{\sl{Root}}}
\begin{algorithmic}[1]
\setcounter{ALG@line}{36}
\Statex
\Statex $\leadsto$ \textsf{Precondition: \tt{root.blocks[end].sum\sub{enq} $\geq$ \tt{e}}}
\Statex \com\ \textmd{Returns \tt{<b,i>} such that $E_\nf{e}(\nf{root})=E_\nf{i}(\nf{root},\nf{b})$, i.e., the \nf{e}th \nf{Enqueue} in the \nf{root} is the \nf{i}th \nf{Enqueue} within \\
\com\ the \nf{b}th block in the \nf{root}.}

\Function{<int, int>}{DoublingSearch}{\sl{int} e, \sl{int} end}
\State \tt{start= end-1} \label{dsearchStart}
\While{\tt{root.blocks[start].sum\sub{enq}}$>=$\tt{e}}
\State \tt{start= max(start-(end-start), 0)} \label{doubling}
\EndWhile \label{dsearchEnd}
\State \tt{b= root.BinarySearch(e, start, end)} \label{dsearchBinarySearch}
\State \tt{i= e- root.blocks[b-1].sum\sub{enq}} \label{DSearchComputei}
\State\Return \tt{<b,i>}
\EndFunction{DoublingSearch}
\end{algorithmic}
\end{algorithm}

\begin{algorithm}
\caption{\tt{\sl{Leaf}}}
\begin{algorithmic}[1]
\setcounter{ALG@line}{45}

\Function{void}{Append}{\sl{block} B} \cmt{Only called by the owner of the leaf.}
\State \tt{blocks[\head]= B} \label{appendLeaf}
\State \tt{\head= \head+1} \label{appendEnd} 
\State \tt{parent.}\Call{Propagate()}{} 
\EndFunction{Append}

\end{algorithmic}
\end{algorithm}




\begin{algorithm}
\caption{\tt{\sl{Node}}}
\begin{algorithmic}[1]
\setcounter{ALG@line}{50}


\Statex \com\ \textmd{\nf{$n$.Propagate} propagates operations  in \nf{this}.children up to \nf{this} when it terminates.}
\Function{void}{Propagate()}{}
\If{\bf{not} \Call{Refresh()}{}} \label{firstRefresh}
\State \Call{Refresh()}{} \label{secondRefresh}
\EndIf
\If{\tt{this} \bf{is not} \tt{root}}
\State \tt{parent.}\Call{Propagate()}{}
\EndIf
\EndFunction{Propagate}

\Statex

\Statex \com\ \textmd{Creates a block containing new operations of \nf{this.}children, and then tries to append it to \nf{this}.}
\Function{boolean}{Refresh()}{}
\State \tt{h= \head} \label{readHead}
\ForEach{\tt{dir} {\keywordfont{in}} \tt{\{left, right\}}} \label{startHelpChild1}
\State \tt{h\sub{dir}= dir.\head} \label{readChildHead}
\If{\nf{dir.blocks[h\sub{dir}]!=null}} \label{ifHeadnotNull}
\State{\tt{dir.\Call{Advance}{h\sub{dir}}}} \label{helpAdvance}
\EndIf
\EndFor \label{endHelpChild1}
\State \tt{new= \Call{CreateBlock}{h}} \label{invokeCreateBlock}
\If{\tt{new.num==0}} \Return{\tt{true}} \label{addOP} 
\EndIf
\State{\tt{result= blocks[h].CAS(null, new)}} \label{cas}
\State{\tt{this.\Call{Advance}{h}}} \label{advance}
\State \Return{ \tt{result}}

\EndFunction{Refresh}


\end{algorithmic}
\end{algorithm}

\begin{algorithm}
\caption{\tt{\sl{Node}}}
\begin{algorithmic}[1]
\setcounter{ALG@line}{73}

\Function{void}{Advance}{\sl{int} h} \cmt{Sets \nf{blocks[h].super} and increments \nf{head} from \nf{h} to \nf{h+1}.}
\State \tt{h\sub{p}= parent.\head} \label{readParentHead}
\State \tt{blocks[h].super.CAS(null, h\sub{p})} \label{setSuper1}
\State \tt{head.CAS(h, h+1)} \label{incrementHead}
\EndFunction{Advance}

\Statex

\Function{Block}{CreateBlock}{\sl{int} i} \cmt{Creates and returns the block to be installed in \tt{blocks[i]}.}
\State \tt{block new= \Call{new}{\sl{InternalBlock}}} \label{initNewBlock}
\ForEach{\tt{dir} {\keywordfont{in}} \tt{\{left, right\}}}
\State \tt{index\sub{prev}= blocks[i-1].\edir} \label{prevLine}
\State \tt{new.\edir= dir.\head-1} \label{lastLine} \cmt{\nf{new} contains \tt{dir.blocks[blocks[i-1].\edir..dir.\head-1]}.}
\State \tt{block\sub{prev}= dir.blocks[index\sub{prev}]}
\State \tt{block\sub{last}= dir.blocks[new.\edir]}
\State \tt{new.sum\sub{enq-dir}= blocks[i-1].sum\sub{enq-dir} + block\sub{last}.sum\sub{enq} - block\sub{prev}.sum\sub{enq}} \label{setSumEnqLeft}
\State \tt{new.sum\sub{deq-dir}= blocks[i-1].sum\sub{deq-dir} + block\sub{last}.sum\sub{deq} - block\sub{prev}.sum\sub{deq}} \label{setSumEnqRight}
\EndFor
\If{\tt{this} \bf{is} \tt{root}}
\State \tt{new.type= \sl{InternalBlock}-->\sl{RootBlock}}
\State \tt{new.size= max(root.blocks[i-1].\size { }+ new.num\sub{enq}- new.num\sub{deq}, 0)}\label{computeLength}
\EndIf

\State \Return \tt{new}
\EndFunction{CreateBlock}


\end{algorithmic}
\end{algorithm}



\begin{algorithm}
\caption{\tt{\sl{Node}}}
\begin{algorithmic}[1]
\setcounter{ALG@line}{94}

\Statex $\leadsto$ \textsf{Precondition:~\tt{blocks[b].num\sub{enq}$\geq$i$\geq 1$}}
\Function{element}{GetEnqueue}{\sl{int} b, \sl{int} i} \cmt{Returns the \tt{element} of $E_\tt{i}(\tt{this},\tt{b})$.}
\If{\tt{this} \bf{is} \tt{leaf}}
\State\Return \tt{blocks[b].element} \label{getBaseCase}
\ElsIf{\tt{i <= blocks[b].num\sub{enq-left}}} \label{leftOrRight} \cmt{$E_\tt{i}(\tt{this},\tt{b})$ is in the left child of this node.}
\State \tt{subblockIndex= left.BinarySearch(i+blocks[b-1].sum\sub{enq-left}, blocks[b-1].\eleft+1,} \label{leftChildGet}
\Statex \hspace{10.7em}\tt{blocks[b].\eleft)} 
\State \Return\tt{left.}\Call{GetEnqueue}{subblockIndex, i} 
\Else
\State \tt{i= i-blocks[b].num\sub{enq-left}}
\State\tt{subblockIndex= right.BinarySearch(i+blocks[b-1].sum\sub{enq-right}, blocks[b-1].\eright+1,} \label{rightChildGet}
\Statex \hspace{11.1em}\tt{blocks[b].\eright)} 
\State \Return\tt{right.}\Call{GetEnqueue}{subblockIndex, i} 
\EndIf
\EndFunction{GetEnqueue}

\Statex
\Statex $\leadsto$ \textsf{Precondition: \tt{b}th block of the node has propagated up to the root and \tt{blocks[b].num\sub{deq}$\geq$i}.}
\Function{<int, int>}{IndexDequeue}{\sl{int} b, \sl{int} i} \cmt{Returns \tt{<x, y>} if $D_\tt{i}(\tt{this}, \tt{b})=D_\tt{y}(\tt{root},\tt{x})$.}
\If{\tt{this} \bf{is} \tt{root}}
\State\Return \tt{<b, i>} \label{indexBaseCase}
\Else
\State \tt{dir= (parent.left==n ? left: right)} 
\State \tt{superblockIndex= parent.blocks[blocks[b].super].sum\sub{deq-dir} > blocks[b].sum\sub{deq} ? \label{computeSuper}
\Statex \hspace{11.3em} blocks[b].super: blocks[b].super+1}

\If{\tt{dir {\keywordfont is} left}} \label{computeISuperStart}
\State \tt{i+= blocks[b-1].sum\sub{deq}-parent.blocks[superblockIndex-1].sum\sub{deq-left}} \label{considerPreviousLeft}
\Else \label{considerRight}
\State \tt{i+= blocks[b-1].sum\sub{deq}-parent.blocks[superblockIndex-1].sum\sub{deq-right}}  \label{considerPreviousRight}
\State \tt{i+= parent.blocks[superblockIndex].num\sub{deq-left}}  \label{considerLeftBeforeRight}
\EndIf \label{computeISuperEnd}
\State \Return\tt{this.parent.}\Call{IndexDequeue}{superblockIndex, i}
\EndIf
\EndFunction{IndexDequeue}

\end{algorithmic}
\end{algorithm}

% !TEX root =  podc-submission.tex

\section{Proof of Correctness}
\label{sec::correctness}

After proving some basic properties in Section \ref{sec::basicProperties},
we show in Section \ref{sec::propagating} that a double refresh at each node
suffices to propagate an operation to the root.
In Section \ref{sec::tracingCorrect} we show \op{GetEnqueue} and \op{IndexDequeue}
correctly navigate through the  tree.
Finally, we prove linearizability in Section \ref{sec::linearizability}.

\subsection{Basic Properties}
\label{sec::basicProperties}

A \typ{Block} object's fields, except for \fld{super}, are immutable:  they are written only 
when the block is created.
% at line \ref{enqNew} or \ref{deqNew} (for a leaf's block) or lines \ref{initNewBlock}--\ref{computeLength} (for an internal node's block).  
Moreover, only a \op{CAS} at line \ref{setSuper1}  modifies  \fld{super}  
(from \nl\ to a non-\nl\ value), so it is changed only once.
Similarly, only a \op{CAS} at line \ref{cas} modifies an element of a node's \fld{blocks} array 
(from \nl\ to a non-\nl\ value), so blocks are permanently added to nodes.
Only a \op{CAS} at line \ref{incrementHead} can update a node's \head\ field by incrementing it,
which implies the following.

\begin{observation} \label{nonDecreasingHead}
For each node \var{v},  \var{v}.\fld{head} is non-decreasing over time.
\end{observation}

\begin{observation} \label{lem::headInc}
Let $R$ be an instance of \opa{Refresh}{v} whose call to \op{CreateBlock} returns a non-\nl\ block.  When $R$ terminates, \var{v}.\head\ is strictly greater than the value $R$ reads from it at line \ref{readHead}.
\end{observation}
\begin{proof}
After $R$'s \op{CAS} at line \ref{incrementHead}, \var{v}.\head\ is no longer equal to the value \var{h}
read at line \ref{readHead}.  The claim follows from \Cref{nonDecreasingHead}.
\end{proof}

Now we show $\var{v}.\fld{blocks}[\var{v}.\head]$ is either the last non-\nl\ block or the first \nl\ block in node $\var{v}$.

\begin{invariant}\label{lem::headPosition} 
For $0 \leq i < \var{v}.\head$, $\var{v}.\fld{blocks}[i]\neq\nl$.  For $i>\var{v}.\head$, $\var{v}.\fld{blocks}[i]=\nl$.
If $\var{v}\neq \var{root}$,  $\var{v}.\fld{blocks}[i].super \neq \nl$ for $0<i<\var{v}.\head$.
\end{invariant}

\begin{proof}
Initially, $\var{v}.\head=1$, $\var{v}.\fld{blocks}[0]\neq\nl$  and $\var{v}.\fld{blocks}[i]=\nl$ for  $i>0$, so the claims~hold.

Assume the claims hold before a change to $\var{v}.\fld{blocks}$, which can be made only
by a successful \op{CAS} at line \ref{cas}.
The \op{CAS} changes $\var{v}.\fld{blocks}[h]$ from \nl\ to a non-\nl\ value.
Since $\var{v}.\fld{blocks}[h]$ is \nl\ before the CAS, $\var{v}.\head \leq h$ by the hypothesis.
Since $h$ was read from $\var{v}.\head$ earlier at line \ref{readHead}, the current value of 
$\var{v}.\head$ is at least $h$ by \Cref{nonDecreasingHead}.
So, $\var{v}.\head=h$ when the \op{CAS} occurs and a change to $\var{v}.\fld{blocks}[\var{v}.\head]$ preserves the invariant.

Now, assume the claim holds before a change to $\var{v}.\head$, which can only be an increment from $h$ to $h+1$
by a successful \op{CAS} at line \ref{incrementHead} of \op{Advance}.
For the first two claims, it suffices to show that $\var{v}.\fld{blocks}[head] \neq \nl$.
\nf{Advance} is called either at line \ref{helpAdvance} 
after testing that $\var{v}.\fld{blocks}[h]\neq\nl$ at line \ref{ifHeadnotNull},
or at line \ref{advance} after the \op{CAS} at line \ref{cas} ensures $\var{v}.\fld{blocks}[h]\neq\nl$.
For the third claim, observe that prior to incrementing $\var{v}.\head$ to $i+1$ at line \ref{incrementHead},
the \op{CAS} at line \ref{setSuper1} ensures that $\var{v}.\fld{blocks}[i].super\neq \nl$.
\end{proof}

It follows that blocks accessed by the \op{Enqueue}, \op{Dequeue} and \op{CreateBlock} routines are non-\nl.

The following two lemmas show that no operation appears in more than one block of the root.
\begin{lemma} \label{lem::headProgress}
 If $b>0$ and $\var{v}.\fld{blocks}[b] \neq \nl$, then
 \begin{eqnarray*}
 \var{v}.\fld{blocks}[b-1].\fld{end\sub{left}} &\leq& \var{v}.\fld{blocks}[b].\fld{end\sub{left}} \mbox{ and}\\
 \var{v}.\fld{blocks}[b-1].\fld{end\sub{right}} &\leq& \var{v}.\fld{blocks}[b].\fld{end\sub{right}}.
 \end{eqnarray*}
\end{lemma}
\begin{proof}
Let $B$ be the block in $\var{v}.\fld{blocks}[b]$.
Before creating $B$ at line \ref{invokeCreateBlock}, the \op{Refresh} that installed $B$
read $b$ from $\var{v}.\head$ at line \ref{readHead}.
At that time, $\var{v}.\fld{blocks}[b-1]$ contained a block $B'$, by \Cref{lem::headPosition}.
Thus, the \op{CreateBlock}($\var{v},b-1$) that created $B'$ terminated before the \op{CreateBlock}($\var{v},b$) that
created $B$ started.
It follows from \Cref{nonDecreasingHead} that the value that 
line \ref{createEndLeft} of \op{CreateBlock}($\var{v},b-1$) stores in $B'.\fld{end\sub{left}}$   
is less than or equal to the value that line \ref{createEndLeft} of \op{CreateBlock}($\var{v},b$) 
stores in $B.\fld{end\sub{left}}$.
Similarly, the values stored in $B'.\eright$ and $B.\eright$ at line \ref{createEndRight} 
%of these calls to \op{CreateBlock} 
satisfy the claim.
\end{proof}

\begin{lemma} \label{lem::subblocksDistinct}
If $B$ and $B'$ are two blocks in nodes at the same depth, their sets of subblocks are disjoint.
\end{lemma}
\begin{proof}
We prove the lemma by reverse induction on the depth.
If $B$ and $B'$ are in leaves, they have no subblocks, so the claim holds.
Assume the claim holds for nodes at depth $d+1$ and let $B$ and $B'$ be two blocks in nodes at depth $d$.
Consider the direct subblocks of $B$ and $B'$ defined by~(\ref{defsubblock}).
If $B$ and $B'$ are in different nodes at depth $d$, then their direct subblocks are disjoint.
If $B$ and $B'$ are in the same node, it follows from \Cref{lem::headProgress} that their direct subblocks are disjoint.
Either way, their direct subblocks (at depth $d+1$) are disjoint, so the claim follows from the induction hypothesis.
\end{proof}

It follows that each block has at most one superblock.
Moreover, we can now prove each operation is contained in at most one block of each node,
and hence appears at most once in the linearization~$L$.

\here{Might be able to get rid of this corollary and just cite previous lemma instead to save space}
\begin{corollary}\label{lem::noDuplicates}
For  $i\neq j$, $\var{v}.\fld{blocks}[i]$ and $\var{v}.\fld{blocks}[j]$ cannot both contain the same operation.
\end{corollary}
\begin{proof}
A block $B$ contains the operations in $B$'s subblocks in leaves of the tree.
An operation by process $P$ appears in just one block of $P$'s leaf, so
an operation 
cannot be in two different leaf blocks. 
By \Cref{lem::subblocksDistinct}, $\var{v}.\fld{blocks}[i]$ and $\var{v}.\fld{blocks}[j]$ have no common subblocks, so the claim follows.
\end{proof}



%\begin{definition}
%$n\nf{.blocks[}i\nf{]}$ is \emph{established} if $n\nf{.head}>i$. An operation is \it{established} in node $n$ 
%if it is in an established block of $n$. $EST^t_n$ is the set of established operations in node $n$ at time $t$.
%\end{definition}
%
%Now we want to say that blocks of a node grow over time.
%\begin{observation}\label{lem::blocksOrder}
%  If  time $t<$ time $t^\prime$ ($t$ is before $t^\prime$), then $ops(n.blocks)$ at time $t$ is a subset of 
%$ops(n.blocks)$ at time $t^\prime$.
%\end{observation}
%\begin{proof}
%Blocks are only appended (not modified) with \nf{CAS} to $n\nf{.blocks[}n\nf{.head]}$, so the set of the blocks of a node after the \nf{CAS} contains the set of the blocks before the \nf{CAS}.
%\end{proof}

% \begin{corollary}\label{lem::establishedOrder}
%   If  time $t<$ time $t^\prime$, then $EST_n^t\subseteq EST_n^{t^\prime}$.
% \end{corollary}
% \begin{proof}
% From Observations \ref{nonDecreasingHead}, \ref{lem::blocksOrder}.  
% \end{proof}

The accuracy of the values stored in the \fld{sum\sub{enq}} and \fld{sum\sub{deq}} fields
on lines \ref{enqNew}, \ref{deqNew}, \ref{createSumEnq} and \ref{createSumDeq} follows easily
from the definition of subblocks.  See Appendix 
\ref{app::tracingDetails} for a detailed proof of \Cref{lem::sum}.

\begin{restatable}{invariant}{sumRes}
\label{lem::sum}
If $B$ is a block stored in $\var{v}.\fld{blocks}[i]$, then
\begin{eqnarray*}
B.\fld{sum\sub{enq}} &=& | E(\var{v}.\fld{blocks}[0])\cdots E(\var{v}.\fld{blocks}[i]) | \mbox{ and}\\
B.\fld{sum\sub{deq}} &=& | D(\var{v}.\fld{blocks}[0])\cdots D(\var{v}.\fld{blocks}[i]) |.
\end{eqnarray*}
\end{restatable}

This allows us to prove that every block a \op{Refresh} installs contains at least one operation.

\begin{corollary}\label{blockNotEmpty}
If a block $B$ is in $\var{v}.\fld{blocks}[i]$ where $i>0$, then $E(B)$ and $D(B)$ are not both empty.
\end{corollary}
\begin{proof}
The \op{Refresh} that installed $B$ got $B$ as the response to its call to \op{CreateBlock} on line \ref{invokeCreateBlock}.
Thus, at line \ref{testEmpty}, $\var{num\sub{enq}}+\var{num\sub{deq}}\neq 0$.
By \Cref{lem::sum}, $\var{num\sub{enq}} = |E(B)|$ and $\var{num\sub{deq}} = |D(B)|$,
so these sequences cannot both be empty.
\end{proof}



\subsection{Propagating Operations to the Root}
\label{sec::propagating}

Next, we show two \op{Refresh}es suffice to propagate operations from a child to its parent.
We say that node $\var{v}$ \emph{contains} an operation $op$ if some block in $\var{v}.\fld{blocks}$ contains $op$.
%\here{move this defn earlier and just recall it here?}
Since blocks are permanently added to nodes, if $\var{v}$ contains $op$ at some time, $v$ contains $op$ at all later times too.

\begin{lemma}\label{successfulRefresh}
Let $R$ be a call to \op{Refresh}($\var{v}$) that performs a successful \op{CAS} on line \ref{cas} (or terminates at line \ref{addOP}).
After that CAS (or termination, respectively), $\var{v}$ contains all operations that $\var{v}$'s children contained 
when $R$ executed line~\ref{readHead}.
\end{lemma}
\begin{proof}
Suppose $\var{v}$'s child (without loss of generality, $\var{v}.\fld{left}$) contained an operation $op$ 
when $R$ executed line \ref{readHead}.
Let $i$ be the index such that the block $B=\var{v}.\fld{left.blocks}[i]$ contains $op$.
By \Cref{nonDecreasingHead} and \Cref{lem::headProgress}, the value of $childHead$ that $R$ reads from
$\var{v}.\fld{left.head}$ in line \ref{readChildHead} is at least $i$.
If it is equal to $i$, $R$ calls \op{Advance} at line \ref{helpAdvance}, which ensures that 
$\var{v}.\fld{left.head} > i$.
Then, $R$ calls \op{CreateBlock}($\var{v},h$) in line \ref{invokeCreateBlock}, where $h$ is the value $R$ reads at line \ref{readHead}.
\op{CreateBlock} reads a value greater than $i$ from $\var{v}.\fld{left.head}$ at line \ref{createEndLeft}.
Thus, $new.\eleft \geq i$.  We consider two cases.

Suppose $R$'s call to \op{CreateBlock} returns the new block $B'$ and $R$'s \op{CAS} at line \ref{cas} 
installs $B'$ in $\var{v}.\fld{blocks}$.
Then, $B$ is a subblock of some block in $\var{v}$, since  $B'.\eleft$ is greater than or equal to $B$'s index
$i$ in $\var{v}.\fld{left.blocks}$.
Hence $\var{v}$ contains $op$, as required.

%\Eric{I think this case was missing from the proof in the thesis}
Now suppose $R$'s call to \op{CreateBlock} returns \nl, causing $R$ to terminate at line \ref{addOP}.
Intuitively, since there are no operations in $\var{v}$'s children to promote, $op$ is already in $\var{v}$.
We formalize this intuition.
The value computed at line \ref{createSumEnq} is
\begin{eqnarray*}
\var{num\sub{enq}} \hspace*{-2mm}
&=& \hspace*{-2mm} \var{v}.\fld{left.blocks}[new.\eleft].\fld{sum\sub{enq}} \\
&& + \var{v}.\fld{right.blocks}[new.\eright].\fld{sum\sub{enq}}\\
&&  - \var{v}.\fld{blocks}[h-1].\fld{sum\sub{enq}} \\
&=& \hspace*{-2mm}\var{v}.\fld{left.blocks}[new.\eleft].\fld{sum\sub{enq}}\\
&& + \var{v}.\fld{right.blocks}[new.\eright].\fld{sum\sub{enq}} \\
&&- \var{v}.\fld{left.blocks}[\var{v}.\fld{blocks}[h-1].\eleft].\fld{sum\sub{enq}} \\
&& - \var{v}.\fld{right.blocks}[\var{v}.\fld{blocks}[h-1].\eright].\fld{sum\sub{enq}}.
\end{eqnarray*}
It follows from \Cref{lem::sum} that $num\sub{enq}$ is the total  number of enqueues  in 
$\var{v}.\fld{left.blocks}[\var{v}.\fld{blocks}[h-1].\eleft+1..new.\eleft]$ and
$\var{v}.\fld{right.blocks}[\var{v}.\fld{blocks}[h-1].\eright+1..new.\eright]$.
Similarly, $num\sub{deq}$ is the total number of dequeues contained in these blocks.
Since $num\sub{enq}+num\sub{deq}=0$ at line \ref{testEmpty},
these blocks contain no operations.
By \Cref{blockNotEmpty}, this means the ranges of blocks are empty, so that $\var{v}.\fld{blocks}[h-1].\eleft \geq \var{new}.\eleft \geq i$.
Hence, $B$ is already a subblock of some block in $\var{v}$, so $\var{v}$ contains $op$.
\end{proof}

We now show  a double \op{Refresh} propagates blocks as required.

\begin{lemma}\label{lem::doubleRefresh}
Consider two consecutive terminating calls $R_1$, $R_2$ to \op{Refresh}($\var{v}$) by the same process.
All operations contained $\var{v}$'s children when $R_1$ begins
are contained in $\var{v}$ when $R_2$ terminates.
\end{lemma}
\begin{proof}
If either $R_1$ or $R_2$ performs a successful \op{CAS} at line \ref{cas} or terminates at line \ref{addOP}, the claim follows
from \Cref{successfulRefresh}.
So suppose both $R_1$ and $R_2$ perform a failed \op{CAS} at line \ref{cas}.
Let $h_1$ and $h_2$ be the values $R_1$ and $R_2$ read from $\var{v}.\head$ at line \ref{readHead}.
By \Cref{lem::headInc}, $h_2>h_1$.
By \Cref{lem::headProgress}, $\var{v}.blocks[h_2]=\nl$ when $R_1$ executes line \ref{readHead}.
Since $R_2$ fails its \op{CAS} on $\var{v}.blocks[h_2]$, some other \op{Refresh} $R_3$ must have done
a successful \op{CAS} on $\var{v}.blocks[h_2]$ before $R_2$'s \op{CAS}.
$R_3$ must have executed line \ref{readHead} after $R_1$, since $R_3$ read the value $h_2$ from $\var{v}.\head$ and the value of $\var{v}.\head$ is non-decreasing, by \Cref{nonDecreasingHead}.
Thus, all operations contained in $\var{v}$'s children when $R_1$ begins
are also contained in $\var{v}$'s children when $R_3$ later executes line \ref{readHead}.
By \Cref{successfulRefresh}, these operations are contained in $\var{v}$ when $R_3$ performs its successful \op{CAS},
which is before $R_2$'s failed \op{CAS}.
\end{proof}

\begin{lemma} \label{lem::appendExactlyOnce}
When an \op{Append}($B$) terminates, $B$'s operation is contained in exactly one block in each node along the path from the process's leaf to the root.
\end{lemma}
\begin{proof}
\op{Append} adds $B$ to the process's leaf and calls \op{Propagate}, which
does a double \op{Refresh}~on each internal node on the path $P$ from the leaf to the root.
By \Cref{lem::doubleRefresh}, this ensures a block in each node on $P$ contains $B$'s operation.
There is at most one such block in each node, by \Cref{lem::noDuplicates}.
\end{proof}


\subsection{Correctness of \opemph{GetEnqueue} and \opemph{IndexDequeue}}
\label{sec::tracingCorrect}

See Appendix \ref{app::tracingDetails} for detailed proofs for this section.

We first show the \fld{super} field is accurate, since \op{IndexDequeue} uses it to trace superblocks up the tree.  This is proved by showing that the \fld{super} field of a block $B$ in node $\var{v}$ is read from  
\var{v}.\fld{parent}'s \var{head} field close to the time that $B$'s superblock $B_s$ is installed in the parent node.
On one hand, $B.super$ is written before $B_s$ is installed:
\op{Advance} writes $B.super$ before advancing $\var{v}.\head$  past $B$'s index, which must happen before the \op{CreateBlock} that creates
$B_s$ gets the value of $B_s.\eleft$ or $B_s.\eright$.
On the other hand, $B.\fld{super}$ cannot be written too long before $B_s$ is installed:
$B.\fld{super}$ is written after $B$ is installed, and \Cref{successfulRefresh} ensures that $B$ is propagated to the
parent soon after.

\begin{restatable}{lemma}{superRelationRes}
\label{superRelation}
Let $B=\var{v}.\var{blocks}[b]$.
  If $\var{v}.\fld{parent.blocks}[s]$ is the superblock of $B$ then $s-1\leq B.\fld{super}\leq s$.
\end{restatable}


%The reader may wonder when the case $b\nf{.super}=s$ happens. This can happen when $
%\nf{$n$.parent.blocks[$B$.super]}=\nf{null}$ when $B$\nf{.super} is written and $R_p$ puts its created block 
%into \nf{$n$.parent.blocks[$B$\nf{.super}]} afterwards.
To show  \op{GetEnqueue} and \op{IndexDequeue} work correctly, we  just  check that they correctly compute the index of the required block
and the operation's rank within the block.  
For \op{IndexDequeue}, we use \Cref{superRelation} each time \op{IndexDequeue} goes one step up the tree.
\here{mention somewhere why precondition of IndexDequeue is true? also check that block of parent indexed by sup in that routine is non-null}

\begin{restatable}{lemma}{indexDequeueRes}
\label{lem::indexDequeue}
If $\var{v}.\fld{blocks}[b]$ has been propagated to the root and $1\leq i\leq |D(\var{v}.\fld{blocks}[b])|$, 
 then \op{IndexDequeue}($\var{v}, b, i$) returns $\langle b',i' \rangle$ such that the \var{i}th dequeue in $D(\var{v}.\fld{blocks}[\var{b}])$ is the $(i')$th dequeue of $D(\var{root}.\fld{blocks}[b'])$.
\end{restatable}

\begin{restatable}{lemma}{getEnqRes}
\label{lem::get}
If $1\leq i\leq |E(\var{v}.\fld{blocks}[b])|$ then \op{getEnqueue}($\var{v},b,i$) returns the argument of the $i$th enqueue in $E(\var{v}.\fld{blocks}[b])$.
\end{restatable}

\subsection{Linearizability}
\label{sec::linearizability}

We show that the linearization ordering $L$ defined in 
(\ref{linearization}) is a legal permutation of a subset of the operations in 
the execution, i.e., that it includes all operations that terminate and 
if one operation $op_1$ terminates before another operation $op_2$ begins, then $op_2$ does not precede $op_1$ in $L$.  We also show the result each dequeue returns is the same as in the sequential execution  $L$.

\begin{lemma} \label{linearSat}
$L$ is a legal linearization ordering.
\end{lemma}
\begin{proof}
By \Cref{lem::noDuplicates}, $L$ is a permutation of a subset of the operations in the execution.
By \Cref{lem::appendExactlyOnce}, each terminating operation is propagated to the root before it terminates,
so it appears in $L$.
Also, if $op_{1}$ terminates before $op_{2}$ begins, then $op_{1}$ 
is propagated to the root before $op_2$ begins, so $op_1$ appears before $op_2$ in $L$.
\end{proof}

A simple proof (in Appendix \ref{app::tracingDetails}) shows that \fld{size} fields are computed correctly.
\begin{restatable}{lemma}{sizeCorrectRes}
\label{sizeCorrectness}
If the operations of $\var{root}.\fld{blocks}[0..b]$ are applied sequentially in the order of~$L$ on an initially empty queue, the resulting queue has $\var{root}.\fld{blocks}[b].\size$ elements.  
\end{restatable}

Next, we show operations return the same response as they would in the sequential execution $L$.

\begin{lemma}\label{linearCorrect}
Each terminating dequeue returns the response it would in the sequential execution $L$.
\end{lemma}
\begin{proof}
If a dequeue $D$ terminates, it is contained in some block in the root, by \Cref{lem::appendExactlyOnce}.
By \Cref{lem::indexDequeue}, $D$'s call to \op{IndexDequeue} on line \ref{invokeIndexDequeue}
returns a pair $\langle b,i\rangle$ such that $D$ is the $i$th dequeue in the block 
$B=\var{root}.\fld{blocks}[b]$.
$D$ then calls \op{FindResponse}($b,i$) on line \ref{deqRest}.
By \Cref{sizeCorrectness}, the queue contains $\var{root}.\fld{blocks}[b-1].\fld{size}$ elements
after the operations in $\var{root}.\fld{blocks}[1..b-1]$ are performed sequentially 
in the order given by $L$.
By \Cref{lem::sum}, the value of \var{num\sub{enq}} computed on line \ref{FRNum}
is the number of enqueues in $B$.
Since the enqueues in block $B$ precede the dequeues,
the queue is empty when the $i$th dequeue of $B$ occurs if 
$\var{root}.\fld{blocks}[b-1].\fld{size} + \var{num\sub{enq}} < i$.
So $D$ returns \nl\ on line \ref{returnNull} if and only if it would do so in the sequential
execution $L$.
Otherwise, the size of the queue after doing the operations in $\var{root}.\fld{blocks}[0..b-1]$
in the sequential execution $L$ is $\var{root}.\fld{blocks}[b-1].\fld{sum\sub{enq}}$ minus
the number of non-\nl\ dequeues in that prefix of $L$.
Hence, line \ref{computeE} sets $e$ to the rank of $D$ among all the non-\nl\ dequeues in $L$.
Thus, in the sequential execution~$L$, $D$ returns the value enqueued by the $e$th enqueue in $L$.
By \Cref{lem::sum}, this enqueue is the $i_e$th enqueue 
in $E(\var{root}.\fld{blocks}[b_e])$, where
$b_e$ and $i_e$ are the values $D$ computes on line \ref{FRb} and \ref{FRi}.
By \Cref{lem::get}, the call to \op{GetEnqueue} returns the argument of the required enqueue.
\end{proof}

Combining \Cref{linearSat} and \Cref{linearCorrect} provides our main result.

\begin{mytheorem}
The queue implementation is linearizable.
\end{mytheorem}


\here{Should there be a lemma somewhere in this section that explicitly shows that each
object we dereference is non-null?}

% !TEX root =  podc-submission.tex

\section{Analysis}
In this section, we analyze the number of \nf{CAS} invocations and the step complexity of our algorithm.


First, we prove some claims about the size and operations of a block. These lemmas will be used later for the correctness and analysis of \nf{GetEnqueue()}.

\begin{lemma}\label{blockSize}
Each block $B$ in each node contains at most one operation of each process.
If $c$ is the execution's maximum point contention, $B$ has at most $c$ direct~subblocks.
\end{lemma}
\begin{proof}
Suppose $B$ contains an operation of process $p$.
Let $op$ be the earliest operation by $p$ contained in $B$.
When $op$ terminates, $op$ is contained in $B$ by \Cref{lem::appendExactlyOnce}.
Thus, $B$ cannot contain any later operations by $p$, since $B$ is created before
those operations are invoked.

Let $t$ be the earliest termination of any operation contained in $B$.
By \Cref{lem::appendExactlyOnce}, $B$ is created before $t$, so all operations contained in $B$
are invoked before $t$.  Thus, they are all concurrently running at $t$, so there are at most
$c$ operations contained in $B$.
By definition, the direct subblocks of $B$ contain these $c$ operations, and each operation is contained
in exactly one of these subblocks, by Lemma \ref{lem::subblocksDistinct}.
By \Cref{blockNotEmpty}, each direct subblock of $B$ contains at least one operation,
so there are at most $c$ direct subblocks.
\end{proof}

\begin{proposition}
An \nf{Enqueue} or \nf{Dequeue} operation does at most $14\log p$ \nf{CAS} operations.
\end{proposition}
\begin{proof}
  In each level of the tree  \nf{Refresh} is invoked at most two times, and every \nf{Refresh} invokes at most seven \nf{CAS}es, one in Line \ref{cas} and two from each \nf{Advance} in Line \ref{helpAdvance} or \ref{advance}.
\end{proof}

\begin{lemma}[\nf{DoublingSearch} Analysis]\label{dsearchTime}
If the \nf{element} enqueued by $E_i(root,b)=E_e(root)$ is the response to some \nf{Dequeue} operation in \nf{root.blocks[$end$]}, then \nf{DoublingSearch($e$, $end$)} takes $O\big(\log ( \nf{root.blocks[$b$].size}+ \nf{root.blocks[$end$].size})\big )$ steps.
\end{lemma}
\begin{proof}
First we show $end - b -1\leq 2 \times \nf{root.blocks[$b-1$].size}+\nf{root.blocks[$end$].size}$. There can be at most \nf{root.blocks[$b$].size}  \nf{Dequeue}s in \nf{root.blocks[$b+1\cdots end-1$]}; otherwise all elements enqueued by \nf{root.blocks[$b$]} would be dequeued before \nf{root.blocks[$end$]}. Furthermore, in the execution of queue operations in the linearization ordering, the size of the queue becomes \nf{root.blocks[$end$].size} after the operations of \nf{root.blocks[$end$]}. The final size of the queue after \nf{root.blocks[$1\cdots end$]} is \nf{root.blocks[$end$].size}. After an execution on a queue, the $size$ of the queue is greater than or equal to $\#enqueues -\#dequeues$ in the execution. We know the number of dequeues in \nf{root.blocks[$b+1\cdots end-1$]} is less than \nf{root.blocks[$b$].size}, therefore in \nf{root.blocks[$b+1\cdots end-1$]} there cannot be more than $\nf{root.blocks[$b$].size} + \nf{root.blocks[$end$].size}$ \nf{Enqueue}s. Overall there can be at most $2 \times\nf{root.blocks[$b$].size}+ \nf{root.blocks[$end$].size}$ operations in \nf{root.blocks[$b+1\cdots end-1$]} and since from Line \ref{addOP} we know that the \nf{num} field of every block in the tree is greater than 0, each block has at least one operation, so there are at most $2 \times\nf{root.blocks[$b$].size}+ \nf{root.blocks[$end$].size}$ blocks in between \nf{root.blocks[$b$]} and \nf{root.blocks[$end$]}. So, $end-b-1\leq 2 \times\nf{root.blocks[$b$].size}+\nf{root.blocks[$end$].size}$.

Thus, the doubling search reaches \nf{start} such that the \nf{root.blocks[start].sum\sub{enq}} is less than $e$ in $O \big(\log(\nf{root.blocks[$b$].size}+\nf{root.blocks[$end$].size})\big)$ steps. See Figure \ref{fig::doubling}. After Line \ref{dsearchEnd}, the binary search that finds $b$ also takes $O\big(\log(\nf{root.blocks[$b$].size}+\nf{root.blocks[$end$].size})\big)$. Next, \nf{i} is computed via the definition of \nf{sum\sub{enq}} in constant time (Line \ref{DSearchComputei}).
\end{proof}
\begin{figure}[hbt]  
  \center\includegraphics[width=6in]{pics/doubling.png}
  \caption{Distance relations between \nf{start}$,b,end$.}
  \label{fig::doubling}
\end{figure}

\begin{lemma}[Worst Case Time Analysis] \label{enqDeqTime}
The worst case number of steps for an \nf{Enqueue} is $O(\log^2 p)$ and for a \nf{Dequeue}, is $O(\log^2 p + \log q_e+ \log q_d)$, where $q_d$ is the size of the queue when the \nf{Dequeue} is linearized and $q_e$ is the size of the queue at the time the response of the \nf{Dequeue} is linearized.
\end{lemma}
\begin{proof}
\nf{Enqueue} consists of creating a block and appending it to the tree. The first part takes constant time. To propagate the operation to the root the algorithm tries at most two \nf{Refresh}es in each node of the path from the leaf to the root (Lines \ref{firstRefresh}, \ref{secondRefresh}). We can see from the code  that each \nf{Refresh} takes a constant number of steps and does $O(1)$ \nf{CAS}es. Since the height of the tree is $\Theta(\log p)$, \nf{Enqueue} takes $O(\log p)$ steps.

A \nf{Dequeue} creates a block whose \nf{element} is \nf{null}, appends it to the tree, computes its rank among non-null dequeues, finds the corresponding enqueue and returns the response. The first two parts are similar to an \nf{Enqueue} operation and take $O(\log p)$ steps. To compute the rank of a \nf{Dequeue} in $D(n)$, the \nf{Dequeue} calls \nf{IndexDequeue()}. \nf{IndexDequeue} does $O(1)$ steps in each level which takes $O(\log p)$ steps. If the response to the \nf{Dequeue} is \nf{null}, \nf{FindResponse} returns \nf{null} in $O(1)$ steps. Otherwise, if the response to a dequeue in \nf{root.blocks[end]} is in \nf{root.blocks[b]} the \nf{DoublingSearch} takes $\Theta(\log$(\nf{root.blocks[b].size+root.blocks} \nf{[end].size}) by Lemma~\ref{dsearchTime}, which is $O(\log q_e+\log q_d)$. Each search in \nf{GetEnqueue()} takes $O(\log p)$ steps since there are at most $p$ subblocks in a block (Lemma \ref{subBlocksBound}), so \nf{GetEnqueue()} takes $O(\log^2 p)$ steps.
\end{proof}


\begin{lemma}[Amortized Worst-case Analysis]
The amortized number of steps for an \nf{Enqueue} or \nf{Dequeue} is $O(\log^2 p + \log q)$,  where $q$ is the size of the queue when the operation is linearized.
\end{lemma}
\begin{proof}
If we split the \nf{DoublingSearch} time cost between the corresponding \nf{Enqueue} and \nf{Dequeue}, each operation takes $O(\log^2 p +q)$ steps.
\end{proof}

\begin{observation}
    If the maximum number of concurrent processes at any time in an execution is $c$, then the amortized worst-case step complexity is $O(\log p\log c + \log q)$ per operations. Furthermore, in a sequential, execution where $c=1$, the step complexity of our algorithm is $\Theta(\log p + \log q)$ per operation.
\end{observation}
\begin{proof}
    The analysis is similar to the two previous Lemmas, but by Lemma \ref{subBlocksBound} each \nf{BinarySearch} in each call of \nf{GetEnqueue} takes $O(\log c)$ steps.
\end{proof}

\begin{theorem}
The queue implementation is wait-free.
\end{theorem}
\begin{proof}
To prove the claim, it is sufficient to show that every \nf{Enqueue} and \nf{Dequeue} operation terminates after a finite number of its own steps. This is directly concluded from Lemma \ref{enqDeqTime}.
\end{proof}


% !TEX root =  podc-submission.tex

\section{Reducing Space}
\label{reducing}
\paragraph{Reducing Space Usage}
The \nf{blocks} arrays defined in our algorithm are unbounded. To use $O(n)$ space in each node where $n$ is the total number of operations, instead of unbounded arrays, we could use the memory model of the wait-free vector introduced by Feldman, Valera-Leon and Damian~\cite{7073592}. We can create an array called \nf{arr} of pointers to array segments (see Figure \ref{fig::doublingArray}). When a process wishes to write into location \nf{head} it checks whether \nf{arr[$\lfloor\log $ head$\rfloor$]} points to an array or not. If not, it creates a shared array of size $2^{\lfloor\log \nf{head} \rfloor}$ and tries to \nf{CAS} a pointer to the created array into \nf{arr[$\lfloor\log $ head$\rfloor$]}. Whether the \nf{CAS} is successful or not, \nf{arr[$\lfloor\log $ head$\rfloor$]} points to an array. When a process wishes to access the $i$th element it looks up \nf{arr[$\lfloor\log i\rfloor$][$i-2^{\lfloor\log i\rfloor}$]}, which takes $O(1)$ steps. The CAS Retry Problem does not happen here because if $n$ elements are appended to the array, then only $O(p\times\log n)$ \nf{CAS} steps have happened on the array \nf{arr}. Furthermore, at most $p$ arrays with size $2^{\lfloor\log i\rfloor}$ are allocated by processes while processes try to do the \nf{CAS} on \nf{arr[$i$]}. Jayanti and Shun \cite{DBLP:conf/wdag/JayantiS21} present a way to initialize wait-free arrays in constant steps. The time taken to allocate arrays in an execution containing $n$ operations is $O(\frac{p\log n}{n})$  per operation, which is negligible if $n>>p$. The vector implementation also has a mechanism for doubling \nf{arr} when necessary, but this happens very rarely since increasing \nf{arr} from $s$ to $2s$ increases the capacity of the vector from $2^s$ to $2^{2s}$.
\begin{figure}[hbt]  
  \center\includegraphics[width=6.5in]{pics/doublingArray.png}
  \caption{Array segments.}
  \label{fig::doublingArray}
\end{figure}

\paragraph{Garbage Collection}
We did not handle garbage collection: \nf{Enqueue} operations  remain in the nodes even after their elements have been dequeued.
We can keep track of the \nf{block}s in the \nf{root} whose operations are all terminated, i.e., all enqueues have been dequeued, and the responses of all dequeues have been computed. We call these blocks \it{finished blocks}. If we help the operations of all processes to compute their responses, then we can say if block $B$ is finished, then all blocks before $B$ are also finished. Knowing the most recent finished block in a node, we can reclaim the memory taken by finished blocks. We cannot use arrays (or vectors) to throw the garbage blocks away. We need a data structure that supports \nf{tryAppend()}, \nf{read(i)}, \nf{write(i)} and \nf{split(i)} operations in $O(\log n)$ time, where \nf{split(i)} removes all the indices less than~\nf{i}. If each process tries to do the garbage collection once every $p^2$ operations on the queue, then the amortized complexity remains the same. We can use a concurrent implementation of a persistent red-black trees for this~\cite{DBLP:conf/afp/Okasaki96}. Bashari and Woelfel ~\cite{DBLP:conf/podc/BashariW21} used persistent red-black trees in a similar way.

Tarjan \cite[Sec.~4.2]{Tar83} described a split algorithm for red-black trees that runs in logarithmic time.

\begin{figure}[hbt]  
  \center\includegraphics[width=3.5in]{pics/finishedBlocks.png}
  \caption[Blocks that can be safely garbage collected.]{Finished blocks are shown with red color and unfinished blocks are shown with green color. All the subblocks of a finished block are also finished.}
  \label{fig::finishedBlock}
\end{figure}


% !TEX root =  podc-submission.tex

\section{Future Directions}

Our focus was on optimizing step complexity for worst-case executions.
However, our queue has a higher cost than the MS-queue in the best case (when an operation
runs by itself).
Perhaps our queue could be made adaptive by having an operation capture a starting node
in the \ordering\ tree (as in \cite{DBLP:conf/stoc/AfekDT95}) rather than starting at a statically assigned leaf.
A possible application of our queue  might be to use it as the slow path in the
fast-path slow-path methodology  \cite{10.1145/2370036.2145835} to
get a queue that has good performance in practice while also having good worst-case step complexity.

There is a gap between our implementation, which takes $O(\log^2 p + \log q)$ steps per operation,
and Attiya and Fouren's $\Omega(\min(c,\log\log p))$ lower bound \cite{DBLP:conf/opodis/AttiyaF17}.
It would be interesting to determine how the true step complexity of lock-free queues (or, more generally, bags)
depends on $p$.
Since a queue is also a bag, it is the first lock-free bag we know of that has polylogarithmic step complexity.

We believe the approach used here to implement a lock-free queue 
could be applied to obtain other lock-free
data structures with a polylogarithmic step complexity.
For example, we can easily adapt our routines to implement a  vector data structure that stores a sequence and
provides three operations: \opa{Append}{e} to add an element \var{e} to the end of the sequence,
\opa{Get}{i} to read the \var{i}th element in the sequence, and
\opa{Index}{e} to compute the position of element \var{e} in the sequence.
%\here{Can omit rest of this paragraph to save space}
%An \opa{append}{e} is implemented like \opa{Enqueue}{e} in $O(\log p)$ steps.  
%A \opa{get}{i} is similar to \op{GetEnqueue}, taking $O(\log n + \log^2p)$ steps when the vector has $n$ elements.  
%An \opa{index}{e} is similar to \op{IndexDequeue} (except operating on enqueues instead of dequeues) and would take $O(\log p)$ steps.
\here{doublecheck all preceding claims for accuracy.}
We would like to investigate whether a similar approach could be used for stacks, deques or even priority queues.



\newpage

\bibliographystyle{plain}
\bibliography{queues.bib}

\appendix
\end{document}
