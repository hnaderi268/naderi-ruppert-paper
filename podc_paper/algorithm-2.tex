% !TEX root =  podc-submission.tex

\section{Detailed Description of the Bounded-Space Implementation}
\label{reducing-details}


\renewcommand{\algorithmiccomment}[1]{\hfill\eqparbox{COMMENTSINGLEAPP}{\com\ #1}}

Here, we give more details of the bounded-space construction sketched in Section \ref{reducing}.
We first describe the modifications to the implementation.
To avoid confusion, we use nodes to refer to the nodes of the ordering tree, and blocks to refer
to the nodes of a RBT (since the RBT stores blocks).

The space-bounded implementation uses an additional shared array so that processes can access one another's leaves to perform helping.
\begin{itemize}
\item \typ{Node[]} \var{leaf}$[1..p]$ \Comment{\var{leaf}$[k]$ is the leaf assigned to process $k$}
\end{itemize}
Each \typ{Block} has an additional field.
\begin{itemize}
\item \typ{int} \var{index} \Comment{position this block would have in the \fld{blocks} array}
\end{itemize}
To facilitate helping, each \typ{Block} in a leaf has one more additional field.
\begin{itemize}
\item \typ{Object} \var{response} \Comment{response of the operation in the block, if it is a dequeue}
\end{itemize}
Each \typ{Node} has an additional array field to help determine which of its blocks are safe to discard.
\begin{itemize}
\item \typ{int[]} \var{last}$[1..p]$ \Comment{\var{last}$[k]$ is the largest \fld{index} of a block in the node that process $k$ }\\
\mbox{ }\Comment{has accessed, either to find the enqueued item for its dequeue}\\
\mbox{ }\Comment{or during a \nl\ dequeue}
\end{itemize}
The \fld{blocks} field of each node in the \ordering\ tree is implemented as a pointer to the root of a RBT of \typ{Blocks} rather than an infinite array.  
Each RBT is initialized with an empty block with index 0.
The \head\ field, which previously gave the next position to insert into the \fld{blocks} array is no
longer needed; we can instead find the maximum \fld{index} of any block in the RBT.
To facilitate this, \op{MaxBlock} is a query operation on the RBT that 
returns the block with the maximum \fld{index}.
We can store, in the root of the RBT, a pointer to the maximum block so that \op{MaxBlock}
can be done in constant time, without affecting the time of other RBT operations.

\here{Talk about what happens if a block is not found in an RBT}
\here{Talk about \emph{marking} a block to indicate that any blocks with smaller indices will no longer
be needed by operations that haven't started yet.}

Pseudocode for the space-bounded implementation appears in Figure \ref{pseudocode1GC}--\ref{pseudocode3GC}.
The \op{Propagate} and \op{FindResponse} routines are unchanged.
Minor additions to \op{IndexDequeue} and \op{GetEnqueue} update the \var{last} field of nodes
visited by the routines to mark the blocks they visit in those nodes.
Only minor modifications have been made to \op{Enqueue}, \op{Append}, \op{CreateBlock}
to accomodate the switch from an array of blocks to a RBT of blocks (and the corresponding disappearance
of the \head\ field).
\op{Dequeue} has been similarly modified, and the second half of the routine is now in a
separate routine called \op{CompleteDequeue} so that it can also be used by other processes
helping to complete the operation.

Once a dequeue has been propagated to the root, 
\bf{\op CompleteDequeue} computes its answer.  This routine can also be called by helpers 
who want to help complete the dequeue.
In addition to calling \op{FindResponse} and \op{IndexDequeue} to compute the response,
it also updates \var{last} entries if the computed response is \nl.
It does this by calling \op{IndexDequeue} again, this time with \tr\ as its final parameter.
This extra parameter has been added to 
\op{IndexDequeue} to indicate whether it should mark blocks that it traverses (which all contain the dequeue).

\here{Discuss new routines AddBlock, SetSuper, Help, and largely modified Refresh}


\here{does RBT have to have pointer to block rather than block itself so that updates to super field work correctly?}

\renewcommand{\algorithmiccomment}[1]{\hfill\eqparbox{COMMENTDOUBLE}{\com\ #1}}

\begin{figure}
\begin{minipage}[t]{0.465\textwidth}
\begin{algorithmic}[1]
\setcounter{ALG@line}{0}

\Function{void}{Enqueue}{\typ{Object} \var{e}} 
    \gc{
    \State \var{h} \assign \Call{MaxBlock}{\var{leaf}.\fld{blocks}}.\fld{index}+1
	\State \hangbox{let \var{B} be a new \typ{Block} with \fld{element} \assign\ \var{e},\\
		$\fld{sum\sub{enq}} \assign\ \var{leaf}.\fld{blocks}[h-1].\fld{sum\sub{enq}}+1,$\\
		$\fld{sum\sub{deq}} \assign\ \var{leaf}.\fld{blocks}[h-1].\fld{sum\sub{deq}}$,\\
		$\fld{index}\assign h$}\label{enqNewGC}
	}
    \State \Call{Append}{\var{B}}
\EndFunction{Enqueue}

\spac

\Function{Object}{Dequeue()}{} 
    \gc{
    \State \var{h} \assign \Call{MaxBlock}{\var{leaf}.\fld{blocks}}.\fld{index}+1
    \State \hangbox{let \var{B} be a new \typ{Block} with \fld{element} \assign\ \nl,\\
	    $\fld{sum\sub{enq}} \assign\ \var{leaf}.\fld{blocks}[h-1].\fld{sum\sub{enq}},$\\
	    $\fld{sum\sub{deq}} \assign\ \var{leaf}.\fld{blocks}[h-1].\fld{sum\sub{deq}}+1$,\\
	    $\fld{index}\assign h$}\label{deqNewGC}
	}
    \State \Call{Append}{\var{B}}
    \gc{\State \Return \Call{CompleteDequeue}{\var{leaf}, $h$}}
\EndFunction{Dequeue}

\spac

\gc{
\Function{Object}{CompleteDequeue}{\var{leaf}, $h$}
    \State \linecomment Finish propagated dequeue in $\var{leaf}.\fld{blocks}[h]$
    \State $\langle \var{b}, \var{i}\rangle \assign\ \Call{IndexDequeue}{\var{leaf}, h, 1, \fa}$
    \State $\var{res} \assign \Call{FindResponse}{b, i}$
	\If{$\var{res} = \nl$} \linecomment{update \var{last} values}
		\State $\Call{IndexDequeue}{\var{leaf}, h, 1, \tr}$
	\EndIf
	\State \Return{\var{res}}
\EndFunction{CompleteDequeue}
}

\spac

\Function{void}{Append}{\typ{Block} \var{B}} 
    \State \linecomment{append block to leaf and propagate to root}
    \gc{\State \var{leaf}.\fld{blocks} \assign \Call{AddBlock}{\var{leaf}, \var{leaf}.\fld{blocks}, \var{B}}\label{appendLeafGC}}
    \State \Call{Propagate}{\var{leaf}.\fld{parent}} 
\EndFunction{Append}

\spac

\gc{
\Function{RBT}{AddBlock}{\typ{Node} \var{v}, \typ{RBT} \var{T}, \typ{Block} \var{B}}
    \State \linecomment{add block $\var{B}\neq\nl$ to \var{T}, doing GC if necessary}
    \If{\var{B}.\fld{index} is a multiple of $p^2$}
    	\State \linecomment{Do garbage collection}
		\State $m \assign 0$
		\For{$k\assign 1..p$} $m\assign \max(m, v.\fld{last}[k])$
		\EndFor
        \State \Call{Help}{}
        \State $T' \assign\ \Call{Split}{T,m-1}$
        \State \linecomment{\op{Split} removes blocks with $\fld{index} < m-1$}
        \State \Return \Call{Insert}{$T',B$}
	\Else\ \Return \Call{Insert}{$T, B$}
    \EndIf
\EndFunction{AddBlock}
}
\spac

\Function{void}{Propagate}{\typ{Node} \var{v}}
    \State \linecomment{propagate blocks from \var{v}'s children to root}
    \If{\bf{not} \Call{Refresh}{\var{v}}} \label{firstRefreshGC}  \hfill \com\ double refresh
        \State \Call{Refresh}{\var{v}} \label{secondRefreshGC}
    \EndIf
    \If{$\var{v} \neq \var{root}$} \hfill \com\ recurse up tree
        \State \Call{Propagate}{\var{v}.\fld{parent}}
    \EndIf
\EndFunction{Propagate}



\end{algorithmic}
\end{minipage}
\begin{minipage}[t]{0.529\textwidth}

\begin{algorithmic}[1]
\setcounter{ALG@line}{48}
\Function{boolean}{Refresh}{\typ{Node} \var{v}}
	\State \linecomment{try to append a new block to $v.\fld{blocks}$}
	\gc{
	\State $T \assign v.\fld{blocks}$
    \State \Call{SetSuper}{$v$, $T$}
    \State $\var{h} \assign \Call{MaxBlock}{T}.\fld{index}+1$ \label{refreshReadMax}
    }
    \State \var{new} \assign\ \Call{CreateBlock}{\var{v, h}} \label{invokeCreateBlockGC}
    \If{\var{new = \nl}} \Return{\nl} \label{addOPGC}
    \Else  
    	\gc{
	    \State $T' \assign \Call{AddBlock}{v, T, \var{new}}$
		\State \Return{\Call{CAS}{\var{v}.\fld{blocks}, $T$, $T'$}}
		}
    \EndIf

\EndFunction{Refresh}

\spac
\gc{
\Function{void}{SetSuper}{\typ{Node} \var{v}, \typ{RBT} $T$} 
	\State \linecomment{set \fld{super} field of last block in $T$}
	\State \linecomment{and in $v$'s children's RBTs}
	\State $\var{lastBlock} \assign \Call{MaxBlock}{T}$
	\If{$v\neq \var{root}$}	
		\State \var{sup} \assign\ \Call{MaxBlock}{\var{v}.\fld{parent.blocks}}.\fld{index}
		\State \Call{CAS}{\var{lastBlock}.\fld{super}, \nl, \var{sup}}
	\EndIf
	\ForEach{\fld{dir} in \{\fld{left}, \fld{right}\}} \label{startHelpChild1GC}
    	\State \var{lastChildBlock} \assign\ \Call{MaxBlock}{\var{v}.\fld{dir}.\fld{blocks}}
		\State \Call{CAS}{\var{lastChildBlock}.\fld{super}, \nl, \var{lastBlock}.\fld{index}}
    \EndFor \label{endHelpChild1GC}
\EndFunction{SetSuper}
}

\spac

\Function{Block}{CreateBlock}{\typ{Node} \var{v}, \typ{int} \var{i}} 
    \State\linecomment{create new block for a \op{Refresh} to install in \var{v}.\fld{blocks}[\var{i}]}
    \State let \var{new} be a new \typ{Block} \label{initNewBlockGC}
    \gc{
    \State \var{new}.\eleft \assign\ $\Call{MaxBlock}{\var{v}.\fld{left}.\fld{blocks}}.\fld{index}$\label{createEndLeftGC}
    \State \var{new}.\eright \assign\ $\Call{MaxBlock}{\var{v}.\fld{right}.\fld{blocks}}.\fld{index}$\label{createEndRightGC}
    \State $\var{new}.\fld{index} \assign i$
    }
	\State \hangbox{\var{new}.\fld{sum\sub{enq}} \assign\ \var{v}.\fld{left.blocks}[\var{new}.\eleft].\fld{sum\sub{enq}} + \\
			\var{v}.\fld{right.blocks}[\var{new}.\eright].\fld{sum\sub{enq}}}
	\State \hangbox{\var{new}.\fld{sum\sub{deq}} \assign\ \var{v}.\fld{left.blocks}[\var{new}.\eleft].\fld{sum\sub{deq}} + \\
			\var{v}.\fld{right.blocks}[\var{new}.\eright].\fld{sum\sub{deq}}}
    \State \var{num\sub{enq}} \assign\ $\var{new}.\fld{sum\sub{enq}} - \var{v}.\fld{blocks}[\var{i}-1].\fld{sum\sub{enq}}$\label{computeNumEnqGC}
    \State \var{num\sub{deq}} \assign\ $\var{new}.\fld{sum\sub{deq}} - \var{v}.\fld{blocks}[\var{i}-1].\fld{sum\sub{deq}}$
    \If{$\var{v} = \var{root}$}
        \State \hangbox{\var{new}.\fld{size} \assign\ max(0, $\var{v}.\fld{blocks}[\var{i}-1].\size\ + \var{num\sub{enq}} - \var{num\sub{deq}}$)}\label{computeLengthGC}
    \EndIf
    \If{$\var{num\sub{enq}} + \var{num\sub{deq}} = 0$}
        \State \Return \nl \hfill \com\ no blocks need to be propagated to \var{v}\label{CBnullGC}
    \Else
        \State \Return \var{new}
    \EndIf
\EndFunction{CreateBlock}

\spac


\Function{element}{FindResponse}{\typ{int} \var{b}, \typ{int} \var{i}}
    \State \linecomment{find response to \var{i}th dequeue in $D(\var{root}.\fld{blocks}[\var{b}])$}
    \State \linecomment{Precondition:  $1\leq i\leq |D(\var{root}.\fld{blocks}[\var{b}])|$}
    \State \hangbox{\var{num\sub{enq}} \assign\ $\var{root}.\fld{blocks}[\var{b}].\fld{sum\sub{enq}} - \mbox{ }$\\
    		$\var{root}.\fld{blocks}[\var{b}-1].\fld{sum\sub{enq}}$}
    \If{$\var{root}.\fld{blocks}[\var{b}-1].\size + \var{num\sub{enq}} < \var{i}$}\label{checkEmptyGC}
        \State \Return \nl \hfill \com\ queue is empty when dequeue occurs\label{returnNullGC}
    \Else \ \linecomment{response is the \var{e}th enqueue in the root}
        \State \var{e} \assign\ \var{i} + \var{root}.\fld{blocks}[\var{b}-1].\fld{sum\sub{enq}} - 
			\var{root}.\fld{blocks}[\var{b}-1].\size\label{computeEGC}
		\State \linecomment{compute enqueue's block using binary search}
		\State find min $b_e \leq \var{b}$ with $\var{root}.\fld{blocks}[b_e].\fld{sum\sub{enq}} \geq \var{e}$
		\State \linecomment{find rank of enqueue within its block}
		\State $i_e \assign\ \var{e} - \var{root}.\fld{blocks}[b_e-1].\fld{sum\sub{enq}}$
        \State \Return \Call{GetEnqueue}{\var{root}, $b_e$, $i_e$}\label{findAnswerGC}
    \EndIf
\EndFunction{FindResponse}

\end{algorithmic}
\end{minipage}
\caption{Bounded-space queue implementation.\label{pseudocode1GC}}
\end{figure}

\begin{figure}
\begin{minipage}[t]{0.465\textwidth}
\begin{algorithmic}[1]
\setcounter{ALG@line}{48}
\gc{
\Function{void}{Help}{}
    \State \linecomment{help pending operations}
    
    \For{$\ell$ in $\var{leaf}[1..k]$}
        \State $\var{B} \assign \Call{LastBlock}{\ell.\fld{blocks}}$
    	\State \Call{Propagate}{$\ell.\fld{parent}$}
        \If{$\var{B}.\fld{element} = \nl$ and $\var{B}.\fld{index}>0$} 
            \State \linecomment{operation is a  dequeue}
            \State $\var{B}.\fld{response} \assign\ \Call{CompleteDequeue}{\ell, \var{B}.\fld{index}}$ 
        \EndIf
    \EndFor
\EndFunction{Help}
}
\end{algorithmic}
\end{minipage}
\begin{minipage}[t]{0.529\textwidth}
\begin{algorithmic}[1]
\setcounter{ALG@line}{48}

\Function{$\langle\typ{int}, \typ{int}\rangle$}{IndexDequeue}{\typ{Node} \var{v}, \typ{int} \var{b}, \typ{int} \var{i}, \gc{\typ{boolean} \var{mark}}}
    \State \linecomment{return $\langle\var{x, y}\rangle$ such that \var{i}th dequeue in $D(\var{v}.\fld{blocks}[\var{b}])$}
    \State \linecomment{is \var{y}th dequeue of $D(\var{root}.\fld{blocks}[\var{x}])$}
    \State \linecomment{Precondition: \var{v}.\fld{blocks}[\var{b}] has been propagated to root}
    \State \linecomment{and $D(\var{v}.\fld{blocks}[\var{b}])$ has at least \var{i} dequeues}
    \gc{
    \If{\var{mark} and $b>\var{v}.\fld{last}[\mbox{process id}]$}
    	\State $\var{v}.\fld{last}[\mbox{process id}] \assign b$
    \EndIf
    }
    \If{$\var{v} = \var{root}$} \Return $\langle\var{b, i}\rangle$ \label{indexBaseCaseGC}
    \Else
	    \State \fld{dir} \assign\ (\var{v}.\fld{parent.left} = \var{v} ? \fld{left} : \fld{right}) 
    	\State \var{sup} \assign\ \var{v}.\fld{blocks}[\var{b}].\fld{super}
	    \If{$\var{b} > \var{v}.\fld{parent.blocks}[\var{sup}].\fld{end\sub{dir}}$} \var{sup} \assign\ $\var{sup}+1$
	    \EndIf
	    \State \linecomment{compute index \var{i} of dequeue in superblock}
	    \State \hangbox{\var{i} += $\var{v}.\fld{blocks}[\var{b}-1].\fld{sum\sub{deq}} -$ \\
	    		$\var{v}.\fld{blocks}[\var{v}.\fld{parent.blocks}[\var{sup}-1].\edir].\fld{sum\sub{deq}}$}
        \If{$\fld{dir} = \fld{right}$} 
        	\State \hangbox{\var{i} += $\var{v}.\fld{blocks}[\var{v}.\fld{parent.blocks}[\var{sup}].\eleft].\fld{sum\sub{deq}} - \mbox{ }$\\
					$\var{v}.\fld{blocks}[\var{v}.\fld{parent.blocks}[\var{sup}-1].\eleft].\fld{sum\sub{deq}}$}\label{considerLeftBeforeRightGC}
        \EndIf \label{computeISuperEndGC}
        \State \Return\Call{IndexDequeue}{\var{v}.\fld{parent}, \var{sup}, \var{i}}
    \EndIf
\EndFunction{IndexDequeue}

\end{algorithmic}
\end{minipage}

\caption{Bounded-space queue implementation, continued.\label{pseudocode2GC}}
\end{figure}



%\begin{algorithm}
%\caption{\tt{\sl{Node}}}
%\begin{algorithmic}[1]
%\setcounter{ALG@line}{25}
%
%\Statex $\leadsto$ \textsf{Precondition: \tt{blocks[start..end]} contains a block with \tt{sum\sub{enq}} greater than or equal to \tt{x}}
%\Statex \com\ \textmd{Does a binary search for~the value \tt{x} of \tt{sum\sub{enq}} field and returns the index of the leftmost block in\\
%\com\ \tt{blocks[start..end]} whose \tt{sum\sub{enq}} is $\geq$ \tt{x}}.
%\Function{int}{BinarySearch}{\sl{int} x, \sl{int} start, \sl{int} end}
% \State \Return \tt{min\{j: blocks[j].sum\sub{enq}$\geq$x\}}
%\While{\nf{start<end}}
%\State \tt{\sl{int}} \tt{mid \assign\ floor((start+end)/2)}
%\If{\nf{blocks[mid].sum\sub{enq}<x}}
%\State \nf{start \assign\ mid+1}
%\Else
%\State \nf{end \assign\ mid}
%\EndIf
%\EndWhile
%\State\Return \nf{start}
%\EndFunction{BinarySearch}
%
%\end{algorithmic}
%\end{algorithm}

%\begin{algorithm}
%\caption{\tt{\sl{Root}}}
%\begin{algorithmic}[1]
%\setcounter{ALG@line}{36}
%\Statex
%\Statex $\leadsto$ \textsf{Precondition: \tt{root.blocks[end].sum\sub{enq} $\geq$ \tt{e}}}
%\Statex \com\ \textmd{Returns \tt{<b,i>} such that $E_\nf{e}(\nf{root})$ is $E_\nf{i}(\nf{root},\nf{b})$, i.e., the \nf{e}th \nf{Enqueue} in the \nf{root} is the \nf{i}th \nf{Enqueue} within \\
%\com\ the \nf{b}th block in the \nf{root}.}
%
%\Function{<int, int>}{DoublingSearch}{\sl{int} e, \sl{int} end}
%\State \tt{start \assign\ end-1} \label{dsearchStart}
%\While{\tt{root.blocks[start].sum\sub{enq}}$\geq$\tt{e}}
%\State \tt{start \assign\ max(start-(end-start), 0)} \label{doubling}
%\EndWhile \label{dsearchEnd}
%\State \tt{b \assign\ root.BinarySearch(e, start, end)} \label{dsearchBinarySearch}
%\State \tt{i \assign\ e- root.blocks[b-1].sum\sub{enq}} \label{DSearchComputei}
%\State\Return \tt{<b,i>}
%\EndFunction{DoublingSearch}
%\end{algorithmic}
%\end{algorithm}


\begin{figure}
\begin{algorithmic}[1]
\setcounter{ALG@line}{101}

\Function{element}{GetEnqueue}{\typ{Node} \var{v}, \typ{int} \var{b}, \typ{int} \var{i}} \Comment{returns argument of \var{i}th enqueue in $E(\var{v}.\fld{blocks}[\var{b}])$}
    \State \linecomment{Preconditions: $\var{i}\geq 1$ and \var{v}.\fld{blocks}[\var{b}] is non-\nl\ and contains at least \var{i} enqueues}
	\gc{
	\If{$b > \var{v}.\fld{last}[\mbox{process id}]$}
		\State $\var{v}.\fld{last}[\mbox{process id}] \assign b$
	\EndIf
	}
    \If{\var{v} is a leaf node}
        \State\Return \var{v}.\fld{blocks}[\var{b}].\fld{element} \label{getBaseCaseGC}
    \Else 
        \State \var{sum\sub{left}} \assign\ \var{v}.\fld{left.blocks}[\var{v}.\fld{blocks}[\var{b}].\eleft].\fld{sum\sub{enq}} \Comment{\#\ of enqueues in \var{v}.\fld{blocks}[1..$\var{b}$] from left child}
        \State \var{prev\sub{left}} \assign\ \var{v}.\fld{left.blocks}[\var{v}.\fld{blocks}[$\var{b}-1$].\eleft].\fld{sum\sub{enq}} \Comment{\#\ of enqueues in \var{v}.\fld{blocks}[1..$\var{b}-1$] from left child}
        \State \var{prev\sub{right}} \assign\ \var{v}.\fld{right.blocks}[\var{v}.\fld{blocks}[$\var{b}-1$].\eright].\fld{sum\sub{enq}} \Comment{\#\ of enqueues in \var{v}.\fld{blocks}[1..$\var{b}-1$] from right child}
        \If{$\var{i} \leq \var{sum\sub{left}} - \var{prev\sub{left}}$} \label{leftOrRightGC} \cmt{required enqueue is in \var{v}.\fld{left}}
            \State \fld{dir} \assign\ \fld{left}
        \Else
            \State \fld{dir} \assign\ \fld{right}
            \State $\var{i}\ \assign\ \var{i} - (\var{sum\sub{left}} - \var{prev\sub{left}})$
        \EndIf
        \State \linecomment{Use binary search to find enqueue's block in \var{v}.\fld{dir.blocks} and its rank within block}
        \State find minimum $\var{b}'$ in range [\var{v}.\fld{blocks}[\var{b}-1].\edir+1..\var{v}.\fld{blocks}[\var{b}].\edir] s.t. $\var{v}.\fld{dir.blocks}[\var{b}'].\fld{sum\sub{enq}} \geq \var{i} + \var{prev\sub{dir}}$\label{getChildGC}
        \State $\var{i}'$ \assign\ $\var{i} - (\var{v}.\fld{dir.blocks}[\var{b}'-1].\fld{sum\sub{enq}} - \var{prev\sub{dir}})$
        \State \Return\Call{GetEnqueue}{\var{v}.\fld{dir}, $\var{b}'$, $\var{i}'$}
    \EndIf
\EndFunction{GetEnqueue}

\end{algorithmic}
\caption{\label{pseudocode3GC}Bounded-space queue implementation, continued. }
\end{figure}

\subsection{Correctness}
Unfortunately, there are enough changes to the algorithm that a new proof of correctness
is required.  Its structure mirrors the proof of the original algorithm, but requires additional
reasoning to ensure GC does not interfere with other routines.

\begin{observation}\label{RBTinsertions}
When a block is inserted into an RBT $T$, its \fld{index} is one greater than the maximum \fld{index} in $T$.
\end{observation}
\begin{proof}
If line \ref{appendLeafGC} inserts a block into a leaf's RBT, then it was created at 
line \ref{enqNewGC} or \ref{deqNewGC} with \fld{index} $h+1$, where $h$ is the maximum \fld{index}
in the leaf's RBT.  When the new block is inserted, the maximum \fld{index} in the RBT is still $h$,
since only the owner of the leaf modifies the leaf's RBT.

If line \ref{refreshInsertGC} inserts a block  into an internal node's RBT $T$,
it is created by the call to \op{CreateBlock} at line \ref{invokeCreateBlockGC} and 
has \fld{index} $h$, which is set at line \ref{refreshReadMax} to  the maximum \fld{index} in $T$ plus one.
\end{proof}

\begin{observation}\label{lastinroot}
Each entry in \var{last} is the \var{index} of a block that has appeared in the root's RBT.
\end{observation}
\begin{proof}
Only line \ref{setlast1} and \ref{setlast2} in \op{FindResponse} write values into \var{last}.
The value written is either $b$ or $b_e \leq b$.
By the precondition of \op{FindResponse}, $\var{root}.\fld{blocks}[b]$ exists, so the index $b$ has appeared
in the root's RBT.
By \Cref{RBTinsertions}, all values less than $b$ have appeared in the root's RBT too.
\end{proof}

\begin{observation}
\label{edirGC} 
If $B$ is a block in the RBT of an internal node $v$, then for $\var{dir} \in \{\fld{left},\fld{right}\}$,
$B.\edir$ was read from the \fld{index} field of some node in the RBT of $v.\var{dir}$.
\end{observation}
\begin{proof}
The \eleft\ and \eright\ fields are set by reading the \fld{index} field of a node in the appropriate child
on line \ref{createEndLeftGC} and \ref{createEndRightGC}.
\end{proof}

\begin{invariant}\label{nonemptyConsec}
For each RBT $T$, 
$T$ is non-empty and the blocks of $T$ have consecutive \fld{index} fields.
\end{invariant}
\begin{proof}
The claim is true initially, since each RBT initially contains only the empty block with \fld{index} 0.
We show that each time a new RBT is constructed, it preserves the invariant.
When a new RBT is created by inserting an element into an existing RBT, \Cref{RBTinsertions}
implies that the invariant is preserved.

Suppose a new RBT is created by a \op{Split}($T,t$) operation $S$ on at line \ref{split} of a \op{Discard}($v,t$), 
and that the invariant held prior to $S$.
Clearly, $S$ preserves the property that indices are consecutive.  To show that the new tree
is non-empty we first prove that the call to \op{Discard}($v,t$) has a parameter $t$ that is at least as big as a value previously
read from the index field of some block in $v$'s RBT.
If \op{Discard} was called from line \ref{discard}, then $t+1$ was read from an entry of \var{last}, so the claim follows from \Cref{lastinroot}.
If \op{Discard} was called from line \ref{discardrecurse1} or \ref{discardrecurse2}, the claim follows
from \Cref{edirGC}.
Since the invariant held prior to $S$, no split before $S$ emptied $v$'s tree.  Therefore,
some block with index at least $t$ is still in $T$ when $S$ occurs.  That block will be in the new tree created by $S$.
\end{proof}

Since RBTs are always non-empty, calls to \op{MaxBlock} are well-defined.  
Throughout the proof, we use $v.\blocks[b]$ to refer to the block with \fld{index} $b$ that appeared
in $v$'s tree at some time during the execution.
It follows from \Cref{RBTinsertions} and \Cref{nonemptyConsec} that this block is unique, if 
it exists.
We also use this notation in the code to indicate that a search of the RBT should be performed for the block
with \fld{index} $b$.
We describe later what should happen if the block is not found by the search because it has been removed
by a GC phase.
\here{remember to come back to this point}

\begin{corollary}\label{maxNonDecreasing}
The maximum \fld{index} in $v$'s RBT is non-decreasing over time.
\end{corollary}
\begin{proof}
The CAS steps (on line \ref{casGC} and \ref{splitCASGC}) that update $v.\fld{blocks}$ always replace 
a tree $T$ by a tree that is built from $T$ by inserting an element and/or removing the elements with 
the smallest \fld{index} values, without emptying the tree.  Thus, the maximum \fld{index} in the tree cannot decrease.
\end{proof}

We now prove the analogue of Lemma \ref{lem::headProgress}.
\begin{lemma}
 If $b>0$ and a block with index $b$ has been inserted into $v.\fld{blocks}$ then 
 $v.\fld{blocks}[b-1].\fld{end\sub{left}} \leq v.\fld{blocks}[b].\fld{end\sub{left}}$ and 
 $v.\fld{blocks}[b-1].\fld{end\sub{right}} \leq v.\fld{blocks}[b].\fld{end\sub{right}}$.
\end{lemma}
\begin{proof}
The block $B$ with index $b$ was installed into $v$'s RBT by the CAS at line \ref{casGC} or \ref{splitCASGC} 
that changed the tree from $T$ to $T_{new}$.
Before this CAS, the tree $T$ was read from $v.blocks$ at line \ref{readTGC} or \ref{readT2GC}, respectively,
before the \op{RefreshUpdate} on line \ref{RU1} or \ref{RU2} that inserted $B$ to get $T_{new}$.
Since the previous block $B'=v.\fld{blocks}[b-1]$ was already in $T$ (by \Cref{RBTinsertions}),
the \op{CreateBlock}$(v,b-1)$ that created $B'$ terminated before the \op{CreateBlock}$(v,b)$ that created $B$ started.
It follows from \Cref{maxNonDecreasing} that the value that line \ref{createEndLeftGC} of \op{CreateBlock}$(v,b-1)$ stores in $B'.\eleft$ is less than or equal to the value that line \ref{createEndLeftGC} of \op{CreateBlock}$(v,b)$ stores in $B.\eleft$.  Similarly, the values stored in $B'.\eright$ and $B.\eright$ at line \ref{createEndRightGC} satisfy the claim.
\end{proof}

\Cref{lem::subblocksDistinct}, \Cref{lem::noDuplicates}, \Cref{lem::sum}, \Cref{blockNotEmpty} 
hold true for the modified algorithm.  Their proofs are identical to those given in 
Section \ref{sec::basicProperties} since they only depend on \Cref{lem::headProgress} 
and the definition of subblocks given in (\ref{defsubblock}).
The following pair of lemmas are analogous to \Cref{successfulRefresh} for the modified \op{Refresh} routine.
We say that an operation has been \emph{propagated} to node $v$ in configuration $C$ if
some block containing the operation has been in $v.blocks$ at or before $C$.

\begin{lemma}\label{nullRefreshGC}
Suppose $v.\fld{blocks}=T$ at some configuration $C$ and \op{RefreshUpdate}($T,v$) is invoked after $C$ and
returns \nl.  Then, all operations that were propagated to $v$'s
children before $C$ have been propagated to $v$ before $C$.
\end{lemma}
\begin{proof}
Suppose an operation $op$ has been propagated to $v$'s child (without loss of generality, $v.\fld{left}$) before $C$.
Then, for some $i$, a block $B$ with \fld{index} $i$ has been in $v.\fld{left.blocks}$ before $C$.
By Lemma \ref{maxNonDecreasing}, 
the maximum \fld{index} in $v.\fld{left.blocks}$ is at least $i$ at all times during the \op{RefreshUpdate}.
So, the value of $\var{new}.\eleft$ chosen at line \ref{createEndLeftGC} of the \op{RefreshUpdate}'s call
to \op{CreateBlock} is at least $i$.

Let $h$ be the value set at line \ref{refreshReadMax} of the \op{RefreshUpdate}.
The \op{RefreshUpdate} returns \nl\ at line \ref{addOPGC}
because the \op{CreateBlock}$(v,h)$  at line \ref{CBnullGC} returns \nl.
By reasoning identical to the last paragraph of \Cref{successfulRefresh},
it follows from the fact that $\var{num\sub{enq}}+\var{num\sub{deq}}=0$ that
the blocks $v.\fld{left.blocks}[v.\fld{blocks}[h-1].\eleft+1..new.\eleft]$ and
$v.\fld{right.blocks}[v.\fld{blocks}[h-1].\eright+1..new.\eright]$
contain no operations.
Since each block contains at least one operation by \Cref{blockNotEmpty}
these ranges must be empty, so $v.\fld{blocks}[h-1].\eleft \geq \var{new}.\eleft\geq i$.
This implies that the block $B$ containing $op$ is a subblock of some block that has appeared in $v.\fld{blocks}$, so $op$ has been propagated to $v$.
\end{proof}

\begin{lemma}\label{successfulRefreshGC}
If a \op{CAS} at line \ref{casGC} or \ref{splitCASGC} modifies $v.\fld{blocks}$ from $T$ to $T'$,
then all operations that were propagated to $v$'s children before $v.\fld{blocks}$ was set to $T$
have been propagated to $v$ in the configuration after the CAS.
\end{lemma}
\begin{proof}
Suppose an operation $op$ has been propagated to $v$'s child (without loss of generality, $v.\fld{left}$) before $v.\fld{blocks}$ was set to $T$.
Then, for some $i$, a block $B$ with \fld{index} $i$ that contains $op$
has been in $v.\fld{left.blocks}$ before $v.\fld{blocks}$ was set to $T$.
Let $c$ be the \op{CAS} instruction that changes $v.\fld{blocks}$ from $T$ to $T'$.
If line \ref{casGC} performs $c$, then the \op{RefreshUpdate} $R$ at line \ref{RU1} that creates
$T'$ began after $T$ was read from $v.\fld{blocks}$ at line \ref{readTGC}.
Otherwise, line \ref{splitCASGC} performs $c$ and the \op{RefreshUpdate} $R$ at line \ref{RU2},
which is invoked after $T$ was read from $v.\fld{blocks}$ at line \ref{readT2GC},
either returned \nl\ or $T'$.

If the \op{RefreshUpdate} $R$ returned \nl, the lemma follows from \Cref{nullRefreshGC}.
So, assume \op{RefreshUpdate} $R$ returned $T'$.
Then $T'$ contains a newly inserted block created by the call
to \op{CreateBlock} on line \ref{invokeCreateBlockGC}.
Since $R$ was invoked after $T$ was stored in $v.\fld{blocks}$, 
the maximum \fld{index} in $v.\fld{left.blocks}$ is at least $i$ throughout $R$, by \Cref{maxNonDecreasing}.
Thus. the \eleft\ field of the new block added to $T'$ by $R$'s call to \op{CreateBlock} is
set to a value greater than or equal to $i$ at line \ref{createEndLeftGC}.
By the definition of subblocks, it follows that the block $B$ containing $op$ is a subblock of a block
that has been in $v.blocks$, as required.
\end{proof}

This allows us to show that a double \op{Refresh} propagates operations up the tree, 
as in \Cref{lem::doubleRefresh}.

\begin{lemma}\label{lem::doubleRefreshGC}
Consider two consecutive terminating calls $R_1$, $R_2$ to \op{GCRefresh}($v$) by the same process.
All operations propagated to $v$'s children when $R_1$ begins
are propagated to $v$ when $R_2$ terminates.
\end{lemma}
\begin{proof}
\end{proof}






-------

In our design a process attempts to collect the garbage, when it is going to to append a block in the $kp^2$ position in \nf{root.blocks} (see Line \ref{}). Every $p^2$  block appended to the root, one \nf{GarbageCollect} terminates because \nf{root.head} cannot advance until a \nf{GarbageCollect} garbage collect is done (see Lines ).

%\begin{lemma}
%Size of \nf{root.blocks} after \nf{GarbageCollect} is $O(p^2+q)$.
%\end{lemma}
%
%\begin{lemma}
%Total number of the blocks in the tree is $O(p^3+pq)$.
%\end{lemma}

\nf{Enqueue} operation $e$ can be removed from the tree after termination of \nf{Dequeue} $d$ where $Resp(d)=e$. It is safe to remove \nf{Dequeue} $d$ from the tree after the $d$ terminates. We can remove a block after the conditions told are satisfied for all of its \nf{Enqueue}s and \nf{Dequeue}s. A \nf{Dequeue} may go to sleep for a long time and prevent a \nf{block} to be removed (the block it is in or the block its response is in). In that situation other processes can help the \nf{Dequeue} by computing its response and writing it down somewhere. After writing down the response of th \nf{Dequeue} can be removed, since if the process wanted the response it can read from the helped response.

\begin{definition}
A block  is \it{finished} if all of its \nf{Enqueue}s have been computed to be the response of some \nf{Dequeue} and all of its \nf{Dequeues} responses have been computed.  
\end{definition}

\begin{corollary}
If a block is finished, then all of its subblocks are also finished.  
\end{corollary}

\begin{lemma}
It is safe to remove a finished block from the non-leaf nodes.
\end{lemma}

If the $i$th \nf{Enqueue} gets dequeued in a FIFO queue, it means the first $i-1$ \nf{Enqueue} operations have been already dequeued. This gives us the idea that if a block is finished then all the blocks before it are also finished. If an operation in a block goes to sleep for a long time then other processes help the operation so the block gets finished. There are less than $p$ idle operations, we can help them before garbage collection and then remove all the finished blocks safely in an amortized poly-log time.

\begin{lemma}
    If all \nf{Dequeue} operations in the root are helped, then all the blocks before a finished block in the root are also finished.
\end{lemma}

\begin{lemma}
  Blocks before the most recent block that has been dequeued(computed to be dequeue) from are finished. If the most current \nf{Dequeue} returned(computed) \nf{null} then all the blocks before the block containing the \nf{Dequeue} are finished.
\end{lemma}

The idea above leads us to a poly-log data structure that supports throwing away all the blocks with keys smaller than an index. Red-black trees do this for us. \nf{Get(i)}, \nf{Append()} and \nf{Split(i)} are logarithmic in block trees.
We can create a shared red-black tree just creating a new path for the operation and then using \nf{CAS} to change the root of the tree. See [this] for more.

\begin{observation}
PBRT supports poly-log operations ....
\end{observation}

\begin{lemma}
    If we replace the arrays we used to implement \nf{blocks} with red-black trees the amortized complexity of the algorithm would be $PolyLog(p,q)$. And also the algorithm is correct.
\end{lemma}

We can help a \nf{Dequeue} by computing its response and writing it down. If the process in future failed to execute, it can read the helped value written down.

\begin{lemma}
The \nf{response} written is correct.
\end{lemma}

But how can we know which blocks in each node are finished or not? 

\begin{observation}
Every $p^2$ block appended to the root, \nf{FreeMemory} is invoked.  
\end{observation}

To know the last block dequeied from we can implement a shared array among processes which they write the last root block they have dequeued from. 

\begin{lemma}
    $\nf{Max(Last)} - \text{index of the last finished block}$ in the node $n$ is $O(p)$.
\end{lemma}

\begin{lemma}
    After \nf{FreeMemory}, the space taken by each node of the tree is $O(p+q)$.
\end{lemma}

\begin{corollary}
The space taken by each node of the tree is $O(p^2+q)$. The total space in the tree is $PolyLog(p+q)$.
\end{corollary}

\begin{lemma}
  The amortized step per process for the algorithm with garbage collection is $PolyLog(p+q)$.
\end{lemma}

\begin{lemma}
  Algorithm is wait-free and linearizable.
\end{lemma}

\here{Remark on an optimization:  instead of storing \head\ field separate from \fld{blocks}, we could store
the maximum index of the RBT in the root of the RBT.  This would simplify things, but require modifying the proof more extensively.}

