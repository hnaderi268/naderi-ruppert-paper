\documentclass[compress]{beamer}

\usepackage{color}
\usepackage{amssymb}
\usepackage{times}
\usepackage[T1]{fontenc}
\usepackage[normalem]{ulem} % for striking through text

\mode<presentation>
{
  \usetheme{Warsaw}
  \usecolortheme{beaver}
  \usefonttheme{structurebold}

  % or ...

  \setbeamercovered{transparent}
  % or whatever (possibly just delete it)
}
\logo{\includegraphics[width=15mm]{YorkLogo.pdf}}

\newcommand{\Red}[1]{{\color{red}#1}}
\newcommand{\Blue}[1]{{\color{blue}#1}}
\newcommand{\Green}[1]{{\color{green}#1}}

\newcommand{\tb}{\hspace*{10mm}}

\newcommand{\ignore}[1]{}

\title{A Wait-Free Queue with Polylogarithmic Step Complexity}

\author{Hossein Naderibeni \and Eric Ruppert}
\date{June 21, 2023}


\begin{document}

\begin{frame}
\maketitle
\end{frame}

\begin{frame}{Lock-Free Queue using CAS}
Michael and Scott Queue [PODC 1996]

\bigskip

% We store the elements of the queue in a singly-linked list.
% Enqueues add elements at the left end and dequeues remove elements from the right end.
% Each enqueue creates new node and performs a CAS on the tail node's next pointer to add it to the left end (and then advances the Tail pointer)
% Each dequeue uses a CAS to advance the Head pointer

\only<1>{\input{MSQueue-48to50.pdf_t}}
\only<2>{\input{MSQueue-48to51.pdf_t}}
\only<3>{\input{MSQueue-48to52.pdf_t}}
\only<4>{\input{MSQueue-48-50to53.pdf_t}}
\only<5>{\input{MSQueue-50to54.pdf_t}}
\end{frame}

\begin{frame}{CAS Retry Problem}



\end{frame}



\end{document}