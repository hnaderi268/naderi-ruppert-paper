% !TEX root =  queue.tex

\section{Analysis}
We now analyze the number of steps and the number of \op{CAS} instructions performed by operations.

\begin{proposition}\label{casbound}
Each \op{Enqueue} or \op{Dequeue} operation performs $O(\log p)$ \op{CAS} instructions.
\end{proposition}
\begin{proof}
An operation invokes \op{Refresh} at most twice at each of the $\ceil{\log_2 p}$ levels of the tree.
A \op{Refresh} does at most 5 \op{CAS} steps: one in line \ref{cas} and two during each \op{Advance} in line \ref{helpAdvance} or~\ref{advance}.
\end{proof}



\begin{lemma}\label{dSearchTime}
The search that \op{FindResponse}$(b,i)$ does at line \ref{FRb} to find the index $b_e$ of the block in the root containing the $e$th enqueue takes $O(\log (\var{root}.\fld{blocks}[b_e].\size + \var{root}.\fld{blocks}[b-1].\size))$ steps.
\end{lemma}
\begin{proof}
Let the blocks in the root be $B_1, \ldots, B_\ell$.
The doubling search for $b_e$ takes $O(\log (b-b_e))$ steps,
so we prove $b - b_e \leq 2 \cdot B_{b_e}.\size + B_{b-1}.\size + 1$.
If $b \leq b_e+1$, then this is trivial, so assume  $b>b_e+1$.
%
As shown in Lemma \ref{linearCorrect}, the dequeue that calls \op{FindResponse} is in $B_b$ and is supposed to return an enqueue in $B_{b_e}$.
Thus, there can be at most $B_{b_e}.\size$ dequeues in 
$D(B_{b_e+1}) \cdots D(B_{b-1})$; otherwise in the sequential execution $L$,
all elements enqueued before the end of
$E(B_{b_e})$ would be dequeued before $D(B_b)$. 
Furthermore, by Lemma \ref{sizeCorrectness}, the size of the queue  after the prefix of $L$ corresponding to 
$B_1,\ldots,B_{b-1}$  is 
$B_{b-1}.\size \geq B_{b_e}.size + |E(B_{b_e+1})\cdots E(B_{b-1})| - |D(B_{b_e+1})\cdots D(B_{b-1})|$.
Thus, $|E(B_{b_e+1})\cdots E(B_{b-1})| \leq B_{b-1}.\size + |D(B_{b_e+1})\cdots D(B_{b-1})| \leq B_{b-1}.\size + B_{b_e}.\size$.
So, the total number of operations in $B_{b_e+1}, \ldots, B_{b-1}$ is at most
$B_{b-1}.\size + 2\cdot B_{b_e}.\size$.
Each of these $b-1-b_e$ blocks contains at least one operation, by Corollary \ref{blockNotEmpty}.
So, $b-1-b_e \leq B_{b-1}.\size + 2\cdot B_{b_e}.\size$.
\end{proof}

The following lemma helps bound the time for \op{GetEnqueue}.
\begin{lemma}\label{blockSize}
Each block $B$ in each node contains at most one operation of each process.
If $c$ is the execution's maximum point contention, $B$ has at most $c$ direct~subblocks.
\end{lemma}
\begin{proof}
Suppose $B$ contains an operation of process $p$.
Let $op$ be the earliest operation by $p$ contained in $B$.
When $op$ terminates, $op$ is contained in $B$ by Lemma \ref{lem::appendExactlyOnce}.
$B$ cannot contain any later operations by $p$, since $B$ is created before
those operations are invoked.

Let $t$ be the earliest termination of any operation contained in $B$.
By Lemma \ref{lem::appendExactlyOnce}, $B$ is created before $t$, so all operations contained in $B$
are invoked before $t$.  Thus, all are  running concurrently at $t$, so $B$ contains at most $c$ operations.
By definition, the direct subblocks of $B$ contain these $c$ operations, and each operation is contained
in exactly one of these subblocks, by Lemma \ref{lem::subblocksDistinct}.
By Corollary \ref{blockNotEmpty}, each direct subblock of $B$ contains at least one operation,
so $B$ has at most $c$ direct subblocks.
\end{proof}

We now bound step complexity in terms of the number of processes $p$, the maximum contention $c\leq p$, and the size of the queue. 

\begin{mytheorem}\label{enqDeqTime}
Each \op{Enqueue} and null Dequeue takes $O(\log p)$ steps 
and each non-null \op{Dequeue} takes
$O(\log p\log c + \log q_e+ \log q_d)$ steps,
where $q_d$ is the size of the queue when the \op{Dequeue} is linearized and 
$q_e$ is the size of the queue when the \op{Enqueue} of the value returned is linearized.
\end{mytheorem}
\begin{proof}
An \op{Enqueue} or null \op{Dequeue} creates a block, appends it to the process's 
leaf and propagates it to the root.  The \op{Propagate}  does $O(1)$ steps 
at each node on the path from the process's leaf to the root.
A null \op{Dequeue} additionally calls \op{IndexDequeue}, which also does $O(1)$ steps
at each node on this path. 
So, the total number of steps for either type of operation is $O(\log p)$.

A non-null \op{Dequeue} must also search at line \ref{FRb} and call \op{GetEnqueue}
at line \ref{findAnswer}.
By Lemma \ref{dSearchTime}, the doubling search takes $O(\log(q_e+q_d+p))$ steps,
since the size of the queue can change by at most $p$ within one block (by Lemma \ref{blockSize}).
\op{GetEnqueue} does a binary search within each node on a path from the root to a leaf.
Each node $\var{v}$'s search is within the subblocks of one block in $\var{v}$'s parent.
By Lemma \ref{blockSize}, each such search takes $O(\log c)$ steps, for a total of $O(\log p\log c)$ steps.
\end{proof}

\begin{corollary}
The queue implementation is wait-free.
\end{corollary}
