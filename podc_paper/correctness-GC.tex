% !TEX root =  podc-submission.tex

\subsection{Correctness}
There are enough changes to the algorithm that a new proof of correctness
is required.  Its structure mirrors the proof of the original algorithm, but requires additional
reasoning to ensure GC does not interfere with other routines.

\subsubsection{Basic Properties}

The following observation describes
how the set of blocks in a node's RBT can be modified.

\begin{lemma}\label{RBTupdates}
Suppose a step of the algorithm changes $\var{v}.\fld{blocks}$ from a non-empty tree $T$ to $T'$.
If the set of \fld{index} values  in $T$ is $I$, then the set of \fld{index} values in $T'$ is 
$(I \cap [m-1,\infty))\cup \{\max(I) + 1\}$ for some $m$.
\end{lemma}
\begin{proof}
The RBT of a node is updated only at line \ref{appendLeafGC} or \ref{casGC}.

If line \ref{appendLeafGC} of an \op{Append} operation modifies $\var{v}.\fld{blocks}$,
then \var{v} is a leaf node, and no other process ever modifies $\var{v}.\fld{blocks}$.
$T'$ was obtained from $T$ by calling \op{AddBlock}($T,B$).
$B$ was created either by the \op{Enqueue} or \op{Dequeue} that called \op{Append}.
Either way, $B.\fld{index} = \max(I)+1$.
The \op{AddBlock} that creates $T'$ may optionally \op{Split} the RBT and then add $B$ to it.
So the claim is satisfied.

If line \ref{casGC} of a \op{Refresh} modifies $\var{v}.\fld{blocks}$, then \var{v} is an internal node.
After reading $T$ from $\var{v}.\fld{blocks}$ at line \ref{refreshReadTGC},
then creates the  block \var{new},
and then calls \op{AddBlock}($T,new$) to create $T'$.
Line \ref{createIndexGC} sets $\var{new}.\var{index} = \max(I)+1$.
The \op{AddBlock} that creates $T'$ may optionally \op{Split} the RBT and then add $new$ to it.
So the claim is satisfied.
\end{proof}

Since each RBT starts with a single block with \fld{index} 0, the following is an easy consequence of \Cref{RBTupdates}.
\begin{corollary}\label{nonEmptyIncreasing}
The RBT stored in each node \var{v} is never empty and always stores a set of blocks with consecutive indices.
Moreover, its maximum \fld{index} can only increase over time.
\end{corollary}

Since RBTs are always non-empty, calls to \op{MaxBlock} have well-defined answers.  
Throughout the proof, we use $\var{v}.\blocks[b]$ to refer to the block with \fld{index} $b$ that appeared
in \var{v}'s tree at some time during the execution.
It follows from \Cref{RBTupdates} and \Cref{nonEmptyIncreasing} that 
each time a new block appears in \var{v}'s RBT, its \fld{index} is greater than any block
that has appeared in \var{v}'s RBT earlier.
Thus, $\var{v}.\blocks[b]$ is unique, if it exists.
We also use this notation in the code to indicate that a search of the RBT should be performed for the block
with \fld{index} $b$.

We now establish that Definition (\ref{defsubblock}) of a block's subblocks
still makes sense by proving the analogue of Lemma \ref{lem::headProgress}.

%\here{If next obs is only used once, can fold it into the proof that needs it}
%\begin{observation}
%\label{edirGC} 
%If $B$ is a block in the RBT of an internal node \var{v}, then for $\var{dir} \in \{\fld{left},\fld{right}\}$,
%$B.\edir$ was read from the \fld{index} field of some node in the RBT of $\var{v}.\var{dir}$.
%\end{observation}
%\begin{proof}
%The \eleft\ and \eright\ fields are set by reading the \fld{index} field of a node in the appropriate child
%on line \ref{createEndLeftGC} and \ref{createEndRightGC}.
%\end{proof}


\begin{customlemma}{\ref{lem::headProgress}$'$}\label{lem::headProgressGC}
 If $h>0$ and a block with index $h$ has been inserted into $\var{v}.\fld{blocks}$ then 
 $\var{v}.\fld{blocks}[h-1].\fld{end\sub{left}} \leq \var{v}.\fld{blocks}[h].\fld{end\sub{left}}$ and 
 $\var{v}.\fld{blocks}[h-1].\fld{end\sub{right}} \leq \var{v}.\fld{blocks}[h].\fld{end\sub{right}}$.
\end{customlemma}
\begin{proof}
The block $B$ with index $h$ was installed into \var{v}'s RBT by the \op{CAS} at line \ref{casGC}.
Suppose that \op{CAS} changed the tree from $T$ to $T'$.
Before this \op{CAS}, line \ref{refreshReadTGC} read the tree $T$ from $\var{v}.blocks$,
line \ref{refreshReadMax} found a block $B'$  with \fld{index} $h-1$ in $T$,
and then line \ref{invokeCreateBlockGC} created the block $B$ with $\fld{index}=h$.
Since $B'$ was already in $T$ before $B$ was created, the
\op{CreateBlock}$(\var{v},b-1)$ that created $B'$ terminated before the 
\op{CreateBlock}$(\var{v},b)$ that created $B$ started.
By \Cref{nonEmptyIncreasing}, the value that line \ref{createEndLeftGC} of \op{CreateBlock}$(\var{v},b-1)$ 
stores in $B'.\eleft$ is less than or equal to the value that line \ref{createEndLeftGC} 
of \op{CreateBlock}$(\var{v},b)$ stores in $B.\eleft$.  
Similarly, the values stored in $B'.\eright$ and $B.\eright$ at line \ref{createEndRightGC} satisfy the claim.
\end{proof}

\Cref{lem::headProgressGC} implies that 
the nodes of an in-order traversal of any RBT have non-decreasing values of
$\eleft$ (and of $\eright$).
Thus, the searches for a block based on \eleft\ or \eright\ values 
at lines \ref{searchsuper3}, \ref{searchsuper1} and \ref{searchsuper2},
which are used to look for the superblock of a node or its predecessor, can be done in logarithmic time.

\Cref{lem::subblocksDistinct}, \Cref{lem::noDuplicates}, \Cref{lem::sum}, \Cref{blockNotEmpty} 
hold for the modified algorithm.  Their proofs are identical to those given in 
Section \ref{sec::basicProperties} since they  depend only on \Cref{lem::headProgress} (which can be replaced by \Cref{lem::headProgressGC})
and the definition of subblocks given in (\ref{defsubblock}).
In particular, \Cref{lem::sum} says that nodes in an in-order traversal of a RBT have non-decreasing
values of \fld{sum\sub{enq}}  so the searches of a RBT based on
\fld{sum\sub{enq}} values in lines
\ref{getChildGC} and \ref{FRsearchGC} can be done in logarithmic time.

\subsubsection{Propagating Operations to the Root}

Next, we prove an analogue of \Cref{successfulRefresh}.
We say a node \var{v} \emph{contains} an operation
if some block containing the operation has previously appeared in \var{v}'s RBT (even if 
the block has been removed from the RBT by a subsequent \op{Split}).

\begin{customlemma}{\ref{successfulRefresh}$'$}
\label{successfulRefreshGC}
Let $R$ be a call to \op{Refresh}(\var{v}) that performs a successful \op{CAS} on line \ref{casGC} (or terminates at line \ref{addOPGC}).
In the configuration after that CAS (or termination, respectively), \var{v} contains all operations that \var{v}'s children contained 
when $R$ executed line~\ref{refreshReadTGC}.
\end{customlemma}
\begin{proof}
Suppose \var{v}'s child (without loss of generality, $\var{v}.\fld{left}$) contained an operation $op$ 
when $R$ executed line \ref{refreshReadTGC}.
Let $i$ be the index of the block containing $op$ that was in \var{v}'s RBT before 
$R$ executed line \ref{refreshReadTGC}.
We consider two cases.

Suppose $R$'s call to \op{CreateBlock} returns a new block $B'$.  Then, $R$'s \op{CAS} at line \ref{cas} 
installs $B'$ in $\var{v}.\fld{blocks}$.
The \op{CreateBlock} set $B'.\eleft$ to the maximum \fld{index} in $\var{v}.\fld{left}$'s RBT at line \ref{createEndLeftGC}.
By \Cref{nonEmptyIncreasing}, this maximum \fld{index} is bigger than $i$.
By the definition of subblocks, some block in \var{v} contains $B$ as a subblock
and therefore \var{v} contains $op$.

Now suppose $R$'s call to \op{CreateBlock} returns \nl, causing $R$ to terminate at line \ref{addOPGC}.
Let $h$ be the maximum \fld{index} in $T$ plus 1.
By reasoning identical to the last paragraph of \Cref{successfulRefresh}'s proof,
it follows from the fact that $\var{num\sub{enq}}+\var{num\sub{deq}}=0$ at line \ref{testEmptyGC}
that the blocks $\var{v}.\fld{left.blocks}[\var{v}.\fld{blocks}[h-1].\eleft+1..new.\eleft]$ and
$\var{v}.\fld{right.blocks}[\var{v}.\fld{blocks}[h-1].\eright+1..new.\eright]$
contain no operations.
By \Cref{blockNotEmpty}, each block contains at least one operation, so
these ranges must be empty, and $\var{v}.\fld{blocks}[h-1].\eleft \geq \var{new}.\eleft\geq i$.
This implies that the block $B$ containing $op$ is a subblock of some block that has appeared
in \var{v}'s RBT, so $op$ is contained in \var{v}.
\end{proof}

This allows us to show that a double \op{Refresh} propagates operations up the tree, 
as in \Cref{lem::doubleRefresh}.

\begin{customlemma}{\ref{lem::doubleRefresh}$'$}\label{lem::doubleRefreshGC}
Consider two consecutive terminating calls $R_1$, $R_2$ to \op{Refresh}(\var{v}) by the same process.
All operations contained \var{v}'s children when $R_1$ begins
are contained in \var{v} when $R_2$ terminates.
\end{customlemma}
\begin{proof}
If either $R_1$ or $R_2$ performs a successful \op{CAS} at line \ref{casGC} or terminates at line \ref{addOPGC}, the claim follows
from \Cref{successfulRefreshGC}.
So suppose both $R_1$ and $R_2$ perform a failed \op{CAS} at line \ref{casGC}.
Then some other \op{CAS} on $\var{v}.\fld{blocks}$ succeeds between the time each \op{Refresh}
reads $\var{v}.\fld{blocks}$ at line \ref{refreshReadTGC} and performs its \op{CAS} at line \ref{casGC}.
Consider the \op{Refresh} $R_3$ that does this successful \op{CAS} during $R_2$.
$R_3$ must have read $\var{v}.\fld{blocks}$ after the successful \op{CAS} during $R_1$.
The claim follows from \Cref{successfulRefreshGC} applied to $R_3$.
\end{proof}

\Cref{lem::appendExactlyOnce} can then be proved in the same way as in Section \ref{sec::propagating}.


\subsubsection{GC Keeps Needed Blocks}

The correctness of \op{GetEnqueue} and \op{IndexDequeue}, which are very similar to the 
original implementation, are dependent only on the fact that GC does not 
discard blocks needed by those routines.
The following results are used to show this.

We say a block is \emph{finished} if 
\begin{itemize}
\item
it has been propagated to the root,
\item
the value of each enqueue contained in the block has either been returned by a dequeue
or written in the \fld{response} field of a dequeue, and
\item
each dequeue contained in the block has terminated or some process has written to the \fld{response} field in the leaf block that represents it.
\end{itemize}
Intuitively, once a block is finished, operations no longer need to pass through the block
to compute responses to operations.
The following is an immediate consequence of the definition of finished and what it means for an operation to be contained in a block.

\begin{observation}\label{subblocksFinished}
A block is finished if and only all of its subblocks are finished.
\end{observation}

\here{Technically, following proof might have to be wrapped up in a big induction with the argument
that outputs are linearized correctly, to argue that the matchup between enqueues and dequeues
is the same as it would be in the linearization ordering.  (Check whether this is really needed.)
Or alternatively, we could add an assumption that if every operation agrees with the linearization
so far, then the invariant is preserved by each step.
Then, when proving the linearization is correct, everything will work out.
But just leave it as is for the submission, since this is a subtle point}

\begin{invariant}\label{lem::finished}
If the minimum \fld{index} of any block in \var{v}'s RBT is $b_{min}$, then
each block with $\fld{index}$ at most $b_{min}$ that was ever added to \var{v} is finished.
\end{invariant}
\begin{proof}
The invariant is true initially, since the minimum \fld{index} block in \var{v}'s RBT is the empty block, which is (vacuously) finished.

We show that every step preserves the invariant.  
We need only consider a step $st$ that modifies a node's RBT.
The minimum \fld{index} of \var{v}'s RBT can only change when \var{v}'s RBT changes, either at line 
\ref{appendLeafGC} (if \var{v} is a leaf) or at line \ref{casGC} (if \var{v} is an internal node).
In either case, the step $st$ changes $\var{v}.\fld{blocks}$ is changed
from $T$ to $T'$, where $T'$ is obtained by  a call $A$ to \op{AddBlock}(\var{v}, $T$, $B$).
(In the case of a leaf \var{v}, this is true because only the process that owns the leaf ever
writes to $\var{v}.\fld{blocks}$.)
If $A$ does not do GC (lines \ref{GCstart}--\ref{GCend}), 
then $T'$ is obtained by adding a new block to $T$, so by \Cref{RBTupdates} the minimum \fld{index}
is unchanged and the invariant is trivially preserved.
So consider the case where $A$ performs GC.
We must show that any block of \var{v} whose index is less than or equal to the
minimum \fld{index} in $T'$ is finished when $T'$ is installed in $\var{v}.\fld{blocks}$.
Since $T'$ is obtained by discarding all blocks with \fld{index} at most $s$ (and adding
a block with a larger index), it suffices to show that all blocks that were ever added
to \var{v}'s RBT with \fld{index} at most $s$ are finished.

We must examine how $A$'s call to the recursive algorithm \op{SplitBlock} computes the value of $s$.

\begin{claim}\label{splitfinished}
If one of the recursive calls to $\op{SplitBlock}(x)$ within $A$'s call to \op{SplitBlock},
then that block and all earlier blocks in $x$ are finished when $st$ occurs.
\end{claim}
\begin{proof}[Proof of Claim]
We prove this claim  by induction on the depth of $x$.

For the base case, suppose $x$ is the root.
\op{SplitBlock} finds the maximum value $m$ in \var{last}, which is the index of some block
that contains an operation that is either a null dequeue or an enqueue whose value is the response for a dequeue that has been propagated to the root (since these are the only ways that an entry of \var{last} can be set to $m$).
By the FIFO property of queues, the values enqueued by enqueues in $\var{root}.\fld{blocks}[1..m-1]$
are all dequeued by operations of $\var{root}.\fld{blocks}[1..m]$.
Between the termination of $A$'s call to \op{SplitBlock} at line \ref{callSplitBlock}
and the \op{CAS} step $st$ after $A$ terminates, the $A$ helps all pending dequeues at line \ref{invokeHelp}.
Thus, after this helping (and before step $st$),
$\var{root}.\fld{blocks}[1..m-1]$ are all finished blocks, so the claim is true.

For the induction step, we assume the claim holds for $x$'s parent, and prove it for $x$.
We consider two cases.

If \op{SplitBlock}($x$) returns the minimum block of $x$'s RBT at line \ref{substitute},
then the claim follows from the assumption that \Cref{lem::finished} holds at all times before $st$.

Otherwise, \op{SplitBlock}($x$) returns the block $B$ at line \ref{substitute}.
By the induction hypothesis, the block $B_p$ computed at line \ref{SBrecurse}
(and all earlier blocks of $x$) are finished when $st$ occurs.
By \Cref{subblocksFinished}, the block $B$ in $x$ indexed by $B_p.\eleft$ or $B_p.\eright$
is also finished when $st$ occurs.
\renewcommand{\qedsymbol}{$\diamondsuit$}
\end{proof}

If the \op{Split} at line \ref{splitGC} of $A$ modifies the RBT, then it discards
all the blocks older than the one returned by \op{SplitBlock} at line \ref{callSplitBlock}.
By \Cref{splitfinished}, the minimum block in the new tree will satisfy the invariant when $st$
installs the new tree in $\var{v}.\fld{blocks}$.
\end{proof}

We remark that \Cref{lem::finished} guarantees that GC keeps one block that is finished and discards all blocks with smaller indices.
This is because the first unfinished block may still be traversed by an operation in the future,
and when examining that block the operation may need information from the preceding block.  For example,
when \op{FindResponse} is called on a block with index $b$, line \ref{FRnumenqGC} looks up the block with
index $b-1$.

\begin{lemma}\label{blocknotfound}
If a \op{Dequeue} operation fails to find a block in an RBT,
%during its calls to \op{IndexDequeue} or \op{GetEnqueue}, 
then it has been propagated to the root and its result has been written in its \fld{response} field.
If an \op{Enqueue} operation fails to find a block in an RBT, it has been propagated to the root.
\end{lemma}
\begin{proof}
Any block that GC removes from an RBT is finished, by \Cref{lem::finished}, so if an 
\op{Enqueue} or \op{Dequeue} fails to find a block while it is propagating itself up to
the root (for example, during the \op{CreateBlock} routine), then a block containing
the operation itself has been removed from a RBT, so the operation has propagated to the root,
by the definition of finished.
Moreover, by \Cref{lem::finished}, if the operation is a dequeue, then its result is in its \op{response} field.

After propagation to the root, a \op{Dequeue} must access blocks that contain the dequeue,
as well as the enqueue whose value it will return (if it is not a null dequeue).
If any of those blocks have been removed, it follows from \Cref{lem::finished} that the
\op{Dequeue}'s result is written in its \fld{response} field.
\end{proof}

By \Cref{blocknotfound}, an operation that fails to find a block in an RBT can terminate.
If it is a \op{Dequeue}, it can return the result written in its \fld{response} field.
Since no other process updates the RBT in a process's leaf, the block containing the \fld{response}
will be the last block in the leaf's RBT, and the last block of an RBT is never removed by GC.
Thus, the \fld{response} field will still be there when a \op{Dequeue} needs it.

\subsubsection{Linearizability}

The correctness of the \op{IndexDequeue} and \op{GetEnqueue}  operations
can be proved in the same way as in Section \ref{sec::tracingCorrect},
since they are largely unchanged (except for the simplification that \op{IndexDequeue}
can simply search for a block's superblock instead of using the block's \fld{super} 
field to calculate the superblock's position).
They will give the correct response, provided none of the blocks they need to access have been 
removed by GC.  But as we have seen above, if that happens, the \op{Enqueue} or \op{Dequeue} can simply terminate.

Similarly, the results of Section \ref{sec::linearizability} 
can be reproved in exactly the same way as for the original algorithm to establish
that the space-bounded algorithm is linearizable.

\subsection{Analysis}
\label{sec::GCanalysis}

\here{need to prove a lemma analogous to \Cref{blockSize}}

We first bound the size of RBTs.  Let $q_{max}$ be the maximum size of the queue at any time during the sequential execution given by the linearization $L$.
Recall that GC is done on a node every $G$ times its RBT is updated, and we chose $G$ to be $p^2\ceil{\log p}$.
Part of the proof of the following lemma is similar to the proof of \Cref{dSearchTime}.
%We say the index $b$ \emph{is marked in node \var{v}} if some process has tested
%whether $b>\var{v}.\fld{last}[*]$ at line \ref{marktest1} or \ref{marktest2}.
%Intuitively, a block's index value is marked 
%when a dequeue passes through this block to find the enqueue
%whose argument it should return, or when a dequeue that returns null passes through the block.
%Thus, $b$ gets marked in \var{v} when $\var{v}.\fld{blocks}[b]$ 
%contains either an enqueue whose value is being dequeued or a \nl\ dequeue.
%If the index of block $B$ is marked in \var{v}, we also say $B$ is marked in \var{v}.

\begin{lemma}\label{boundAfterGC}
If the maximum \fld{index} in a node's RBT is a multiple of $G$, then
it contains at most $2q_{max}+4p+1$ blocks.
\end{lemma}
\begin{proof}
Consider the invocation $A$ of \op{AddBlock} that updates a node \var{v}'s RBT with the insertion of a block whose 
\fld{index} is a multiple of $G$.
$A$ performs a GC phase.
Let $C$ be the configuration before $A$ invokes \op{SplitBlock} on line \ref{callSplitBlock}.
That call to \op{SplitBlock} recurses up to the root, where it computes $m$ by reading the \var{last}
array.
Let $L_1$ be the prefix of the linearization $L$ corresponding to blocks $1..m$ of the root.
Let $L_2$ be the next segment of the linearization corresponding to blocks $m+1..\ell$ of the root, where
$\ell$ is the last block added to the root's RBT before $C$.

We first bound the number of operations in $L_2$.

The number of enqueues in $L_2$ whose values are still in the queue at the end of $L_2$ is at most $q_{max}$.
Any enqueue in $L_2$ whose value is not still in the queue at the end of $L_2$
must still be in progress at $C$; otherwise the process that dequeued it would have
set its \var{last} entry to a value greater than the index of the block of the root that contains the enqueue prior to $C$,
contradicting the definition of $m$.
So, there are at most $p$ enqueues in $L_2$ whose values are still in the queue at the end of $L_2$.
Thus, there are at most $q_{max}+p$ enqueues in $L_2$.

If a dequeue in $L_2$ returns a non-null value in the sequential execution $L$,
then the value it returns was either in the queue at the end of $L_1$ or it was enqueued during $L_2$.
Thus, there are at most $q_{max}+(q_{max}+p)$ dequeues in $L_2$ that return non-null values.
Any dequeue in $L_2$ that returns a null value in the sequential execution $L$ must still be in progress
at $C$; otherwise the process that performed the dequeue would have set its \var{last} entry
to a value greater than the index of the block of the root that contains the dequeue prior to $C$, 
contradicting the definition of $m$.
So, there are at most $p$ null dequeues in $L_2$.
Thus, there are at most $2q_{max} + 2p$ dequeues in $L_2$.

$A$'s call to \op{SplitBlock} determines the index $s$ used to split \var{v}'s RBT by following 
\eleft\ and \eright\ pointers from the root down to \var{v}.
So, the block returned is a subblock of block $m$ of the root, unless
at some point along the path of subblocks the subblock has already been removed by a split,
in which case \op{SplitBlock} returns a subblock of a block $m'>m$ in the root.

Next, we bound the number of operations in \var{v}'s blocks that are retained when $A$  sets
$T'\assign\op{Split}(T,s)$.
Since $T$ was read before $C$, any operation in $T$ is either in progress at $C$ or has been propagated to the root before $C$.\here{cite a lemma about propagation before operation terminates}
Thus, there are at most $p$ operations in $T$ that do not appear in $L_1\cdot L_2$.
All the rest of the operations in blocks of $T'$ have been propagated to blocks $m..\ell$ of the root.
There are at most $p$ operations in block $m$ of the root %by \Cref{blockSizeGC}
and we showed above that there are at most $2q_{max} + 2p$ in blocks 
$m+1..\ell$ of the root.
Thus, there are at most $2q_{max}+4p$ operations in blocks of $T'$.
Since each block is non-empty by \Cref{blockNotEmpty}, $T'$ contains at most $2q_{max}+4p$ blocks, and one more 
block is inserted before $A$ sets \var{v}'s \fld{blocks} to the resulting RBT.
\end{proof}


\begin{corollary}\label{RBTbound}
At all times, the size of a node's RBT is $O(q_{max}+p+G)$. 
\end{corollary}
\begin{proof}
Each update to a node's RBT adds at most one block to it, increasing its maximum \fld{index} by 1.
Thus, there are at most $G$ updates since the last time its maximum \fld{index} was a multiple of $G$.
The claim follows from \Cref{boundAfterGC}.
\end{proof}

The following theorem bounds the space that is reachable (and therefore cannot be freed by the environment's garbage collector) at any time.
\here{For later version:  There may be room to improve this bound by a more careful analysis since the bounds so far
didn't use the fact that operations only appear along one path in the tree; in other words
doing a bound on total size of all RBTs at one level of the ordering tree at once, rather than
on each node separately, might reduce it to qlogp + poly(p).  Might also be possible to be
more precise about how it depends on "current" size of queue rather than max size.}

\begin{mytheorem}\label{spaceBound}
The queue data structure uses a maximum of $O(pq_{max}+p^3\log p)$ words of memory at any time.
\end{mytheorem}
\begin{proof}
There are $2p-1$ nodes in the ordering tree.  Aside from the RBT, each node uses $O(1)$ memory words.
Each process may hold pointers to $O(1)$ RBTs that are no longer current in local variables.
So the space bound follows from \Cref{RBTbound} and the fact that $G$ is chosen
to be $p^2\ceil{\log p}$.
\end{proof}

\here{It may be possible, via a more careful analysis, to reduce the bound on space usage to $O(\log^2 p(q_{max}+p^3)$.}

Although individual operations may now become more expensive because they have to perform GC,
we show that operations still have polylogarithmic amortized step complexity.

\begin{mytheorem}
The amortized step complexity of each operation is $O(\log p \log(p+q_{max}))$.
\end{mytheorem}
\begin{proof}
\here{this proof may need some modification since code was changed to change how split point is computed}
It follows from \Cref{RBTbound} and the fact that $G=p^2\ceil{\log p}$
that all the RBT routines we use to perform \op{Split}, \op{Insert} and searches for
blocks with a particular index or for a \fld{sum\sub{enq}} value (in line \ref{FRsearchGC} or \ref{getChildGC}) can all be done in $O(\log(p+q_{max}))$ steps.

First, we bound the number of steps taken \emph{excluding} the GC phase in line \ref{GCstart}--\ref{GCend}.
An \op{Enqueue} or null \op{Dequeue} does $O(1)$ RBT operations and other work at each level of the tree during \op{Propagate},
for a total of $O(\log p \log(p+q_{max}))$ steps.
A non-null \op{Dequeue} must also search for a block in the root at line \ref{FRsearchGC}
and call \op{GetEnqueue}.  At each level of the tree, \op{GetEnqueue} does $O(1)$ RBT operations (including a search at line \ref{getChildGC}) and $O(1)$ other steps.
Thus, a \op{Dequeue} also takes $O(\log p \log(p+q_{max}))$ steps.

Now we consider the additional steps a process takes while doing GC in line \ref{GCstart}--\ref{GCend}
and show that the amortized number of GC steps each operation takes is also $O(\log p\log(p+q_{max}))$.
If a process does GC in a call to \op{AddBlock}($\var{v},T,B$) where $B$ has \fld{index} $r\cdot G$ for some integer $r$, we call this the process's \emph{$r$th helping phase on \var{v}}.
We argue that each process $P$ can do an $r$th helping phase on \var{v} at most once as follows.
Consider $P$'s first call $A$ to \op{AddBlock} that does an $r$th helping phase on \var{v}.
Let $\var{v},T,B$ be the arguments of $A$.
Any call to \op{AddBlock} on internal node \var{v} is from line \ref{refreshABGC} of \op{Refresh}, so $B.\fld{index}$
is the maximum \fld{index} in $T$ plus 1.
The \op{Refresh} that called $A$ performed a \op{CAS} at line \ref{casGC}.  Either the \op{CAS} succeeds
or it fails because 
some other \op{CAS} changes \var{v}.\fld{blocks} from $T$ to another tree.
Either way, by \Cref{RBTupdates}, \var{v}'s RBT's maximum \fld{index} will be at least $r\cdot G$ at all times
after this CAS.
So if a subsequent \op{Refresh} by process $P$ 
ever calls \op{AddBlock} on \var{v} again, the block it passes as the third argument
will have $\index>r\cdot G$, so $P$ will not perform an $r$th helping phase on \var{v} again.

Each helping phase takes $O(p)$ steps to read the \var{last} array,
$O(p \log p \log(p+q_{max}))$ steps to \op{Help},
followed by $O(\log(p+q_{max}))$ steps to split and insert a new node into a RBT.
Thus, for each integer $r$ and each node \var{v}, a total of $O(p^2\log p\log(p+q_{max}))$ steps
are performed by all processes during their $r$th helping phase on \var{v}.
We can amortize these steps over the operations that appear in 
$\var{v}.\fld{blocks}[(r-1)p^2+1..p^2]$.
By \Cref{blockNotEmpty}, there are at least $G$ such operations, 
so each operation's amortized number of steps for GC at each node along the path from its leaf to the root
is $O(p^2\log p\log(p+q_{max})/G)=O(\log(p+q_{max}))$.
Hence each operation's amortized number of GC steps is $O(\log p\log(p+q_{max}))$.
\end{proof}

The implementation remains wait-free:  the depth of recursion in each routine is  bounded
by the height of the tree and the only loop is the counted loop in the \op{Help} routine.
Since each operation still only does two \op{CAS} instructions at each level of the tree (at line \ref{casGC}), the following proposition still holds for the space-bounded version of the queue.

\begin{customprop}{\bf{\ref{casbound}}$'$}
Each operation performs $O(\log p)$ \op{CAS} instructions in the worst case.
\end{customprop}
