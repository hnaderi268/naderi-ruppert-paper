% !TEX root =  podc-submission.tex

\section{Proof of Correctness}
\label{sec::correctness}

After proving some basic properties in Section \ref{sec::basicProperties},
we show in Section \ref{sec::propagating} that a double refresh at each node
suffices to propagate an operation to the root.
In Section \ref{sec::tracingCorrect} we show \op{GetEnqueue} and \op{IndexDequeue}
correctly navigate through the  tree.
Finally, we prove linearizability in Section \ref{sec::linearizability}.

\subsection{Basic Properties}
\label{sec::basicProperties}

A \typ{Block} object's fields, except for \fld{super}, are immutable:  they are written only 
when the block is created.
% at line \ref{enqNew} or \ref{deqNew} (for a leaf's block) or lines \ref{initNewBlock}--\ref{computeLength} (for an internal node's block).  
Moreover, only a \op{CAS} at line \ref{setSuper1} can modify the \fld{super} field 
(from \nl\ to a non-\nl\ value), so it is written only once.
Similarly, only a \op{CAS} at line \ref{cas} can modify an element of a node's \fld{blocks} array 
(from \nl\ to a non-\nl\ value), so blocks are permanently added to nodes.
Only a \op{CAS} at line \ref{incrementHead} can update a node's \head\ field by incrementing it,
which implies the following.

\begin{observation} \label{nonDecreasingHead}
For each node \var{v},  \var{v}.\fld{head} is non-decreasing over time.
\end{observation}

\begin{lemma} \label{lem::headInc}
Let $R$ be an instance of \opa{Refresh}{v} whose call to \op{CreateBlock} returns a non-\nl\ block.  When $R$ terminates, \var{v}.\head\ is greater than the value $R$ reads from it at line \ref{readHead}.
\end{lemma}
\begin{proof}
After $R$'s \op{CAS} at line \ref{incrementHead}, \var{v}.\head\ is no longer equal to the value \var{h}
read at line \ref{readHead}.  The claim follows from \Cref{nonDecreasingHead}.
\end{proof}

Now we show $\var{v}.\fld{blocks}[\var{v}.\head]$ is either the last non-\nl\ block or the first \nl\ block in node $\var{v}$.

\begin{invariant}\label{lem::headPosition} 
For $0 \leq i < \var{v}.\head$, $\var{v}.\fld{blocks}[i]\neq\nl$.  For $i>\var{v}.\head$, $\var{v}.\fld{blocks}[i]=\nl$.
If $\var{v}\neq \var{root}$,  $\var{v}.\fld{blocks}[i].super \neq \nl$ for $0<i<\var{v}.\head$.
\end{invariant}

\begin{proof}
Initially, $\var{v}.\head=1$, $\var{v}.\fld{blocks}[0]\neq\nl$  and $\var{v}.\fld{blocks}[i]=\nl$ for  $i>0$, so the claims~hold.

Assume the claim holds before a change to $\var{v}.\fld{blocks}$, which can be made only
by a successful \op{CAS} at line \ref{cas}.
The \op{CAS} changes $\var{v}.\fld{blocks}[h]$ from \nl\ to a non-\nl\ value.
Since $\var{v}.\fld{blocks}[h]$ is \nl\ before the CAS, $\var{v}.\head \leq h$ by the hypothesis.
Since $h$ was read from $\var{v}.\fld{blocks}[h]$ earlier at line \ref{readHead}, 
$\var{v}.\head \geq h$ by \Cref{lem::headPosition}.
So, $h=\var{v}.\head$ and a change to $\var{v}.\fld{blocks}[\var{v}.\head]$ preserves the invariant.

Now, assume the claim holds before a change to $\var{v}.\head$, which can only be an increment from $h$ to $h+1$
by a successful \op{CAS} at line \ref{incrementHead} of \op{Advance}.
For the first two claims, it suffices to show that $\var{v}.\fld{blocks}[head] \neq \nl$.
\nf{Advance} is called either at line \ref{helpAdvance} 
after testing that $\var{v}.\fld{blocks}[h]\neq\nl$ at line \ref{ifHeadnotNull},
or at line \ref{advance} after the \op{CAS} at line \ref{cas} ensures $\var{v}.\fld{blocks}[h]\neq\nl$.
For the third claim, observe that prior to incrementing $\var{v}.\head$ at line \ref{incrementHead},
the \op{CAS} at line \ref{setSuper1} ensures that $\var{v}.\fld{blocks}[i].super\neq \nl$.
\end{proof}

It follows that blocks accessed by the \op{Enqueue}, \op{Dequeue} and \op{CreateBlock} routines are non-\nl.

The following two lemmas show that no operation appears in more than one block of the root.
\begin{lemma} \label{lem::headProgress}
 If $b>0$ and $\var{v}.\fld{blocks}[b] \neq \nl$ then 
 $\var{v}.\fld{blocks}[b-1].\fld{end\sub{left}} \leq \var{v}.\fld{blocks}[b].\fld{end\sub{left}}$ and 
 $\var{v}.\fld{blocks}[b-1].\fld{end\sub{right}} \leq \var{v}.\fld{blocks}[b].\fld{end\sub{right}}$.
\end{lemma}
\begin{proof}
Let $B$ be the block in $\var{v}.\fld{blocks}[b]$.
Before creating $B$ at line \ref{invokeCreateBlock}, the \op{Refresh} that installed $B$
read $b$ from $\var{v}.\head$ at line \ref{readHead}.
At that time, $\var{v}.\fld{blocks}[b-1]$ contained a block $B'$, by \Cref{lem::headPosition}.
Thus, the \op{CreateBlock}($\var{v},b-1$) that created $B'$ terminated before the \op{CreateBlock}($\var{v},b$) that
created $B$ started.
It follows from \Cref{nonDecreasingHead} that the value that 
line \ref{createEndLeft} of \op{CreateBlock}($\var{v},b-1$) stores in $B'.\fld{end\sub{left}}$   
is less than or equal to the value that line \ref{createEndLeft} of \op{CreateBlock}($\var{v},b$) 
stores in $B.\fld{end\sub{left}}$.
Similarly, the values stored in $B'.\eright$ and $B.\eright$ at line \ref{createEndRight} 
%of these calls to \op{CreateBlock} 
satisfy the claim.
\end{proof}

\begin{lemma} \label{lem::subblocksDistinct}
If $B$ and $B'$ are two blocks in nodes at the same depth, their sets of subblocks are disjoint.
\end{lemma}
\begin{proof}
We prove the lemma by reverse induction on the depth.
If $B$ and $B'$ are in leaves, they have no subblocks, so the claim holds.
Assume the claim holds for nodes at depth $d+1$ and let $B$ and $B'$ be two blocks in nodes at depth $d$.
Consider the direct subblocks of $B$ and $B'$ defined by~(\ref{defsubblock}).
If $B$ and $B'$ are in different nodes at depth $d$, then their direct subblocks are disjoint.
If $B$ and $B'$ are in the same node, it follows from \Cref{lem::headProgress} that their direct subblocks are disjoint.
Either way, their direct subblocks (at depth $d+1$) are disjoint, so the claim follows from the induction hypothesis.
\end{proof}

It follows that each block has at most one superblock.
Moreover, we can now prove each operation is contained in at most one block of each node,
and hence appears at most once in the linearization~$L$.

\here{Might be able to get rid of this corollary and just cite previous lemma instead to save space}
\begin{corollary}\label{lem::noDuplicates}
For  $i\neq j$, $\var{v}.\fld{blocks}[i]$ and $\var{v}.\fld{blocks}[j]$ cannot both contain the same operation.
\end{corollary}
\begin{proof}
A block $B$ contains the operations in $B$'s subblocks in leaves of the tree.
An operation by process $P$ appears in one block of $P$'s leaf, so
an operation 
cannot be in two different leaf blocks. 
By \Cref{lem::subblocksDistinct}, $\var{v}.\fld{blocks}[i]$ and $\var{v}.\fld{blocks}[j]$ have no common subblocks, so the claim follows.
\end{proof}



%\begin{definition}
%$n\nf{.blocks[}i\nf{]}$ is \emph{established} if $n\nf{.head}>i$. An operation is \it{established} in node $n$ 
%if it is in an established block of $n$. $EST^t_n$ is the set of established operations in node $n$ at time $t$.
%\end{definition}
%
%Now we want to say that blocks of a node grow over time.
%\begin{observation}\label{lem::blocksOrder}
%  If  time $t<$ time $t^\prime$ ($t$ is before $t^\prime$), then $ops(n.blocks)$ at time $t$ is a subset of 
%$ops(n.blocks)$ at time $t^\prime$.
%\end{observation}
%\begin{proof}
%Blocks are only appended (not modified) with \nf{CAS} to $n\nf{.blocks[}n\nf{.head]}$, so the set of the blocks of a node after the \nf{CAS} contains the set of the blocks before the \nf{CAS}.
%\end{proof}

% \begin{corollary}\label{lem::establishedOrder}
%   If  time $t<$ time $t^\prime$, then $EST_n^t\subseteq EST_n^{t^\prime}$.
% \end{corollary}
% \begin{proof}
% From Observations \ref{nonDecreasingHead}, \ref{lem::blocksOrder}.  
% \end{proof}

The accuracy of the values stored in the \fld{sum\sub{enq}} and \fld{sum\sub{deq}} fields
on line \ref{enqNew}, \ref{deqNew}, \ref{createSumEnq} and \ref{createSumDeq} follows easily
from the definition of subblocks.  See Appendix 
\ref{app::tracingDetails} for a detailed proof.

\begin{restatable}{invariant}{sumRes}
\label{lem::sum}
If $B$ is a block stored in $\var{v}.\fld{blocks}[i]$,
$B.\fld{sum\sub{enq}} = | E(\var{v}.\fld{blocks}[0])\cdots E(\var{v}.\fld{blocks}[i]) |$ and
$B.\fld{sum\sub{deq}} = | D(\var{v}.\fld{blocks}[0])\cdots D(\var{v}.\fld{blocks}[i]) |$.
\end{restatable}

This allows us to prove that every block a Refresh installs contains at least one operation.

\begin{corollary}\label{blockNotEmpty}
If a block $B$ is in $\var{v}.\fld{blocks}[i]$ where $i>0$, then $E(B)$ and $D(B)$ are not both empty.
\end{corollary}
\begin{proof}
The \op{Refresh} that installed $B$ got $B$ as the response to its call to \op{CreateBlock} on line \ref{invokeCreateBlock}.
Thus, at line \ref{testEmpty} $\var{num\sub{enq}}+\var{num\sub{deq}}\neq 0$.
By \Cref{lem::sum}, $\var{num\sub{enq}} = |E(B)|$ and $\var{num\sub{deq}} = |D(B)|$,
so these sequences cannot both be empty.
\end{proof}



\subsection{Propagating Operations to the Root}
\label{sec::propagating}

Next, we show two \op{Refresh}es suffice to propagate operations from a child to its parent.
We say that node $\var{v}$ \emph{contains} an operation $op$ if some block in $\var{v}.\fld{blocks}$ contains $op$.
%\here{move this defn earlier and just recall it here?}
Since blocks are permanently added to nodes, if $\var{v}$ contains $op$ at some time, $v$ contains $op$ at all later times too.

\begin{lemma}\label{successfulRefresh}
Let $R$ be a call to \op{Refresh}($\var{v}$) that performs a successful \op{CAS} on line \ref{cas} (or terminates at line \ref{addOP}).
In the configuration after that CAS (or termination, respectively), $\var{v}$ contains all operations that $\var{v}$'s children contained 
when $R$ executed line~\ref{readHead}.
\end{lemma}
\begin{proof}
Suppose $\var{v}$'s child (without loss of generality, $\var{v}.\fld{left}$) contained an operation $op$ 
when $R$ executed line \ref{readHead}.
Let $i$ be the index such that the block $B=\var{v}.\fld{left.blocks}[i]$ contains $op$.
By \Cref{nonDecreasingHead} and \Cref{lem::headProgress}, the value of $childHead$ that $R$ reads from
$\var{v}.\fld{left.head}$ in line \ref{readChildHead} is at least $i$.
If it is equal to $i$, $R$ calls \op{Advance} at line \ref{helpAdvance}, which ensures that 
$\var{v}.\fld{left.head} > i$.
Then, $R$ calls \op{CreateBlock}($\var{v},h$) in line \ref{invokeCreateBlock}, where $h$ is the value $R$ reads at line \ref{readHead}.
\op{CreateBlock} reads a value greater than $i$ from $\var{v}.\fld{left.head}$ at line \ref{createEndLeft}.
Thus, $new.\eleft \geq i$.  We consider two cases.

Suppose $R$'s call to \op{CreateBlock} returns the new block $B'$ and $R$'s \op{CAS} at line \ref{cas} 
installs $B'$ in $\var{v}.\fld{blocks}$.
Then, $B$ is a subblock of some block in $\var{v}$, since  $B'.\eleft$ is greater than or equal to $B$'s index
in $\var{v}.\fld{left.blocks}$.
Hence $\var{v}$ contains $op$, as required.

%\Eric{I think this case was missing from the proof in the thesis}
Now suppose $R$'s call to \op{CreateBlock} returns \nl, causing $R$ to terminate at line \ref{addOP}.
Intuitively, since there are no operations in $\var{v}$'s children to promote, $op$ is already in $\var{v}$.
We formalize this intuition.
The value computed at line \ref{createSumEnq} is
\begin{eqnarray*}
\var{num\sub{enq}} \hspace*{-2mm}
&=& \hspace*{-2mm} \var{v}.\fld{left.blocks}[new.\eleft].\fld{sum\sub{enq}} + \var{v}.\fld{right.blocks}[new.\eright].\fld{sum\sub{enq}} - \var{v}.\fld{blocks}[h-1].\fld{sum\sub{enq}} \\
&=& \hspace*{-2mm}\var{v}.\fld{left.blocks}[new.\eleft].\fld{sum\sub{enq}} + \var{v}.\fld{right.blocks}[new.\eright].\fld{sum\sub{enq}} \\
&&- \var{v}.\fld{left.blocks}[\var{v}.\fld{blocks}[h-1].\eleft].\fld{sum\sub{enq}} - \var{v}.\fld{right.blocks}[\var{v}.\fld{blocks}[h-1].\eright].\fld{sum\sub{enq}}
\end{eqnarray*}
By \Cref{lem::sum}, $num\sub{enq}$ is the  number of enqueues  in 
$\var{v}.\fld{left.blocks}[\var{v}.\fld{blocks}[h-1].\eleft+1..new.\eleft]$ and
$\var{v}.\fld{right.blocks}[\var{v}.\fld{blocks}[h-1].\eright+1..new.\eright]$.
Similarly, $num\sub{deq}$ is the total number of dequeues contained in these blocks.
Since $num\sub{enq}+num\sub{deq}=0$ at line \ref{testEmpty},
these blocks contain no operations.
By \Cref{blockNotEmpty}, this means the ranges of blocks are empty, so that $\var{v}.\fld{blocks}[h-1].\eleft \geq \var{new}.\eleft \geq i$.
Hence, $B$ is already a subblock of some block in $\var{v}$, so $\var{v}$ contains $op$.
\end{proof}

We now show that if a process fails its \op{Refresh}($\var{v}$) twice, some other process must have succeeded.

\begin{lemma}\label{lem::doubleRefresh}
Consider two consecutive terminating calls $R_1$, $R_2$ to \op{Refresh}($\var{v}$) by the same process.
All operations contained $\var{v}$'s children when $R_1$ begins
are contained in $\var{v}$ when $R_2$ terminates.
\end{lemma}
\begin{proof}
If either $R_1$ or $R_2$ performs a successful \op{CAS} at line \ref{cas} or terminates at line \ref{addOP}, the claim follows
from \Cref{successfulRefresh}.
So suppose both $R_1$ and $R_2$ perform a failed \op{CAS} at line \ref{cas}.
Let $h_1$ and $h_2$ be the values $R_1$ and $R_2$ read from $\var{v}.\head$ at line \ref{readHead}.
By \Cref{lem::headInc}, $h_2>h_1$.
By \Cref{lem::headProgress}, $\var{v}.blocks[h_2]=\nl$ when $R_1$ executes line \ref{readHead}.
Since $R_2$ fails its \op{CAS} on $\var{v}.blocks[h_2]$, some other \op{Refresh} $R_3$ must have done
a successful \op{CAS} on $\var{v}.blocks[h_2]$ before $R_2$'s \op{CAS}.
$R_3$ must have executed line \ref{readHead} after $R_1$, since $R_3$ read the value $h_2$ from $\var{v}.\head$ and the value of $\var{v}.\head$ is non-decreasing, by \Cref{nonDecreasingHead}.
Thus, all operations contained in $\var{v}$'s children when $R_1$ begins
are also contained in $\var{v}$'s children when $R_3$ later executes line \ref{readHead}.
By \Cref{successfulRefresh}, these operations are contained in $\var{v}$ when $R_3$ performs its successful \op{CAS},
which is before $R_2$ terminates.
\end{proof}

\begin{lemma} \label{lem::appendExactlyOnce}
When an \op{Append}($B$) terminates, $B$'s operation is contained in exactly one block in each node along the path from the process's leaf to the root.
\end{lemma}
\begin{proof}
\op{Append} adds $B$ to the process's leaf and calls \op{Propagate}, which
does a double \op{Refresh}~on each internal node on the path $P$ from the leaf to the root.
By \Cref{lem::doubleRefresh}, this ensures a block in each node on $P$ contains $B$'s operation.
There is at most one such block in each node, by \Cref{lem::noDuplicates}.
\end{proof}


\subsection{Correctness of \opemph{GetEnqueue} and \opemph{IndexDequeue}}
\label{sec::tracingCorrect}

See Appendix \ref{app::tracingDetails} for detailed proofs for this section.

We first show the \fld{super} field is accurate, since \op{IndexDequeue} uses it to trace superblocks up the tree.  This is proved by showing that the \fld{super} field of a block $B$ in node $\var{v}$ is read from the 
\var{v}.\fld{parent}'s \var{head} field close to the time that $B$'s superblock $B_s$ is installed in the parent node.
On one hand, $B.super$ is written before $B_s$ is installed:
\op{Advance} writes $B.super$ before advancing $\var{v}.\head$  past $B$'s index, which must happen before the \op{CreateBlock} that creates
$B_s$ gets the value of $B_s.\eleft$ or $B_s.\eright$.
On the other hand, $B.\fld{super}$ cannot be written too long before $B_s$ is installed:
$B.\fld{super}$ is written after $B$ is installed, and \Cref{successfulRefresh} ensures that $B$ is propagated to the
parent soon after.

\begin{restatable}{lemma}{superRelationRes}
\label{superRelation}
Let $B=\var{v}.\var{blocks}[b]$.
  If $\var{v}.\fld{parent.blocks}[s]$ is the superblock of $B$ then $s-1\leq B.\fld{super}\leq s$.
\end{restatable}


%The reader may wonder when the case $b\nf{.super}=s$ happens. This can happen when $
%\nf{$n$.parent.blocks[$B$.super]}=\nf{null}$ when $B$\nf{.super} is written and $R_p$ puts its created block 
%into \nf{$n$.parent.blocks[$B$\nf{.super}]} afterwards.
Showing  \op{GetEnqueue} and \op{IndexDequeue} work correctly is mostly just checking that they correctly compute the index of the required block
and the operation's rank within the block.  
For \op{IndexDequeue}, we use \Cref{superRelation} each time \op{IndexDequeue} goes one step up the tree.
\here{mention somewhere why precondition of IndexDequeue is true? also check that block of parent indexed by sup in that routine is non-null}

\begin{restatable}{lemma}{indexDequeueRes}
\label{lem::indexDequeue}
If $\var{v}.\fld{blocks}[b]$ has been propagated to the root and $1\leq i\leq |D(\var{v}.\fld{blocks}[b])|$, 
 then \op{IndexDequeue}($\var{v}, b, i$) returns $\langle b',i' \rangle$ such that the \var{i}th dequeue in $D(\var{v}.\fld{blocks}[\var{b}])$ is the $(i')$th dequeue of $D(\var{root}.\fld{blocks}[b'])$.
\end{restatable}

\begin{restatable}{lemma}{getEnqRes}
\label{lem::get}
If $1\leq i\leq |E(\var{v}.\fld{blocks}[b])|$ then \op{getEnqueue}($\var{v},b,i$) returns the argument of the $i$th enqueue in $E(\var{v}.\fld{blocks}[b])$.
\end{restatable}

\subsection{Linearizability}
\label{sec::linearizability}

We show that the linearization ordering $L$ defined in 
(\ref{linearization}) is a legal permutation of a subset of the operations in 
the execution, i.e., that it includes all operations that terminate and 
if one operation $op_1$ terminates before another operation $op_2$ begins, then $op_2$ does not precede $op_1$ in $L$.  We also show result each dequeue returns is the same as in the sequential execution  $L$.

\begin{lemma} \label{linearSat}
$L$ is a legal linearization ordering.
\end{lemma}
\begin{proof}
By \Cref{lem::noDuplicates}, $L$ is a permutation of a subset of the operations in the execution.
By \Cref{lem::appendExactlyOnce}, each terminating operation is propagated to the root before it terminates,
so it appears in $L$.
Also, if $op_{1}$ terminates before $op_{2}$ begins, then $op_{1}$ 
is propagated to the root before $op_2$ begins, so $op_2$ appears before $op_2$ in $L$.
\end{proof}

A simple proof (in Appendix \ref{app::tracingDetails}) shows that \fld{size} fields are computed correctly.
\begin{restatable}{lemma}{sizeCorrectRes}
\label{sizeCorrectness}
If the operations of $\var{root}.\fld{blocks}[0..b]$ are applied sequentially in the order of~$L$ on an initially empty queue, the resulting queue has $\var{root}.\fld{blocks}[b].\size$ elements.  
\end{restatable}

Next, we show operations return the same response as they would in the sequential execution $L$.

\begin{lemma}\label{linearCorrect}
Each terminating dequeue returns the response it would in the sequential execution $L$.
\end{lemma}
\begin{proof}
If a dequeue $D$ terminates, it is contained in some block in the root, by \Cref{lem::appendExactlyOnce}.
By \Cref{lem::indexDequeue}, $D$'s call to \op{IndexDequeue} on line \ref{invokeIndexDequeue}
returns a pair $\langle b,i\rangle$ such that $D$ is the $i$th dequeue in the block 
$B=\var{root}.\fld{blocks}[b]$.
$D$ then calls \op{FindResponse}($b,i$) on line \ref{deqRest}.
By \Cref{sizeCorrectness}, the queue contains $\var{root}.\fld{blocks}[b-1].\fld{size}$ elements
after the operations in $\var{root}.\fld{blocks}[1..b-1]$ are performed sequentially 
in the order given by $L$.
By \Cref{lem::sum}, the value of \var{num\sub{enq}} computed on line \ref{FRNum}
is the number of enqueues in $B$.
Since the enqueues in block $B$ precede the dequeues,
the queue is empty when the $i$th dequeue of $B$ occurs if 
$\var{root}.\fld{blocks}[b-1].\fld{size} + \var{num\sub{enq}} < i$.
So $D$ returns \nl\ on line \ref{returnNull} if and only if it would do so in the sequential
execution $L$.
Otherwise, the size of the queue after doing the operations in $\var{root}.\fld{blocks}[0..b-1]$
in the sequential execution $L$ is $\var{root}.\fld{blocks}[b-1].\fld{sum\sub{enq}}$ minus
the number of non-\nl\ dequeues in that prefix of $L$.
Hence, line \ref{computeE} sets $e$ to the rank of $D$ among all the non-\nl\ dequeues in $L$.
Thus, in the sequential execution~$L$, $D$ returns the value enqueued by the $e$th enqueue in $L$.
By \Cref{lem::sum}, this enqueue is the $i_e$th enqueue 
in $E(\var{root}.\fld{blocks}[b_e])$, where
$b_e$ and $i_e$ are the values $D$ computes on line \ref{FRb} and \ref{FRi}.
By \Cref{lem::get}, the call to \op{GetEnqueue} returns the argument of the required enqueue.
\end{proof}

Combining \Cref{linearSat} and \Cref{linearCorrect} provides our main result.

\begin{mytheorem}
The queue implementation is linearizable.
\end{mytheorem}


\here{Should there be a lemma somewhere in this section that explicitly shows that each
object we dereference is non-null?}