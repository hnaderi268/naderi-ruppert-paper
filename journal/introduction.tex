% !TEX root =  queue.tex

\section{Introduction}

There has been a great deal of research in the past several decades on the design of shared queues.
Besides being a fundamental data structure, queues are used in
significant concurrent applications, including OS kernels \cite{MP91}, memory management \cite{BBFRSW21}\here{try to find an older, more canonical reference},
%packet processing \cite{DPDK}, 
synchronization \cite{KAE23},\here{is this a good citation for this?}
and sharing resources or tasks.
We focus on shared queues that are \emph{linearizable} \cite{HW90}, meaning that operations
appear to take place atomically, and \emph{lock-free}, meaning that some operation on the queue
is guaranteed to complete regardless of how asynchronous processes are scheduled to take~steps.

The lock-free MS-queue of Michael and Scott \cite{MS98} is a classic shared queue implementation.
It uses a singly-linked list with pointers to the front and back nodes.
To dequeue or enqueue an element, the front or back pointer is updated by a 
compare-and-swap (CAS) instruction.
If this CAS fails, the operation must retry.
In the worst case, this means that each successful CAS may cause all other processes to
fail and retry, leading to an amortized step complexity of $\Omega(p)$ per operation in a system of $p$ processes.
(To measure amortized step complexity of a lock-free implementation, we consider all possible finite executions
and divide the number of steps in the execution by the number of operations  in the execution.)
Numerous papers have suggested modifications to the MS-queue \cite{DBLP:conf/opodis/HoffmanSS07,DBLP:conf/podc/KoganH14,DBLP:conf/ppopp/KoganP11,DBLP:journals/dc/Ladan-MozesS08,MKLLP22,DBLP:conf/spaa/MoirNSS05,RC17}, but 
all still have $\Omega(p)$ amortized step complexity as a result of
contention on the front and back of the queue.
Morrison and Afek \cite{DBLP:conf/ppopp/MorrisonA13} called this the \emph{CAS retry problem}.
The same problem occurs in array-based implementations of queues \cite{DBLP:conf/iceccs/ColvinG05,DBLP:conf/icdcn/Shafiei09,DBLP:conf/spaa/TsigasZ01,DBLP:conf/opodis/GidenstamST10}.
Solutions that tried to sidestep this problem using fetch\&increment \cite{DBLP:conf/ppopp/MorrisonA13,DBLP:conf/ppopp/YangM16,Nik19,10.1145/3490148.3538572}
rely on slower mechanisms to handle worst-case executions and still have $\Omega(p)$ step complexity.

Many concurrent data structures that keep track of a set of elements also have an $\Omega(p)$ term in their step complexity, as observed by Ruppert \cite{Rup16}.
For example, lock-free lists \cite{FR04,Sha15}, stacks \cite{Tre86} and search trees \cite{EFHR14} 
have an $\Omega(c)$ term in their step complexity, where $c$ represents contention,
the number of processes that access the data structure concurrently, which can be $p$ in the worst case.
Attiya and Fouren \cite{DBLP:conf/opodis/AttiyaF17} proved 
that amortized $\Omega(c)$ steps per operation are indeed necessary
for any CAS-based implementation of a lock-free bag data structure, which provides operations
to insert an element or remove an arbitrary element (chosen non-deterministically).
Since a queue trivially implements a bag, this lower bound also applies to queues.
Although this might seem to settle the step complexity of lock-free queues, the lower bound
holds only if $c$ is $O(\log\log p)$ so it should be stated more precisely as
%At first glance, this would seem to settle the question of the step complexity of lock-free queues.
%However, the lower bound leaves a loophole:  it holds only if $c$ is $O(\log\log p)$.
%Thus, the lower bound could be stated more precisely as 
an amortized bound of $\Omega(\min(c,\log\log p))$ steps per operation.

We exploit this loophole.  We show  it is, in fact, possible for a linearizable queue
to have step complexity sublinear in $p$.
Our queue is the first whose step complexity  is polylogarithmic in $p$ and in $q$, the number of elements in the queue.
It is \emph{wait-free}, meaning that every operation is guaranteed to complete within a finite number of its own steps.
For ease of presentation, we first give an unbounded-space construction where enqueues take $O(\log p)$ steps and
dequeues take $O(\log^2 p + \log q)$ steps,
and then modify it to bound the space
while  having $O(\log p\log( p+ q))$ amortized step complexity  per operation.
Moreover, each operation does $O(\log p)$ CAS instructions in the worst case, whereas previous
lock-free queues use 
%an unbounded number of CAS instructions in the worst case and 
$\Omega(p)$ CAS instructions, even in an amortized sense.
Since a queue is also a bag, our queue is the first lock-free bag that we know of that has polylogarithmic step complexity.
Both versions of our queue use single-word CAS on reasonably-sized 
words.
We assume that a word is large enough to store an item to be enqueued (or at least a pointer to it).  We also assume that the number of operations performed on the queue can be stored (in binary) in $O(1)$ words.
This is analogous to the assumption for the classical RAM model that the number of bits per word is logarithmic in the problem size.
For the space-bounded version, we unlink unneeded objects from our data structure.
We do not address the orthogonal problem of reclaiming memory; we assume a safe
garbage collector, such as the highly optimized one that Java provides.

Our queue uses a binary tree, called the \emph{\ordering\ tree}, where each process has its own leaf.
A process adds its operations to its leaf.
As in previous work (e.g., \cite{DBLP:conf/stoc/AfekDT95,DBLP:conf/fsttcs/JayantiP05}), operations are propagated from the leaves up to the root in a cooperative way that ensures wait-freedom
and avoids the CAS retry problem.
Operations in the root are ordered, 
and this order is used to linearize the operations and compute their responses.
\here{Either here or in related work section, talk about previous usage of ordering tree and how ours differs from it}
Explicitly storing  operations in the tree nodes would be too costly.
Instead, we use a novel implicit representation of sets
of operations that allows us to quickly merge two sets from the children of a node,
and quickly access any  operation in a~set.
A preliminary version of this work appeared in~\cite{Nad22}.
\here{Maybe say a little more about techniques used in the implementation, if space permits}
