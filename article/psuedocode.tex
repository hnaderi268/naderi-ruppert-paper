\documentclass[10pt]{article}
\renewcommand{\baselinestretch}{1.8}

\usepackage[b4paper,left=0.8in,right=0.8in,top=0.8in,bottom=0.8in]{geometry}
\usepackage{tikz-qtree}
\usepackage{algorithm}
\usepackage{algpseudocode}
\makeatletter
\renewcommand{\ALG@beginalgorithmic}{\footnotesize}
\makeatother
\usepackage{graphicx}
\usepackage{subcaption}
\usepackage{showkeys}
\usepackage{amsmath}


\usepackage{multicol}
\setlength\columnsep{24pt}

\algnewcommand\algorithmicforeach{\textbf{for each}}
\algdef{S}[FOR]{ForEach}[1]{\algorithmicforeach\ #1\ \algorithmicdo}

\algdef{S}[FUNCTION]{Function}
   [3]{{\texttt{\textsl{#1}}} \textproc{\texttt{#2}}\ifthenelse{\equal{#3}{}}{}{\texttt{(#3)}}}
  
\algdef{E}[FUNCTION]{EndFunction}
   [1]{\algorithmicend\ \texttt{\textproc{#1}}}

\algrenewcommand\Call[2]{\texttt{\textproc{#1}\ifthenelse{\equal{#2}{}}{}{(#2)}}}
   
\newcommand\keywordfont{\sffamily\bfseries}
\algrenewcommand\algorithmicend{{\keywordfont end}}
\algrenewcommand\algorithmicfor{{\keywordfont for}}
\algrenewcommand\algorithmicforeach{{\keywordfont for each}}
\algrenewcommand\algorithmicdo{{\keywordfont do}}
\algrenewcommand\algorithmicuntil{{\keywordfont until}}
\algrenewcommand\algorithmicfunction{{\keywordfont function}}
\algrenewcommand\algorithmicif{{\keywordfont if}}
\algrenewcommand\algorithmicthen{{\keywordfont then}}
\algrenewcommand\algorithmicelse{{\keywordfont else}}
\algrenewcommand\algorithmicreturn{{\keywordfont return}}

\renewcommand\thealgorithm{}
\newcommand{\setalglineno}[1]{%
  \setcounter{ALC@line}{\numexpr#1-1}}


\usepackage{amsmath,amssymb,amsthm}
\newtheorem{theorem}{Theorem}
\newtheorem{lemma}[theorem]{Lemma}
\theoremstyle{definition}
\newtheorem{definition}[theorem]{Definition}

\begin{document}

\begin{algorithm}
\caption{Queue \label{algQ}}
\begin{algorithmic}[1]
\begin{multicols}{2}

%\Statex \textbf{Structure}

\Statex $\diamondsuit$ \texttt{\textsl{Local}}
\begin{itemize}
\item \texttt{\textsl{*Node} leaf\textsf{: pointer the the process's leaf in the tree}}
\end{itemize}

\Statex

\Statex $\diamondsuit$ \texttt{\textsl{Shared}}
\begin{itemize}
\item \texttt{\textsl{Tree}} \textit{to complete, how?}
\end{itemize}

\Statex
\Statex $\diamondsuit$ \texttt{\textsl{Structures}}

\Statex $\blacktriangleright$ \texttt{\textsl{Node}}
\begin{itemize}
\item \texttt{\textsl{*Node} left, right, parent}
\item \texttt{\textsl{Block[]} blocks}\textsf{: index 0 contains an empty block with all fields equal to 0 and \texttt{en} pointers to the first block of the coresponding children. \texttt{blocks[i]} returns the $i$th block stored.}
\item \texttt{\textsl{int} head= 1}\textsf{: index of the first empty cell of \texttt{blocks}}
\item \texttt{\textsl{int} counter= 0}\textsf{}
\item \texttt{\textsl{int[]} super}\textsf{: \texttt{super[i]} stores the index of a superblock in parent that contains some block of this node whose \texttt{time} is field \texttt{i}}
\end{itemize}

\Statex $\blacktriangleright$ \texttt{\textsl{leaf} extends Node}
\begin{itemize}
  \item \texttt{\textsl{int[]} response}
  
  \textsf{leaf.response[i] stores response of leaf.ops[i]}
  
  \item \texttt{\textsl{int} maxOld}

  \textsf{Index of the youngest old block in the root that this process has seen yet.}
  
\end{itemize}


\Statex $\blacktriangleright$ \texttt{\textsl{Block}}

\begin{itemize}
  \item \texttt{\textsl{int} num\textsubscript{enq-left}, sum\textsubscript{enq-left}}
  \textsf{:~~\#enqueues from subblocks in left child, prefix sum of \texttt{num\textsubscript{enq-left}}}
  \item \texttt{\textsl{int} num\textsubscript{deq-left}, sum\textsubscript{deq-left}}
  \textsf{:~~\#dequeues from subblocks in left child, prefix sum of \texttt{num\textsubscript{deq-left}}}
  \item \texttt{\textsl{int} num\textsubscript{enq-right}, \texttt{sum\textsubscript{enq-right}}}
  \textsf{: ~\#enqueues from subblocks in right child, prefix sum of \texttt{num\textsubscript{enq-right}}}
  \item \texttt{\textsl{int} num\textsubscript{deq-right}, \texttt{sum\textsubscript{deq-right}}}
  \textsf{: ~\#dequeues from subblocks in right child, prefix sum of \texttt{num\textsubscript{deq-right}}}
  \item \texttt{\textsl{int} num\textsubscript{enq}, num\textsubscript{deq}}
  \textsf{: \# enqueue, dequeue operations in the block}
  \item \texttt{\textsl{int} sum\textsubscript{enq}, sum\textsubscript{deq}}
  \textsf{: sum of \# enqueue, dequeue operations in blocks up to this one}
  \item \texttt{\textsl{int} num, sum}
  \textsf{: total \# operations in block, prefix sum of \texttt{num}}
  \item \texttt{\textsl{int} end\textsubscript{left}, end\textsubscript{right}}
  \textsf{:~~index of the last subblock in the left and right child}
  \item \texttt{\textsl{int} group}
  \textsf{: id of the group of blocks including this propagated together, more precisely the value of the node's counter when propagating this block.}
  \item \texttt{\textsl{int} order}
  \textsf{: the index of the block in the node containing it}
\end{itemize}

\pagebreak

\Statex $\blacktriangleright$ \texttt{\textsl{Leaf Block} extends \textsl{Block}}
\begin{itemize}
  \item \texttt{\textsl{Object} element}
  \textsf{Each block in a leaf also represents an operation. The \texttt{element} shows the operations argument if it is an enqueue, and if it is a dequeue the value is \texttt{null}.}
\end{itemize}

\Statex $\blacktriangleright$ \texttt{\textsl{Root Block} extends \textsl{Block}}
\begin{itemize}
  \item \texttt{\textsl{int} size}
  \textsf{: size of queue after this block's operations finish}
  \item \texttt{\textsl{int} sum\textsubscript{non-null deq}}
  \textsf{: count of non-null dequeus up to this block}
  \item \texttt{\textsl{int} age}
  \textsf{: number of finished operations in the block}
\end{itemize}

\Statex

\Function{void}{Enqueue}{\textsl{Object} e}
\State \texttt{block b= \Call{new}{\textsl{block}}}
\State \texttt{b.element= e}
\State \texttt{b.sum\textsubscript{enq}=1}
\State \Call{Append}{b}
\EndFunction{Enqueue}

\Statex

\Function{Object}{Dequeue()}{}
\State \texttt{block b= \Call{new}{\textsl{block}}}
\State \texttt{b.element= null}
\State \texttt{b.sum\textsubscript{deq}=1}
\State \Call{Append}{b}
\State \texttt{<i, b>=} \Call{Index}{l\textsubscript{pid}, b.order, 1}
\State \texttt{res=} \Call{ComputeHead}{i, b} \Comment{Index of the enqueue whose argument should be returned} \label{deqRest}
\State \Return \Call{Get}{res}
\State \texttt{b\textsubscript{i}= b}\Comment{block in the root contains the invocation of dequeue}
\State \texttt{b\textsubscript{r}= root.blocks.get(sum\textsubscript{enq}==i)}\Comment{block in the root contains the invocation of dequeue}
\State \texttt{b\textsubscript{i}.age= b\textsubscript{i}.age+1}
\State \texttt{b\textsubscript{r}.age= b\textsubscript{r}.age+1}
\If{\texttt{b\textsubscript{i} or b\textsubscript{r} become old}} \texttt{update maxOld}
\EndIf
\EndFunction{Dequeue}

\Statex

\Function{int}{ComputeHead}{int i, int b}\Comment{Computes head of the queue when \texttt{i}th dequeue in \texttt{b}th block occurs. The dequeue should return the argument of the head enqueue.}
\If{\texttt{root.blocks[b-1].size + root.blocks[b].num\textsubscript{enq} - i $<$ 0}}
\State \Return \texttt{-1}
\Else{ \Return \texttt{root.blocks[b-1].sum\textsubscript{non-null deq} + i}}
\EndIf
\EndFunction{ComputeHead}

\Statex

\Function{void}{Append}{\textsl{block} b} 
\State \texttt{b.group= this.leaf.head}
\State \texttt{l\textsubscript{pid}.blocks[this.leaf.head]= b}
\State \texttt{this.leaf.head+=1}
\State \Call{Propagate}{this.leaf.parent} 
\EndFunction{Append}

\end{multicols}
\end{algorithmic}
\end{algorithm}


%##########################################


\begin{algorithm}
%\caption{Queue Continued}
\begin{algorithmic}[1]
\setcounter{ALG@line}{33}
\begin{multicols}{2}

\Function{void}{Propagate}{\textsl{node} n}
\If{\texttt{not} \Call{Refresh}{n}}
\State \Call{Refresh}{n}
\EndIf
\If{\texttt{n.parent} \textbf{is not} \texttt{null}} 
\State \Call{Propagate}{n.parent}
\EndIf
\EndFunction{Propagate}

\Statex

\Function{boolean}{Refresh}{\textsl{node} n}
\State \texttt{h= n.head}
%\If{\texttt{n.blocks[h]!=null?}} \texttt{h+=1}\EndIf
\State \texttt{c= n.counter}
\State \texttt{<new, c\textsubscript{left}, c\textsubscript{right}>= \Call{CreateBlock}{n, h}}
\State \texttt{new.group= c}
\If{\texttt{new.num==0}} \Return{\texttt{true}}
\ElsIf{\texttt{n} \textbf{is} \texttt{root}}
\If{\texttt{root.blocks.append(new)}}\State\texttt{goto \ref{okcas}}\EndIf
\ElsIf{\Call{CAS}{n.blocks[h], null, new}}
\ForEach{\texttt{dir} {\keywordfont{in}} \texttt{\{left, right\}}} \label{okcas}
\State \texttt{\Call{CAS}{n.dir.super[c\textsubscript{dir}], null, h+1}}
\State \texttt{\Call{CAS}{n.dir.counter, c\textsubscript{dir}, c\textsubscript{dir}+1}}
\EndFor
\State \texttt{\Call{CAS}{n.head, h, h+1}}
\State \Return{\texttt{true}}
\Else
\State \texttt{\Call{CAS}{n.head, h, h+1}}
\State \Return{ \texttt{false}}
\EndIf
\EndFunction{Refresh}

\Statex

\Function{element}{Get}{\textsl{int} i}
\Comment{Returns $i$th Enqueue.}
\If{\texttt{i }\textbf{is }\texttt{null}}
\State \Return \texttt{null}
\EndIf
\State \texttt{res= root.blocks.get(sum\textsubscript{enq}==i).order}
\State \Return{\Call{Get}{root, res, i-root.blocks[res-1].sum\textsubscript{enq}}}
\EndFunction{Get}

\Statex

\Statex $\leadsto$ \textsf{Precondition: \texttt{n.blocks[start..end]} contains a block with field \texttt{f} $\geq$ \texttt{i}}
\Function{int}{BSearch}{\textsl{node} n, \textsl{field} f, \textsl{int} i, \textsl{int} start, \textsl{int} end}

\Statex \Comment{\textmd{Does binary search for~the value \texttt{i} of the given prefix sum \texttt{feild}. Returns the index of the leftmost block in \texttt{n.blocks[start..end]} whose \textsl{field} \texttt{f} is $\geq$ \texttt{i}}.}
%\State \Return \texttt{result block's index}
\EndFunction{BSearch}

\pagebreak

\Function{<Block, int, int>}{CreateBlock}{\textsl{node} n, \textsl{int} i} 
\Statex\Comment{Creates a block to insert into \texttt{n.blocks[i]}. Returns the created block as well as values read from each child counter feild.}
\State \texttt{block b= \Call{new}{\textsl{block}}}
\State \texttt{b.order= i}
\ForEach{\texttt{dir} {\keywordfont{in}} \texttt{\{left, right\}}}
\State \texttt{lastIndex= n.dir.head} \label{lastLine}
\State \texttt{prevIndex= n.blocks[i-1].end\textsubscript{dir}} \label{prevLine}
\State \texttt{lastBlock= n.dir.blocks[lastIndex]}
\State \texttt{prevBlock= n.dir.blocks[prevIndex]}
\State \texttt{c\textsubscript{dir}= n.dir.counter}
\State \texttt{b.end\textsubscript{dir}= lastIndex}
\State \texttt{b.num\textsubscript{enq-dir}= lastBlock.sum\textsubscript{enq} - prevBlock.sum\textsubscript{enq}}
\State \texttt{b.num\textsubscript{deq-dir}= lastBlock.sum\textsubscript{deq} - prevBlock.sum\textsubscript{deq}}
\State \texttt{b.sum\textsubscript{enq-dir}= n.blocks[i-1].sum\textsubscript{enq-dir} + b.num\textsubscript{enq-dir}}
\State \texttt{b.sum\textsubscript{deq-dir}= n.blocks[i-1].sum\textsubscript{deq-dir} + b.num\textsubscript{deq-dir}}
\EndFor
\State \texttt{b.num\textsubscript{enq}= b.num\textsubscript{enq-left} + b.num\textsubscript{enq-right}}
\State \texttt{b.num\textsubscript{deq}= b.num\textsubscript{deq-left} + b.num\textsubscript{deq-right}}
\State \texttt{b.num= b.num\textsubscript{enq} + b.num\textsubscript{deq}}
\State \texttt{b.sum= n.blocks[i-1].sum + b.num}

\If{\texttt{n.parent} \textbf{is} \texttt{null}}
\State \texttt{b.size= max(root.blocks[i-1].size + b.num\textsubscript{enq} - b.num\textsubscript{deq}, 0)}
\State \texttt{b.sum\textsubscript{non-null deq}= root.blocks[i-1].sum\textsubscript{non-null deq} + max( b.num\textsubscript{deq} - root.blocks[i-1].size - b.num\textsubscript{enq}, 0)}
\EndIf

\State \Return \texttt{b, c\textsubscript{left}, c\textsubscript{right}}
\EndFunction{CreateBlock}

\end{multicols}
\end{algorithmic}
\end{algorithm}

\begin{algorithm}
\begin{algorithmic}[1]
\setcounter{ALG@line}{83}

\Statex $\leadsto$ \textsf{Precondition:~\texttt{n.blocks[b]} contains $\geq$\texttt{i} enqueues.}
\Function{element}{Get}{\textsl{node} n, \textsl{int} b, \textsl{int} i} 
\Comment{\textmd{Returns the \texttt{i}th Enqueue in \texttt{b}th block of node \texttt{n}}}
\If{\texttt{n {\keywordfont is} leaf}} \Return \texttt{n.ops[b]}
\Else
\If{\texttt{i $\leq$ n.blocks[b].num\textsubscript{enq-left}}} \Comment{\texttt{i} exists in the left child of \texttt{n}}
\State \texttt{subBlock= \Call{BSearch}{n.left, sum\textsubscript{enq}, i, n.blocks[b-1].end\textsubscript{left}+1, n.blocks[b].end\textsubscript{left}}}
\State \Return\Call{Get}{n.left, subBlock, i-n.left.blocks[subBlock-1].sum\textsubscript{enq}} 
\Else
\State \texttt{i= i-n.blocks[b].num\textsubscript{enq-left}}
\State\texttt{subBlock=\Call{BSearch}{n.right, sum\textsubscript{enq}, i, n.blocks[b-1].end\textsubscript{right}+1, n.blocks[b].end\textsubscript{right}}}
\State \Return\Call{Get}{n.right, subBlock, i-n.right.blocks[subBlock-1].sum\textsubscript{enq}} 
\EndIf
\EndIf
\EndFunction{Get}

\Statex
\Statex $\leadsto$ \textsf{Precondition: \texttt{b}th block of node \texttt{n} has propagated up to the root and \texttt{i}th dequeue resides in node \texttt{n} is in block \texttt{b} of node \texttt{n}.}
\Function{<int, int>}{Index}{\textsl{node} n, \textsl{int} b, \textsl{int} i} \Comment{Returns the order in the root among dequeus, of ith dequeue in bth block of node n}
\If{\texttt{n {\keywordfont is} root}} \Return \texttt{root.blocks.get(order==b-1).sum\textsubscript{deq}+i, b}
\Else
\State \texttt{dir= (n.parent.left==n)? left: right}
\State \texttt{superBlock= \Call{BSearch}{n.parent, n.sum\textsubscript{deq-dir}, i, super[n.blocks[b].group]-p, super[n.blocks[b].group]+p}}
\If{\texttt{dir {\keywordfont is} left}}
\State \texttt{i+= n.parent.blocks[superBlock-1].sum\textsubscript{deq-right}}
\Else \State \texttt{i+= n.parent.blocks[superBlock-1].sum\textsubscript{deq} + n.blocks[superBlock].sum\textsubscript{deq-left}}
\EndIf
\State \Return\Call{Index}{n.parent, superBlock, i}
\EndIf
\EndFunction{Index}

\end{algorithmic}
\end{algorithm}


%##########################################


\begin{algorithm}
\begin{algorithmic}[1]
\begin{multicols}{2}

\Statex $\blacktriangleright$ \texttt{\textsl{PRBTree[rootBlock]}}
  \Statex \textsf{A persistant red-black tree supporting \texttt{append(b, key),get(key=i),split(j)}}. \texttt{append(b, key)} returns \texttt{true} in case successful.






\Function{void}{RBTAppend}{block b} \Comment{\textsf{adds block b to the \texttt{root.blocks}}}
\State \texttt{step= root.head}
\If{\texttt{step\%$p^2$==0}}
\State \texttt{Help()}
\State \texttt{CollectGarbage()}
\EndIf
\State \texttt{b.age= 0}
\State \Return \texttt{root.blocks.append(b, b.order)}
\EndFunction{RBTAppend}
\Statex


\Function{void}{Help}{}\Comment{Helps pending operations}
\For{\texttt{leaf l} \textbf{in leaves}}\Comment{\textit{how to iterate over them?}}
\State{\texttt{last= l.head-1}}
\If{\texttt{l.blocks[last]} \textbf{is not} \texttt{null}}
\If{\texttt{l.blocks[last].element==null}} \Comment{operation is dequeue}
\State \texttt{goto \ref{deqRest} with} \Comment{run \texttt{Dequeue()}  for \texttt{l.ops[last]} after Propagate()}
\State \texttt{l.responses[last]= response}
\EndIf
\EndIf
\EndFor
\EndFunction{Help}
\Statex

\Function{void}{CollectGrabage}{}\Comment{Collects the old root blocks.}
\State \texttt{l=FindYoungestOld(Root.Blocks.root)}
\State \texttt{t1,t2= RBT.split(l)}
\State \texttt{RBTRoot.CAS(t2.root)}
\EndFunction{CollectGrabage}

\Statex

\Function{Block}{FindYoungestOld}{b}
\State\Return  \texttt{read all maxOld values among leaves and decide the largest one} \Comment{There is no need to do a sasDK;Lnapgshot.}
\EndFunction{findYoungestOld}

\Statex

\Function{response}{FallBack}{op i}

\If{operation i in leaf l cannot find its desired RootBlock}

\State \Return \texttt{l.response[i]}
\EndIf

\EndFunction{FallBack}

\end{multicols}
\end{algorithmic}
\end{algorithm}




\end{document}






