% !TEX root =  queue.tex

\section{Detailed Proofs for Section \ref{sec::correctness}}
\label{app::tracingDetails}

Detailed proofs omitted from Section \ref{sec::correctness} for lack of space appear here.

\sumRes*
\begin{proof}
Initially, each \fld{blocks} array  contains only an empty block $B_0$ in location 0.
By definition, $E(B_0)$ and $D(B_0)$ are empty sequences.
Moreover, $B_0.\fld{sum\sub{enq}} = B_0.\fld{sum\sub{deq}} = 0$, so the claim is true.

We show that each installation of a block $B$ into some location $\var{v}.\fld{blocks}[i]$ preserves the claim,
assuming the claim holds before this installation.  We consider two cases.

If $\var{v}$ is a leaf, $B$ was created at line \ref{enqNew} or \ref{deqNew}.
For line \ref{enqNew}, $B$ represents a single enqueue, so $|E(B)|=1$ and $|D(B)|=0$.
Since $B.\fld{sum\sub{enq}}$ is set to $\var{v}.\fld{blocks}[i-1].\fld{sum\sub{enq}}+1$ and
$B.\fld{sum\sub{deq}}$ is set to $\var{v}.\fld{blocks}[i-1].\fld{sum\sub{deq}}$, the claim follows from the hypothesis.
The proof for line~\ref{deqNew}, where $B$ has a single dequeue, is similar.

Now suppose $\var{v}$ is an internal node. By the definition of subblocks in (\ref{defsubblock}) and \Cref{lem::headProgress}, the
subblocks of $\var{v}.\fld{blocks}[1..i]$ are $\var{v}.\fld{left.blocks}[1..B.\eleft]$ 
and $\var{v}.\fld{right.blocks}[1..B.\eright]$.
Thus, the enqueues in $E(\var{v}.\fld{blocks}[0])\cdots E(\var{v}.\fld{blocks}[i])$ are those in\linebreak
$E(\var{v}.\fld{left.blocks}[0]) \cdots E(\var{v}.\fld{left.blocks}[B.\eleft])$ and those in \linebreak
$E(\var{v}.\fld{left.blocks}[0]) \cdots E(\var{v}.\fld{left.blocks}[B.\eright])$.
By the hypothesis, the total number of these enqueues is \linebreak
$\var{v}.\fld{left.blocks}[B.\eleft].\fld{sum\sub{enq}} + \var{v}.\fld{right.blocks}[B.\eright].\fld{sum\sub{enq}}$, \linebreak
which is the value that line \ref{createSumEnq} stored in $B.\fld{sum\sub{enq}}$ when $B$ was created.
The proof for \fld{sum\sub{deq}} (stored on line~\ref{createSumDeq}) is similar.
\end{proof}

\superRelationRes*
\begin{proof}
We first show that $B.\fld{super}\leq s$.
Let $R_s$ be the instance of \op{Refresh}($\var{v}.\fld{parent}$) that installs $B$'s superblock 
in $\var{v}.\fld{parent.blocks}[s]$.
By the definition of subblocks (\ref{defsubblock}), $R_s$'s read $r$ of $\var{v}.head$ at line \ref{createEndLeft} or \ref{createEndRight} obtains a value greater than $b$.
By \Cref{lem::headPosition}, $B.\fld{super}$ is not $\nl$ when $r$ occurs, which means
that $B.\fld{super}$ was set (by line~\ref{setSuper1}) to a value read from $\var{v}.\fld{parent}.\head$ before $r$.
When $r$ occurs, $\var{v}.\fld{parent.blocks}[s] = \nl$, since the later \op{CAS} by $R_s$ at line
\ref{cas} succeeds.
So, by \Cref{lem::headPosition}, $\var{v}.\fld{parent}.\head \leq s$ when $r$ occurs.
Since the value stored in $B.\fld{super}$ was read from $\var{v}.\fld{parent.head}$ before $r$ and the \head\ field is non-decreasing by \Cref{nonDecreasingHead}, it follows that $B.super\leq s$.

Next, we show that $B.\fld{super}\geq s-1$.
The value stored in $B.\fld{super}$ at line \ref{setSuper1} is read from $\var{v}.\fld{parent}.\head$ at line \ref{readParentHead} and \head\ is always at least 1, so $B.\fld{super} \geq 1$.
So, if $s\leq 2$, the claim is trivial.  Assume $s>2$ for the rest of the proof.
By \Cref{lem::headProgress}, $\var{v}.\fld{parent.blocks}[s-1]$ is not $\nl$.  Let $R_{s-1}$ be the call to
$\op{Refresh}(\var{v}.\fld{parent})$ that installed the block in $\var{v}.\fld{parent.blocks}[s-1]$.
Let $r'$ be the step when $R_{s-1}$ reads $s-1$ in $\var{v}.\fld{parent}.\head$ at line \ref{readHead}.
This read $r'$ must be before $B$ is installed in $\var{v}$;
otherwise, \Cref{successfulRefresh} would imply that $B$ is a subblock of one of 
$\var{v}.\fld{parent.blocks}[1..s-1]$, contrary to the hypothesis.
Now, consider the call to \op{Advance}($\var{v}, b$) that writes $B.\fld{super}$.
It is invoked either 
at line \ref{helpAdvance} after seeing $\var{v}.\fld{blocks}[b]\neq \nl$ at line \ref{ifHeadnotNull}
or at line \ref{advance} after ensuring $\var{v}.\fld{blocks}[b]\neq \nl$ at line~\ref{cas}.
Either way, the \op{Advance} is invoked after $B$ is installed, and therefore after $r'$.
By \Cref{nonDecreasingHead}, $\var{v}.\fld{parent}.\head$ is non-decreasing, so 
the value this \op{Advance} reads in $\var{v}.\fld{parent}.\head$ and
writes in $B.\fld{super}$ is greater than or equal to the value $s-1$ that $r'$ reads in $\var{v}.\fld{parent}.\head$.
\end{proof}

\indexDequeueRes*
\begin{proof}
We prove the claim by induction on the depth of node $\var{v}$. The base case where $\var{v}$ is the root is trivial (see Line \ref{indexBaseCase}).
Assuming the claim holds for $\var{v}$'s parent, we prove it for $\var{v}$.
Let $B=\var{v}.\fld{blocks}[b]$ and $B'$ be the superblock of $B$.
\op{IndexDequeue}($\var{v}, b, i$) first computes the index $sup$ of $B'$ in $\var{v}.\fld{parent}$.
By \Cref{superRelation}, this index is either $B.super$ or $B.super+1$.
The correct index is determined by testing on line \ref{supertest} whether $B$ is a subblock of $\var{v}.\fld{parent.blocks}[B.\fld{super}]+1$.

Next, the position of the required dequeue in $D(B')$ is computed in 
lines \ref{computeISuperStart}--\ref{computeISuperEnd}. 
We first add the number of dequeues in the subblocks of $B'$ in $\var{v}$ that precede $B$ on line \ref{computeISuperStart}.
If $\var{v}$ is the right child of its parent, then all of the subblocks of $B'$ from $\var{v}$'s left sibling
also precede the required dequeue, so we add the number of dequeues in those subblocks in line \ref{considerLeftBeforeRight}.

Finally, \op{IndexDequeue} is called recursively on $\var{v}$'s parent.
Since $B$ has been propagated to the root, so has its superblock $B'$.
Thus, all preconditions of the recursive call are met.
By the induction hypothesis, the recursive call returns the location of the required dequeue in the root.\end{proof}

\getEnqRes*
\begin{proof}
We prove the claim by induction on the height of node $\var{v}$.
If $\var{v}$ is a leaf, the hypothesis implies that $i=1$ and the block $\var{v}.\fld{blocks}[b]$ represents 
an enqueue whose argument is stored in $\var{v}.\fld{blocks}[b].\fld{element}$.
\op{GetEnqueue} returns the argument of this enqueue at line \ref{getBaseCase}.

Assuming the claim holds for $\var{v}$'s children, we prove it for $\var{v}$.
Let $B$ be $\var{v}.\fld{blocks}[b]$.
By (\ref{defSeqs}),
$E(B)$ is obtained by concatenating the enqueue sequences of the direct subblocks
of $B$, which are listed in (\ref{defsubblock}).
By \Cref{lem::sum}, $\var{sum\sub{left}}-\var{prev\sub{left}}$ is the number
of enqueues in $E(B)$ that come from $B$'s subblocks in $\var{v}$'s left child.
Thus, $dir$ is set to the direction for the child of $\var{v}$ that contains the required enqueue.
Moreover, when line \ref{endChooseDir} is reached, $i$ is the position of the required enqueue within the portion $E'$ of $E(B)$ that comes from that child.
Thus,  line \ref{getChild} finds the index $b'$ of the subblock $B'$ containing the required enqueue.
By \Cref{lem::sum}, $\var{v}.\var{dir.blocks}[b'-1].\var{sum\sub{enq}} - \var{prev\sub{dir}}$ is the number of 
enqueues in $E'$ before the enqueues of block $B'$, so
the value $i'$ computed on line \ref{getChildIndex} is the position of the required enqueue within $E(B')$.
Thus, the recursive call on line \ref{getRecurse} satisfies its precondition, and 
returns the required result, by the induction hypothesis.
\end{proof}

\sizeCorrectRes*
\begin{proof}
We prove the claim by induction on $b$. 
The base case when ${b=0}$ is trivially true, since the queue is initially empty and 
$\var{root}.\fld{blocks}[0]$ contains an empty block whose \size\ field is $0$. 
Assuming the claim holds for $b-1$, we prove it for $b$.
The \fld{size} field of the block $B$ installed in $\var{root}.\fld{blocks}[b]$ is computed
at line \ref{computeLength} of a call to \op{CreateBlock}(\var{root, b}).
By the induction hypothesis, $root.\fld{blocks}[b-1].\size$ gives the size of the queue before the operations
of block $B$ are performed.
By \Cref{lem::sum}, the values of \var{num\sub{enq}} and \var{num\sub{deq}}
are the number of enqueues and dequeues contained in $B$.
Hence, the size of the queue after the operations of $B$ are performed (with enqueues before dequeues as specified by $L$)
is $\max(0, \var{root}.\fld{blocks}[b-1].\size + \var{num\sub{enq}} - \var{num\sub{deq}})$.
\end{proof}

